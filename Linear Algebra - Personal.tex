\documentclass[10pt, a4paper]{article}
\usepackage{preamble}

\title{Linear Algebra}
\author{Luke Phillips}
\date{September 2024}

\begin{document}

\maketitle

\newpage

\tableofcontents

\newpage

\section{Introduction}

\subsection{Basic properties of matrices}

If $n$ and $m$ are positive integers then by a matrix of size $m$ by $n$, or an $n \times n$ matrix, some examples of matrices are
\[
\text{size}\quad 1\times 5 : \begin{bmatrix} 10 & 9 & 8 & 7 & 6 \end{bmatrix}
\]
\[
\text{size}\quad 3\times 2 : \begin{bmatrix} 1 & 2 \\ 3 & 4 \\ 5 & 6 \end{bmatrix}
\]
We display a general $m \times n$ matrix as
\[
\begin{bmatrix}
    x_{11} & x_{12} & x_{13} & \dots & x_{1n} \\
    x_{21} & x_{22} & x_{23} & \dots & x_{2n} \\
    x_{31} & x_{32} & x_{33} & \dots & x_{3n} \\
    \vdots & \vdots & \vdots & \ddots & \vdots \\
    x_{m1} & x_{m2} & x_{m3} & \dots & x_{mn}
\end{bmatrix}
\]
The first suffix gives the number of the row and the second suffix gives the column, so $x_{ij}$ appears at the intersection of the $i$-th row and the $j$-th column. 

Often the above display is abbreviated to simply
\[
[x_{ij}]_{m\times n}
\]
and refer to $x_{ij}$ as the $(i, j)$-th element or the $(i, j)$-th entry of the matrix.

Hence we can use $X = [x_{ij}]_{m\times n}$ which means that $X$ is the $m\times n$ matrix whose $(i, j)$-th element is $x_{ij}$.

\begin{example}
    The $3\times 3$ matrix $X = \begin{bmatrix} a & a & a \\ 0 & a & a \\ 0 & 0 & a \end{bmatrix}$ can be expressed as $X = [x_{ij}]_{3\times 3}$ where
    \[
    x_{ij} = 
    \begin{cases}
        a & \text{if } i \leq j; \\
        0 & \text{otherwise}.
    \end{cases}
    \]
\end{example}

\begin{definition}
    If $A = [a_{ij}]_{m\times n}$ and $B = [b_{ij}]_{p\times q}$ then we shall say that $A$ and $B$ are equal (and write $A = B$) if and only if
    \begin{enumerate}[label = (\arabic*)]
        \item $m = p$ and $n = q$
        \item $a_{ij} = b_{ij}$ for all $i,j$.
    \end{enumerate}
\end{definition}

\begin{definition}
    Given $m\times n$ matrices $A = [a_{ij}]$ and $B = [b_{ij}]$, we define the sum $A + B$ to be the $m\times n$ matrix whose $(i, j)$-th element is $a_{ij} + b_{ij}$.
\end{definition}

\begin{theorem}\label{thm:matr_add_comm}
    Addition of matrices is
    \begin{enumerate}[label = (\arabic*)]
        \item commutative [in the sense that if $A, B$ are the same size then we have $A + B = B + A$]
        \item associative [in the sense that if $A, B, C$ are of the same size then we have $A + (B + C) = (A + B) + C$].
    \end{enumerate}
    \begin{proof}
    \phantom{}
    \begin{enumerate}[label = (\arabic*)]
        \item If $A$ and $B$ are each of size $m\times n$ then $A + B$ and $B + A$ are both of size $m\times n$ by the definition of matrix addition
        \[
        A + B = [a_{ij} + b_{ij}],\qquad B + A = [b_{ij} + a_{ij}].
        \]
        Since addition is commutative $a_{ij} + b_{ij} = b_{ij} + a_{ij}$, for all $i, j$ and so, by the definition of equality for matrices, we can conclude that $A + B = B + A$.
        \item If $A, B$ and $C$ are each of size $m\times n$ then $A + (B + C)$ and $(A + B) + C$ are both of size $m\times n$. Now the $(i, j)$-th element of $A + (B + C)$ is $a_{ij} + (b_{ij} + c_{ij})$ and the $(i, j)$-th element of $(A + B) + C$ is $(a_{ij} + b_{ij}) + c_{ij}$. Since addition is associative $a_{ij} + (b_{ij} + c_{ij}) = (a_{ij} + b_{ij}) + c_{ij}$ for all $i, j$ and so, by the definition of equality for matrices, we can conclude that $A + (B + C) = (A + B) + C$.
    \end{enumerate}
    \end{proof}
\end{theorem}

\begin{theorem}\label{thm:matr_add_identity}
    There is a unique $m\times n$ matrix $M$ such that, for every $m\times n$ matrix $A$, $A + M = A$.
    \begin{proof}
        Consider the matrix $M = [m_{ij}]_{m\times n}$ all of whose entries are $0$; i.e. $m_{ij} = 0$ for all $i, j$. For every matrix $A = [a_{ij}]_{m\times n}$ we have
        \[
        A + M = [a_{ij} + m_{ij}]_{m\times n} = [a_{ij} + 0]_{m\times n} = [a_{ij}]_{m\times n} = A.
        \]
        Suppose that $B = [b_{ij}]_{m\times n}$ is also such that $A + B = A$ for every matrix $m\times n$ matrix $A$. In particular we have $M + B = M$. But, taking $B$ instead of $A$ in the property for $M$, we have $B + M = B$. It follows by \autoref{thm:matr_add_comm}(1) that $B = M$.
    \end{proof}
\end{theorem}

\begin{definition}
    The unique matrix arising from \autoref{thm:matr_add_identity} is called the $m\times n$ zero matrix and will be denoted by $0_{m\times n}$, or simply by $0$ if no confusion arises.
\end{definition}

\begin{theorem}\label{thm:matr_add_inv}
    For every $m\times n$ matrix $A$ there is a unique $m\times n$ matrix $B$ such that $A + B = 0$.

    \begin{proof}
        Given $A = [a_{ij}]_{m\times n}$, consider $B = [b_{ij}]_{m\times n}$. Clearly, we have
        \[
        A + B = [a_{ij} + (-a_{ij})]_{m\times n} = 0.
        \]
        To establish uniqueness of such a matrix $B$, suppose that $C = [c_{ij}]_{m\times n}$ is also such that $A + C = 0$. Then for all $i, j$ we have $a_{ij} + c_{ij} = 0$ and consequently $c_{ij} = -a_{ij}$ which means, by the above definition of equality, that $C = B$.
    \end{proof}
\end{theorem}

\begin{definition}
    The unique matrix $B$ arising from \autoref{thm:matr_add_inv} is called the additive inverse of $A$ denoted by $-A$. Thus $-A$ is the matrix whose elements are the additive inverses of the corresponding elements of $A$.
\end{definition}

\begin{definition}
    Given a matrix $A$ and a number $\lambda$ we define the product of $A$ by $\lambda$ to be the matrix denoted by $\lambda A$, that is obtained from $A$ by multiplying every element of $A$ by $\lambda$.

    Thus, if $A = [a_{ij}]_{m\times n}$ then $\lambda A = [\lambda a_{ij}]_{m\times n}$.
\end{definition}

\begin{theorem}
    If $A$ and $B$ are $m\times n$ matrices then, for any scalars $\lambda$ and $\mu$,
    \begin{enumerate}[label = (\arabic*)]
        \item $\lambda (A + B) = \lambda A + \lambda B$
        \item $(\lambda  + \mu)A = \lambda A + \mu A$
        \item $\lambda (\mu A) = (\lambda \mu) A$
        \item $(-1)A = -A$
        \item $0A = 0_{m\times n}$.
    \end{enumerate}
    \begin{proof}
        Let $A = [a_{ij}]_{m\times n}$ and $B = [b_{ij}]_{m\times n}$. Then the above equalities follow from the following observations
        \begin{enumerate}[label = (\arabic*)]
            \item $\lambda (a_{ij} + b_{ij}) = \lambda a_{ij} + \lambda b_{ij}$
            \item $(\lambda  + \mu)a_{ij} = \lambda a_{ij} + \mu a_{ij}$
            \item $\lambda (\mu a_{ij}) = (\lambda \mu) a_{ij}$
            \item $(-1)a_{ij} = -a_{ij}$
            \item $0a_{ij} = 0$.
        \end{enumerate}
    \end{proof}
\end{theorem}

\begin{definition}
    Let $A = [a_{ij}]_{m\times n}$ and $B = [b_{ij}]_{n\times p}$. Then we define the product $AB$ to be the $m\times p$ matrix whose $(i, j)$-th element is
    \[
    [AB]_{ij} = a_{i1}b_{1j} +a_{i2}b_{2j} + a_{i3}b_{3j} + \dotsc + a_{in}b_{nj}.
    \]
    The $(i, j)$-th element of the product $AB$ is obtained by summing the products of the elements of the $i$-th row of $A$ with the corresponding elements in the $j$-th column of $B$\footnote{It is easier to remember this as "rows by columns".}.
\end{definition}
The above expression for $[AB]_{ij}$ can be abbreviated to the following
\[
[AB]_{ij} = \sum_{k = 1}^{n}{a_{ik}b_{kj}}.
\]

\begin{theorem}
    Matrix multiplication is associative [when the products are defined, $A(BC) = (AB)C$].

    \begin{proof}
        For $A(BC)$ to be defined we require the respective sizes to be $m\times n,\, n\times p,\, p\times q$ in which case the product $A(BC)$ is also defined. Computing the $(i, j)$-th element of this product, we obtain
        \begin{align*}
            [A(BC)]_{ij} &= \sum_{k = 1}^{n}{a_{ik}[BC]_{kj}} \\
            &= \sum_{k = 1}^{n}{a_{ik}\left(\sum_{l = 1}^{p}{b_{kl}c_{lj}}\right)} \\
            &= \sum_{k = 1}^{n}{\sum_{l = 1}^{p}{a_{ik}b_{kl}c_{lj}}}
        \end{align*}
        Similarly computing the $(i, j)$-th element of the $(AB)C$, we obtain the same
        \begin{align*}
            [(AB)C]_{ij} &= \sum_{l = 1}^{p} [AB]_{il}c_{lj} \\
            &= \sum_{l = 1}^{p}\left(\sum_{k = 1}^{n} a_{il} b_{kl}\right)c_{kj} \\
            &= \sum_{l = 1}^{p}{\sum_{k = 1}^{n}{a_{il}b_{kl}c_{kj}}}
        \end{align*}
    \end{proof}
    Consequently we see that $A(BC) = (AB)C$.
\end{theorem}

\end{document}