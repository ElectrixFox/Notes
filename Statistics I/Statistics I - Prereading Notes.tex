\documentclass[10pt, a4paper]{article}
\usepackage{preamble}

\title{Statistics I \\
    \large Prereading}
\author{Luke Phillips}
\date{January 2025}

\begin{document}

\maketitle

\newpage

\tableofcontents

\newpage

\section{Introduction to Statistics}

\subsection{Foundations of Statistics and Interpretation of Probability}

\subsubsection{Probability: Revision}

\begin{definition}[Axioms of Probability]
    For a sample space $\Omega$,
    with collection $\mathcal{F}$ of events,
    the probability $\P(A)$ satisfies the axioms:
    \begin{enumerate}[label = A\arabic*]
        \item $\P(A) \geq 0$,
        for every $A \in \mathcal{F}$.

        \item $\P(\Omega) = 1$.

        \item For $A$ and $B$ disjoint then:
        \[
        \P(A \cup B) = \P(A) + \P(B)
        \]
    \end{enumerate}
\end{definition}

The axioms lead to some of the following consequences
\begin{proposition}
    \begin{enumerate}[label = (\roman*)]
        \item $0 \leq \P(A) \leq 1$.
        
        \item $\P(A ^ c) = 1 - \P(A)$.
        
        \item $\P(\emptyset) = 0$.
    \end{enumerate}
\end{proposition}

\subsubsection{Interpretations of Probability}
\textbf{Classical}

This is based on an assumption underlying equally likely events.

\textbf{Frequentist}

An event $A$ has probability $\P(A)$ given by:
\[
\P(A) = \liminfty\frac{n_A}{n},
\]
where $n_A$ is the number of times event $A$ occurred in $n$ repetitions of the experiment.

\textbf{Subjective}

Probabilities are viewed as subjective about the likelihood of an event $A$ occurring.
This can be defined in a precise way and Probability Axioms are actually derived as requirements of coherence.

\subsubsection{A Summary of Useful Probability Results}
\underline{Combining Events}

The event $A, B$ or both occur is $A \cup B$.
The event $A$ and $B$ occur is $A \cap B$.
These are related by the following rule
\[
\P(A \cup B) = \P(A) + \P(B) - \P(A \cap B).
\]
Events are disjoint if they cannot occur at the same time.

Addition law for disjoint events:
If $A$ and $B$ are disjoint events
\[
\P(A \cup B) = \P(A) + \P(B).
\]

\underline{Conditional Probability and Independence}

For any two events $A, B$,
the notation $\P(A | B)$ means the conditional probability that $A$ occurs,
assuming that the event $B$ has already occurred.

Conditional probability is obtained directly or by using the conditional probability rule:
\[
\P(A | B) = \frac{\P(A | B)}{\P(B)},
\]
for $\P(B) > 0$.

Rearranging we get the general multiplication rule
\[
\P(A | B) = \P(A | B)\P(B).
\]

Two events are independent when the occurrence of one has no bearing on the occurrence of the other.
If $A, B$ are independent then
\[
\P(A | B) = \P(A).
\]

\underline{Partitions}

Suppose $E_1, \dotsc, E_n$ are mutually disjoint events,
and suppose exactly one must happen.
Such a collection of events is called a
(sure)
partition.

We can write any other event $A$ in combination with this event:
in general,
\[
\P(A) = \P(A \cap E_1) + \P(A \cap E_2) + \dotsc + \P(A \cap E_n),
\]
which simplifies to
\[
\P(A) = \P(A | E_1)\P(E_1) + \dotsc + \P(A | E_n)\P(E_n)
\]
using the multiplication rule.

\underline{Bayes Theorem}

For any two events $A, B$,
the multiplication rule gives the formula
\[
\P(A \cap B) = \P(A | B)\P(B).
\]
Another equivalent formula is obviously
\[
\P(A \cap B) = \P(B | A)\P(A).
\]

Equating these we get the formula known as Bayes theorem:
\[
\P(A | B) = \frac{\P(B | A)\P(A)}{\P(B)}.
\]
Often the probability in the denominator must be calculated using the simplifying method shown in the last section;
i.e. via a partition.






\end{document}