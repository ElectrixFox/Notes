\documentclass[10pt, a4paper]{article}
\usepackage{preamble}

\title{Statistics I}
\author{Luke Phillips}
\date{January 2025}

\begin{document}

\maketitle

\newpage

\tableofcontents

\newpage

\section{}

\subsection{High Profile Applications}
\textbf{Useless films}.

\subsection{What is Statistics?}
Statistics involves the mathematical representation of key real world quantities of interest and their associated uncertainties
(often,
but not always using probability),
the coherent incorporation of any
(uncertain)
knowledge,
information or observed data into this framework and the subsequent learning,
prediction,
future experimental design and decision making in the presence of uncertainty that this structure facilitates.

\subsection{Frequentists and Bayesian Statistics}
Bayesian statistics requires more input.

The relative frequency interpretation of probability is linked to frequentist statistics.

The subjective interpretation of probability is linked to frequentist Bayesian statistics.

\textit{Genuine waffle for about half an hour}.

\begin{example}[Motivating example 1 - Covid-19 Disease Test]
    A new test for Covid-19 has been developed.
    It is fast and cheap,
    but has moderate accuracy.
    It has been tested on a limited set of people with known Covid status.
    
    You are selected at random from the UK population in August $2020$ and you test positive.
    What is the probability you have Covid?
    How about if you were selected from the London population in January $2021$.
\end{example}

\begin{example}[Motivating example 2 - US Presidential Election polling]
    $22$nd of October $2024$,
    two weeks before the $2024$ US presidential election.
    You run a polling company and you have conducted a poll of $1000$ people form Pennsylvania state.
    Out of $1000$ people:
    $485$ said they'd vote for Harris,
    $515$ said they'd vote for Trump.
    Everyone wants to know your prediction.
\end{example}




















\end{document}