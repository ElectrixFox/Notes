\documentclass[10pt, a4paper]{article}
\usepackage{preamble}

\title{Statistics I}
\author{Luke Phillips}
\date{January 2025}

\begin{document}

\maketitle

\newpage

\tableofcontents

\newpage

\section{}

\subsection{High Profile Applications}
\textbf{Useless films}.

\subsection{What is Statistics?}
Statistics involves the mathematical representation of key real world quantities of interest and their associated uncertainties
(often,
but not always using probability),
the coherent incorporation of any
(uncertain)
knowledge,
information or observed data into this framework and the subsequent learning,
prediction,
future experimental design and decision making in the presence of uncertainty that this structure facilitates.

\subsection{Frequentists and Bayesian Statistics}
Bayesian statistics requires more input.

The relative frequency interpretation of probability is linked to frequentist statistics.

The subjective interpretation of probability is linked to frequentist Bayesian statistics.

\textit{Genuine waffle for about half an hour}.

\begin{example}[Motivating example 1 - Covid-19 Disease Test]
    A new test for Covid-19 has been developed.
    It is fast and cheap,
    but has moderate accuracy.
    It has been tested on a limited set of people with known Covid status.
    
    You are selected at random from the UK population in August $2020$ and you test positive.
    What is the probability you have Covid?
    How about if you were selected from the London population in January $2021$.
\end{example}

\begin{example}[Motivating example 2 - US Presidential Election polling]
    $22$nd of October $2024$,
    two weeks before the $2024$ US presidential election.
    You run a polling company and you have conducted a poll of $1000$ people form Pennsylvania state.
    Out of $1000$ people:
    $485$ said they'd vote for Harris,
    $515$ said they'd vote for Trump.
    Everyone wants to know your prediction.
\end{example}

\newpage

\section{Disease Testing using Bayesian Methods}

\subsection{Testing for Covid-19}

Remember motivating example $1$.
\begin{enumerate}[label = (\roman*)]
    \item 
    We have data from a limited sample of $620$ patients given the new test is the 'gold standard'
    (i.e. perfect).

    Events
    \begin{align*}
        D ^ {+} &= \text{person has disease}. \\
        D ^ {-} &= \text{person doesn't have disease}. \\
        T ^ {+} &= \text{person tests positive}. \\
        T ^ {-} &= \text{person tests negative}.
    \end{align*}
    \begin{table}[H]
        \begin{tabular}{c|c|c|c}
             & Yes $D ^ {+}$ & No $D ^ {-}$ & Total \\
             \hline
             Test result positive $T ^ {+}$ & $209$ & $6$ & $215$ \\
             Test result negative $T ^ {-}$ & $11$ & $394$ & $405$ \\
             \hline
             Total & $220$ & $400$ & $620$
        \end{tabular}
    \end{table}

    Probability distribution
    \begin{align*}
        \P_t &= \text{probability distribution for test group of $620$ people}. \\
        \P &= \text{probability distribution of general UK population}.
    \end{align*}

    Probabilities of interest
    \begin{align*}
        \P_t\left(T ^ {+} \mmid D ^ {+}\right) &= \text{sensitivity of the test} \\
        \P_t\left(T ^ {-} \mmid D ^ {-}\right) &= \text{specificity of the test} \\
        \P_t\left(T ^ {+} \mmid D ^ {-}\right) &= \text{probability of a false positive} \\
        \P_t\left(T ^ {-} \mmid D ^ {+}\right) &= \text{probability of a false negative}
    \end{align*}

    Sensitivity:
    \begin{align*}
        \P_t\left(T ^ {+} \mmid D ^ {+}\right) &= \frac{\P_t(T ^ {+} \cap D ^ {+})}{\P_t(D ^ {+})} \\
        &= \frac{\frac{209}{620}}{\frac{270}{620}} \\
        &= \frac{209}{270} \\
        &= 0.95.
    \end{align*}

    Specificity:
    \begin{align*}
        \P_t\left(T ^ {-} \mmid D ^ {-}\right) &= \frac{\P_t(T ^ {-} \cap D ^ {-})}{\P_t(D ^ {-}} \\
        &= \frac{\frac{396}{620}}{\frac{400}{620}} \\
        &= \frac{396}{400} \\
        &= 0.8985.
    \end{align*}

    False positive:
    \begin{align*}
        \P_t\left(T ^ {+} \mmid D ^ {-}\right) &= 1 - \P_t\left(T ^ {-} \mmid D ^ {-}\right) \\
        &= 1 - 0.8985 \\
        &= 0.015.
    \end{align*}

    False negative:
    \begin{align*}
        \P_t\left(T ^ {-} \mmid D ^ {+}\right) &= 1 - \P_t\left(T ^ {+} \mmid D ^ {+}\right) \\
        &= 1 - 0.95 \\
        &= 0.05.
    \end{align*}

    Question:

    We care about the whole U.K. population.
    How does it relate to our test set?

    Assumption

    The sensitivity and specificity are a property of the test only.

    Hence
    \begin{align*}
        \P\left(T ^ {+} \mmid D ^ {+}\right) &= \P_t\left(T ^ {+} \mmid D ^ {+}\right). \\
        \P\left(T ^ {-} \mmid D ^ {-}\right) &= \P_t\left(T ^ {-} \mmid D ^ {-}\right). \\
        \P\left(T ^ {+} \mmid D ^ {-}\right) &= \P_t\left(T ^ {+} \mmid D ^ {-}\right). \\
        \P\left(T ^ {-} \mmid D ^ {+}\right) &= \P_t\left(T ^ {-} \mmid D ^ {+}\right).
    \end{align*}
    This is not true for other probabilities such as $\P(D ^ {+}) \neq \P_t(D ^ {+})$.
\end{enumerate}

\subsection{Bayesian Inference: from prior to posterior}
\begin{definition}
    Prior,
    likelihood,
    posterior.
    \[
    \P(D ^ {+}) = \text{prior probability of having Covid-19.\footnotemark}
    \]
    \footnotetext{Prevalence in the relevant population.}
    \[
    \P\left(T ^ {+} \mmid D ^ {+}\right), \P\left(T ^ {+} \mmid D ^ {-}\right) = \text{likelihood of positive test given disease states.}
    \]
    \[
    \P\left(D ^ {+} \mmid T ^ {+}\right) = \text{posterior probability of having Covid-19 given a positive test.}
    \]
    The posterior is critical and found using Bayes Theorem:
    \begin{align*}
        \P\left(D ^ {+} \mmid T ^ {+}\right) &= \frac{\P\left(T ^ {+} \mmid D ^ {+}\right)\P\left(D ^ {+}\right)}{\P\left(T ^ {+}\right)} \quad\text{(by Bayes theorem)} \\
        &= \frac{\P\left(T ^ {+} \mmid D ^ {+}\right)\P\left(D ^ {+}\right)}{\P\left(T ^ {+} \mmid D ^ {+}\right)\P\left(D ^ {+}\right) + \P\left(T ^ {+} \mmid D ^ {-}\right)\P\left(D ^ {-}\right)} \quad\text{(by partition)}\\
        &= \frac{\P_t\left(T ^ {+} \mmid D ^ {+}\right)\P\left(D ^ {+}\right)}{\P_t\left(T ^ {+} \mmid D ^ {+}\right)\P\left(D ^ {+}\right) + \P_t\left(T ^ {+} \mmid D ^ {-}\right)\P\left(D ^ {-}\right)} \quad\text{(by assumptions)} \\
    \end{align*}
    Covid-19 Example:
    August $2020$ U.K.

    In August $2020$ U.K. it was thought that $0.00025$ of the U.K. population had Covid-19.
    \[
    \P(D ^ {+}) = 0.00025
    \]
    and
    \[
    \P(D ^ {-}) = 1 - \P(D ^ {+})
    \]
    \begin{align*}
        \P\left(D ^ {+} \mmid T ^ {+}\right) &= \frac{\P\left(T ^ {+} \mmid D ^ {+}\right)\P\left(D ^ {+}\right)}{\P\left(T ^ {+}\right)} \\
        &= \frac{\P\left(T ^ {+} \mmid D ^ {+}\right)\P\left(D ^ {+}\right)}{\P\left(T ^ {+} \mmid D ^ {+}\right)\P\left(D ^ {+}\right) + \P\left(T ^ {+} \mmid D ^ {-}\right)\P\left(D ^ {-}\right)} \\
        &= \frac{0.95 \times 0.00025}{0.95 \times 0.00025 + 0.015 \times (1 - 0.00025)} \\
        &= \frac{0.0002375}{0.0002375 + 0.01499625} \\
        &= 0.0156.
    \end{align*}
\end{definition}































\end{document}