\documentclass[10pt, a4paper]{article}
\usepackage{amsmath, amsfonts, amssymb, amsthm}
\usepackage{hyperref}
\usepackage{enumitem}
\usepackage{thmtools}
\usepackage{parskip}
\usepackage{microtype}
\usepackage{tikz}
\usepackage{pgfplots}
\pgfplotsset{compat=1.18}

\newcommand{\N}{\mathbb{N}}
\newcommand{\Z}{\mathbb{Z}}
\newcommand{\R}{\mathbb{R}}
\newcommand{\limas}[3][n]{#2 \rightarrow #3 \text{ as } #1 \rightarrow \infty}
\newcommand{\sumfrto}[3][n = 1]{\sum_{#1}^{#2}{#3}} % this is for set start
\newcommand{\sumto}[2][\infty]{\sumfrto{#1}{#2}}
\newcommand{\seq}[1][x_n]{\left\langle #1 \right\rangle}

\declaretheoremstyle[
    notefont = \normalfont\itshape,
    notebraces={(}{)}
]{avgstyle}
\declaretheoremstyle[
    bodyfont = \normalfont,
    notefont = \normalfont\itshape,
    spaceabove = 1em,
    spacebelow = 1em
]{defstyle}
\declaretheoremstyle[
    bodyfont = \normalfont,
    notefont = \normalfont,
    spaceabove = 1em,
    spacebelow = 1em
]{exampstyle}

\declaretheorem[style = avgstyle]{theorem}
\declaretheorem[]{lemma}
\declaretheorem{proposition, corollary, remark}[style=plain]
\declaretheorem{definition}[style=defstyle, numbered=unless unique]
\declaretheorem{outline}[style=plain, numbered=no]
\declaretheorem{example}[style=exampstyle, numberwithin=section]

\title{Linear Algebra}
\author{Luke Phillips}
\date{September 2024}

\begin{document}

\maketitle

\section{Introduction}

If $n$ and $m$ are positive integers then by a matrix of size $m$ by $n$, or an $n \times n$ matrix, some examples of matrices are
\[
\text{size}\quad 1\times 5 : \begin{bmatrix} 10 & 9 & 8 & 7 & 6 \end{bmatrix}
\]
\[
\text{size}\quad 3\times 2 : \begin{bmatrix} 1 & 2 \\ 3 & 4 \\ 5 & 6 \end{bmatrix}
\]
We display a general $m \times n$ matrix as
\[
\begin{bmatrix}
    x_{11} & x_{12} & x_{13} & \dots & x_{1n} \\
    x_{21} & x_{22} & x_{23} & \dots & x_{2n} \\
    x_{31} & x_{32} & x_{33} & \dots & x_{3n} \\
    \vdots & \vdots & \vdots & \ddots & \vdots \\
    x_{m1} & x_{m2} & x_{m3} & \dots & x_{mn}
\end{bmatrix}
\]
The first suffix gives the number of the row and the second suffix gives the column, so $x_{ij}$ appears at the intersection of the $i$-th row and the $j$-th column. 

Often the above display is abbreviated to simply
\[
[x_{ij}]_{m\times n}
\]
and refer to $x_{ij}$ as the $(i, j)$-th element or the $(i, j)$-th entry of the matrix.


\end{document}