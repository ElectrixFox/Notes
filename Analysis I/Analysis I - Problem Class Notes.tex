\documentclass[10pt, a4paper]{article}
\usepackage{preamble}

\title{Analysis I \\
    \large Problem Class}
\author{Luke Phillips}
\date{Feburary 2025}

\begin{document}

\maketitle

\newpage

\tableofcontents

\newpage

\section{Problems Class \texorpdfstring{$1$}{}}

\begin{problem}
    Let
    \[
    f(x) = \begin{cases}
        e ^ {-\frac{1}{x ^ 2}} & \text{if } x \neq 0 \\
        0 & \text{if } x = 0
    \end{cases}.
    \]
    Show
    (by induction)
    that there exists for each $n \in \N$ a polynomial $p_n(x)$ such that for the $n$-th derivative we have
    \[
    f ^ {(n)}(x) = p_n\left(\frac{1}{x}\right)e ^ {-\frac{1}{x ^ 2}}
    \]
    for $x \neq 0$.
    Use this to compute to show that $f$ is also infinitely often differentiable at $x = 0$ with $f ^ {(n)}(0) = 0$ for all $n$.

    \begin{solution}
        For $x \neq 0$.
        \begin{align*}
            f'(x) &= \lim_{x \rightarrow 0}\frac{f(x) - f(0)}{x - 0} \\
            &= \frac{x \rightarrow 0}{\frac{e ^ {-\frac{1}{x ^ 2}}}{x}} \\
            &= \lim_{x \rightarrow \pm\infty}\frac{e ^ {-x ^ 2}}{\frac{1}{x ^ 2}} \\
            &= \lim_{x \rightarrow \pm\infty}x ^ 2e ^ {-x ^ 2} = 0
        \end{align*}
        exponentials beat powers.

        $n = 0$:
        $p_0(x) = 1$,
        $n = 1$:
        $p_1(x) = 2x ^ 3$.

        at $n + 1$
        \begin{align*}
            \left[p_n\left(\frac{1}{x}\right)e ^ {-\frac{1}{x ^ 2}}\right]' &= \left[-\frac{1}{x ^ 2}p_n'\left(\frac{1}{x}\right) + p_n\left(\frac{1}{x}\right)2x ^ {-3}\right]e ^ {-\frac{1}{x ^ 2}}
        \end{align*}
        this is a polynomial in $\frac{1}{x}$,
        $p_{n + 1}(x) = -x ^ 2p_n'(x) + p_n(x)2x ^ 3$
        all at $x \neq 0$.

        $x = 0$.

        Induction:
        $n = 0$,
        $n = 1$ are fine.

        \begin{align*}
            f ^ {(n + 1)}(0) &= \lim_{x \rightarrow 0}\frac{p_n\left(\frac{1}{x}\right)e ^ {-\frac{1}{x ^ 2}} - 0}{x - 0} \\
            &= \lim_{x \rightarrow \pm\infty}xp_n(x)e ^ {-x ^ 2} \\
            &= 0
        \end{align*}
        since exponentials beat polynomials.
    \end{solution}
\end{problem}

\begin{problem}
    Know $\log(x)$ is differentiable as inverse function of $e ^ x$ with derivative
    \[
    \log'(x) = \frac{1}{e ^ {\log{x}}} = \frac{1}{x}.
    \]

    \begin{align*}
        \lim_{h \rightarrow 0}\frac{\log(x + h) - \log{x}}{h} &= \lim_{h \rightarrow 0}\frac{\log\left(\frac{x + h}{h}\right)}{h} \\
        &= \lim_{h \rightarrow 0}\frac{\log\left(1 + \frac{x}{h}\right)}{h}
        \intertext{if $h > 0$
        ($h < 0$ goes same)} \\
        \frac{\frac{\frac{h}{x}}{1 + \frac{h}{x}}}{h} &\leq \lim_{h \rightarrow 0}\frac{\log\left(1 + \frac{x}{h}\right)}{h} \leq \frac{\frac{h}{x}}{h}
    \end{align*}
    by squeezing it goes to $\frac{1}{x}$.
\end{problem}

\newpage

\section{Problems Class \texorpdfstring{$2$}{}}

\begin{problem}[$1$]
    \begin{enumerate}[label = (\roman*)]
        \item Calculate the radius of convergence $R$ of the power series $\sum[(-1) ^ n /  2 ^ n]x ^ {2n}$.

        \item In general,
        if $\sum a_nx ^ n$ has finite radius of convergence $R$,
        what is the radius of convergence of $\sum a_nx ^ {2n}$?
        (Give a proof of your answer).
    \end{enumerate}

    \begin{solution}
        \begin{enumerate}[label = (\roman*)]
            \item Root test:
            \[
            \sqrt[n]{\frac{|x| ^ {2n}}{2 ^ n}} = \frac{|x| ^ 2}{2}
            \]
            which as $n \rightarrow \infty$ is equal to $\frac{|x| ^ 2}{2} < 1$,
            this is equivalent to $|x| < \sqrt{2}$.
            So $R = \sqrt{2}$.

            \item
            $\sum a_nx ^ n$ $R < \infty$,
            hence this converges if $|x| < R$,
            diverges if $|x| > R$.
            Then
            \[
            \sum a_nx ^ {2n} = \sum a_n(x ^ 2) ^ n
            \]
            converges if $x ^ 2 < R$,
            i.e. $|x| < \sqrt{R}$,
            diverges if $x ^ 2 > R$,
            i.e. $|x| > \sqrt{R}$.
        \end{enumerate}
    \end{solution}
\end{problem}

\begin{problem}[5]
    Let $\sum a_nx ^ n$ and $\sum b_nx ^ n$ with $|a_n| \leq b_n$.
    Show that the radius of convergence of $\sum b_nx ^ n$ must be smaller or equal to the radius of convergence of $\sum a_nx ^ n$.

    \begin{proof}[Proof $1$]
        Let $\sum a_nx ^ n$ have radius of convergence $R_1$,
        and likewise $R_2$ for $\sum b_nx ^ n$.
        We have
        \[
        |a_n| < b_n.
        \]

        \[
        R_1 = \frac{1}{\limsup\sqrt[n]{|a_n|}}
        \]
        and
        \[
        R_2 = \frac{1}{\limsup\sqrt[n]{|b_n|}}
        \]
        the $\limsup{\sqrt[n]{|a_n|}} \leq \limsup{\sqrt[n]{|b_n|}}$ hence $R_1 \geq R_2$. 
    \end{proof}

    \begin{proof}[Proof $2$]
        Take $x \in (-R_2, R_2)$ such that
        \[
        \infsum[n = 0]b_nx ^ n
        \]
        converges,
        and $\infsum[n = 0]b_n|x| ^ n$ converges.
        By comparison
        \[
        \infsum[n = 0]|a_n||x| ^ n
        \]
        converges.
        So $|x| \leq R_1$.

       Since $|x| < R_2$ and $|x| \leq R_1$ so $R_2 \leq R_1$.
    \end{proof}
\end{problem}

\begin{problem}[6]
    Show $\infsum[n = 0]x ^ {2n + 1} = \frac{x}{1 - x ^ 2}$ for $|x| < 1$ and use this to explicitly compute
    \[
    \infsum[n = 0](2n + 1)\left(\frac{1}{2}\right) ^ {2n + 1}.
    \]

    \begin{solution}
        \begin{align*}
            \infsum[n = 0]x ^ {2n + 1} &= x\infsum[n = 0](x ^ 2) ^ n \\
            &= \frac{x}{1 - x ^ 2}
        \end{align*}
        for $|x ^ 2| < 1$ i.e. $|x| < 1$.

        Need to evaluate
        \[
        \sum(2n + 1)x ^ {2n + 1}\text{ at $x = \frac{1}{2}$}.
        \]
        Let
        \[
        f(x) = \infsum[n = 0]x ^ {2n + 1}
        \]
        \begin{align*}
            f'(x) &= \infsum[n = 0](2n + 1)x ^ {2n} \\
            &= \frac{1}{x}\infsum[n = 0](2n + 1)x ^ {2n + 1}.
        \end{align*}
        But also
        \[
        f'(x) = \left(\frac{x}{1 - x ^ 2}\right)' = \frac{1 - x ^ 2 - (-2x)}{(1 - x ^ 2) ^ 2} = \frac{1 + x ^ 2}{(1 - x ^ 2) ^ 2}.
        \]
        So
        \[
        \infsum[n = 0](2n + 1)x ^ {2n + 1} = x \cdot \frac{1 + x ^ 2}{(1 - x ^ 2) ^ 2}.
        \]
        Now put in $x = \frac{1}{2}$.
    \end{solution}
\end{problem}

\begin{problem}[6]
    Prove that
    \[
    \frac{1}{(n + 1)!} + \frac{1}{(n + 2)!} + \dotsc < \frac{1}{(n + 1)!}\cdot\frac{n + 2}{n + 1}
    \]
    for all $n \in \N$.
    Deduce that
    \[
    0 < e - \left(1 + \frac{1}{1!} + \frac{1}{2!} + \frac{1}{3!} + \frac{1}{4!}\right) < \frac{1}{100}
    \]
    and conclude that $2.7083 < 2.7184$.

    Use Lagrange remainder in Taylor's Theorem to show the weaker inequality
    \[
    \frac{1}{(n + 1)!} + \frac{1}{(n + 2)!} + \dotsc < \frac{3}{(n + 1)!}.
    \]
    What do you then get as an estimate for $e$?

    \begin{solution}
        \[
        e ^ x = \exp(x) = \infsum[n = 0]\frac{x ^ n}{n!}
        \]
        \begin{align*}
            e &= e ^ 1 \\
            &= \sum\frac{1}{n!} \\
            &= 1 + \frac{1}{1!} + \frac{1}{2!} + \frac{1}{3!} + \frac{1}{4!} + \dotsc.
        \end{align*}

        Lagrange remainder
        \[
        e = e ^ 1 = \infsumo\frac{1}{k!} + \frac{e ^ {\xi}}{(k + 1)!}1 ^ {k + 1}
        \]
        with $\xi \in (0, 1)$.
        So
        \begin{align*}
            \left|e - \infsumo\frac{1}{k!}\right| &= \frac{e ^ {\xi}}{(n + 1)!} \\
            &< \frac{e ^ 1}{(n + 1)!} \\
            &< \frac{3}{(n + 1)!}.
        \end{align*}

        We can see $\frac{3}{(n + 1)!} = \infsum[k = n + 1]\frac{1}{k!}$.

        For $n = 4$,
        the error is at most $\frac{3}{5!} = \frac{3}{120} = \frac{1}{40}$.
        So
        \[
        \frac{65}{24} < e < \frac{65}{24} + \frac{1}{40}.
        \]

        For the final inequality.
        \begin{align*}
            \infsum[k = n + 1]\frac{1}{k!} &= \frac{1}{(n + 1)!}\left(1 + \frac{1}{n + 2} + \frac{1}{(n + 2)(n + 3)} + \frac{1}{(n + 2)(n + 3)(n + 4)} + \dotsc\right) \\
            &<\frac{1}{(n + 1)!}\left(1 + \frac{1}{n + 2} + \frac{1}{(n + 2) ^ 2} + \frac{1}{(n + 2) ^ 3} + \dotsc\right) \\
            &= \frac{1}{(n + 1)!}\infsumo\frac{1}{(n + 2)}k \\
            &= \frac{1}{(n + 1)!}\cdot\frac{1}{1 - \frac{1}{n + 2}} \\
            &= \frac{1}{(n + 1)!}\cdot \frac{n + 2}{n + 2 - 1} \\
            &= \frac{1}{(n + 1)!}\cdot \frac{n + 2}{n + 1}.
        \end{align*}
        For $n = 4$:
        \[
        \frac{1}{5!}\cdot\frac{6}{5} = \frac{6}{120 \cdot 5} = \frac{1}{20 \cdot 5} = \frac{1}{100}.
        \]
    \end{solution}
\end{problem}

\begin{problem}[7]
    Prove $e$ is irrational.
    Use the weaker inequality $\infsum[k = n + 1]\frac{1}{k!} < \frac{3}{(n + 1)!}$ to prove this.
    
    \begin{solution}
        Note the following:
        \begin{align*}
            \infsum[k = n + 1]\frac{1}{k!} &< \frac{3}{(n + 1)!} \\
            &< \frac{1}{(n + 1)!}\cdot\frac{n + 2}{n + 1}.
        \end{align*}
        Assume $e = \frac{p}{q}$.
        Then
        \begin{align*}
            \left|e - \infsumo\frac{1}{k!}\right| &= \infsum[k = n + 1]\frac{1}{k!} \\
            &< \frac{3}{(n + 1)!}
        \end{align*}
        we have
        \begin{align*}
            \left|\frac{p}{q} - \sum_{k = 0}^{n}\frac{1}{k!}\right| &= \left|\frac{p\cdot n! - \sum\frac{n!}{k!}}{q\cdot n!}\right| \\
            &= \frac{a_n}{q\cdot n!}&\text{with $a_n \in \Z$, but $a_n \neq 0$}.
        \end{align*}
        So
        \[
        |a_n| < \frac{3 \cdot q \cdot n!}{(n + 1)!} = \frac{3\cdot q}{n + 1} \xrightarrow[n \rightarrow \infty]{} 0
        \]
        in particular $\frac{3\cdot q}{n + 1} < 1$ for large $n$.
        Contradiction since $a_n \neq 0$ integers.
    \end{solution}
\end{problem}

















\end{document}