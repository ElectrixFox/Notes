\documentclass[10pt, a4paper]{article}
\usepackage{preamble}

\newcommand{\limas}[3][n]{#2 \rightarrow #3 \text{ as } #1 \rightarrow \infty}

\title{Analysis I \\
    \large Important Results}
\author{Luke Phillips}
\date{April 2025}

\begin{document}

\maketitle

\newpage

\tableofcontents

\newpage

\section{Definitions}

\begin{definition}
    Let $X$ be a set which is bounded above.
    A number $C \in \R$ is called the supremum of $X$ if:
    \begin{enumerate}[label = (\roman*)]
        \item $C$ is an upper bound of $X$.

        \item Whenever $B \in \R$ is another upper bound of $X$,
        then $C \leq B$.
    \end{enumerate}
\end{definition}

\begin{definition}
    A real sequence $(x_n)_{n \in \N}$ converges to the limit $x \in \R$,
    if for every $\varepsilon > 0$,
    there exists an $N \in \N$ with
    \[
    |x_n - x| < \varepsilon\text{ for all } n \geq N.
    \]
\end{definition}

\begin{definition}
    Let $(x_n)_{n \in \N}$ be a sequence.
    A subsequence of $(x_n)_{n \in \N}$ is a sequence $(x_{n_j})_{j \in \N}$ with $n_1 < n_2 < n_3 < \dotsi$,
    $n_j$ is strictly monotone increasing.
\end{definition}

\begin{definition}
    A sequence $(x_n)_{n \in \N}$ is called a Cauchy sequence,
    if for every $\varepsilon > 0$ there exists an $N \in \N$ such that
    \[
    |x_m - x_n| < \varepsilon\quad\forall n, m \geq N.
    \]
\end{definition}

\begin{definition}
    Let $\infsum a_k$ be a series,
    it is absolutely convergent if $\infsum|a_k|$ is convergent.
\end{definition}

\begin{definition}
    Let $\infsum a_k$ be a series.
    It is conditionally convergent if it is convergent but not absolutely convergent.
\end{definition}

\begin{definition}
    Let $f : (a, b) \to \R$ be a function.
    Let $c \in (a, b)$ and $f$ is possibly not defined at $c$.
    We say
    \[
    \lim_{x \to c}f(x) = L
    \]
    if for all $\varepsilon > 0$ there exists a $\delta > 0$ such that
    \[
    |f(x) - L| < \varepsilon
    \]
    for all $x \neq c$ with
    \[
    |x - c| < \delta.
    \]
    I.e.
    \[
    \forall \varepsilon > 0, \exists \delta > 0, |x - c| < \delta \implies |f(x) - L| < \varepsilon.
    \]
\end{definition}

\textbf{Very important!}
\begin{definition}[Continuous at a point]
    $f : X \to \R$,
    $X = (a, b)$,
    $c \in (a, b)$.
    $f(x)$ is continuous at $x = c$ if
    \[
    \forall \varepsilon > 0, \exists \delta > 0, \forall x \in X, |x - c| < \delta \implies |f(x) - f(c)| < \varepsilon.
    \]
\end{definition}

\begin{definition}[Continuity on a set]
    Let $X$ be a set.
    \[
    \forall c \in X, \forall \varepsilon > 0, \exists \delta > 0, \forall x \in X, |x - c| < \delta \implies |f(x) - f(c)| < \varepsilon.
    \]
\end{definition}

\begin{definition}[Uniform continuity]
    $f : X \to \R$ is uniform continuous if
    \[
    \forall \varepsilon > 0, \exists \delta > 0, \forall x, y \in X, |x - y| < \delta \implies |f(x) - f(y)| < \varepsilon.
    \]
\end{definition}

\textbf{Very important!}
\begin{definition}[Differentiability]
    $f : X \to \R$
    ($X$ open).
    $f$ is differentiable at $c \in X$ if
    \[
    \lim_{x \to c}\frac{f(x) - f(c)}{x - c}
    \]
    with the limit equal to $f'(c)$.
\end{definition}

\begin{definition}
    \[
    \infsumo a_kx ^ k
    \]
    is a power series.
    $a_k \in \R$ and $x \in \R$.
\end{definition}

\begin{definition}[Pointwise limit]
    $f(x) : I \to \R$ is the pointwise limit of $(f_n(x))$ if
    \[
    \forall x \in I, \forall \varepsilon > 0, \exists N > 0, \forall n \geq N, |f_n(x) - f(x)| < \varepsilon.
    \]
\end{definition}

\begin{definition}[Uniform convergence]
    $f_n \to f$ uniformly on $I$
    \[
    \forall\varepsilon > 0, \exists N > 0, \forall x \in I, \forall n \geq N, |f_n(x) - f(x)| < \varepsilon.
    \]
\end{definition}

\begin{definition}[Uniform convergence on compact subsets]
    $f_n : I \to \R$,
    $f_n \to f$ uniformly in compact subsets.
    If $f_n \to f$ uniformly on all $[a, b] \subset I$.
\end{definition}

\begin{definition}[Step function]
    $f : [a, b] \to \R$.
    Exists a partition
    \[
    x_0 = a < x_1 < x_2 < \dotsi < x_N = b
    \]
    of $[a, b]$ such that on each open subinterval $(x_k, x_{k + 1})$ the function $f(x)$ is constant.
\end{definition}

\begin{definition}[Regulated functions]
    $f : [a, b] \to \R$ is called regulated if there exists a sequence $f_n(x) : [a, b] \to \R$ of step functions such that $f_n \to f$ uniformly.
\end{definition}

\begin{definition}[Regulated integral]
    $f : [a, b] \to \R$ regulated,
    say $f_n \to f$,
    $f_n$ step functions.
    Define the integral by
    \[
    I(f) = \liminfty I(f_n).
    \]
    Where
    \[
    I(f) = \sum_{k = 0}^{N - 1}(x_{k + 1} - x_k)f(x_k ^ {*})
    \]
    $x_k ^ {*}$ is any element in $(x_k, x_{k + 1})$.
\end{definition}

\end{document}