\documentclass[10pt, a4paper]{article}
\usepackage{preamble}

\newcommand{\lhopital}[0]{L' H\^opital}
\newcommand{\limas}[3][n]{#2 \rightarrow #3 \text{ as } #1 \rightarrow \infty}

\title{Analysis I \\
    \large Important Results}
\author{Luke Phillips}
\date{April 2025}

\begin{document}

\maketitle

\newpage

\tableofcontents

\newpage

\section{Definitions}

\begin{definition}
    Let $X$ be a set which is bounded above.
    A number $C \in \R$ is called the supremum of $X$ if:
    \begin{enumerate}[label = (\roman*)]
        \item $C$ is an upper bound of $X$.

        \item Whenever $B \in \R$ is another upper bound of $X$,
        then $C \leq B$.
    \end{enumerate}
\end{definition}

\begin{definition}
    A real sequence $(x_n)_{n \in \N}$ converges to the limit $x \in \R$,
    if for every $\varepsilon > 0$,
    there exists an $N \in \N$ with
    \[
    |x_n - x| < \varepsilon\text{ for all } n \geq N.
    \]
\end{definition}

\begin{definition}
    Let $(x_n)_{n \in \N}$ be a sequence.
    A subsequence of $(x_n)_{n \in \N}$ is a sequence $(x_{n_j})_{j \in \N}$ with $n_1 < n_2 < n_3 < \dotsi$,
    $n_j$ is strictly monotone increasing.
\end{definition}

\begin{definition}
    A sequence $(x_n)_{n \in \N}$ is called a Cauchy sequence,
    if for every $\varepsilon > 0$ there exists an $N \in \N$ such that
    \[
    |x_m - x_n| < \varepsilon\quad\forall n, m \geq N.
    \]
\end{definition}

\begin{definition}
    Let $\infsum a_k$ be a series,
    it is absolutely convergent if $\infsum|a_k|$ is convergent.
\end{definition}

\begin{definition}
    Let $\infsum a_k$ be a series.
    It is conditionally convergent if it is convergent but not absolutely convergent.
\end{definition}

\begin{definition}
    Let $f : (a, b) \to \R$ be a function.
    Let $c \in (a, b)$ and $f$ is possibly not defined at $c$.
    We say
    \[
    \lim_{x \to c}f(x) = L
    \]
    if for all $\varepsilon > 0$ there exists a $\delta > 0$ such that
    \[
    |f(x) - L| < \varepsilon
    \]
    for all $x \neq c$ with
    \[
    |x - c| < \delta.
    \]
    I.e.
    \[
    \forall \varepsilon > 0, \exists \delta > 0, |x - c| < \delta \implies |f(x) - L| < \varepsilon.
    \]
\end{definition}

\textbf{Very important!}
\begin{definition}[Continuous at a point]
    $f : X \to \R$,
    $X = (a, b)$,
    $c \in (a, b)$.
    $f(x)$ is continuous at $x = c$ if
    \[
    \forall \varepsilon > 0, \exists \delta > 0, \forall x \in X, |x - c| < \delta \implies |f(x) - f(c)| < \varepsilon.
    \]
\end{definition}

\begin{definition}[Continuity on a set]
    Let $X$ be a set.
    \[
    \forall c \in X, \forall \varepsilon > 0, \exists \delta > 0, \forall x \in X, |x - c| < \delta \implies |f(x) - f(c)| < \varepsilon.
    \]
\end{definition}

\begin{definition}[Uniform continuity]
    $f : X \to \R$ is uniform continuous if
    \[
    \forall \varepsilon > 0, \exists \delta > 0, \forall x, y \in X, |x - y| < \delta \implies |f(x) - f(y)| < \varepsilon.
    \]
\end{definition}

\textbf{Very important!}
\begin{definition}[Differentiability]
    $f : X \to \R$
    ($X$ open).
    $f$ is differentiable at $c \in X$ if
    \[
    \lim_{x \to c}\frac{f(x) - f(c)}{x - c}
    \]
    with the limit equal to $f'(c)$.
\end{definition}

\begin{definition}
    \[
    \infsumo a_kx ^ k
    \]
    is a power series.
    $a_k \in \R$ and $x \in \R$.
\end{definition}

\begin{definition}[Pointwise limit]
    $f(x) : I \to \R$ is the pointwise limit of $(f_n(x))$ if
    \[
    \forall x \in I, \forall \varepsilon > 0, \exists N > 0, \forall n \geq N, |f_n(x) - f(x)| < \varepsilon.
    \]
\end{definition}

\begin{definition}[Uniform convergence]
    $f_n \to f$ uniformly on $I$
    \[
    \forall\varepsilon > 0, \exists N > 0, \forall x \in I, \forall n \geq N, |f_n(x) - f(x)| < \varepsilon.
    \]
\end{definition}

\begin{definition}[Uniform convergence on compact subsets]
    $f_n : I \to \R$,
    $f_n \to f$ uniformly in compact subsets.
    If $f_n \to f$ uniformly on all $[a, b] \subset I$.
\end{definition}

\begin{definition}[Step function]
    $f : [a, b] \to \R$.
    Exists a partition
    \[
    x_0 = a < x_1 < x_2 < \dotsi < x_N = b
    \]
    of $[a, b]$ such that on each open subinterval $(x_k, x_{k + 1})$ the function $f(x)$ is constant.
\end{definition}

\begin{definition}[Regulated functions]
    $f : [a, b] \to \R$ is called regulated if there exists a sequence $f_n(x) : [a, b] \to \R$ of step functions such that $f_n \to f$ uniformly.
\end{definition}

\begin{definition}[Regulated integral]
    $f : [a, b] \to \R$ regulated,
    say $f_n \to f$,
    $f_n$ step functions.
    Define the integral by
    \[
    I(f) = \liminfty I(f_n).
    \]
    Where
    \[
    I(f) = \sum_{k = 0}^{N - 1}(x_{k + 1} - x_k)f(x_k ^ {*})
    \]
    $x_k ^ {*}$ is any element in $(x_k, x_{k + 1})$.
\end{definition}

\newpage

\section{Important theorems}

\begin{proposition}[Inverse function rule]
    Let $f : I \to \R$ be continuous on the interval $I$,
    differentiable at a point $d$.
    Assume that $f$ is invertible with inverse $f ^ {-1}$ and that $f'(d) \neq 0$.
    Then $f ^ {-1}$ is differentiable at $c \coloneqq f(d)$ with
    \[
    (f ^ {-1})'(c) = \frac{1}{f'(f ^ {-1}(c))}.
    \]
\end{proposition}

\begin{theorem}[Mean Value Theorem]
    Let $f : [a, b] \to \R$ be continuous and differentiable on $(a, b)$.
    Then there exists $c \in (a, b)$ such that
    \[
    f'(c) = \frac{f(b) - f(a)}{b - a}.
    \]
\end{theorem}

\begin{theorem}[Growth Theorem]
    Let $f : I \to \R$ be continuous function on an interval $I$,
    differentiable in its interior points.
    Then if $f'(x) = 0$ for all $x$,
    then $f$ is constant.
\end{theorem}

\begin{theorem}[\lhopital's Rule]
    Let $f, g$ differentiable functions on $(a, b)$.
    Assume $\lim_{x \to a ^ {+}}f(x) = 0 = \lim_{x \to a ^ {+}}g(x)$ and $g(x) \neq 0, g'(x) \neq 0$ for all $x$.
    If $\lim_{x \to a ^ {+}}f'(x) / g'(x)$ exists,
    then also $\lim_{x \to a ^ {+}}f(x) / g(x)$ exists,
    and we have
    \[
    \lim_{x \to a ^ {+}}\frac{f(x)}{g(x)} = \lim_{x \to a ^ {+}}\frac{f'(x)}{g'(x)}.
    \]
\end{theorem}

\begin{theorem}[Ratio Test]
    Let $(a_k)$ be a sequence $a_k \neq 0$ for all but some $k$ if
    \[
    \liminfty[k]\frac{|a_{k + 1}|}{|a_k|} < 1
    \]
    then $\infsumo a_k$ converges absolutely.
\end{theorem}

\begin{theorem}[Root Test]
    For a sequence $(a_k)$ if
    \[
    \limsup_{k \to \infty}|a_k| ^ {1 / k} < 1
    \]
    then $\infsumo a_k$ converges absolutely.
\end{theorem}

\textbf{Both of the above tell nothing for equality to $1$.}

\begin{theorem}[Cauchy-Hadamard]
    Let $\infsumo a_kx ^ k$ be a power series.
    Then there exists a constant $R \in [0, \infty]$ such that:
    if $R = 0$,
    $\infsumo a_kx ^ k$ converges only for $x = 0$.
    If $R > 0$,
    then
    \[
    \infsumo a_kx ^ k\text{ converges absolutely for $x \in (-R, R)$}.
    \]
\end{theorem}

\begin{theorem}
    Let $f_n$ be a sequence of continuous functions on an interval $I$ such that $f_n \to f$ uniformly.
    Then the limit function $f$ is also continuous.
\end{theorem}

\begin{theorem}[Weierstrass M-Test]
    Let $I \subset \R$ be an interval and $(f_k)$ be a sequence of functions $f_k : I \to \R$.
    Let $M_k$ be a sequence of real numbers
    (not depending on $x \in I$!)
    satisfying
    \begin{center}
        \begin{enumerate}[label = (\roman*)]
            \item $|f_k(x)| \leq M_k$ for all $x \in I$;
    
            \item $\displaystyle \infsumo M_k$ is convergent.
        \end{enumerate}
    \end{center}
    Then $\infsumo f_k(x)$ converges uniformly
    (and absolutely)
    to a limit function $f(x)$.
\end{theorem}

\newpage

\section{Methods}

\textbf{Important}

Remember that when we look at functions $f : I \to \R$ if you have $(x_n)_{n \in \N} \in I$ then we can look at $f(x_n)$ in the same way as we would $f(x)$.


\subsection{Functions by derivatives}
We can define functions through their derivatives.

\begin{example}
    Let
    \[
    \exp(x) \coloneqq \infsumo\frac{x ^ k}{k!}
    \]
    then we can show $\exp(x + y) = \exp(x)\exp(y)$ by the following.

    Consider $f(t) \coloneqq \exp(x + t)\exp(y - t)$ then,
    $f(0) = \exp(x)\exp(y)$ and $f(y) = \exp(x + y)$.
    But
    \[
    f'(t) = \exp(x + t)\exp(y - t) - \exp(x + t)\exp(y - t) = 0
    \]
    meaning $f(t)$ is constant by the growth theorem,
    thus we can see that
    \[
    f'(t) = f(t) + C
    \]
    since $f'(0) = 0$ and $f(0) = \exp(x)\exp(y)$,
    $C = -\exp(x)\exp(y)$ we can conclude
    \[
    0 = f'(y) = \exp(x + y) - \exp(x)\exp(y) \iff \exp(x + y) = \exp(x)\exp(y).
    \]
\end{example}

We can do a similar procedure for $\sin(x + y), \cos(x + y)$.
\begin{example}
    Define
    \[
    f(t) \coloneqq \sin(x + t)\cos(y - t) + \cos(x + t)\sin(y - t).
    \]
    \begin{align*}
        f(0) &= \sin(x)\cos(y) + \cos(x)\sin(y) \\
        f(t) &= \sin(x + y) \\
        f'(t) &= \cos(x + t)\cos(y - t) + \sin(x + t)\sin(y - t) \\
        &- \sin(x + t)\sin(y - t) - \cos(x + t)\cos(y - t) = 0.
    \end{align*}
    Then by the growth theorem this is constant hence
    \begin{align*}
        f'(t) &= f(t) + C \\
        &\iff \\
        0 = f'(0) &= \sin(x)\cos(y) + \cos(x)\sin(y) + C  \\
        &\iff \\
        C &= -\sin(x)\cos(y) - \cos(x)\sin(y)
        \intertext{which means that}
        0 = f'(t) &= f(t) + C \\
        &\iff \\
        f(t) &= -C \\
        &\iff \\
        \sin(x + y) &= \sin(x)\cos(y) + \cos(x)\sin(y).
    \end{align*}
    We can then just see that $\sin(x)' = \cos(x)$ so $\sin(x + y)' = \cos(x + y)$ then apply the product rule.
\end{example}





\end{document}