\documentclass[10pt, a4paper]{article}
\usepackage{preamble}

\newcommand{\seq}[1][x]{(#1_n)_{n \in \N}}
\newcommand{\dseq}[2][n]{(#2_#1)_{#1 \in \N}}
\newcommand{\limas}[3][n]{#2 \rightarrow #3 \text{ as } #1 \rightarrow \infty}

\title{Analysis I \\
    \large Prereading}
\author{Luke Phillips}
\date{December 2024}

\begin{document}

\maketitle

\newpage

\section{Functions, Limits, and Continuity}

\subsection{Open and closed sets}
Let $a \leq b$.
Recall that we call
\[
(a, b) := \{t \in \R;\,a < t < b\}
\]
an open interval inside $\R$.
We do allow $a = -\infty$ or $b = \infty$.
Similarly,
we have the closed
(or compact)
interval inside $\R$
\[
[a, b] := \{t \in \R;\, a \leq t \leq b\}.
\]
We also have half-open intervals:
$[a, b)$ or $(a, b]$.

We call a subset $X \subseteq \R$ open if for each $c \in X$ there exists an open interval $(c - \delta, c + \delta)$ with $\delta > 0$ which is contained in $X : (c - \delta, c + \delta) \subseteq X$.
\textit{Note:
that $\delta$ typically depends on the point $c$ and that $\delta$ isn't certainly unique.}

If you have found one $\delta > 0$ then all positive reals less than $\delta$ will also work.
\textit{Note:
This definition of an open interval $(a, b)$ is indeed open:
Given $c \in (a, b)$ one can take $\delta = \min\{b - c, c - a\}$.}

Finally,
$c \in X$ is an interior point if there exists an open subset $U$ or an open interval $(a, b)$ containing $c$ which lies completely in $X$.

\hfill

\begin{lemma}\label{pre_analy_lem_limofconvseqinclosedintinint}
    Let $\seq$ be a convergent sequence in the closed interval $[a, b]$.
    Then its limit $L$ lies also in $[a, b]$.
    \begin{proof}
        Assume $L \notin [a, b]$.
        Let $\varepsilon = \min\{L - b, a - L\} > 0$.
        Then by the definition of the limit there exists an $N$ such that $|x_n - L| < \varepsilon$ for all $n \geq N$.
        But this means in particular $x_n \notin [a, b]$ for $n \geq N$.
        Contradiction.
    \end{proof}
\end{lemma}
\begin{center}
\fbox{
\begin{minipage}{0.9\textwidth}
The proof is sort of saying
(by my understanding)
that if $\varepsilon$ is in the interval of the limit and $L$ is outside of the limit,
then $x_n$ must not be in $[a, b]$,
but this contradicts our assumption.
\end{minipage}
}
\end{center}
\hfill

Of course this does not hold for open intervals $(a, b)$.
The sequence $(1/n)_{n \in \N}$ lies on $(0, 1]$ but has limit $0$ outside the set.
However,
considering $(a, b) \subset [a, b]$ we immediately obtain.

\begin{corollary}
    Let $\seq$ be a converging sequence in the open interval $(a, b)$.
    Then its limit $L$ lies in the closed interval $[a, b]$.
\end{corollary}

$[a, b]$ are the only potential limit points for a sequence in $(a, b)$.
We generalise now that a given set $X \subseteq \R$ we let $\bar{X}$ be the set of all limits of all convergent sequences in $X$.
We call $\bar{X}$ the closure of $X$.

In general,
this can be quite complicated to describe but we will be mainly concerned with situations where $X$ is an interval or finite union of such.

\begin{corollary}[Bolzano-Weierstrass]
    Let $\seq$ be a sequence in the compact interval $[a, b]$.
    Then the sequence has a subsequence which converges in $[a, b]$.

    \begin{proof}
        The sequence $\seq$ is bounded,
        hence by Bolzano-Weierstrass a converging subsequence.
        By \autoref{pre_analy_lem_limofconvseqinclosedintinint} its limit must lie in $[a, b]$.
    \end{proof}
\end{corollary}

\subsection{Limits of functions}
Throughout we let $f : X \rightarrow \R$ be a function on a subset $X \subseteq \R$ of the reals.
Sometimes we might also write $D = D(f)$ for the domain of $f$.

\begin{definition}
    Let $f : X \rightarrow \R$ be a function and assume $X$ contains the open interval $(a, b)$ with the possible exception of a point $c \in (a, b)$.
    We say
    \[
    \lim_{x \rightarrow c}f(x) = L
    \]
    for some $L \in \R$,
    if for all $\varepsilon > 0$ there exists a $\delta > 0$ such that
    \[
    |f(x) - L| < \varepsilon
    \]
    for all $x \in (a, b) \setminus \{c\}$ with
    \[
    |x - c| < \delta.
    \]
\end{definition}

\textit{Note:
if $c \in X$,
then $L$ can be $f(c)$,
but need not be.
(If it is we will call $f$ continuous at $c$,
see below).}
One often also writes $\lim_{x \rightarrow c}f(x) = L$ in the form
\[
f(x) \rightarrow L\quad\text{as}\quad x \rightarrow c.
\]

\begin{proposition}\label{pre_analy_prop_funclimcriterion}
    Let $f : X \rightarrow \R$ be a function and assume $X$ contains the open interval $(a, b)$ with the possible exception of a point $c \in (a, b)$.
    
    Then $\lim_{x \rightarrow c}f(x) = L$ if and only if for all sequences $\seq$ in $X \setminus \{c\}$ with $x_n \rightarrow c$ as $n \rightarrow \infty$ we have $\lim_{n \rightarrow \infty}f(x_n) = L$.
    \begin{proof}
        First assume $\lim_{x \rightarrow c}f(x) = L$ and let $(x_n)$ be a sequence in $X \setminus \{c\}$ converging to $c$.
        Pick $\varepsilon > 0$.
        Then by assumption there exists a $\delta > 0$ such that
        \[
        |f(x) - L| < \varepsilon\quad\text{provided that}\quad|x - c| < \delta
        \]
        and $x \in (a, b) \setminus \{c\}$.
        Since $x_n \rightarrow c$ as $n \rightarrow \infty$ there exists an $N \in \N$ with
        \[
        |x_n - c| < \delta
        \]
        for all $n \geq N$.
        But then $|f(x_n) - L| < \varepsilon$ for all $n \geq N$.
        This means that $f(x_n) \rightarrow L$ as $n \rightarrow \infty$.

        Now let us assume that $\lim_{x \rightarrow c}f(x) \neq L$
        (or does not exist).
        We need to find a sequence $(x_n)$ in $X \setminus \{c\}$ converging to $c$ such that the sequence $(f(x_n))$ does not converge to $L$.
        By assumption there exists an $\varepsilon > 0$ such that for all $\delta > 0$ there exists a $x \neq c$ with $|x - c| < \delta$ and $|f(x) - L| \geq \varepsilon$.
        So given this $\varepsilon$ we can choose $\delta = \frac{1}{n}$ and then there exists an $x_n \neq c$ with $|x_n - c| < \frac{1}{n}$,
        but $|f(x_n) - L| \geq \varepsilon$.
        But then $x_n \rightarrow c$ as $n \rightarrow \infty$,
        while $f(x_n) \nrightarrow L$.
    \end{proof}
\end{proposition}

We now will extend the definition of a limit of a function to one-sided limits and also to $c = \pm\infty$.
\begin{definition}[Right-sided Limit]
    Let $f : (a, b) \rightarrow \R$ be a function.
    We define the right-sided limit as
    \[
    \lim_{x \rightarrow a ^ {+}}f(x) = L
    \]
    if for all $\varepsilon > 0$ there exists a $\delta > 0$ such that
    \[
    |f(x) - L| < \varepsilon\qquad\text{for all } x \in (a, b)\text{ with }|x - a| = x - a < \delta.
    \]
\end{definition}

\begin{definition}[Left-sided Limit]
    Let $f : (a, b) \rightarrow \R$ be a function.
    We define the left-sided limit as
    \[
    \lim_{x \rightarrow a ^ {-}}f(x) = L
    \]
    if for all $\varepsilon > 0$ there exists a $\delta > 0$ such that
    \[
    |f(x) - L| < \varepsilon\qquad\text{for all } x \in (a, b)\text{ with }|x - b| < \delta.
    \]
\end{definition}

We can also define the left/right-sided limit for an interior point $c \in (a, b)$ in same way
(formally,
pick $a = c$ respectively.
$b = c$ in the definition.
Naturally,
the one-sided limits if they exist don't have to be the same!)

If the domain $X$ of $f$ is not bounded above or below
(containing intervals of the form $(b, \infty)$ or $(-\infty, a)$),
we say
\begin{definition}
    \[
    \lim_{x \rightarrow \infty}f(x) = L
    \]
    if for every $\varepsilon > 0$ there exists a $K > 0$ such that
    \[
    |f(x) - L| < \varepsilon\text{ for every } x \in X\text{ with } x \geq K.
    \]
    If $X$ is not bounded below,
    we can define
    \[
    \lim_{x \rightarrow -\infty}f(x) = L
    \]
    analogously.
\end{definition}

We can use the limit criterion \autoref{pre_analy_prop_funclimcriterion} to easily carry over COLT to the limit of functions
\begin{theorem}
    Let $f, g : X \rightarrow \R$ be two functions with $\lim_{x \rightarrow c}f(x) = L_1$ and $\lim_{x \rightarrow c}g(x) = L_2$.
    \begin{enumerate}[label = (\roman*)]
        \item For any $\alpha, \beta \in \R$ the limit $\lim_{x \rightarrow c}\alpha\cdot f(x) + \beta\cdot g(x)$ exists and is equal to $\alpha L_1 + \beta L_2$.
        \item The limit $\lim_{x \rightarrow c}f(x)g(x)$ exists and is equal to $L_1L_2$.
        \item Assume $L_2 \neq 0$.
        The limit $\lim_{x \rightarrow c}f(x) / g(x)$ exists and is equal to $L_1 / L_2$.
    \end{enumerate}
    \begin{proof}
        \begin{enumerate}[label = (\roman*)]
        \item
            We will use the criterion \autoref{pre_analy_prop_funclimcriterion}.
            For this let $(x_n)$ be a sequence in $X$ with $\limas{x_n}{c}$.
    
            By COLT for sequences and the hypothesis we then have that $\alpha \cdot f(x_n) + \beta \cdot g(x_n)$ converges to $\alpha \cdot L_1 + \beta \cdot L_2$ which then implies the first claim by \autoref{pre_analy_prop_funclimcriterion}.
        
        \textit{The other claims follow in the same way.}
        \item \textbf{Prove!}
        \item \textbf{Prove!}
        \end{enumerate}
    \end{proof}
\end{theorem}

Similarly we also have,
\begin{theorem}
    Let $X \subset \R$,
    $c \in X$ and $f, g, h : X \rightarrow \R$ be three functions $f(x) \leq g(x) \leq h(x)$ for all $x \in X$ and also $\lim_{x \rightarrow c}f(x) = \lim_{x \rightarrow c}h(x) = L$.
    Then also
    \[
    \lim_{x \rightarrow c}g(x) \text{ exists with } \lim_{x \rightarrow c}g(x) = L.
    \]
    \begin{proof}
        This follows from \autoref{pre_analy_prop_funclimcriterion} in exactly the same way.
        So let $(x_n)$ be a sequence in $X$ with $\limas{x_n}{c}$.
        Hence by hypothesis $f(x_n) \leq g(x_n) \leq h(x_n)$ for all $x \in X$ and also $\liminfty f(x_n) = \liminfty h(x_n) = L$.
        Now apply squeezing for sequences to conclude that $\liminfty g(x_n)$ exists and is equal to $L$.
        This implies the claim by \autoref{pre_analy_prop_funclimcriterion}.
    \end{proof}
\end{theorem}

\subsection{Continuity}





\end{document}