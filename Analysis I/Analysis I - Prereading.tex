\documentclass[10pt, a4paper]{article}
\usepackage{preamble}

\newcommand{\seq}[1][x]{(#1_n)_{n \in \N}}
\newcommand{\dseq}[2][n]{(#2_#1)_{#1 \in \N}}

\title{Analysis I \\
    \large Prereading}
\author{Luke Phillips}
\date{December 2024}

\begin{document}

\maketitle

\newpage

\section{Functions, Limits, and Continuity}

\subsection{Open and closed sets}
Let $a \leq b$.
Recall that we call
\[
(a, b) := \{t \in \R;\,a < t < b\}
\]
an open interval inside $\R$.
We do allow $a = -\infty$ or $b = \infty$.
Similarly,
we have the closed
(or compact)
interval inside $\R$
\[
[a, b] := \{t \in \R;\, a \leq t \leq b\}.
\]
We also have half-open intervals:
$[a, b)$ or $(a, b]$.

We call a subset $X \subseteq \R$ open if for each $c \in X$ there exists an open interval $(c - \delta, c + \delta)$ with $\delta > 0$ which is contained in $X : (c - \delta, c + \delta) \subseteq X$.
\textit{Note:
that $\delta$ typically depends on the point $c$ and that $\delta$ isn't certainly unique.}

If you have found one $\delta > 0$ then all positive reals less than $\delta$ will also work.
\textit{Note:
This definition of an open interval $(a, b)$ is indeed open:
Given $c \in (a, b)$ one can take $\delta = \min\{b - c, c - a\}$.}

Finally,
$c \in X$ is an interior point if there exists an open subset $U$ or an open interval $(a, b)$ containing $c$ which lies completely in $X$.

\hfill

\begin{lemma}\label{pre_analy_lem_limofconvseqinclosedintinint}
    Let $\seq$ be a convergent sequence in the closed interval $[a, b]$.
    Then its limit $L$ lies also in $[a, b]$.
    \begin{proof}
        Assume $L \notin [a, b]$.
        Let $\varepsilon = \min\{L - b, a - L\} > 0$.
        Then by the definition of the limit there exists an $N$ such that $|x_n - L| < \varepsilon$ for all $n \geq N$.
        But this means in particular $x_n \notin [a, b]$ for $n \geq N$.
        Contradiction.
    \end{proof}
\end{lemma}
\begin{center}
\fbox{
\begin{minipage}{0.9\textwidth}
The proof is sort of saying
(by my understanding)
that if $\varepsilon$ is in the interval of the limit and $L$ is outside of the limit,
then $x_n$ must not be in $[a, b]$,
but this contradicts our assumption.
\end{minipage}
}
\end{center}
\hfill

Of course this does not hold for open intervals $(a, b)$.
The sequence $(1/n)_{n \in \N}$ lies on $(0, 1]$ but has limit $0$ outside the set.
However,
considering $(a, b) \subset [a, b]$ we immediately obtain.

\begin{corollary}
    Let $\seq$ be a converging sequence in the open interval $(a, b)$.
    Then its limit $L$ lies in the closed interval $[a, b]$.
\end{corollary}

$[a, b]$ are the only potential limit points for a sequence in $(a, b)$.
We generalise now that a given set $X \subseteq \R$ we let $\bar{X}$ be the set of all limits of all convergent sequences in $X$.
We call $\bar{X}$ the closure of $X$.

In general,
this can be quite complicated to describe but we will be mainly concerned with situations where $X$ is an interval or finite union of such.

\begin{corollary}[Bolzano-Weierstrass]
    Let $\seq$ be a sequence in the compact interval $[a, b]$.
    Then the sequence has a subsequence which converges in $[a, b]$.

    \begin{proof}
        The sequence $\seq$ is bounded,
        hence by Bolzano-Weierstrass a converging subsequence.
        By \autoref{pre_analy_lem_limofconvseqinclosedintinint} its limit must lie in $[a, b]$.
    \end{proof}
\end{corollary}


\end{document}