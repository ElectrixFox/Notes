\documentclass[10pt, a4paper]{article}
\usepackage{preamble}

\title{Analysis I}
\author{Luke Phillips}
\date{October 2024}

\begin{document}

\maketitle

\newpage

\section{Basic logic and sets}

\subsection{Logic}
\begin{definition}
    A statement is a sentence which is either true or false.
\end{definition}
\begin{example}
    \phantom{}
    \begin{itemize}
        \item There exist infinitely many prime numbers.
        \item There exists a rational number $x$ with $x ^ 2 = 2$.
    \end{itemize}
\end{example}

\begin{definition}
    Let $A$ and $B$ be statements.
    \begin{enumerate}[label = (\alph*)]
        \item "$A$ and $B$" is the statement that is true when exactly both $A$ and $B$ are true, otherwise false.
        \item "$A$ or $B$" is the statement which is false exactly when both $A$ and $B$ are false, otherwise true.
        \item "Not $A$"\footnote{Called the negation.} is the statement which is true when $A$ is false, and false when $A$ is true.
    \end{enumerate}

    The truth tables are as follows 
    \begin{table}[h!]
        \centering
        \begin{tabular}{|c|c|c|c|c|}
            \hline
            $A$ & $B$ & $A \text{ and } B$ & $A \text{ or } B$ & $\text{not } A$ \\
            \hline
            T & T & T & T & F \\
            T & F & F & T & F \\
            F & T & F & T & T \\
            F & F & F & F & T \\
            \hline
        \end{tabular}
        \caption{Graph 6}
        \label{tab:Gr6}
    \end{table}
\end{definition}

\begin{definition}
    Let $A$ and $B$ be statements. The statement "if $A$ then $B$" is defined to be false when $A$ is true and $B$ is false, and true in all other cases.

    The truth table for this is as follows
    \begin{table}[h!]
        \centering
        \begin{tabular}{|c|c|c|}
            \hline
            $A$ & $B$ & $\text{if } A \text{ then } B$ \\
            \hline
            T & T & T \\
            T & F & F \\
            F & T & T \\
            F & F & T \\
            \hline
        \end{tabular}
        \caption{Graph 7}
        \label{tab:Gr7}
    \end{table}
\end{definition}
\begin{example}
    If $p$ is a prime number, then $p$ is an odd number. (this is false)

    If $p \geq 3$ is a prime number, then $p$ is an odd number. (this is true)
\end{example}

\begin{definition}
    Let $A$ and $B$ be statements. The statement $A$ is equivalent to $B$ for the statement $A \implies B$ and $B \implies A$, or $A \iff B$.
    
    The truth table for this is as follows
    \begin{table}[h!]
        \centering
        \begin{tabular}{|c|c|c|c|c|}
            \hline
            $A$ & $B$ & $\text{if } A \text{ then } B$ & $\text{if } B \text{ then } A$ & $A \iff B$ \\
            \hline
            T & T & T & T & T \\
            T & F & F & T & F \\
            F & T & T & F & F \\
            F & F & T & T & T \\
            \hline
        \end{tabular}
        \caption{Graph 8}
        \label{tab:Gr8}
    \end{table}
\end{definition}

Often statements depend on a variable $x$. This is called a conditional statement $A(x)$.

We can turn a conditional statement $A(x)$ into an unconditional statement using the quantifiers "for all" and "there is"

\begin{example}
    \phantom{}
    
    $\forall x$, we have $A(x)$
    
    $\exists x$ such that $A(x)$
\end{example}


\subsection{Sets}
A set $X$ is a collection of objects. We write $x \in X$ if the object $x$ is contained in the set $X$\footnote{$x \notin X$ for $x$ not in $X$}.

\[
X = \{x\,|\,A(x)\}
\]
is a set of all objects $x$ for which the conditional statement $A(x)$ is true.

\begin{definition}
    Let $X, Y$ be sets.
    \begin{enumerate}[label = (\alph*)]
        \item We say $X$ is a subset of $Y$, write $X \subset Y$, if $x \in X$ implies $x \in Y$
        \item We say that $X$ is equal to $Y$, write $X = Y$, if $X \subset Y$ and $Y \subset X$.
    \end{enumerate}
\end{definition}

\begin{remark}
    \begin{enumerate}[label = (\alph*)]
        \item The order of elements in a set is irrelevant.
        \item There is a set $\emptyset$ which doesn't contain any elements. In particular $\emptyset \subset X$ for every $X$ a set.
        \item There is no biggest set $U$, such that $X \subset U$ for every $X$.
    \end{enumerate}
\end{remark}

\begin{definition}
    Let $X, Y$ be sets.
    \begin{enumerate}[label = (\alph*)]
        \item The union of $X$ and $Y$ is given by
        \[
        X \cup Y = \{x\, |\, x \in X\text{ or } x \in Y\}.
        \]
        \item The intersection of $X$ and $Y$ is given by
        \[
        X \cap Y = \{x\, |\, x \in X\text{ and } x \in Y\}.
        \]
        \item The complement of $Y$ in $X$ is given by
        \[
        X \setminus Y = \{x\, |\, x \in X \text{ and } x \notin Y\}.
        \]
    \end{enumerate}
\end{definition}

Given two objects $a, b$ there is an object $(a, b)$ called the ordered pair of $a$ and $b$.

We require $(a_1, b_1) = (a_2, b_2) \iff a_1 = a_2 \text{ and } b_1 = b_2$

\begin{definition}[Cartesian product]
    The Cartesian product of two sets $X, Y$ is given by
    \[
    X \times Y = \{(x, y)\, |\, x \in X, y \in Y\}.
    \]
\end{definition}

\begin{definition}[Function]
    Let $X, Y$ be sets.
    A function $f$ from $X$ to $Y$ is a subset $f \subset X \times Y$ such that
    for every $x \in X$ there exists exactly one $y \in Y$ with $(x, y) \in f$.
    
    We write $f : X \rightarrow Y$ for a function and
    write $f(x)$ for the unique element of $y$ with $(x, f(x)) \in f$

    $X$ is called the domain,
    $Y$ is called the co-domain of $f$.
\end{definition}
We write function often with a formula
\[
f(x) = x ^ 2 + 3.
\]
Need a domain $X$ e.g. $X = \R$ or $X = (0, \infty)$.

\begin{example}
    $\sqrt{\phantom{1}} : [0, \infty) \rightarrow \R$

    $\sqrt{4} = +2$ not $\pm2$
\end{example}
\begin{example}
    \[
    g(x) = \frac{1}{x}
    \]
    only works for domains not containing $0$, $g: \R \rightarrow \R$ not defined by formula.
\end{example}

\begin{definition}
    Let $f : X \rightarrow Y$ be a function between sets $X$ and $Y$. If $A \subset X$ we define the image of $A$ as
    \[
    f(A) = \{f(a) \in Y\,|\, a \in A\} \subset Y.
    \]
    For $B \subset Y$ we define the pre-image of $B$ as
    \[
    f^{-1}(B) = \{x \in X\,|\,f(x) \in B\} \subset X.
    \]
\end{definition}

\begin{example}
    $f: [-1, 1] \rightarrow \R$ $f(x) = x ^ 2 + 3$

    $f([-1, 1]) = [3, 4]$

    $f\left(\left(\frac{1}{2}, 1\right)\right) = \left(\frac{13}{4}, 4\right)$

    $B = (-\infty, 0)$
    $f^{-1}(B) = \emptyset$

    $B = [3, 4]$
    $f^{-1}(B) = [-1, 1]$
    
    $f^{-1}\left(\left(\frac{13}{4}, 4\right)\right) =
    \left(\frac{1}{2}, 1\right) \cup \left(-1, -\frac{1}{2}\right)$
\end{example}

\begin{proposition}
    Let $f : X \rightarrow Y$ be a function, and assume that $A, B \subset X$. Then
    \begin{itemize}
        \item $f(A \cap B) \subset f(A) \cap f(B)$
        \item $f(A \cup B) = f(A) \cup f(B)$
        \item $f(X \setminus A) \supset f(X) \setminus f(A)$
    \end{itemize}
    Assume that $C, D \subset Y$. Then
    \begin{itemize}
        \item $f ^ {-1}(C \cap D) = f^{-1}(C) \cap f^{-1}(D)$
        \item $f^{-1}(C \cup D) = f^{-1}(C) \cup f^{-1}(D)$
        \item $f^{-1}(Y \setminus C) = X \setminus f^{-1}(C)$
    \end{itemize}

    \begin{proof}
        Look at $f(A \cap B) \subset f(A) \cap f(B)$.
        
        Take $y \in f(A \cap B)$. Then there is $x \in A \cap B$ with $f(x) = y$.
        The $x \in A$ and $x \in B$. The $y = f(x) \in f(A)$ and $y = f(x) \in f(B)$.
        So $y \in f(A) \cap f(B)$ \\

        Look at $f(A \cup B) = f(A) \cup f(B)$.

        Firstly, we need to examine the two conditions for equality, $f(A \cup B) \subset f(A) \cup f(B)$ and $f(A \cup B) \supset f(A) \cup f(B)$.
        
        Take $y \in f(A \cup B)$. There is a unique $x \in A \cup B$ such that $f(x) = y$\footnote{By definition of image.}, this implies that $x \in A$ or $x \in B$. Hence, $y = f(x) \in f(A)$ or $y = f(x) \in f(B)$, which is equivalent to saying $y \in f(A) \cup f(B)$ which proves that $y \in f(A \cup B) \implies y \in f(A) \cup f(B)$.

        Now take $y \in f(A) \cup f(B)$. There is a unique $x \in A$ or $x \in B$ such that $f(x) = y$, therefore $x \in A \cup B$ so $y = f(x) \in f(A \cup B)$. This proves the goal that $f(A \cup B) \supset f(A) \cup f(B)$.

        Finally, we have proven that $f(A \cup B) = f(A) \cup f(B)$ by showing that $f(A \cup B) \subset f(A) \cup f(B)$ and that $f(A \cup B) \supset f(A) \cup f(B)$. \\

        Look at $f(X \setminus A) \supset f(X) \setminus f(A)$.

        Take $y \in f(X) \setminus f(A)$. Then there is a unique $x \in X$ and $x \notin A$ such that $f(x) = y$, this means that there is an $x \in X \setminus A$ such that $f(x) = y$, therefore $y = f(x) \in f(X \setminus A)$. \\

        Look at $f^{-1}(C \cap B) = f^{-1}(C) \cap f^{-1}(D)$.

        Take $x \in f^{-1}(C \cap B)$. The $f(x) \in C \cap D$. So $f(x) \in C$ and $f(x) \in D$. So $x \in f^{-1}(C)$ and $x \in f^{-1}(D)$. Hence $x \in f^{-1}(C) \cap f^{-1}(D)$.

        Take $x \in f^{-1}(C) \cap f^{-1}(D)$. So $x \in f^{-1}(C)$ and $x \in f^{-1}(D)$ so $f(x) \in C$ and $f(x) \in D$. So $f(x) \in C \cap D$. So $x \in f^{-1}(C \cap D)$. \\

        Look at $f^{-1}(C \cup D) = f^{-1}(C) \cup f^{-1}(D)$.

        To prove equality we need to prove both that $f^{-1}(C \cup B) \subset f^{-1}(C) \cup f^{-1}(D)$ and that $f^{-1}(C \cup B) \supset f^{-1}(C) \cup f^{-1}(D)$.

        Take $x \in f^{-1}(C \cup D)$. This implies that $f(x) \in C \cup D$ which means $f(x) \in C$ or $f(x) \in D$ therefore $x \in f^{-1}(C)$ or $x \in f^{-1}(D)$, hence $x \in f^{-1}(C) \cup f^{-1}(D)$, proving $f^{-1}(C \cup B) \subset f^{-1}(C) \cup f^{-1}(D)$.

        Next, take $x \in f^{-1}(C) \cup f^{-1}(D)$. This implies that $x \in f^{-1}(C)$ or $x \in f^{-1}(D)$. Therefore $f(x) \in C$ or $f(x) \in D$ in other words $f(x) \in C \cup D$ so $x \in f^{-1}(C \cup D)$. Finally, this proves that $f^{-1}(C \cup D) = f^{-1}(C) \cup f^{-1}(D)$. \\

        Look at $f^{-1}(Y \setminus C) = X \setminus f^{-1}(C)$.

        We have to prove that $f^{-1}(Y \setminus C) \subset X \setminus f^{-1}(C)$ and $f^{-1}(Y \setminus C) \supset X \setminus f^{-1}(C)$ to prove the equality of the sets.

        Take $x \in f^{-1}(Y \setminus C)$. $f(x) \in Y \setminus C$ therefore $f(x) \in Y$ and $f(x) \notin C$, $x \in f^{-1}(Y)$ and $f(x) \notin f^{-1}(C)$, but $x \in f^{-1}(Y) \implies x \in X$ hence, $x \in X \setminus f^{-1}(C)$.

        Take $x \in X \setminus f^{-1}(C)$. $x \in X$ and $x \notin f^{-1}(C)$. But $x \in X \implies x \in f^{-1}(Y)$, so $f(x) \in Y$ and $f(x) \notin C$, therefore $f(x) \in Y \setminus C$. Hence $x \in f^{-1}(Y \setminus C)$.

        This completes the proof that $f^{-1}(Y \setminus C) = X \setminus f^{-1}(C)$.
    \end{proof}
\end{proposition}


\subsection{The set of real numbers}
There is a set of real numbers $\R$ which admits a function $+: \R \times \R \rightarrow \R$, called addition, write $+(x, y) = x + y$

Addition satisfies,
\begin{enumerate}[label = (A.\arabic*)]
    \item Commutativity: $a + b = b + a$ for all $a, b \in \R$.
    \item Associativity: $(a + b) + c = a + (b + c)$ for all $a, b, c \in \R$.
    \item Existence of $0$: There is a real number $0$ with $0 + a = a$ for all $a \in \R$.
    \item Existence of the negative: For every $a \in \R$ there is a unique $x \in \R$ such that $a + x = 0$.
    
    We write $-a$ for the unique $x$ with $a + x = 0$. $a + (-a) = 0$ by (A.2) $-a + a = 0$ so by uniqueness of the negative $-(-a) = a$.
\end{enumerate}

There is also a multiplication, a function $\cdot: \R \times \R \rightarrow \R$, write $a \cdot b = ab$ for $\cdot(a, b)$.

Multiplication satisfies,
\begin{enumerate}[label = (M.\arabic*)]
    \item Commutativity: $a \cdot b = b \cdot a$ for all $a, b \in \R$.
    \item Associativity: $a \cdot (b \cdot c) = (a \cdot b) \cdot c$ for all $a, b, c \in \R$.
    \item Existence of $1$: There is a real number $1$ different from $0$ such that $1 \cdot a = a$ for all $a \in \R$.
    \item Existence of the inverse: For every $a \in \R \setminus \{0\}$ there is a unique $x \in \R$ such that $a \cdot x = 1$. We write $a ^ {-1}$ for the inverse (or $\frac{1}{a}$).
\end{enumerate}
\begin{enumerate}[label = (D.\arabic*)]
    \item Distributivity : We have
    \[
    a \cdot (b + c) = (a \cdot b) + (a \cdot c)
    \]
    we'll write $a \cdot b + a\cdot c$ as multiplication has precedence over addition.
\end{enumerate}


Let $a \in \R$, $a \cdot 0 = a \cdot (0 + 0) = a \cdot 0 + a \cdot 0$. Recall $a \cdot 0 \in \R$ so there is a negative $-(a \cdot 0)$, add to equation to get, $0 = a \cdot 0$.

Similarly we can show $-a = (-1) \cdot a$ for $a \in \R$. Also $(-a) \cdot (-a) = (-a) ^ 2 = a ^ 2$.


The next property of the reals is an existence of an order $<$. For any $a, b \in \R$ there is a statement $a < b$. We need to specify when this is true. We read this as "$a$ is smaller than $b$".

We require
\begin{enumerate}[label = (O.\arabic*)]
    \item Trichotomy: We either have $a < b,\, a = b,\,\text{or } b < a$ (as a true statement).
    \item Transitivity: If $a < b$ and $b < c$, then $a < c$.
    \item Monotony of Addition: If $a < b$, then $a + c < b + c$ for all $c \in \R$.
    \item Monotony of Multiplication: If $a < b$ and $0 < c$ then $c \cdot a < c \cdot b$
\end{enumerate}
If $a < b$ we also write $b > a$, we also write $a \leq b$ for ($a < b$ or $a = b$).

If $a \in \R$ different from $0$, then either $a < 0$ or $0 < a$. Then $-a$ is also different from $0$ and by   (O.3) we have $0 < -a$ or $-a < 0$.

\begin{proposition}
    Let $a, b, c, d \in \R$. Then
    \begin{enumerate}[label = (\alph*)]
        \item If $a \neq 0$, then $a ^ 2 > 0$. In particular $1 > 0$.
        \item If $a < b$ and $c < d$, then $a + c < b + d$.
        \item If $0 < a < b$ and $0 < c < d$, then $a \cdot c < b \cdot d$.
        \item If $a < b$ and $c < 0$, then $c \cdot b < c \cdot a$.
        \item If $0 < a < b$ then $0 < b ^ {-1} < a ^ {-1}$.
    \end{enumerate}

    \begin{proof}
        \begin{enumerate}[label = (\alph*)]
            \item If $a > 0$, then $a ^ 2 = a \cdot a > a \cdot 0 = 0$
            if $a < 0$, then $-a > 0$ now we use $(-a) ^ 2 = a ^ 2$ and $(-a) ^ 2 > 0$. $1 = 1 \cdot 1 = 1 ^ 2 > 0$.

            \item b
            \item c
            \item Use $-c > 0$ so $-c \cdot a < -c \cdot b$ now apply (O.3) adding $c \cdot a + c \cdot b$.
            \item Note $b ^ {-1} \neq 0$
        \end{enumerate}
    \end{proof}
\end{proposition}


\end{document}