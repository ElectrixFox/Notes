\documentclass[10pt, a4paper]{article}
\usepackage{preamble}

\title{Analysis I}
\author{Luke Phillips}
\date{October 2024}

\begin{document}

\maketitle

\newpage

\section{Basic logic and sets}

\subsection{Logic}
\begin{definition}
    A statement is a sentence which is either true or false.
\end{definition}
\begin{example}
    \phantom{}
    \begin{itemize}
        \item There exist infinitely many prime numbers.
        \item There exists a rational number $x$ with $x ^ 2 = 2$.
    \end{itemize}
\end{example}

\begin{definition}
    Let $A$ and $B$ be statements.
    \begin{enumerate}[label = (\alph*)]
        \item "$A$ and $B$" is the statement that is true when exactly both $A$ and $B$ are true, otherwise false.
        \item "$A$ or $B$" is the statement which is false exactly when both $A$ and $B$ are false, otherwise true.
        \item "Not $A$"\footnote{Called the negation.} is the statement which is true when $A$ is false, and false when $A$ is true.
    \end{enumerate}

    The truth tables are as follows
    \textbf{Gr6}
\end{definition}

\begin{definition}
    Let $A$ and $B$ be statements. The statement "if $A$ then $B$" is defined to be false when $A$ is true and $B$ is false, and true in all other cases.

    The truth table for this is as follows
    \textbf{Gr7}
\end{definition}
\begin{example}
    If $p$ is a prime number, then $p$ is an odd number. (this is false)

    If $p \geq 3$ is a prime number, then $p$ is an odd number. (this is true)
\end{example}

\begin{definition}
    Let $A$ and $B$ be statements. The statement $A$ is equivalent to $B$ for the statement $A \implies B$ and $B \implies A$, or $A \iff B$.
    
    The truth table for this is as follows
    \textbf{Gr8}
\end{definition}

Often statements depend on a variable $x$. This is called a conditional statement $A(x)$.

We can turn a conditional statement $A(x)$ into an unconditional statement using the quantifiers "for all" and "there is"

\begin{example}
    \phantom{}
    
    $\forall x$, we have $A(x)$
    
    $\exists x$ such that $A(x)$
\end{example}


\subsection{Sets}
A set $X$ is a collection of objects. We write $x \in X$ if the object $x$ is contained in the set $X$\footnote{$x \notin X$ for $x$ not in $X$}.

\[
X = \{x\,|\,A(x)\}
\]
is a set of all objects $x$ for which the conditional statement $A(x)$ is true.

\begin{definition}
    Let $X, Y$ be sets.
    \begin{enumerate}[label = (\alph*)]
        \item We say $X$ is a subset of $Y$, write $X \subset Y$, if $x \in X$ implies $x \in Y$
        \item We say that $X$ is equal to $Y$, write $X = Y$, if $X \subset Y$ and $Y \subset X$.
    \end{enumerate}
\end{definition}

\end{document}