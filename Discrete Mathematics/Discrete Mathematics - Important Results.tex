\documentclass[10pt, a4paper]{article}
\usepackage{preamble}

\title{Discrete Mathematics I \\
    \large Important Results}
\author{Luke Phillips}
\date{April 2025}

\begin{document}

\maketitle

\newpage

\tableofcontents

\newpage

\section{Counting}

\subsection{Basic definitions}

\begin{definition}[Arrangement]
    An arrangement is a sequence of items in which order matters.
\end{definition}
\begin{example}
    Here are some arrangements
    (of length $4$)
    of the numbers $123$:
    \begin{align*}
        1111 && 1122 && 1233 \\
        2112 && 2312 && 2231 \\
        3221 && 3332 && 3121
    \end{align*}
    These are not all of the arrangements,
    there are $3 ^ 4$ of them for $123$ of size $4$.

    $1221$ and $2112$ are not the same arrangement,
    as order matters.
\end{example}

\begin{definition}[Permutation]
    A permutation of a set $S$ is an arrangement which uses each item in $S$A exactly once.
\end{definition}
\begin{example}
    The permutations of $123$:
    \begin{align*}
        123 && 213 && 312 \\
        132 && 231 && 321
    \end{align*}
\end{example}

\begin{definition}[Combination]
    A combination is an unordered selection of elements of a set.
\end{definition}
\begin{example}
    The following different arrangements of $123$
    (size $4$)
    are the same combination
    \begin{align*}
        1233 && 3312 && 3123.
    \end{align*}
    Where as
    \begin{align*}
        1233 && 3212 && 1123
    \end{align*}
    are different combinations of $123$.
\end{example}

\begin{definition}[Configuration]
    A configuration can refer to how objects are arranged,
    ordered,
    or laid out.
\end{definition}

\textbf{Binomial coefficient notation}

We use the following notation to denote the number of combinations
(size $k$)
of a set $n$ without repetition
\[
\binom{n}{k} = \frac{n!}{(n - k)!k!}.
\]

\subsection{Notes about the binomial coefficient}
No proofs :C.

\begin{proposition}
    \[
    \binom{n}{k} = \binom{n}{n - k}.
    \]
\end{proposition}

\begin{proposition}
    $1 \leq k \leq n - 1$,
    \[
    \binom{n}{k} = \binom{n - 1}{k - 1} + \binom{n - 1}{k}.
    \]
\end{proposition}

\begin{proposition}
    \[
    (a + b) ^ n = \sum_{k = 0}^{n}\binom{n}{k}a ^ kb ^ {n - k}.
    \]
\end{proposition}

\textbf{Remember that we can find identities using the last proposition}
i.e.
\[
\sum_{k = 0}^{n}\binom{n}{k} = \sum_{k = 0}^{n}\binom{n}{k}1 ^ k1 ^ {n - k} = (1 + 1) ^ n = 2 ^ n.
\]

\subsection{Arrangements and combinations with repetitions}

\begin{theorem}
    Given a list of $n$ objects of $r$ different types,
    in which objects of type $i$ occur $n_i$ times
    (with $n_1 + \dotsc + n_r = n$),
    the number of arrangements of the list is
    \[
    \frac{n!}{n_1!n_2!\dotsi n_r!}.
    \]
\end{theorem}

\begin{center}
\fbox{
\begin{minipage}{0.9\textwidth}
    The above is pretty much saying,
    get the $n$ objects and since all of each type are grouped together we can ignore permutations within that group.

    For example,
    find the arrangements of the word REARRANGE:

    There are $3$ R's,
    $2$ A's,
    $2$ E's,
    $1$ G and $1$ N.

    So looking at one arrangement:
    RERARANGE,
    it doesn't matter which A or R is where since they are the same,
    we can rearrange them the number of times they appear times,
    i.e.
    we get
    \[
    \frac{9!}{3!2!2!1!1!}
    \]
    arrangements.
\end{minipage}
}
\end{center}
\hfill

Looking at combinations,
since order doesn't matter we can group the alike elements together.

\begin{example}
    The number of combinations
    (size $5$)
    with repetition of $\{0, 1, 2\}$ where each number appears at least once.

    Since order doesn't matter lets lay it out as such
    \[
    0\ 0\ 0\ 1\ 2
    \]
    then we can get all different variations of that as long as we have one $0, 1$ and $2$.

    We can separate them using a line
    \[
    0\ 0\ 0\ |\ 1\ |\ 2
    \]
    which we can see we will need $2$ lines to separate the $3$ different types of items.
    Looking at this configuration without numbers:
    \[
    A\ A\ A\ |\ B\ |\ C
    \]
    we can see that we are able to put the lines in $4$ places as we cannot have them on either end as that would eliminate one of our types of item.

    Hence we need to \textbf{choose} \textbf{two} of the \textbf{four} places to put the lines,
    we can do this
    \[
    \binom{4}{2}
    \]
    ways,
    thus this is the number of combinations.
\end{example}

\begin{theorem}
    The number of $k$-combinations from $n$ different objects with repetition and each object occurring at least once is $\binom{k - 1}{n - 1}$ for $k \geq n$.
\end{theorem}
This is just a generalisation of the above.

Strengthening this to disregard the requirement of needing at least one of each object:
\begin{theorem}
    The number of $k$-combinations from $n$ different objects with repetition is
    \[
    \binom{k + n - 1}{n - 1} = \binom{k + n - 1}{k}.
    \]
\end{theorem}
This is since we are allowing a combination of size $k$ from $n$ objects given that each object may be selected more than once.

\subsection{Extending binomial coefficients}
\begin{proposition}
    \[
    \binom{-n}{k} = (-1) ^ k\binom{k + n - 1}{k} = (-1) ^ k\binom{k + n - 1}{n - 1}.
    \]
\end{proposition}

\textbf{Remember} we can differentiate and integrate series term-wise,
\begin{example}
    \[
    \sum_{r = 1}^{n}\binom{n}{r}(-1) ^ rr.
    \]

    \begin{align*}
        (1 + x) ^ n &= \sum_{r = 0}^{n}\binom{n}{r}x ^ r  \\
        &\iff \\
        \frac{d}{dx}[(1 + x) ^ n] &= \frac{d}{dx}\left[\sum_{r = 0}^{n}\binom{n}{r}x ^ r\right] \\
        &\iff \\
        n(1 + x) ^ {n - 1} &= \sum_{r = 0}^{n}\binom{n}{r}rx ^ r
        \intertext{Take $x = -1$}
        &\iff \\
        n(1 - 1) ^ {n - 1} &= \sum_{r = 0}^{n}\binom{n}{r}(-1) ^ rr \\
        &\iff \\
        0 &= \sum_{r = 0}^{n}\binom{n}{r}(-1) ^ rr
    \end{align*}
    so we get
    \[
    \sum_{r = 1}^{n}\binom{n}{r}(-1) ^ rr = 0
    \]
    for all $n > 1$ and $-1$ for $n = 1$.
\end{example}

\subsection{Pigeon-hole and inclusion-exclusion principle}

\textbf{Pigeon-hole principle}

If $n$ envelopes are distributed among $m$ pigeon-holes,
and $n > m$,
then at least one pigeon-hole contains more than one envelope.

\hfill
\textbf{Inclusion-exclusion principle}
\begin{theorem}[Inclusion-exclusion formula]
    For a positive integer $n$ and finite sets $A_1, \dotsc, A_n$
    \begin{align*}
        |A_1 \cup A_2 \cup \dotsi \cup A_n| &= \sum_{k = 1}^{n}(-1) ^ {k + 1}S_k,
        \intertext{where}
        S_k &= \sum_{i_1 < i_2 < \dotsi < i_k}|A_{i_1} \cap A_{i_2} \cap \dotsi \cap A_{i_k}|.
    \end{align*}
\end{theorem}

\begin{definition}[Derangement]
    A derangement is a permutation in which no element is in its correct position.
    That is,
    for every $k$,
    item $k$ is not in the $k$th place.
\end{definition}

\subsection{Generating functions}

\begin{definition}
    Let $a_0, a_1, \dotsc$ be a sequence of real numbers.
    The
    (ordinary)
    generating function for the sequence $(a_k)$ is the formal sum
    \[
    f(x) = \infsumo a_kx ^ k.
    \]
\end{definition}

\begin{lemma}[Infamous Lemma $7.1.$]
    \begin{enumerate}[label = (\alph*)]
        \item
        For $n \geq 0$ an integer,
        \[
        \sum_{k = 0}^{n}x ^ k = \frac{1 - x ^ {n + 1}}{1 - x}.
        \]

        \item
        \[
        \infsumo x ^ k = \frac{1}{1 - x}.
        \]

        \item
        \[
        \infsumo\binom{n + k - 1}{k}x ^ k = \frac{1}{(1 - x) ^ n}.
        \]

        \item
        If $f(x) = \infsumo a_kx ^ k$ and $g(x) = \infsumo b_kx ^ k$ are the generating functions of sequences $(a_k)$ and $(b_k)$ respectively,
        then
        \[
        f(x)g(x) = \infsumo c_kx ^ k
        \]
        is the generating function of the sequence $(c_k)$ where $c_k = \sum_{\ell = 0}^{k}a_{\ell}b_{k - \ell}$.
    \end{enumerate}
\end{lemma}

\newpage

\section{Methods}

\subsection{Recurrence relations}
To solve a recurrence relation
\[
a_n = \dotsc
\]
form the characteristic equation of the recurrence relation and its roots,
$\alpha, \beta$,
will be part of the final solution
\[
a_n = d_1\alpha ^ n + d_2\beta ^ n.
\]

For recurrence relations of the form
\[
a_n = c_1a_{n - 1} + \dotsc + c_ra_{n - r} + f(n)
\]
solve the first order case $a_n = c_1a_{n - 1} + \dotsc + c_ra_{n - r}$ and call it $g(n)$.

Find a particular solution $a_n = p(n)$ to the $f(n)$ part.

The general solution is $a_n = g(n) + p(n)$,
apply the initial conditions to find the unknown constants.

\textit{In short just solve it like an ODE}.

\subsubsection{Stairs example}
\begin{example}
    Climb down stairs
    one or two at a time.
    How many distinct ways are there to climb $n$ stairs.

    \begin{solution}
        Call the answer $f_n$.

        $f_1 = 1$ as there is only one way to go one step,
        $f_2 = 2$ as can go two or one then one.
        $f_3 = 3$,
        two and one,
        one and two;
        one,
        one and one.

        $f_0 = 1$ since you can only go zero steps in one way.

        Suppose $n \geq 2$.
        We can climb one step first then we can climb the other $n - 1$ stairs $f_{n - 1}$ ways.
        Similarly,
        we can climb two steps first then $n - 2$ remain to be done in $f_{n - 2}$ ways.

        Since these have different starts they are disjoint combinations,
        so
        \[
        f_n = f_{n - 1} + f_{n - 2}\quad(n \geq 2).
        \]        
    \end{solution}
\end{example}

One can generalise the above example to being able to take $s_1, \dotsc, s_k$ stairs at once,
then
\[
f_n = f_{n - s_1} + f_{n - s_2} + \dotsc + f_{n - s_k}\quad(n \geq s_k + 1).
\]

\subsection{Counting with generating functions}
Say we have $(e_1, e_2, e_3, e_4)$ solutions to the equation
\[
e_1 + e_2 + e_3 + e_4 = 7,
\]
with $0 \leq e_1 \leq 4$,
$0 \leq e_2 \leq 3$,
$0 \leq e_2 \leq 2$,
and $0 \leq e_4 \leq 1$.

Assign each $e_i$ an expressions with the sum of $x ^ p$ for the $p$ possible values of $e_i$,
e.g.
$e_1$ would be $x ^ 0 + x ^ 1 + x ^ 2 + x ^ 3 + x ^ 4$.

Now we can multiply them all together
\[
f(x) = (x ^ 0 + x ^ 1 + x ^ 2 + x ^ 3 + x ^ 4)(x ^ 0 + x ^ 1 + x ^ 2 + x ^ 3)(x ^ 0 + x ^ 1 + x ^ 2)(x ^ 0 + x ^ 1),
\]
each term representing a possible value for $e_i$.
$x ^ {e_1}\cdot x ^ {e_2}\cdot x ^ {e_3}\cdot x ^ {e_4} = x ^ {e_1 + e_2 + e_3 + e_4} = x ^ 7$.
So the number of solutions corresponds to the coefficient of $x ^ 7$ in $f(x)$.

\subsection{Recurrence relations with generating functions}
Say we have a recurrence relation $f_n = f_{n - 2} + f_{n - 4}$ with the generating function for $f_n$ being $f(x) = \infsum[n = 0]f_nx ^ n$ then we can do the following.

Clearly
\[
f_n = f_{n - 2} + f_{n - 4}
\]
is only true for $n \geq 4$ as we cannot have $f_{-1}$.

Now we need to work with this expression for the generating function
\begin{align*}
    f(x) &= \infsum[n = 0]f_nx ^ n \\
    &= f_0 + f_1x + f_2x ^ 2 + f_3x ^ 3 + \infsum[n = 4]f_nx ^ n \\
    &= f_0 + f_1x + f_2x ^ 2 + f_3x ^ 3 + \infsum[n = 4]f_{n - 2}x ^ n + \infsum[n = 4]f_{n - 4}x ^ n \\
    &= f_0 + f_1x + f_2x ^ 2 + f_3x ^ 3 + x ^ 2\infsum[n = 2]f_nx ^ n + x ^ 4\infsum[n = 0]f_nx ^ n \\
    &= f_0 + f_1x + f_2x ^ 2 + f_3x ^ 3 + x ^ 2(\infsum[n = 0]f_nx ^ n - f_0 - f_1x) + x ^ 4\infsum[n = 0]f_nx ^ n \\
    &= f_0 + f_1x + f_2x ^ 2 + f_3x ^ 3 - f_0x ^ 2 - f_1x ^ 3 + x ^ 2\infsum[n = 0]f_nx ^ n + x ^ 4\infsum[n = 0]f_nx ^ n \\
    &= f_0 + f_1x + f_2x ^ 2 + f_3x ^ 3 - f_0x ^ 2 - f_1x ^ 3 + x ^ 2f(x) + x ^ 4f(x) \\
    \intertext{which we can then substitute in values for $f_0, \dotsc, f_3$ to get an expression}
    &\iff \\
    f(x)(1 - x ^ 2 - x ^ 4) &= f_0 + f_1x + f_2x ^ 2 + f_3x ^ 3 - f_0x ^ 2 - f_1x ^ 3 \\
    &\iff \\
    f(x) &= \frac{f_0 + f_1x + f_2x ^ 2 + f_3x ^ 3 - f_0x ^ 2 - f_1x ^ 3}{1 - x ^ 2 - x ^ 4}.
\end{align*}
This is the expression for this particular sequence.

\begin{center}
\fbox{
\begin{minipage}{0.9\textwidth}  
    In compact terms,
    we expand the series to the starting point of our recurrence relation $n \geq s$.
    Then we apply the recurrence relation,
    expand,
    and relabel the series to get $\infsum[n = 0]a_nx ^ n = f(x)$ and $x$ terms.
    Finally,
    we rearrange for $f(x)$.
\end{minipage}
}
\end{center}
\hfill


\subsection{Counting integer solutions}
Given an expression of the following form
\[
x_1 + x_2 + \dotsc + x_n = k
\]
how many integer solutions are there to this equation where $\forall i \in \N, 0 \leq x_i$?

The answer is
\[
\binom{k + n - 1}{n - 1}.
\]

We also have the case where $n_i \leq x_i$,
in general we can rewrite this equation as follows:
say we have
\[
x_1 + \dotsc + x_n = k
\]
and $n_i \leq x_i$ then to find the number of integer solutions which satisfy these conditions we use the following substitution
\[
y_i = x_i + n_i
\]
then we get
\[
x_1 + \dotsc + x_n = k \iff y_1 + \dotsc + y_n + (n_1 + \dotsc + n_n) = k \iff y_1 + \dotsc + y_n = k - (n_1 + \dotsc + n_n)
\]
meaning if we let $a = k - (n_1 + \dotsc + n_n)$ and simply relabel $y_i = x_i$ then
\[
x_1 + \dotsc + x_i = a
\]
is the same problem as the original which gives us the final answer:
\[
\binom{a + n - 1}{n - 1} = \binom{k - (n_1 + \dotsc + n_n) + n - 1}{n - 1}.
\]

Now let us look at the case where $x_i \leq n_i$ for some $x_i$ we would need to use the inclusion-exclusion principle to find the number of combinations,
e.g.
if we had $x_1 + x_2 + x_3 = 5$ with $0 \leq x_1$,
$0 \leq x_2$ and $x_3 \leq 3$,
we would find the number with only $x_1 + x_2 = 5$,
$x_1 + x_2 = 4$,
and subtract the intersection.





\end{document}