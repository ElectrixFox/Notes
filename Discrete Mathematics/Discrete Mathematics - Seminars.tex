\documentclass[10pt, a4paper]{article}
\usepackage{preamble}

\title{Discrete Mathematics}
\author{}
\date{January 2025}

\begin{document}

\maketitle

\newpage

\section{Flows and Matchings}

\begin{definition}
    Let $V$ be a finite set,
    the vertices,
    containing two special
    (distinct)
    elements $s$
    (the source)
    and $t$
    (the sink).
    Suppose for every ordered pair $u, v \in V$ we have a capacity $c_{uv}$,
    which is a non-negative integer;
    denote the set of capacities by $C = \{c_{uv}\,:\,u, v \in V\}$.
    We call $(V, C)$ a capacitated network.
\end{definition}

If we have some resource we want to move from the source $s$ to the sink $t$.
We have unlimited resource to transport,
we are limited by the fact that we must route the resource from vertex to vertex is bounded by the capacity $c_{uv}$.
If $c_{uv} = 0$,
we cannot move anything directly from $u$ to $v$,
however there may be an indirect route.

\textbf{Application examples}:

Water flows through pipes,
or a river network,
linking the vertices of $V$.
Our capacity of the river or pipe is the volume of water which can pass through per unit time.

In a road network built on the cities located at vertices of $V$,
traffic can get from city $s$ to city $t$.
The capacity of the road would be how many cars can travel the route per unit time.


We should look at directed graph.

\begin{definition}
    Given a capacitated network $(V, C)$,
    let $E$ be the set of directed edges $E = \{(u, v)\,:\, u, v \in V, c_{uv} > 0\}$.
    Then $G = (V, E)$ is the directed graph representing the available routes on vertices $V$ given capacities $C$.
\end{definition}

Usually we write $uv$ for $(u, v)$.












\end{document}