\documentclass[10pt, a4paper]{article}
\usepackage{preamble}

\declaretheorem[style = avgstyle, name = Problem]{problem}

\title{Discrete Mathematics Problem Sheet Solutions}
\author{Luke Phillips}
\date{October 2024}

\begin{document}

\maketitle

\newpage

\section{Introduction}

\begin{problem}[Q10]\phantom{}
    \begin{enumerate}[label = (\alph*)]
        \item In how many ways can $6$ people be seated around dinner table if all rotations of a given arrangement are regarded as the same?
        \item In how many ways can $6$ different spherical beads be arranged on a circular ring?
    \end{enumerate}
    \begin{enumerate}[label = (\alph*)]
        \item 
        \begin{proof}[Solution]\renewcommand{\qedsymbol}{}
            If we assign a number to each seat and a letter to each person then person $A$ has $6$ choices, person $B$ has $5$ then person $C$ has $4$ choices and so on until there is only one seat left. This looks like the following
            \[
            6 \times 5 \times 4 \times 3 \times 2 \times 1 = 6!,
            \]
            hence there are $6!$ ways of seating $6$ people around the table, however, for each way there is a rotation which is similar therefore we would divide by $6$ to get $\frac{6!}{6} = 5!$ ways excluding rotations.
        \end{proof}
        \item
        \begin{proof}[Solution]\renewcommand{\qedsymbol}{}
            This is similar to the previous problem as there are $6$ different beads and they can be arranged in different ways. We have to exclude rotations by the definition of just looking at the circular ring. Therefore we get $5!$ ways of arranging the beads.
        \end{proof}
    \end{enumerate}
\end{problem}

\newpage

\begin{problem}[Q11]
    \begin{enumerate}[label = (\alph*)]
        \item How many $5$ letter words can be made from $\{\text{X, Y, Z}\}$ that include at least one $X$?
        \item How many of the words specified in (a) also contain Y?
        \item How many of the words specified in (b) also contain Z?
    \end{enumerate}

    \begin{proof}[Solution]\renewcommand{\qedsymbol}{}
        
    \end{proof}
\end{problem}

\newpage

\begin{problem}[Q12]\phantom{}
    \begin{enumerate}[label = (\alph*)]
        \item In how many ways can $8$ identical pieces be placed on the squares of an $8$-by-$8$ chess board so that no row or column contains more than one piece?
        \item The same question as in (a) but with $n$ pieces on an $n$-by-$n$ board.
    \end{enumerate}
    \begin{enumerate}[label = (\alph*)]
        \item 
        \begin{proof}[Solution]\renewcommand{\qedsymbol}{}
            To simplify this problem we will think of all of the different positions along the row for each piece on each row. Firstly, the first piece has $8$ options to be placed on, but since this eliminates one of the columns, the second piece (the piece on the row above the first) will only have $7$ options. This pattern continues, where the third piece only has $6$ options, the fourth $5$ options, and so on until we reach the eighth piece with only one option. Therefore using the multiplication principle we can see that the total number of ways is
            \[
            8 \times 7 \times 6 \times 5 \times 4 \times 3 \times 2 \times 1 = 8!.
            \]
        \end{proof}
        \item 
        \begin{proof}[Solution]\renewcommand{\qedsymbol}{}
            Using similar logic to that of the previous question we can see that, for an $n\times n$ grid the choices go as follows,
            \begin{enumerate}[label = {}]
                \item $n$ choices for the $1^{\text{st}}$ row.
                \item $n - 1$ choices for the $2^{\text{nd}}$ row.
                \item $\vdots$
                \item $2$ choices for the $(n - 1)^{\text{th}}$ row.
                \item $1$ choice for the $n^{\text{th}}$ row.
            \end{enumerate}
            We can clearly see that the number of ways will therefore be $n!$.
        \end{proof}
    \end{enumerate}
\end{problem}

\newpage

\begin{problem}[Q14]
    A class consists of $n$ girls and $m$ boys. In how many ways can they stand in a single line so that all the girls are together?

    \begin{proof}[Solution]\renewcommand{\qedsymbol}{}
        
    \end{proof}
\end{problem}

\newpage

\begin{problem}[Q15]
    How many rearrangements of the letters FORMULA contain FOR as three successive letters in that order? How about FOR together but not necessarily in that order?
    \begin{proof}[Solution]\renewcommand{\qedsymbol}{}
        When we are considering rearrangements of the letters of FORMULA which contain FOR as three successive letters we can see that there are $7$ letters in the word FORMULA and $3$ of them have to be taken by FOR hence there are $5$ blocks (or boxes) which can contain either a letter or FOR. The remaining $4$ letters can be placed in the blocks in $4!$ different ways. This means that there are $5$ different ways to place the FOR in the blocks and $4!$ different ways to place the rest of the letters in the blocks, hence the total number of rearrangements is
        \[
        5 \cdot 4! = 120.
        \]
        If we consider FOR still being together but not in that order we get
        \[
        5 \cdot 4! \cdot 3! = 720
        \]
        rearrangements as there are $3!$ ways of rearranging FOR when it is in its block.
    \end{proof}
\end{problem}

\newpage

\begin{problem}[Q16]
    $6$ people sit on a row of $6$ chairs. They stand up and then sit down again (possibly on different chairs). In how many ways can the re-seating be done so that the two end chairs are each not occupied by the same person as sat on them originally.
    \begin{proof}[Solution]\renewcommand{\qedsymbol}{}
        
    \end{proof}
\end{problem}

\end{document}