\documentclass[10pt, a4paper]{article}
\usepackage{preamble}

\title{Discrete Mathematics \\
    \large Prereading}
\author{Luke Phillips}
\date{October 2024}

\begin{document}

\maketitle

\newpage

\section{Introduction}

\subsection{Set notation}
\textbf{Basic notation}

A set is defined by the elements that belong to the set.

If an element $x$ belongs to a set $S$, we write $x \in S$.

Finite or countably infinite sets are those which we can write as a (finite or infinite) list, such as
\[
S = \{e_1, e_2, \dots, e_n\},
\]
(a finite set, with elements labelled by finitely many integers) or
\[
S = \{e_1, e_2, e_3, \dots\},
\]
(representing a countably infinite set, with elements labelled by all positive integers).

The order of elements in a set is unimportant.

A set can be described using the following notation
\[
S = \{n: n \text{ is an integer}, 0\leq n \leq 9\}
\]
which means all $n$ such that the properties to the right of the '$:$' hold.

\textbf{Examples of standard sets}
\begin{itemize}
    \item $\Z = \{\dots, -2, -1, 0, 1, 2, \dots\} = \{0, \pm1, \pm2, \dots\}$, the set of integers (whole numbers).
    \item $\N = \{1, 2, \dots\} = \{n: n\in\Z, n>0\}$, the natural numbers (positive integers).
    \item $\emptyset = \{\}$ is the empty set (the set with no elements).
\end{itemize}

\textbf{More notation}

As mentioned before, $x \in S$ means element $x$ is a member of the set $S$. The negation is $x \notin S$.

If $R$ and $S$ are sets, then $R \subseteq S$ means that $R$ is a subset of $S$, i.e., every element of $R$ is also an element of $S$. That is, for all $x;\ x \in R \implies x \in S$.

Note that $R = S$ if and only if $R \subseteq S$ and $S \subseteq R$.

$R \subset S$ is used to denote that $R$ is a strict subset of $S$, meaning $R \subseteq S$ and $R \neq S$.

If $S$ is a finite set, its size or cardinality, denoted by $|S|$, is the number of elements that it contains.


\textbf{Set operations}

Given several sets $A$ and $B$, $A \cup B$ is the union of $A$ and $B$, which is the set consisting of those elements that are in $A$ or in $B$ (or both)

$A \cap B$ is the intersection of sets $A$ and $B$, which is the set consisting of those elements that are in both $A$ and $B$.

If two sets $A$ and $B$ do not intersect ($A \cap B = \emptyset$) we say that $A$ and $B$ are disjoint.

A collection of sets can be indexed by numbers, e.g. $A_1, A_2, \dots, A_n$.

Then we may use compact notation for multiple unions and intersections:
\begin{align*}
    \bigcup_{i = 1}^{n}{A_i} &= A_1 \cup A_2 \cup \dots \cup A_n \\
    &= \{x: x \in A_i \text{ for at least one } i\};
\end{align*}

\begin{align*}
    \bigcap_{i = 1}^{n}{A_i} &= A_1 \cap A_2 \cap \dots \cap A_n \\
    &= \{x: x \in A_i \text{ for every } i\};
\end{align*}

\textbf{Cartesian products}

If $X$ and $Y$ are two sets, their Cartesian product is
\[
X \times Y = \{(x, y): x \in X \text{ and } y \in Y\},
\]
the set of all ordered pairs $(x, y)$ with first component in $X$ and second in $Y$.

For example, if $X = \{1, 2, 3\}$ and $Y = \{a, b\}$, then the set $X \times Y$ is
\[
\{(1, a), (1, b), (2, a), (2, b), (3, a), (3, b)\},
\]
while the set $Y \times X$ is
\[
\{(a, 1), (a, 2), (a, 3), (b, 1), (b, 2), (b, 3)\}.
\]

\newpage

\section{Arrangements and permutations}

\subsection{Arrangements}
\begin{definition}[Arrangement]
    An \textbf{arrangement} (or list or string) is a (finite) sequence of items in which the \textbf{order} of the items in the sequence \textbf{matters}.
\end{definition}

The length of an arrangement is the number of items in the sequence.

Standard problem \#1.
How many arrangements of length $k$ can be made from elements of a set $S$ of size $|S| = n$?

This would be $n ^ k$ as each arrangement has $k$ places each of which can be occupied by any one of $n$ entries.


Standard problem \#2.
How many arrangements of length $k$ with no repeats can be made from elements of a set $S$ of size $|S| = n$?

This would be $n(n - 1)(n - 2)\dotsi(n - k + 1) = \frac{n!}{(n - k)!}$,
provided $k \leq n$.
This is because every time we make a choice we remove a possible option.


\textbf{Multiplication Principle}. for an arrangement of length $k$, if for $i \in \{1,\,2,\,\dotsc,\,k\}$ there are $n_i$ choices for the $i$th item, regardless of previous choices, then the number of possible arrangements is $n_1n_2\dotsi n_k$.

\subsection{Permutations}
\begin{definition}[Permutation]
    A permutation on a finite set $S$ is an arrangement of length $|S|$ that uses each item from $S$ exactly once.
\end{definition}

\begin{proposition}
    If $|S| = n$, then there are $n!$ permutations on $S$.
    
    \begin{proof}
        $n(n - 1)(n - 2)\dotsi 2 \cdot 1 = n!$\footnote{This is standard problem \#2 (no repeats) with $n = k$.}
    \end{proof}
\end{proposition}

We can generalise this concept. For a non-negative integer $r$, an $r$-permutation on $S$ is an arrangement of $r$ distinct elements from $S$.

We write $P(n,\,r)$ for the number of $r$-permutations on a set of size $n$.
\[
P(n, r) = n(n - 1)\dotsi (n - r + 1) = \frac{n!}{(n - r)!}, \text{ for } 0 \leq r \leq n.
\]

\subsection{Counting examples}

\textbf{Addition Principle}. Let $A$ and $B$ be two finite sets that are disjoint, i.e. $A \cap B = \emptyset$. Then
\[
|A \cup B| = |A| + |B|.
\]
More generally, if $A_1,\,A_2,\,\dotsc,\,A_n$ are pairwise disjoint, meaning that $A_i \cap A_j = \emptyset$ for all $i \neq j$, then
\[
|A_1 \cup A_2 \cup \dotsi \cup A_n| = |A_1| + |A_2| + \dotsi + |A_n|.
\]

\newpage

\section{Combinations}
\begin{definition}[Combination]
    A \textbf{combination} from a set of objects is an \textbf{unordered} selection of elements of that set.
\end{definition}

Formally, a combination on a set $S$ is identified by the multiplicities $m_i,\, i \in S$, where $m_i \in \{0, 1, 2, \dotsc\}$ determines how many times element $i$ appears in the combination.

The size of a combination is the total number of selections, i.e. the sum of all the multiplicities $\sum_{i \in S}m_i$. We sometimes call a combination of size $k$ a $k$-combination.

Standard problem \#3.
How many combinations without repetition do we have of a given size $k$ from a set of size $n$?

Consider the arrangements, without repetitions, of length $k$ from the set of size $n$. There are $P(n, k) = \frac{n!}{(n - k)!}$ of those. These are ordered lists, so sort them into groups of lists using the same letters. Each group contains $k!$ lists. The number of groups is $\frac{P(n, k)}{k!}$, and each corresponds to a combination. So the number of $k$-combinations without repetitions is
\[
\frac{P(n, k}{k!} = \frac{n!}{k!(n - k)!}.
\]
This is the binomial coefficient notation
\[
\binom{n}{k} = \frac{n!}{k!(n - k)!}\quad\text{"$n$ choose $k$"}.
\]

\begin{proposition}
    $\binom{n}{k} = \binom{n}{n - k}$.
    \begin{proof}
        Observe that each choice of $k$ objects from $n$ to take corresponds exactly to a choice of $n - k$ objects to leave.
    \end{proof}
\end{proposition}

\begin{proposition}[Pascal's formula]
    For $1 \leq k \leq n - 1,\, \binom{n}{k} = \binom{n - 1}{k - 1} + \binom{n - 1}{k}$.
    \begin{proof}
        \textbf{To do}.
    \end{proof}
\end{proposition}

\begin{example}
    Let $N$ be the product of $k$ consecutive positive integers. Show that $N$ is (exactly) divisible by $k!$.
    \begin{proof}[Solution]
        We can write $N = (a + 1)(a + 2)\dotsi (a + k)$ for some integer $a \geq 0$. But then
        \[
        N = \frac{(a + k)!}{a!} = \binom{a + k}{a}k!,
        \]
        so $\frac{N}{k!}$ is $\binom{a + k}{a}$, which is a positive integer for all positive integers $k$.
    \end{proof}
\end{example}

\begin{theorem}[The binomial theorem]
    If $n$ is a positive integer and $a, b$ are real numbers, then
    \[
    (a + b) ^ n = \sum_{k = 0}^{n}\binom{n}{k}a ^ {k} b ^ {n - k}.
    \]
    \begin{proof}
        The expression
        \[
        (a + b)(a + b) \dotsi (a + b)
        \]
        when multiplied out, gives various terms of the form $a ^ k b ^ {n - k}$. For fixed $k$, the number of terms $a ^ {k} b ^ {n - k}$ is $\binom{n}{k}$ since one chooses the $k$ brackets in the product from which to choose the factor $a$ (rather than $b$).

        The product gives rise to every arrangement of length $n$ using $a$ and $b$, and grouping these according to the number of times that $a$ appears gives the result.
    \end{proof}
\end{theorem}

Prove that the following identities are valid for integer $n \geq 0$.
\begin{enumerate}[label = (C\arabic*)]
    \item $\sum_{k = 0}^{n}\binom{n}{k} = 2 ^ n$.
    \item $\sum_{k = 0}^{n}\binom{n}{k}(-1) ^ k = 0$ (provided $n \geq 1$).
    \item $\sum_{k = 0}^{n}\binom{n}{k}2 ^ k = 3 ^ n$.
    \item $\sum_{k = 0}^{n}\binom{n}{k}k = n2 ^ {n - 1}$.
\end{enumerate}

\newpage

\section{Arrangements and combinations with repetitions}

\subsection{Arrangements with repetitions}
Standard problem \#4.
How many arrangements are there of a list o objects in which some objects are repeated?

\begin{theorem}
    Given a list of $n$ objects of $r$ different types, in which objects of type $i$ occur $n_i$ times (with $n_1 + \dotsi + n_r = n$), the number of arrangements of the list is
    \[
    P(n; n_1, n_2, \dotsc, n_r) = \frac{n!}{n_1!n_2! \dotsi n_r!}.
    \]
    \begin{proof}
        EXERCISE!!!
    \end{proof}
\end{theorem}

\subsection{Combinations with repetitions}
Standard problem \#6.
How many $k$-combinations with repetitions from $n$ objects are there in which each object is chosen at least once in the combination?

This would be the following theorem.
\begin{theorem}
    The number of $k$-combinations from $n$ different objects with repetition and each object occurring at least once is $\binom{k - 1}{n - 1}$ for $k \geq n$.
    \begin{proof}
        Fix some order for the $n$ objects.
        sorting each $k$-combination so that its objects respect this order.
        There are $k - 1$ gaps between items in the list.
        Each combination is described uniquely by inserting $n - 1$ markers into the $k - 1$ gaps,
        where the markers indicate the boundary between successive types of object.
    \end{proof}
\end{theorem}

Standard Problem \#5.
Given a set of $n$ objects,
how many combinations of size $k$ can be chosen if objects may be selected more than once?

As before,
this solution is the following theorem.
\begin{theorem}
    The number of $k$-combinations from $n$ different objects with repetition is
    \[
    \binom{k + n - 1}{n - 1} = \binom{k + n - 1}{k}.
    \]
\end{theorem}

\subsection{The extended binomial coefficients}

Maclaurin’s theorem says that, for $\alpha \in \R$ and integer $k \geq 0$,
\[
\binom{\alpha}{k} = \frac{\alpha(\alpha - 1)\dotsi (\alpha - k + 1)}{k!}.
\]

Note:
\begin{itemize}
    \item If $\alpha$ is a positive integer and $0 \leq k \leq \alpha$,
    this is just $\alpha$ choose $k$
    \item If $\alpha$ is a positive integer and $k > \alpha$,
    then the numerator includes a factor $0$ so $\binom{\alpha}{k} = 0$:
    in this case the power series for $(1 + x) ^ \alpha$ terminates after finitely many terms.
    \item In general,
    if $\alpha$ is not an integer, then the numerator doesn't include a $0$, so
    $\binom{\alpha}{k} \neq 0$ and the power series for $(1 + x) ^ \alpha$ has infinitely many terms
\end{itemize}

\begin{proposition}[Principle of upper negation]
    If $n$ and $k$ are positive integers, then
    \[
    \binom{-n}{k} = (-1) ^ k \binom{k + n - 1}{k} = (-1) ^ k \binom{k + n - 1}{n - 1}.
    \]
    \begin{proof}
        \begin{align*}
            \binom{-n}{k} &= \frac{(-n)(-n - 1)(-n - 2)\dotsi(-n - k + 1)}{k!} \\
            &= (-1) ^ k\frac{n(n + 1)(n + 2)\dotsi(n + k - 1)}{k!} \\
            &= (-1) ^ k\frac{(n + k - 1)!}{(n - 1)!k!},
        \end{align*}
        which is as shown.
    \end{proof}
\end{proposition}

\newpage

\section{Three principles}

\subsection{Mathematical induction}
Suppose that for each integer $n = 0, 1, 2, \dotsc$ we can make a statement (or proposition) $P(n)$ associated with $n$.
For example $P(n)$ might be the statement $2 ^ n > 2n$.
For each integer $n$,
$P(n)$ will be either true or false.
In this case, $P(1)$ is false and so is $P(2)$,
but $P(3)$ is true.
$P(n)$ is true for any $n \geq 3$.
We will use induction to prove this.

Suppose that we know $P(n)$ is true for some fixed $n \geq 3$.
Consider the statement $P(n + 1) : 2 ^ {n + 1} > 2(n + 1)$.
But
\[
2 ^ {n + 1} = 2 \cdot 2 ^ n > 2 \cdot 2n,
\]
by $P(n)$, which we said was true.
So we have established the truth of $P(n + 1)$ assuming the truth of $P(n)$,
$n \geq 3$.
This proves that $P(n)$ holds for all $n \geq 3$,
by induction.

\begin{theorem}[Principle of mathematical induction]
    Suppose that $P(n)$ is a sequence of propositions labelled by integers $n \geq n_0$.
    Suppose that
    \begin{enumerate}[label = (\alph*)]
        \item $P(n_0)$ is true; and
        \item For any $n \geq n_0$, the truth of $P(n)$ implies the truth of $P(n + 1)$.
    \end{enumerate}
    Then, for every $n \geq n_0$,
    $P(n)$ is true.
    \begin{proof}
        $P(n_0)$ is true by assumption (a).
        Then by (b),
        $P(n_0 + 1)$ is also true.
        Again by (b),
        $P(n_0 + 2)$ is also true.
        And so on.
        This chain of deductions eventually reaches any given $n \geq n_0$,
        so $P(n)$ is true for all $n \geq n_0$.
    \end{proof}
\end{theorem}

A proof by induction using the above principle consists of two parts:
verifying the starting case (a) and then establishing the inductive step (b).
The inductive step involves assuming the inductive hypothesis $P(n)$ and deducting $P(n + 1)$ from that.

\subsection{The pigeon-hole principle}
The pigeon-hole principle.
If $n$ envelopes are distributed among $m$ pigeon-holes,
and $n > m$,
then at least one pigeon-hole contains more than one envelope.



\subsection{The inclusion-exclusion principle}

\begin{proposition}
    We have
    \[
    |A_1 \cup A_2 \cup A_3| = |A_1| + |A_2| + |A_3| - |A_1 \cap A_2| - |A_1 \cap A_3| - |A_2 \cap A_3| + |A_1 \cap A_2 \cap A_3|.
    \]
    \begin{proof}
        Apply the formula
        \[
        |A_1 \cup A_2| = |A_1| + |A_2| - |A_1 \cap A_2|
        \]
        repeatedly.
    \end{proof}
\end{proposition}

\begin{theorem}[Inclusion-exclusion formula]
    For a positive integer $n$ and finite sets $A_1, \dotsc, A_n$,
    \[
    |A_1 \cup A_2 \cup \dotsi \cup A_n| = \sum_{k = 1}^{n}(-1) ^ {k + 1}S_k,
    \]
    where
    \[
    S_k = \sum_{i_1 < i_2 < \dotsi < i_k}|A_{i1} \cap A_{i2} \cap \dotsi \cap A_{ik}|.
    \]
    \begin{proof}
        The base case for this is $n = 2$ and it has been given by the previously mentioned formula
        \[
        |A_1 \cup A_2| = |A_1| + |A_2| - |A_1 \cap A_2|.
        \]
        Now assume the formula is true for all $n$.
        Consider the case $n + 1$,
        we need to show that
        \[
        |A_1 \cup A_2 \cup \dotsi \cup A_n \cup A_{n + 1}| = \sum_{k = 1}^{n + 1}(-1) ^ {k + 1}S_k.
        \]
        Let $A = A_1 \cup A_2 \cup \dotsi \cup A_n$, then
        \[
        |A \cup A_{n + 1}| = |A| + |A_{n + 1}| - |A \cap A_{n + 1}|
        \]
        then by the induction hypothesis
        \large\textbf{Continue!}
    \end{proof}
\end{theorem}

\begin{definition}[Derangement]
    A derangement of the numbers $\{1, 2, \dotsc, n\}$ is a permutation in which no number is in its correct position.
    That is, for every $k$, the number $k$ is not in the $k$th place.
\end{definition}

\textbf{Question}: For $n$ objects,
how many of the $n!$ permutations are derangements?
Denote this number by $d(n)$.

We can use the inclusion-exclusion principle to find a formula for $d(n)$.

Fix $n$.
For $1 \leq i \leq n$,
let $A_i$ be the set of permutations of $1, 2, \dotsc, n$ in which $i$ is in the $i$th place.

Then $|A_i| = (n - 1)!$ since after placing $i$ in the $i$th place,
we can arrange the remaining $n - 1$ numbers in the remaining $n - 1$ places in $(n - 1)!$ ways.

Similarly, for $i \neq j,\ A_i \cap A_j$ consists of permutations with both $i$ and $j$ fixed, so
\[
|A_i \cap A_j| = (n - 2)!,\ \text{for } i \neq j,
\]
since after fixing $i$ and $j$,
there are $n - 2$ free places.

Generally, if $1 \leq i_1 < i_2 < \dotsi < i_k \leq n$, then
\[
|A_{i_1} \cap A_{i_2} \cap \dotsi \cap A_{i_k}| = (n - k)!,
\]
since we place each of the $i_j$ in its proper place and arrange the remaining $n - k$ numbers in $(n - k)!$ ways.

There are $\binom{n}{k}$ ways of selecting the $i_1 < i_2 < \dotsi < i_k$.
So
\[
S_k = \sum_{i_1 < i_2 < \dotsi < i_k}|A_{i_1} \cap A_{i_2} \cap \dotsi \cap A_{i_k}| = \binom{n}{k}(n - k)! = \frac{n!}{k!}.
\]
Then the inclusion-exclusion formula tells us that the number of non-derangements is
\[
|A_1 \cup A_2 \cup \dotsi \cup A_n| = \sum_{k = 1}^{n}(-1) ^ {k + 1}\frac{n!}{k!}.
\]
\begin{proposition}
    The number of derangements of $1, 2, \dotsc, n$ is
    \[
    d(n) = \sum_{k = 0}^{n}(-1) ^ k\frac{n!}{k!}.
    \]
    \begin{proof}
        We subtract from $n!$ the number of non-derangements to get
        \begin{align*}
            d(n) &= n! - \sum_{k = 1}^{n}(-1) ^ {k + 1}\frac{n!}{k!} \\
            &= n!\left(1 + \sum_{k = 1}^{n}(-1) ^ k \frac{1}{k!}\right) \\
            &= n!\sum_{k = 0}^{n}(-1) ^ k\frac{1}{k!}.
        \end{align*}
    \end{proof}
\end{proposition}
In particular, the proportion of permutations that are derangements is
\[
\frac{d(n)}{n!} = \sum_{k = 0}^{n}(-1) ^ k\frac{1}{k!}.
\]

\newpage

\section{Recurrence relations}

\subsection{Three examples}

\begin{example}[H1]
    A student has a flight of $n$ stairs to climb,
    and can take either one or two stairs in a stride.
    How many distinct ways of climbing the stairs does she have?

    Call the answer $f_n$.

    Then
    \begin{align*}
        f_1 &= 1 &(1 \text{ single step}); \\
        f_2 &= 2 &(1, 1 \text{ or } 2); \\
        f_3 &= 3 &(1, 1, 1 \text{ or } 1, 2 \text{ or } 2, 1),
    \end{align*}
    and so on,
    where we are describing the ways of climbing the stairs by a sequence of $1$s and $2$s,
    corresponding to the sequence of strides where a $1$ represents a $1$-stair stride and a $2$ represents a $2$-stair stride.
    The sum of elements in the sequence must be $n$,
    because $n$ stairs are traversed in total.

    An effective way to generate the numbers for $f_n$ recursively by relating $f_n$ to previous values in the sequence.

    Suppose $n \geq 2$.
    The trick is to take one stride and then consider how many ways there are of mounting the remaining stairs.

    This splits the configurations contributing to $f_n$ into two (disjoint) types:
    \begin{itemize}
        \item Those where the first stride climbs $1$ stair ($1, \dotsc$).
        Then $n - 1$ stairs remain,
        and these remaining stairs can, by definition,
        be climbed in $f_{n - 1}$ ways.
        So there are $f_{n - 1}$ configurations of this type.
        \item Those where the first stride climbs $2$ stairs ($2, \dotsc$).
        Then $n - 2$ stairs remain,
        and these remaining stairs can, by definition,
        be climbed in $f_{n - 2}$ ways.
        So there are $f_{n - 2}$ configuration of this type.
    \end{itemize}
    This argument shows that
    \[
    f_n = f_{n - 1} + f_{n - 2},\quad(n \geq 2).
    \]
\end{example}

\begin{example}[H2]
    Draw $n$ infinite straight lines in the plane,
    assumed to be non-parallel and non-concurrent.
    Into how many contiguous regions does this divide the plane?

    Let $r_n$ be the number of regions.
    Looking at some pictures, we find
    \[
    r_0 = 1,\,r_1 = 2,\,r_2 = 4,\,r_3 = 7,\,r_4 = 11,\,r_5 = 16,\dotsc.
    \]
    Considering $n = 4$.
    Each time the new line crosses an old line it crosses into a new region,
    this line crosses $4$ (out of $7$) old regions.
    Each of these $4$ old regions is split into two,
    so the number of regions increases by $4$.

    In general,
    the $n$th line makes $n$ more regions:
    \[
    r_n = r_{n - 1} + n,\quad(n \geq 1).
    \]

\end{example}






\end{document}