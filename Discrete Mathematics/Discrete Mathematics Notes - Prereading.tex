\documentclass[10pt, a4paper]{article}
\usepackage{preamble}

\title{Discrete Mathematics \\
    \large Prereading}
\author{Luke Phillips}
\date{October 2024}

\begin{document}

\maketitle

\newpage

\section{Introduction}

\subsection{Set notation}
\textbf{Basic notation}

A set is defined by the elements that belong to the set.

If an element $x$ belongs to a set $S$, we write $x \in S$.

Finite or countably infinite sets are those which we can write as a (finite or infinite) list, such as
\[
S = \{e_1, e_2, \dots, e_n\},
\]
(a finite set, with elements labelled by finitely many integers) or
\[
S = \{e_1, e_2, e_3, \dots\},
\]
(representing a countably infinite set, with elements labelled by all positive integers).

The order of elements in a set is unimportant.

A set can be described using the following notation
\[
S = \{n: n \text{ is an integer}, 0\leq n \leq 9\}
\]
which means all $n$ such that the properties to the right of the '$:$' hold.

\textbf{Examples of standard sets}
\begin{itemize}
    \item $\Z = \{\dots, -2, -1, 0, 1, 2, \dots\} = \{0, \pm1, \pm2, \dots\}$, the set of integers (whole numbers).
    \item $\N = \{1, 2, \dots\} = \{n: n\in\Z, n>0\}$, the natural numbers (positive integers).
    \item $\emptyset = \{\}$ is the empty set (the set with no elements).
\end{itemize}

\textbf{More notation}

As mentioned before, $x \in S$ means element $x$ is a member of the set $S$. The negation is $x \notin S$.

If $R$ and $S$ are sets, then $R \subseteq S$ means that $R$ is a subset of $S$, i.e., every element of $R$ is also an element of $S$. That is, for all $x;\ x \in R \implies x \in S$.

Note that $R = S$ if and only if $R \subseteq S$ and $S \subseteq R$.

$R \subset S$ is used to denote that $R$ is a strict subset of $S$, meaning $R \subseteq S$ and $R \neq S$.

If $S$ is a finite set, its size or cardinality, denoted by $|S|$, is the number of elements that it contains.


\textbf{Set operations}

Given several sets $A$ and $B$, $A \cup B$ is the union of $A$ and $B$, which is the set consisting of those elements that are in $A$ or in $B$ (or both)

$A \cap B$ is the intersection of sets $A$ and $B$, which is the set consisting of those elements that are in both $A$ and $B$.

If two sets $A$ and $B$ do not intersect ($A \cap B = \emptyset$) we say that $A$ and $B$ are disjoint.

A collection of sets can be indexed by numbers, e.g. $A_1, A_2, \dots, A_n$.

Then we may use compact notation for multiple unions and intersections:
\begin{align*}
    \bigcup_{i = 1}^{n}{A_i} &= A_1 \cup A_2 \cup \dots \cup A_n \\
    &= \{x: x \in A_i \text{ for at least one } i\};
\end{align*}

\begin{align*}
    \bigcap_{i = 1}^{n}{A_i} &= A_1 \cap A_2 \cap \dots \cap A_n \\
    &= \{x: x \in A_i \text{ for every } i\};
\end{align*}

\textbf{Cartesian products}

If $X$ and $Y$ are two sets, their Cartesian product is
\[
X \times Y = \{(x, y): x \in X \text{ and } y \in Y\},
\]
the set of all ordered pairs $(x, y)$ with first component in $X$ and second in $Y$.

For example, if $X = \{1, 2, 3\}$ and $Y = \{a, b\}$, then the set $X \times Y$ is
\[
\{(1, a), (1, b), (2, a), (2, b), (3, a), (3, b)\},
\]
while the set $Y \times X$ is
\[
\{(a, 1), (a, 2), (a, 3), (b, 1), (b, 2), (b, 3)\}.
\]

\section{Arrangements and permutations}

\subsection{Arrangements}
An arrangement (or list or string) is a (finite) sequence of items in which the order of the items in the sequence matters.

The length of an arrangement is the number of items in the sequence.

\textbf{Multiplication Principle}. for an arrangement of length $k$, if for $i \in \{1,\,2,\,\dotsc,\,k\}$ there are $n_i$ choices for the $i$th item, regardless of previous choices, then the number of possible arrangements is $n_1n_2\dotsi n_k$.

\subsection{Permutations}
A permutation on a finite set $S$ is an arrangement of length $|S|$ that uses each item from $S$ exactly once.

\begin{proposition}
    If $|S| = n$, then there are $n!$ permutations on $S$.
    
    \begin{proof}
        $n(n - 1)(n - 2)\dotsi 2 \cdot 1 = n!$\footnote{This is standard problem \#2 (no repeats) with $n = k$.}
    \end{proof}
\end{proposition}

We can generalise this concept. For a non-negative integer $r$, an $r$-permutation on $S$ is an arrangement of $r$ distinct elements from $S$.

We write $P(n,\,r)$ for the number of $r$-permutations on a set of size $n$.
\[
P(n, r) = n(n - 1)\dotsi (n - r + 1) = \frac{n!}{(n - r)!}, \text{ for } 0 \leq r \leq n.
\]

\subsection{Counting examples}

\textbf{Addition Principle}. Let $A$ and $B$ be two finite sets that are disjoint, i.e. $A \cap B = \emptyset$. Then
\[
|A \cup B| = |A| + |B|.
\]
More generally, if $A_1,\,A_2,\,\dotsc,\,A_n$ are pairwise disjoint, meaning that $A_i \cap A_j = \emptyset$ for all $i \neq j$, then
\[
|A_1 \cup A_2 \cup \dotsi \cup A_n| = |A_1| + |A_2| + \dotsi + |A_n|.
\]

\newpage

\section{Combinations}

A combination from a set of objects is an unordered selection of elements of that set.

Formally, a combination on a set $S$ is identified by the multiplicities $m_i,\, i \in S$, where $m_i \in \{0, 1, 2, \dotsc\}$ determines how many times element $i$ appears in the combination.

The size of a combination is the total number of selections, i.e. the sum of all the multiplicities $\sum_{i \in S}m_i$. We sometimes call a combination of size $k$ a $k$-combination.

How many combinations without repetition do we have of a given size $k$ from a set of size $n$?

Consider the arrangements, without repetitions, of length $k$ from the set of size $n$. There are $P(n, k) = \frac{n!}{(n - k)!}$ of those. These are ordered lists, so sort them into groups of lists using the same letters. Each group contains $k!$ lists. The number of groups is $\frac{P(n, k)}{k!}$, and each corresponds to a combination. So the number of $k$-combinations without repetitions is
\[
\frac{P(n, k}{k!} = \frac{n!}{k!(n - k)!}.
\]
This is the binomial coefficient notation
\[
\binom{n}{k} = \frac{n!}{k!(n - k)!}\quad\text{"$n$ choose $k$"}.
\]

\begin{proposition}
    $\binom{n}{k} = \binom{n}{n - k}$.
    \begin{proof}
        Observe that each choice of $k$ objects from $n$ to take corresponds exactly to a choice of $n - k$ objects to leave.
    \end{proof}
\end{proposition}

\begin{proposition}[Pascal's formula]
    For $1 \leq k \leq n - 1,\, \binom{n}{k} = \binom{n - 1}{k - 1} + \binom{n - 1}{k}$.
    \begin{proof}
        \textbf{To do}.
    \end{proof}
\end{proposition}

\begin{example}
    Let $N$ be the product of $k$ consecutive positive integers. Show that $N$ is (exactly) divisible by $k!$.
    \begin{proof}[Solution]
        We can write $N = (a + 1)(a + 2)\dotsi (a + k)$ for some integer $a \geq 0$. But then
        \[
        N = \frac{(a + k)!}{a!} = \binom{a + k}{a}k!,
        \]
        so $\frac{N}{k!}$ is $\binom{a + k}{a}$, which is a positive integer for all positive integers $k$.
    \end{proof}
\end{example}

\begin{theorem}[The binomial theorem]
    If $n$ is a positive integer and $a, b$ are real numbers, then
    \[
    (a + b) ^ n = \sum_{k = 0}^{n}\binom{n}{k}a ^ {k} b ^ {n - k}.
    \]
    \begin{proof}
        The expression
        \[
        (a + b)(a + b) \dotsi (a + b)
        \]
        when multiplied out, gives various terms of the form $a ^ k b ^ {n - k}$. For fixed $k$, the number of terms $a ^ {k} b ^ {n - k}$ is $\binom{n}{k}$ since one chooses the $k$ brackets in the product from which to choose the factor $a$ (rather than $b$).

        The product gives rise to every arrangement of length $n$ using $a$ and $b$, and grouping these according to the number of times that $a$ appears gives the result.
    \end{proof}
\end{theorem}

Prove that the following identities are valid for integer $n \geq 0$.
\begin{enumerate}[label = (C\arabic*)]
    \item $\sum_{k = 0}^{n}\binom{n}{k} = 2 ^ n$.
    \item $\sum_{k = 0}^{n}\binom{n}{k}(-1) ^ k = 0$ (provided $n \geq 1$).
    \item $\sum_{k = 0}^{n}\binom{n}{k}2 ^ k = 3 ^ n$.
    \item $\sum_{k = 0}^{n}\binom{n}{k}k = n2 ^ {n - 1}$.
\end{enumerate}

\end{document}