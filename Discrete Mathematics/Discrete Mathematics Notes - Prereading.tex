\documentclass[10pt, a4paper]{article}
\usepackage{preamble}

\title{Discrete Mathematics \\
    \large Prereading}
\author{Luke Phillips}
\date{October 2024}

\begin{document}

\maketitle

\newpage

\section{Introduction}

\subsection{Set notation}
\textbf{Basic notation}

A set is defined by the elements that belong to the set.

If an element $x$ belongs to a set $S$, we write $x \in S$.

Finite or countably infinite sets are those which we can write as a (finite or infinite) list, such as
\[
S = \{e_1, e_2, \dots, e_n\},
\]
(a finite set, with elements labelled by finitely many integers) or
\[
S = \{e_1, e_2, e_3, \dots\},
\]
(representing a countably infinite set, with elements labelled by all positive integers).

The order of elements in a set is unimportant.

A set can be described using the following notation
\[
S = \{n: n \text{ is an integer}, 0\leq n \leq 9\}
\]
which means all $n$ such that the properties to the right of the '$:$' hold.

\textbf{Examples of standard sets}
\begin{itemize}
    \item $\Z = \{\dots, -2, -1, 0, 1, 2, \dots\} = \{0, \pm1, \pm2, \dots\}$, the set of integers (whole numbers).
    \item $\N = \{1, 2, \dots\} = \{n: n\in\Z, n>0\}$, the natural numbers (positive integers).
    \item $\emptyset = \{\}$ is the empty set (the set with no elements).
\end{itemize}

\textbf{More notation}

As mentioned before, $x \in S$ means element $x$ is a member of the set $S$. The negation is $x \notin S$.

If $R$ and $S$ are sets, then $R \subseteq S$ means that $R$ is a subset of $S$, i.e., every element of $R$ is also an element of $S$. That is, for all $x;\ x \in R \implies x \in S$.

Note that $R = S$ if and only if $R \subseteq S$ and $S \subseteq R$.

$R \subset S$ is used to denote that $R$ is a strict subset of $S$, meaning $R \subseteq S$ and $R \neq S$.

If $S$ is a finite set, its size or cardinality, denoted by $|S|$, is the number of elements that it contains.


\textbf{Set operations}

Given several sets $A$ and $B$, $A \cup B$ is the union of $A$ and $B$, which is the set consisting of those elements that are in $A$ or in $B$ (or both)

$A \cap B$ is the intersection of sets $A$ and $B$, which is the set consisting of those elements that are in both $A$ and $B$.

If two sets $A$ and $B$ do not intersect ($A \cap B = \emptyset$) we say that $A$ and $B$ are disjoint.

A collection of sets can be indexed by numbers, e.g. $A_1, A_2, \dots, A_n$.

Then we may use compact notation for multiple unions and intersections:
\begin{align*}
    \bigcup_{i = 1}^{n}{A_i} &= A_1 \cup A_2 \cup \dots \cup A_n \\
    &= \{x: x \in A_i \text{ for at least one } i\};
\end{align*}

\begin{align*}
    \bigcap_{i = 1}^{n}{A_i} &= A_1 \cap A_2 \cap \dots \cap A_n \\
    &= \{x: x \in A_i \text{ for every } i\};
\end{align*}

\textbf{Cartesian products}

If $X$ and $Y$ are two sets, their Cartesian product is
\[
X \times Y = \{(x, y): x \in X \text{ and } y \in Y\},
\]
the set of all ordered pairs $(x, y)$ with first component in $X$ and second in $Y$.

For example, if $X = \{1, 2, 3\}$ and $Y = \{a, b\}$, then the set $X \times Y$ is
\[
\{(1, a), (1, b), (2, a), (2, b), (3, a), (3, b)\},
\]
while the set $Y \times X$ is
\[
\{(a, 1), (a, 2), (a, 3), (b, 1), (b, 2), (b, 3)\}.
\]

\newpage

\section{Arrangements and permutations}

\subsection{Arrangements}
\begin{definition}[Arrangement]
    An \textbf{arrangement} (or list or string) is a (finite) sequence of items in which the \textbf{order} of the items in the sequence \textbf{matters}.
\end{definition}

The length of an arrangement is the number of items in the sequence.

Standard problem \#1.
How many arrangements of length $k$ can be made from elements of a set $S$ of size $|S| = n$?

This would be $n ^ k$ as each arrangement has $k$ places each of which can be occupied by any one of $n$ entries.


Standard problem \#2.
How many arrangements of length $k$ with no repeats can be made from elements of a set $S$ of size $|S| = n$?

This would be $n(n - 1)(n - 2)\dotsi(n - k + 1) = \frac{n!}{(n - k)!}$,
provided $k \leq n$.
This is because every time we make a choice we remove a possible option.


\textbf{Multiplication Principle}. for an arrangement of length $k$, if for $i \in \{1,\,2,\,\dotsc,\,k\}$ there are $n_i$ choices for the $i$th item, regardless of previous choices, then the number of possible arrangements is $n_1n_2\dotsi n_k$.

\subsection{Permutations}
\begin{definition}[Permutation]
    A permutation on a finite set $S$ is an arrangement of length $|S|$ that uses each item from $S$ exactly once.
\end{definition}

\begin{proposition}
    If $|S| = n$, then there are $n!$ permutations on $S$.
    
    \begin{proof}
        $n(n - 1)(n - 2)\dotsi 2 \cdot 1 = n!$\footnote{This is standard problem \#2 (no repeats) with $n = k$.}
    \end{proof}
\end{proposition}

We can generalise this concept. For a non-negative integer $r$, an $r$-permutation on $S$ is an arrangement of $r$ distinct elements from $S$.

We write $P(n,\,r)$ for the number of $r$-permutations on a set of size $n$.
\[
P(n, r) = n(n - 1)\dotsi (n - r + 1) = \frac{n!}{(n - r)!}, \text{ for } 0 \leq r \leq n.
\]

\subsection{Counting examples}

\textbf{Addition Principle}. Let $A$ and $B$ be two finite sets that are disjoint, i.e. $A \cap B = \emptyset$. Then
\[
|A \cup B| = |A| + |B|.
\]
More generally, if $A_1,\,A_2,\,\dotsc,\,A_n$ are pairwise disjoint, meaning that $A_i \cap A_j = \emptyset$ for all $i \neq j$, then
\[
|A_1 \cup A_2 \cup \dotsi \cup A_n| = |A_1| + |A_2| + \dotsi + |A_n|.
\]

\newpage

\section{Combinations}
\begin{definition}[Combination]
    A \textbf{combination} from a set of objects is an \textbf{unordered} selection of elements of that set.
\end{definition}

Formally, a combination on a set $S$ is identified by the multiplicities $m_i,\, i \in S$, where $m_i \in \{0, 1, 2, \dotsc\}$ determines how many times element $i$ appears in the combination.

The size of a combination is the total number of selections, i.e. the sum of all the multiplicities $\sum_{i \in S}m_i$. We sometimes call a combination of size $k$ a $k$-combination.

Standard problem \#3.
How many combinations without repetition do we have of a given size $k$ from a set of size $n$?

Consider the arrangements, without repetitions, of length $k$ from the set of size $n$. There are $P(n, k) = \frac{n!}{(n - k)!}$ of those. These are ordered lists, so sort them into groups of lists using the same letters. Each group contains $k!$ lists. The number of groups is $\frac{P(n, k)}{k!}$, and each corresponds to a combination. So the number of $k$-combinations without repetitions is
\[
\frac{P(n, k}{k!} = \frac{n!}{k!(n - k)!}.
\]
This is the binomial coefficient notation
\[
\binom{n}{k} = \frac{n!}{k!(n - k)!}\quad\text{"$n$ choose $k$"}.
\]

\begin{proposition}
    $\binom{n}{k} = \binom{n}{n - k}$.
    \begin{proof}
        Observe that each choice of $k$ objects from $n$ to take corresponds exactly to a choice of $n - k$ objects to leave.
    \end{proof}
\end{proposition}

\begin{proposition}[Pascal's formula]
    For $1 \leq k \leq n - 1,\, \binom{n}{k} = \binom{n - 1}{k - 1} + \binom{n - 1}{k}$.
    \begin{proof}
        \textbf{To do}.
    \end{proof}
\end{proposition}

\begin{example}
    Let $N$ be the product of $k$ consecutive positive integers. Show that $N$ is (exactly) divisible by $k!$.
    \begin{proof}[Solution]
        We can write $N = (a + 1)(a + 2)\dotsi (a + k)$ for some integer $a \geq 0$. But then
        \[
        N = \frac{(a + k)!}{a!} = \binom{a + k}{a}k!,
        \]
        so $\frac{N}{k!}$ is $\binom{a + k}{a}$, which is a positive integer for all positive integers $k$.
    \end{proof}
\end{example}

\begin{theorem}[The binomial theorem]
    If $n$ is a positive integer and $a, b$ are real numbers, then
    \[
    (a + b) ^ n = \sum_{k = 0}^{n}\binom{n}{k}a ^ {k} b ^ {n - k}.
    \]
    \begin{proof}
        The expression
        \[
        (a + b)(a + b) \dotsi (a + b)
        \]
        when multiplied out, gives various terms of the form $a ^ k b ^ {n - k}$. For fixed $k$, the number of terms $a ^ {k} b ^ {n - k}$ is $\binom{n}{k}$ since one chooses the $k$ brackets in the product from which to choose the factor $a$ (rather than $b$).

        The product gives rise to every arrangement of length $n$ using $a$ and $b$, and grouping these according to the number of times that $a$ appears gives the result.
    \end{proof}
\end{theorem}

Prove that the following identities are valid for integer $n \geq 0$.
\begin{enumerate}[label = (C\arabic*)]
    \item $\sum_{k = 0}^{n}\binom{n}{k} = 2 ^ n$.
    \item $\sum_{k = 0}^{n}\binom{n}{k}(-1) ^ k = 0$ (provided $n \geq 1$).
    \item $\sum_{k = 0}^{n}\binom{n}{k}2 ^ k = 3 ^ n$.
    \item $\sum_{k = 0}^{n}\binom{n}{k}k = n2 ^ {n - 1}$.
\end{enumerate}

\newpage

\section{Arrangements and combinations with repetitions}

\subsection{Arrangements with repetitions}
Standard problem \#4.
How many arrangements are there of a list o objects in which some objects are repeated?

\begin{theorem}
    Given a list of $n$ objects of $r$ different types, in which objects of type $i$ occur $n_i$ times (with $n_1 + \dotsi + n_r = n$), the number of arrangements of the list is
    \[
    P(n; n_1, n_2, \dotsc, n_r) = \frac{n!}{n_1!n_2! \dotsi n_r!}.
    \]
    \begin{proof}
        EXERCISE!!!
    \end{proof}
\end{theorem}

\subsection{Combinations with repetitions}
Standard problem \#6.
How many $k$-combinations with repetitions from $n$ objects are there in which each object is chosen at least once in the combination?

This would be the following theorem.
\begin{theorem}
    The number of $k$-combinations from $n$ different objects with repetition and each object occurring at least once is $\binom{k - 1}{n - 1}$ for $k \geq n$.
    \begin{proof}
        Fix some order for the $n$ objects.
        sorting each $k$-combination so that its objects respect this order.
        There are $k - 1$ gaps between items in the list.
        Each combination is described uniquely by inserting $n - 1$ markers into the $k - 1$ gaps,
        where the markers indicate the boundary between successive types of object.
    \end{proof}
\end{theorem}

Standard Problem \#5.
Given a set of $n$ objects,
how many combinations of size $k$ can be chosen if objects may be selected more than once?

As before,
this solution is the following theorem.
\begin{theorem}
    The number of $k$-combinations from $n$ different objects with repetition is
    \[
    \binom{k + n - 1}{n - 1} = \binom{k + n - 1}{k}.
    \]
\end{theorem}

\subsection{The extended binomial coefficients}

Maclaurin’s theorem says that, for $\alpha \in \R$ and integer $k \geq 0$,
\[
\binom{\alpha}{k} = \frac{\alpha(\alpha - 1)\dotsi (\alpha - k + 1)}{k!}.
\]

Note:
\begin{itemize}
    \item If $\alpha$ is a positive integer and $0 \leq k \leq \alpha$,
    this is just $\alpha$ choose $k$
    \item If $\alpha$ is a positive integer and $k > \alpha$,
    then the numerator includes a factor $0$ so $\binom{\alpha}{k} = 0$:
    in this case the power series for $(1 + x) ^ \alpha$ terminates after finitely many terms.
    \item In general,
    if $\alpha$ is not an integer, then the numerator doesn't include a $0$, so
    $\binom{\alpha}{k} \neq 0$ and the power series for $(1 + x) ^ \alpha$ has infinitely many terms
\end{itemize}

\begin{proposition}[Principle of upper negation]
    If $n$ and $k$ are positive integers, then
    \[
    \binom{-n}{k} = (-1) ^ k \binom{k + n - 1}{k} = (-1) ^ k \binom{k + n - 1}{n - 1}.
    \]
    \begin{proof}
        \begin{align*}
            \binom{-n}{k} &= \frac{(-n)(-n - 1)(-n - 2)\dotsi(-n - k + 1)}{k!} \\
            &= (-1) ^ k\frac{n(n + 1)(n + 2)\dotsi(n + k - 1)}{k!} \\
            &= (-1) ^ k\frac{(n + k - 1)!}{(n - 1)!k!},
        \end{align*}
        which is as shown.
    \end{proof}
\end{proposition}

\newpage

\section{Three principles}

\subsection{Mathematical induction}
Suppose that for each integer $n = 0, 1, 2, \dotsc$ we can make a statement (or proposition) $P(n)$ associated with $n$.
For example $P(n)$ might be the statement $2 ^ n > 2n$.
For each integer $n$,
$P(n)$ will be either true or false.
In this case, $P(1)$ is false and so is $P(2)$,
but $P(3)$ is true.
$P(n)$ is true for any $n \geq 3$.
We will use induction to prove this.

Suppose that we know $P(n)$ is true for some fixed $n \geq 3$.
Consider the statement $P(n + 1) : 2 ^ {n + 1} > 2(n + 1)$.
But
\[
2 ^ {n + 1} = 2 \cdot 2 ^ n > 2 \cdot 2n,
\]
by $P(n)$, which we said was true.
So we have established the truth of $P(n + 1)$ assuming the truth of $P(n)$,
$n \geq 3$.
This proves that $P(n)$ holds for all $n \geq 3$,
by induction.

\begin{theorem}[Principle of mathematical induction]
    Suppose that $P(n)$ is a sequence of propositions labelled by integers $n \geq n_0$.
    Suppose that
    \begin{enumerate}[label = (\alph*)]
        \item $P(n_0)$ is true; and
        \item For any $n \geq n_0$, the truth of $P(n)$ implies the truth of $P(n + 1)$.
    \end{enumerate}
    Then, for every $n \geq n_0$,
    $P(n)$ is true.
    \begin{proof}
        $P(n_0)$ is true by assumption (a).
        Then by (b),
        $P(n_0 + 1)$ is also true.
        Again by (b),
        $P(n_0 + 2)$ is also true.
        And so on.
        This chain of deductions eventually reaches any given $n \geq n_0$,
        so $P(n)$ is true for all $n \geq n_0$.
    \end{proof}
\end{theorem}

A proof by induction using the above principle consists of two parts:
verifying the starting case (a) and then establishing the inductive step (b).
The inductive step involves assuming the inductive hypothesis $P(n)$ and deducting $P(n + 1)$ from that.

\subsection{The pigeon-hole principle}
The pigeon-hole principle.
If $n$ envelopes are distributed among $m$ pigeon-holes,
and $n > m$,
then at least one pigeon-hole contains more than one envelope.



\subsection{The inclusion-exclusion principle}

\begin{proposition}
    We have
    \[
    |A_1 \cup A_2 \cup A_3| = |A_1| + |A_2| + |A_3| - |A_1 \cap A_2| - |A_1 \cap A_3| - |A_2 \cap A_3| + |A_1 \cap A_2 \cap A_3|.
    \]
    \begin{proof}
        Apply the formula
        \[
        |A_1 \cup A_2| = |A_1| + |A_2| - |A_1 \cap A_2|
        \]
        repeatedly.
    \end{proof}
\end{proposition}

\begin{theorem}[Inclusion-exclusion formula]
    For a positive integer $n$ and finite sets $A_1, \dotsc, A_n$,
    \[
    |A_1 \cup A_2 \cup \dotsi \cup A_n| = \sum_{k = 1}^{n}(-1) ^ {k + 1}S_k,
    \]
    where
    \[
    S_k = \sum_{i_1 < i_2 < \dotsi < i_k}|A_{i1} \cap A_{i2} \cap \dotsi \cap A_{ik}|.
    \]
    \begin{proof}
        The base case for this is $n = 2$ and it has been given by the previously mentioned formula
        \[
        |A_1 \cup A_2| = |A_1| + |A_2| - |A_1 \cap A_2|.
        \]
        Now assume the formula is true for all $n$.
        Consider the case $n + 1$,
        we need to show that
        \[
        |A_1 \cup A_2 \cup \dotsi \cup A_n \cup A_{n + 1}| = \sum_{k = 1}^{n + 1}(-1) ^ {k + 1}S_k.
        \]
        Let $A = A_1 \cup A_2 \cup \dotsi \cup A_n$, then
        \[
        |A \cup A_{n + 1}| = |A| + |A_{n + 1}| - |A \cap A_{n + 1}|
        \]
        then by the induction hypothesis
        \large\textbf{Continue!}
    \end{proof}
\end{theorem}

\begin{definition}[Derangement]
    A derangement of the numbers $\{1, 2, \dotsc, n\}$ is a permutation in which no number is in its correct position.
    That is, for every $k$, the number $k$ is not in the $k$th place.
\end{definition}

\textbf{Question}: For $n$ objects,
how many of the $n!$ permutations are derangements?
Denote this number by $d(n)$.

We can use the inclusion-exclusion principle to find a formula for $d(n)$.

Fix $n$.
For $1 \leq i \leq n$,
let $A_i$ be the set of permutations of $1, 2, \dotsc, n$ in which $i$ is in the $i$th place.

Then $|A_i| = (n - 1)!$ since after placing $i$ in the $i$th place,
we can arrange the remaining $n - 1$ numbers in the remaining $n - 1$ places in $(n - 1)!$ ways.

Similarly, for $i \neq j,\ A_i \cap A_j$ consists of permutations with both $i$ and $j$ fixed, so
\[
|A_i \cap A_j| = (n - 2)!,\ \text{for } i \neq j,
\]
since after fixing $i$ and $j$,
there are $n - 2$ free places.

Generally, if $1 \leq i_1 < i_2 < \dotsi < i_k \leq n$, then
\[
|A_{i_1} \cap A_{i_2} \cap \dotsi \cap A_{i_k}| = (n - k)!,
\]
since we place each of the $i_j$ in its proper place and arrange the remaining $n - k$ numbers in $(n - k)!$ ways.

There are $\binom{n}{k}$ ways of selecting the $i_1 < i_2 < \dotsi < i_k$.
So
\[
S_k = \sum_{i_1 < i_2 < \dotsi < i_k}|A_{i_1} \cap A_{i_2} \cap \dotsi \cap A_{i_k}| = \binom{n}{k}(n - k)! = \frac{n!}{k!}.
\]
Then the inclusion-exclusion formula tells us that the number of non-derangements is
\[
|A_1 \cup A_2 \cup \dotsi \cup A_n| = \sum_{k = 1}^{n}(-1) ^ {k + 1}\frac{n!}{k!}.
\]
\begin{proposition}
    The number of derangements of $1, 2, \dotsc, n$ is
    \[
    d(n) = \sum_{k = 0}^{n}(-1) ^ k\frac{n!}{k!}.
    \]
    \begin{proof}
        We subtract from $n!$ the number of non-derangements to get
        \begin{align*}
            d(n) &= n! - \sum_{k = 1}^{n}(-1) ^ {k + 1}\frac{n!}{k!} \\
            &= n!\left(1 + \sum_{k = 1}^{n}(-1) ^ k \frac{1}{k!}\right) \\
            &= n!\sum_{k = 0}^{n}(-1) ^ k\frac{1}{k!}.
        \end{align*}
    \end{proof}
\end{proposition}
In particular, the proportion of permutations that are derangements is
\[
\frac{d(n)}{n!} = \sum_{k = 0}^{n}(-1) ^ k\frac{1}{k!}.
\]

\newpage

\section{Recurrence relations}

\subsection{Three examples}

\begin{example}[H1]
    A student has a flight of $n$ stairs to climb,
    and can take either one or two stairs in a stride.
    How many distinct ways of climbing the stairs does she have?

    Call the answer $f_n$.

    Then
    \begin{align*}
        f_1 &= 1 &(1 \text{ single step}); \\
        f_2 &= 2 &(1, 1 \text{ or } 2); \\
        f_3 &= 3 &(1, 1, 1 \text{ or } 1, 2 \text{ or } 2, 1),
    \end{align*}
    and so on,
    where we are describing the ways of climbing the stairs by a sequence of $1$s and $2$s,
    corresponding to the sequence of strides where a $1$ represents a $1$-stair stride and a $2$ represents a $2$-stair stride.
    The sum of elements in the sequence must be $n$,
    because $n$ stairs are traversed in total.

    An effective way to generate the numbers for $f_n$ recursively by relating $f_n$ to previous values in the sequence.

    Suppose $n \geq 2$.
    The trick is to take one stride and then consider how many ways there are of mounting the remaining stairs.

    This splits the configurations contributing to $f_n$ into two (disjoint) types:
    \begin{itemize}
        \item Those where the first stride climbs $1$ stair ($1, \dotsc$).
        Then $n - 1$ stairs remain,
        and these remaining stairs can, by definition,
        be climbed in $f_{n - 1}$ ways.
        So there are $f_{n - 1}$ configurations of this type.
        \item Those where the first stride climbs $2$ stairs ($2, \dotsc$).
        Then $n - 2$ stairs remain,
        and these remaining stairs can, by definition,
        be climbed in $f_{n - 2}$ ways.
        So there are $f_{n - 2}$ configuration of this type.
    \end{itemize}
    This argument shows that
    \[
    f_n = f_{n - 1} + f_{n - 2},\quad(n \geq 2).
    \]
\end{example}

\begin{example}[H2]
    Draw $n$ infinite straight lines in the plane,
    assumed to be non-parallel and non-concurrent.
    Into how many contiguous regions does this divide the plane?

    Let $r_n$ be the number of regions.
    Looking at some pictures, we find
    \[
    r_0 = 1,\,r_1 = 2,\,r_2 = 4,\,r_3 = 7,\,r_4 = 11,\,r_5 = 16,\dotsc.
    \]
    Considering $n = 4$.
    Each time the new line crosses an old line it crosses into a new region,
    this line crosses $4$ (out of $7$) old regions.
    Each of these $4$ old regions is split into two,
    so the number of regions increases by $4$.

    In general,
    the $n$th line makes $n$ more regions:
    \[
    r_n = r_{n - 1} + n,\quad(n \geq 1).
    \]

\end{example}

\begin{example}[H3]
    The Tower of Hanoi is played with three pegs and $n$ rings of different sizes.
    
    The initial position is the $n$ rings are stacked in a tower on a single peg in order of decreasing size from the base of the tower.

    A permitted move is to remove a ring from the top of the pile,
    and place it on another peg provided it does not sit atop a smaller ring.

    The objective is to move the whole tower from the starting peg to one of the other pegs in as few moves as possible.

    What is the smallest number of necessary moves?

    Call the number $t_n$.

    We can see for the following cases,
    we get the values for $t_n$:
    \begin{table}[H]
        \centering
        \begin{tabular}{c|ccccc}
             $n$ & $1$ & $2$ & $3$ & $4$ & $5$ \\
             \hline
             $t_n$ & $1$ & $3$ & $7$ & $15$ & $31$
        \end{tabular}
    \end{table}
    How do we obtain a recurrence relation?
    To obtain this relation we need to argue that we are constructing an optimal strategy.
    Here is the strategy to move $n$ rings:
    \begin{enumerate}[label = (\roman*)]
        \item Move the top $n - 1$ rings to a single pile somewhere else - this takes $t_{n - 1}$ moves to do optimally.
        \item Move the largest ring to the empty peg - $1$ move.
        \item Now move the $n - 1$ smaller rings back on top of the largest ring - this takes another $t_{n - 1}$ moves to do optimally.
    \end{enumerate}
    To move all of the rings we need to move the largest ring,
    to move all the other rings first we need to leave an empty peg to perform move (ii).
    So we obtain the following relation
    \[
    t_n = 2t_{n - 1} + 1, (n \geq 2),
    \]
    with initial condition $t_1 = 1$.
    
    We can guess the formula $t_n = 2 ^ n - 1$ for $n \geq 1$,
    and then prove this by induction using the recurrence relation.
\end{example}

\subsection{Towards a general method}
We will start by considering the recursion relation
\[
f_n = f_{n - 1} + f_{n - 2}
\]
with initial conditions $f_0 = f_1 = 1$.

The idea (to motivate a general method):
If we had instead $g_n = g_{n - 1} + g_{n - 1} = 2g_{n - 1}$ then $g_n = c2 ^ n$ is easily seen to be the solution
(for any constant $c$ which would be determined by an initial condition).
Although $f_n$ cannot look exactly like this
(it grows more slowly)
this suggests that we try $f_n = cx ^ n$ as a possible solution.

Substituting $f_n = cx ^ n$ into the recurrence relation for $f_n$ gives
\[
cx ^ n = cx ^ {n - 1} + cx ^ {n - 2},
\]
or equivalently,
\[
cx ^ {n - 2}(x ^ 2 - x - 1) = 0.
\]
This equation has solutions $c = 0$ or $x = 0$
(either gives the trivial solution $f_n = 0$ for all $n$)
or
\[
x ^ 2 - x - 1 = 0.
\]
This is the characteristic equation associated with the relation.
It has the solutions $x = \frac{1 \pm \sqrt{5}}{2}$.
Using $\alpha = \frac{1 + \sqrt{5}}{2}$ and $\beta = \frac{1 - \sqrt{5}}{2}$.
Then $f_n = \alpha ^ n$ and $f_n = \beta ^ n$ each are valid solutions to the recurrence relation,
since by construction
\[
\alpha ^ n = \alpha ^ {n - 1} + \alpha ^ {n - 2}\text{ and } \beta ^ n = \beta ^ {n - 1} + \beta ^ {n - 2}.
\]
But then
\[
f_n = c_1\alpha ^ n + c_2\beta ^ n
\]
is also a solution for any constants $c_1$ and $c_2$.
To solve this,
use the initial conditions $f_0 = f_1 = 1$ to determine the constants $c_1$ and $c_2$.
From the $n = 0$ and $n = 1$ cases of our expression for $f_n$ in terms of $\alpha$ and $\beta$ we get
\[
1 = f_0 = c_1 \alpha ^ 0 + c_2 \beta ^ 0 = c_1 + c_2
\]
and
\[
1 = f_1 = c_1\alpha + c_2\beta.
\]
We can solve these equations in $c_1$ and $c_2$ to find
\[
c_1 = \frac{\alpha}{\sqrt{5}},\text{ and } c_2 = -\frac{\beta}{\sqrt{5}}.
\]
So the Fibonacci sequence is given by
\begin{align*}
    f_n &= \frac{\alpha}{\sqrt{5}}\alpha ^ n - \frac{\beta}{\sqrt{5}}\beta ^ n \\
    &= \frac{1}{\sqrt{5}}(\alpha ^ {n + 1} - \beta ^ {n + 1}) \\
    &= \frac{1}{\sqrt{5}}\left(\left(\frac{1 + \sqrt{5}}{2}\right) ^ {n + 1} - \left(\frac{1 - \sqrt{5}}{2}\right) ^ {n + 1}\right).
\end{align*}

\subsection{General method for solving homogeneous linear recurrence relations}
Let $(a_n)$ be a sequence indexed by integers $n = 0, 1, 2, \dotsc$.
We are given
\begin{itemize}
    \item Initial conditions $a_0, a_1, \dotsc, a_{r - 1}$ ($r$ a positive integer).
    \item Recurrence relation
    \begin{equation}
        a_n = \sum_{i = 1}^{r}c_ia_{n - i} = c_1a_{n - 1} + \dotsi + c_ra_{n - r}, (n \geq r),
    \end{equation}
    where $c_1, \dotsc, c_r$ are specified constants.
\end{itemize}

We want to find an explicit formula for $a_n$ in terms of $n$.

The recurrence relation ($1$) is linear because no term involves more than a single factor of the $a_i$,
and it is homogeneous because every term involves an $a_i$.

We seek solutions to the recurrence relation ($1$) of the form $a_n = x ^ n$, $x \neq 0$.
Substitute this into ($1$) to get
\[
x ^ n = \sum_{i = 1}^{r}c_i x ^ {n - i}, (n \geq r).
\]
Divide through by $x ^ {n - r}$
(the excess powers of $x$)
to get
\begin{equation}
    x ^ r - \sum_{i = 1}^{r}c_ix ^ {r - i} = 0.
\end{equation}
This is the characteristic equation
\begin{enumerate}[label = \textbf{Step \arabic*}]
    \item of the method is to write down the characteristic equation.
    
    In general there are $r$ roots to ($2$)
    (possibly complex, or repeated):
    call them $\alpha_1, \dotsc, \alpha_r$.
    \item of the method is to find the roots $\alpha_1, \dotsc, \alpha_r$.
    
    Then,
    for each $i$,
    $a_n = d\alpha_i ^ n$ satisfies the recurrence relation ($1$) for any constant $d$.
    By linearity,
    any linear combination of the $\alpha_i ^ n$ also solves the recurrence relation.

    If the roots $\alpha_1, \dotsc, \alpha_r$ are all distinct,
    then the general solution to the recurrence relation is
    \[
    a_n = g(n) = d_1\alpha_1 ^ n + d_2\alpha_2 ^ n + \dotsi + d_r\alpha_r ^ n,
    \]
    for arbitrary constants $d_1, d_2, \dotsc, d_r$.
    \item is to find the general solution to the recurrence relation.
    \item is to apply the initial conditions given $r$ linear equations in the $r$ unknown constants $d_i$.
    Solve these to recover the solution $a_n$ to the recurrence relation with initial conditions.
\end{enumerate}

\textbf{Repeated roots}.
If the characteristic equation has repeated roots,
we need to tweak the process.

If $\alpha$ is a root repeated $k$ times,
say $\alpha = \alpha_1 = \alpha_2 = \dotsi = \alpha_k$,
then in place of the expression $d_1\alpha_1 ^ n + \dotsi + d_k\alpha_k ^ n$ in the general solution we use instead $(e_1 + e_2n + \dotsi + e_kn ^ {k - 1}) \alpha ^ n$ for constants $e_1, \dotsc, e_k$.

\textit{Note the similarity with the usual approach to solving ordinary linear differential equations.}

\subsection{Solving inhomogeneous linear recurrence relations}
Assume we have the following
\begin{itemize}
    \item Initial conditions $a_0, a_1, \dotsc, a_{r - 1}$
    ($r$ is a positive integer)
    \item Recurrence relation
    \[
    a_n = c_1a_{n - 1} + \dotsi + c_ra_{n - r} + f(n)
    \]
    where $c_1, \dotsc, c_r$ are constants and $f(n)$ is an inhomogeneous part depending on $n$.
\end{itemize}

The method is now as follows
\begin{enumerate}[label = \textbf{Step \arabic*}]
    \item First consider the associated homogeneous problem
    \[
    a_n = c_1a_{n - 1} + \dotsi + c_ra_{n - r}
    \]
    as above.
    Find the general solution for the homogeneous problem,
    $a_n = g(n)$.
    Do not attempt to find any unknown constants.
    \item Find a specific particular solution,
    $a_n = p(n)$.
    \item The general solution to the inhomogeneous problem is then the sum of these:
    \[
    a_n = g(n) + p(n)
    \]
    \item Apply the initial condition to determine the unknown constants
    (in the $g(n)$ part).
\end{enumerate}

Here is a table of particular solutions,
well forms of them.
\begin{table}[H]
    \centering
    \begin{tabular}{c|c}
         $f(n)$ & form of particular solution $p(n)$ \\
         \hline
         $A$ & $B_0$ (constant) \\
         $An$ & $B_0 + B_1n$ (linear) \\
         $An ^ 2$ & $B_0 + B_1n + B_2n ^ 2$ (quadratic) \\
         $Aq ^ n$ & $B_0q ^ n$ (exponential) \\
         $Anq ^ n$ & $(B_0 + B_1n)q ^ n$ (etc$\dotsc$)
    \end{tabular}
\end{table}

However,
there are some exceptions which need adjustment to the form of $p(n)$:
\begin{enumerate}[label = (\alph*)]
    \item If $f(n)$ contains $q ^ n$ and the constant $q$ is a
    (possibly repeated)
    root of the characteristic equation,
    then the homogeneous part of the solution already contains a $q ^ n$ term,
    so in $p(n)$ we should replace $q ^ n$ by $nq ^ n$
    (or $n ^ kq ^ n$ for appropriate $k$ in the case of a repeated root).
    \item If $1$ is a root of the characteristic equation,
    then what looks like $f(n) = An ^ k$ should be thought of as $f(n) = An ^ k \cdot 1 ^ n$ which now puts us back in case (a),
    and we may need to increase the powers of $n$ in the choice of $p(n)$ to deal with this.
\end{enumerate}

\newpage

\section{Generating functions}

\subsection{What is a generating function?}

Generating function provide a powerful general method of solving counting problems,
including recurrence relations.

Generating functions encode combinatorial information in a convenient analytical form.

For example,
we have already seen that
\[
(1 + x) ^ n = \sum_{k = 0}^{n}\binom{n}{k}x ^ k.
\]
The
(real analytic)
function $f(x) = (1 + x) ^ n$ encodes the counting information
\[
\binom{n}{0}, \binom{n}{1}, \dotsc, \binom{n}{n}
\]
as the coefficients in the corresponding power series.

Generating functions provide a systematic approach to exploiting the link between discrete and continuous mathematics.

\begin{definition}
    Let $a_0, a_1, a_2, \dotsc$ be a sequence of real numbers.
    The
    (ordinary)
    generating function for the sequence $(a_k)$ is the formal sum
    \[
    f(x) = \sum_{k = 0}^{\infty}a_kx ^ k.
    \]
\end{definition}

We are allowed to manipulate these generating functions without worrying about convergence of the power series,
this is what we mean by a formal sum.
We can do addition, multiplication, differentiation, etc.,
in any normal way.

\subsection{Counting with generating functions}
\begin{example}
    $7$ chocolate bars are distributed among $4$ people.
    Alice gets at most $4$,
    Brian gets at most $3$,
    Claire gets at most $2$,
    and Derek gets at most $1$.

    How many ways of distributing the bars are there?
    \begin{proof}[Solution]\renewcommand{\qedsymbol}{}
        We are counting solutions $(e_1, e_2, e_3, e_4)$ to the equation
        \[
        e_1 + e_2 + e_3 + e_4 = 7,
        \]
        where $e_1, e_2, e_3, e_4$ are integers satisfying $0 \leq e_1 \leq 4$, $0 \leq e_2 \leq 3$, $0 \leq e_3 \leq 2$.
        and $0 \leq e_4 \leq 1$.
        
        We construct an associated generating function.

        Associate to each variable $e_i$ an expression with the sum of $x ^ p$ over $p$ the possible values of $e_i$.

        E.g. To $e_1$ is associated the term $x ^ 0 + x ^ 1 + x ^ 2 + x ^ 3 + x ^ 4$.

        Create the product of the expressions for each $e_i$:
        \[
        f(x) = (x ^ 0 + x ^ 1 + x ^ 2 + x ^ 3 + x ^ 4)(x ^ 0 + x ^ 1 + x ^ 2 + x ^ 3)(x ^ 0 + x ^ 1 + x ^ 2)(x ^ 0 + x ^ 1).
        \]
        Then each terms in $f(x)$ represents a possible product $x ^ {e_1} \cdot x ^ {e_2} \cdot x ^ {e_3} \cdot x ^ {e_4}$.
        
        But if the sum of the variables equals one holds,
        we know that
        \[
        x ^ {e_1} \cdot x ^ {e_2} \cdot x ^ {e_3} \cdot x ^ {e_4} = x ^ {e_1 + e_2 + e_3 + e_4} = x ^ 7.
        \]
        So every solution $(e_1, e_2, e_3, e_4)$ to the sum corresponds uniquely to a way of multiplying out the expressions in the product $f(x)$ to accumulate $x ^ 7$.

        For example,
        the solution $(e_1, e_2, e_3, e_4) = (4, 0, 2, 1)$ corresponds to
        \[
        (x ^ 0 + x ^ 1 + x ^ 2 + x ^ 3 + x ^ 4)(x ^ 0 + x ^ 1 + x ^ 2 + x ^ 3)(x ^ 0 + x ^ 1 + x ^ 2)(x ^ 0 + x ^ 1)
        \]
        where $x ^ 4, x ^ 0, x ^ 2, x ^ 1$ are selected from each group of sums respectively,
        which is one way of getting $x ^ 7$.
        So the number of solutions is the coefficient of $x ^ 7$ in the series expansion of the generating function $f(x)$.

        The answer is $15$.
    \end{proof}
\end{example}

\begin{example}
    How many solutions are there to the equation
    \[
    e_1 + e_2 + \dotsi + e_n = k,
    \]
    where each $e_i \in \{0, 1\}$ and $n$ and $k$ are given positive integers?
    \begin{proof}[Solution]\renewcommand{\qedsymbol}{}
        Now in constructing our generating function each $e_i$ is associated with the expression $(x ^ 0 + x ^ 1) = (1 + x)$
        (from now on we write $x ^ 0$ as $1$).
        The generating function is then the $n$-fold product
        \[
        f(x) = (1 + x)\cdot(1 + x)\dotsi(1 + x) = (1 + x) ^ n.
        \]
        If we expand the generating function as a power series we get
        \[
        (1 + x) ^ n = \sum_{k = 0}^{n}\binom{n}{k}x ^ k,
        \]
        by the binomial theorem.
        Each term $x ^ k$ corresponds to a choice of $x ^ {e_1}$ from the first bracket,
        $x ^ {e_2}$ from the second,
        etc.,
        in which the total power $x$ is $e_1 + \dotsi + e_n = k$,
        i.e.,
        a solution $(e_1, \dotsc, e_n)$ to the equation.

        So the number of solutions is the coefficient of $x ^ k$ in the generating function $(1 + x) ^ n$,
        which is $\binom{n}{k}$.
    \end{proof}
\end{example}

\begin{example}
    How many solutions are there to the equation
    \[
    e_1 + e_2 + \dotsi + e_n = k,
    \]
    where each $e_i \in \{0, 1, 2\}$ and $n$ and $k$ are given positive integers?
    \begin{proof}[Solution]\renewcommand{\qedsymbol}{}
        Arguing as in the previous example,
        we want the coefficient of $x ^ k$ in the expansion of $(1 + x + x ^ 2) ^ n$.
        
        In other words,
        if $a_k$ is the number of solutions required,
        then
        \[
        (1 + x + x ^ 2) ^ n = \sum_{k = 0}^{\infty}a_kx ^ k
        \]
        is the generating function of the sequence $(a_k)$.
    \end{proof}
\end{example}

\begin{lemma}\phantom{}
    \begin{enumerate}[label = (\alph*)]
        \item For $n \geq 0$ an integer
        \[
        \sum_{k = 0}^{n}x ^ k = \frac{1 - x ^ {n + 1}}{1 - x}
        \]
        is the generating function for the sequence $1, 1, 1, \dotsc, 1, 1, 0, 0, \dotsc$
        (the first $n + 1$ terms are $1$).
        \item The generating function for the sequence $1, 1, 1, \dotsc$ is
        \[
        \sum_{k = 0}^{\infty}x ^ k = \frac{1}{1 - x}.
        \]
        \item The generating function for the sequence $\binom{n + k - 1}{k}$,
        $k \geq 0$ is
        \[
        \sum_{k = 0}^{\infty}\binom{n + k - 1}{k}x ^ k = \frac{1}{(1 - x) ^ n}.
        \]
        \item If $f(x) = \sum_{k = 0}^{\infty}a_kx ^ k$ and $g(x) = \sum_{k = 0}^{\infty}b_kx ^ k$ are the generating function of sequences $(a_k)$ and $(b_k)$ respectively,
        then
        \[
        f(x)g(x) = \sum_{k = 0}^{\infty}c_kx ^ k
        \]
        is the generating function of the sequence $(c_k)$ where $c_k = \sum_{\ell = 0}^{k}a_{\ell}b_{k - \ell}$.
    \end{enumerate}
    \begin{proof}
        \begin{enumerate}[label = (\alph*)]
            \item This is the sum of a geometric series.
            The formula is easy to prove by induction,
            or directly in the form
            \[
            (1 - x)\sum_{k = 0}^{n}x ^ k = \sum_{k = 0}^{n}(x ^ k - x ^ {k + 1}) = 1 - x ^ {n + 1},
            \]
            since the sum 'telescopes'.
            \item The sum to infinity makes analytical sense only for $|x| < 1$,
            in which case we can take $n \rightarrow \infty$ in (a) to get (b).

            Alternatively,
            write $S = 1 + x + x ^ 2 + \dotsi$ and note that $S = 1 + xS$,
            now solve for $S$.

            \item Recall from the extended form of the binomial theorem that
            \[
            (1 - x) ^ {-n} = \sum_{k = 0}^{\infty}\binom{n + k - 1}{k}x ^ k,
            \]
            as claimed.
            For example,
            the case $n = 2$ gives
            \[
            (1 - x) ^ {-2} = \sum_{k = 0}^{\infty}(k + 1)x ^ k = 1 + 2x + 3x ^ 2 + \dotsi,
            \]
            the generating function for the sequence $1, 2, 3, \dotsc$.
            \item This is just collecting terms.
            For example,
            in
            \[
            (1 + 2x + 4x ^ 2 + 8x ^ 3 + \dotsi)(1 + 5x + 7x ^ 2 + 9x ^ 3 + \dotsi)
            \]
            the coefficient of $x ^ 2$ is
            \[
            1 \cdot 7 + 2 \cdot 5 + 4 \cdot 1 = 21.
            \]
        \end{enumerate}
    \end{proof}
\end{lemma}


















\end{document}