\documentclass[10pt, a4paper]{article}
\usepackage{preamble}

\title{Discrete Mathematics}
\author{Luke Phillips}
\date{October 2024}

\begin{document}

\maketitle

\newpage

\section{Introduction}

\textbf{Starter problems}

\textbf{A1}: A set menu has a choice of 3 possible desserts. A party of 5 diners all order
a dessert each. In how many ways can the party choose their desserts?

This question can be rephrased to treat the people as distinguishable balls and the desserts as distinguishable boxes to put the people in.

\textbf{A2}: A man has 5 identical £1 coins that he intends to put into up to 3 different
charity boxes. In how many ways can he distribute the money among the
charities?

This question can be rephrased to treat the coins as balls which are indistinguishable and the charity boxes as distinguishable boxes.

\textbf{A3}: 5 people on a beach divide themselves into up to 3 teams in order to enter a
sand-castle building competition. How many ways are there of constructing
the teams?

This can be rephrased to have balls as people and boxes as teams and having the teams as the different boxes and the people as balls to put in the boxes.

\textbf{A4}: A woman wants to buy postage stamps to the value of 5 pence. Stamps are
available in all denominations of whole pence. How many different ways are
there of making the purchase with at most 3 stamps?

The number of balls in a box can represent the value of a single stamp, i.e. each box is a stamp and each ball is a 1 pence coin.

\subsection{Set notation}
\begin{itemize}
    \item $x \in S$, $x$ is in $x$.
    \item $x \notin S$, $x$ is not in $x$.
    \item $R \subseteq S$, $R$ is a subset of $S$\footnote{$x \in R \implies x \in S$}.
    \item $S \setminus R$, $S$ minus $R$, $S \setminus R = \{x: x \in S \text{ and } x \notin R\}$.
    \item $R \cup S =  \{x : x \in R \text{ or } x \in S\}$.
    \item $R \cap S =  \{x : x \in R \text{ and } x \in S\}$.
\end{itemize}

If $S$ is a finite set its size or cardinality, denoted as $|S|$, is the number of elements in $S$.

\textbf{Cartesian product}

If $X$ and $Y$ are two sets then, 
\[
X \times Y = \{(x, y) : x \in X \text{ and } y \in Y\},
\]
the set of ordered pairs.

\begin{example}
    $X = \{1, 2, 3\}\quad Y = \{a, b\}$
    \[
    X \times Y = \{(1, a), (1, b), (2, a), (2, b), (3, a), (3, b)\}.
    \]
    \[
    Y \times X = \{(a, 1), (a, 2), (a, 3), (b, 1), (b, 2), (b, 3)\}.
    \]
\end{example}
We can write $X \times X = X ^ 2$.
\[
X ^ n = X \times X \times \dotsi X = \{(x_1, x_2, \dots, x_n) : x_i \in X \text{ for all } i\}
\]

\begin{proposition}
    Let $X$ and $Y$ be finite sets.
    \begin{enumerate}[label = (\alph*)]
        \item $|X \times Y| = |X|\cdot|Y|$
        \item $|X ^ n| = |X| ^ n$
    \end{enumerate}
\end{proposition}

\begin{example}
    Let $A = \{1, 2, 3, 4, 5, 6\}$. How many subsets does $A$ have?
    \[
    2 ^ 6
    \]
    Let $I = \{0, 1\}$ consider $I ^ 6 = \{(a_1, a_2, \dots, a_6) : a_i \in I\}$

    $|I ^ 6| = 2 ^ 6$
    
    Consider the set of all subsets of set $\{1, 2, \dots, 6\}$ we construct a bijection between $I ^ 6$ and the set of subsets. $A \mapsto (1, 1, 1, 1, 1, 1)$, $\{1, 4, 5\} \mapsto (1, 0, 0, 1, 1, 0)$
\end{example}

\end{document}