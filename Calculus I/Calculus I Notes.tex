\documentclass[10pt, a4paper]{article}
\usepackage{preamble}

\DeclareMathOperator{\Dom}{Dom}
\DeclareMathOperator{\Ran}{Ran}

\title{Calculus I}
\author{Luke Phillips}
\date{October 2024}

\begin{document}

\maketitle

\newpage

\section{Functions}

\subsection{Functions, domain \& range}

\textbf{Gr10}

\begin{definition}
    A function $f : D \rightarrow C$ is a map ($f$) from a set $D$ called the domain ($\Dom f$) to a set $C$ called the codomain ($\mathrm{Codom}\ f$).
    
    For any $x \in D$, there is a unique $f(x) \in C$ called the image of $x$ under $f$.
\end{definition}

Sometimes  write $f : D \rightarrow C$ $x \mapsto f(x)$.

\begin{definition}
    The range of a function $f$ is the set of all images of $f$. This is a subset of the codomain.
    
    We can write $\Ran f = \{f(x)\,|\,x \in \Dom f\} \subseteq \mathrm{Codom}\ f$
\end{definition}

Real-valued functions\footnote{Always assume codomain is $\R$.} (i.e. the codomain is $\R$, or a subset) of a single real variable (i.e. the domain is $\R$, or a subset)

\begin{example}
    $f : \R \rightarrow \R$ given by $f(x) = x ^ 2\ \forall x \in \R$

    So $\Dom f = \R$ and $\mathrm{Codom}\ f = \R$ \& $\Ran f = [0, \infty)$.
\end{example}

If domain of $f$ isn't explicitly given, we take it to be the maximal subset of $\R$ for which the function is defined.

\subsection{The graph of a function}
\begin{definition}[Graph]
    The graph of $f$ is the set of all points $(x, y) \in \R ^ 2$ such that $x \in \Dom f$ $y = f(x)$.
\end{definition}
To sketch $f$, we often just restrict to an interval of the domain, $[a, b]$.

We often pick a region such that the graph shows significant features like turning points \& intercepts.

When necessary we use a closed shaded circle to indicate an included point, and an open unshaded circle to denote an excluded point.

\textbf{The vertical line test}

If any vertical line intersects the curve more than once then the curve is not the graph of a function.

\subsection{Even and odd functions}
\begin{definition}
    A function is even if $f(x) = f(-x)$ $\forall x \in \Dom f$
    
    A function is odd if $f(x) = -f(-x)$ $\forall x \in \Dom f$
\end{definition}
\begin{example}
    $f(x) = x ^ 2$ is even

    $f(x) = x ^ 3$ is odd
\end{example}

All functions can be written as the sum of an odd part and an even part.
\[
f(x) = f_{even}(x) + f_{odd}(x)
\]
where $f_{even}(x) = \frac{1}{2}\left(f(x) + f(-x)\right)$,
$f_{odd}(x) = \frac{1}{2}\left(f(x) - f(-x)\right)$

\begin{example}
    $f(x) = e ^ x$ with
    
    $f_{even}(x) = \frac{1}{2}(e ^ x + e ^ {-x}) =: \cosh(x)$
    
    $f_{odd}(x) = \frac{1}{2}(e ^ x - e ^ {-x}) =: \sinh(x)$
\end{example}

\begin{example}
    $f(x) = (\cos x + x)\sin x$ with

    $f_{even}(x) = x\sin x$
    
    $f_{odd}(x) = \cos x \sin x$
\end{example}

Product of two odd functions is an even function, product of an odd function and an even function is an odd function.

\subsection{Piecewise functions}
Some functions may be defined piecewise, i.e. with different expressions for different intervals in the domain.

\begin{example}
    \begin{align*}
        f(x) &= |x| \\
        &= \begin{cases}
            \phantom{-}x\quad \text{if} &x \geq 0 \\
            -x\quad \text{if} &x < 0
        \end{cases}
    \end{align*}
\end{example}

A step function is a piecewise function which is constant on each piece.
\begin{example}
    Heaviside step function
    \[
    H(x) = \begin{cases}
        0 & x < 0 \\
        1 & x > 0
    \end{cases}
    \]
    Here $\Dom H = \R \setminus \{0\}$ though it is sometimes useful to extend the domain to $\R$ by defining $H(0): (0, 1, \frac{1}{2})$
\end{example}

\subsection{Operations with functions}
Given two functions $f, g\  \&\  c \in \R$, we can define the following functions:
\begin{table}[h!]
    \centering
    \begin{tabular}{|c|c|c|}
        \hline
        Name & Definition & Domain \\
        \hline
        Sum $(f + g)$ & $(f + g)(x) = f(x) + g(x)$ & $\Dom (f + g) = \Dom f \cap \Dom g$ \\
        Difference $(f - g)$ & $(f - g)(x) = f(x) - g(x)$ & $\Dom (f - g) = \Dom f \cap \Dom g$ \\
        Product $(fg)$ & $(fg)(x) = f(x)g(x)$ & $\Dom (fg) = \Dom f \cap \Dom g$ \\
        Ratio $\left(\frac{f}{g}\right)$ & $\left(\frac{f}{g}\right)(x) = \frac{f(x)}{g(x)}$ & $\Dom \left(\frac{f}{g}\right) = (\Dom f \cap \Dom g) \setminus \{x\,|\,g(x) = 0\}$ \\
        Composition $(f\circ g)$ & $(f \circ g)(x) = f(g(x))$ & $\Dom (f \circ g) = \{x \in \Dom g\,|\, g(x) \in \Dom f\}$ \\
        Scalar Multiplication $(cf)$ & $(cf)(x) = cf(x)$ & $\Dom (cf) = \Dom f$ \\
        \hline
    \end{tabular}
\end{table}

Unlike multiplication, composing functions in opposite order changes the function (generally).

Using the above we can also form more general linear combinations, such as $(af + bg)(x)$ with $(af + bg)(x) = af(x) + bg(x)$ \& $\Dom (af + bg) = \Dom f \cap \Dom g$.

\subsection{Inverse Functions}
\begin{definition}[Surjective function]
    A function $f : D \rightarrow C$ is surjective if $\Ran f = \mathrm{Codom}\, f = C$, i.e. if $\forall\, y \in C\, \exists\, x \in D$ such that $f(x) = y$
\end{definition}
\begin{example}
    $f : \R \rightarrow \R$, $x \mapsto 2x + 1$ is surjective.

    $f : \R \rightarrow \R$, $x \mapsto x ^ 2$ is not surjective since for example $(-2)$ is not the image of any $x \in \R$ (in fact true $\forall\,u < 0,\, u \in \R$).
\end{example}

\begin{definition}[Injective function]
    $f : D \rightarrow C$ is injective if $\forall\, x_1, x_2 \in \R$ with $x_1 \neq x_2,\, f(x_1) \neq f(x_2)$.
\end{definition}

\begin{example}
    $f(x) = 2x + 1$ is injective since $f(x_1) = f(x_2)$ implies that $x_1 = x_2$.

    $f(x) = x ^ 2$ is not injective since $f(2) = f(-2)$ (in fact $f(x) = f(-x)$ but $x \neq -x\, \forall\, x\neq 0$).
\end{example}

\textbf{Horizontal line test}

If no horizontal line intersects the graph of $f$ more than once then $f$ is injective, otherwise it is not.

\begin{definition}[Bijective function]
    A function $f : D \rightarrow C$ is bijective if it is both injective and surjective.
\end{definition}

\end{document}