\documentclass[10pt, a4paper]{article}
\usepackage{preamble}

\DeclareMathOperator{\Dom}{Dom}
\DeclareMathOperator{\Ran}{Ran}

\title{Calculus I}
\author{Luke Phillips}
\date{October 2024}

\begin{document}

\maketitle

\newpage

\section{Functions}

\subsection{Functions, domain \& range}

\textbf{Gr10}

\begin{definition}
    A function $f : D \rightarrow C$ is a map ($f$) from a set $D$ called the domain ($\Dom f$) to a set $C$ called the codomain ($\mathrm{Codom}\ f$).
    
    For any $x \in D$, there is a unique $f(x) \in C$ called the image of $x$ under $f$.
\end{definition}

Sometimes  write $f : D \rightarrow C$ $x \mapsto f(x)$.

\begin{definition}
    The range of a function $f$ is the set of all images of $f$. This is a subset of the codomain.
    
    We can write $\Ran f = \{f(x)\,|\,x \in \Dom f\} \subseteq \mathrm{Codom}\ f$
\end{definition}

Real-valued functions\footnote{Always assume codomain is $\R$.} (i.e. the codomain is $\R$, or a subset) of a single real variable (i.e. the domain is $\R$, or a subset)

\begin{example}
    $f : \R \rightarrow \R$ given by $f(x) = x ^ 2\ \forall x \in \R$

    So $\Dom f = \R$ and $\mathrm{Codom}\ f = \R$ \& $\Ran f = [0, \infty)$.
\end{example}

\end{document}