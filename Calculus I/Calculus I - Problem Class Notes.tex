\documentclass[10pt, a4paper]{article}
\usepackage{preamble}

\title{Calculus I \\
    \large Problem Class}
\author{Luke Phillips}
\date{Feburary 2025}

\begin{document}

\maketitle

\newpage

\tableofcontents

\newpage

\section{Problems Class \texorpdfstring{$1$}{}}

\begin{problem}
    Find the directional derivative of $f = x ^ 2y ^ 2 / 4 - 2y$ in the direction $\hat{\mbf{n}} = (3 / 5, -4 / 5)$ at $(x, y) = (2, 1)$.

    \begin{solution}
        \[
        \nabla f \cdot \hat{\mbf{n}}
        \]
        \[
        \hat{\mbf{n}} = (\frac{3}{5}, -\frac{4}{5}).
        \]
        \[
        \nabla f = (f_x, f_y) = (\frac{2xy ^ 2}{4}, \frac{2x ^ 2y}{4} - 2).
        \]
        At $(2, 1)$,
        $\nabla f = (1, 0)$.
        $\nabla f \cdot \hat{\mbf{n}} = (1, 0) \cdot (\frac{3}{5}, -\frac{4}{5}) = \frac{3}{5}$.
    \end{solution}
\end{problem}

\begin{problem}
    A change of coordinates $(x, y) \rightarrow (u, v)$ is given by the relations
    \[
    u = \frac{1}{2}(x ^ 2 + y ^ 2), \quad v = \frac{1}{2}(x ^ 2 - y ^ 2)
    \]
    where $x, y > 0$ and $u > v$.
    For an arbitrary differentiable function $f$,
    find expressions for $f_x$,
    $f_y$ and $f_{xx}$ expressed entirely in terms of $u$ and $v$.

    \begin{solution}
        \begin{align*}
            u + v &= x ^ 2 \\
            u - v &= y ^ 2
        \end{align*}
        \[
        \pd[f]{x} = \pd[u]{x}\pd[f]{u} + \pd[v]{x}\pd[f]{v} = x\pd[f]{u} + x\pd[f]{v} = x(f_u + f_v) = \sqrt{u + v}(f_u + f_v).
        \]
        \[
        \pd[f]{y} = \pd[u]{y}\pd[f]{u} + \pd[v]{y}\pd[f]{v} = y\pd[f]{u} - y\pd[f]{v} = y(f_u - f_v) = \sqrt{u - v}(f_u - f_v).
        \]
        \begin{align*}
            f_{xx} &= x\pd{u}f_x + x\pd{v}f_x \\
            &= x\left(\pd{u} + \pd{v}\right)f_x \\
            &= \sqrt{u + v}\left(\pd{u} + \pd{v}\right)[(u + v)(f_u + f_v)] \\
            &= \sqrt{u + v}\left(\frac{1}{2\sqrt{u + v}}(f_u + f_v) + \sqrt{u + v}(f_{uu} + f_{vu})\right. \\
            &+ \left.\frac{1}{2\sqrt{u + v}}(f_u + f_v) + \sqrt{u + v}(f_{uv} + f_{vv})\right) \\
            &= f_u + f_v + (u + v)(f_{uu} + 2f_{uv} + f_{vv}).
        \end{align*}
    \end{solution}
\end{problem}

\begin{problem}
    Define the gradient $\nabla f$ of the function $f(x, y, z)$.
    Calculate $\nabla f$ in the case that
    \[
    f(x, y, z) = 8x ^ 3y + 6xy + 4y ^ 2x + \ln(x + y + 2z),
    \]
    and use your answer to find the tangent plane to the surface $f(x, y, z) = 2$ at the point $(x, y, z) = (1 / 2, 1 / 2, 0)$.

    \begin{solution}
        $\nabla f = (f_x, f_y, f_z)$.
        \[
        \nabla f = \left(24x ^ 2y + 6y + \frac{1}{x + y + 2z}, 8x ^ 3 + 6x + 8yz + \frac{1}{x + y + 2z}, 4y ^ 2 + \frac{2}{x + y + 2z}\right).
        \]
        $\nabla f$ is normal to surface $f = 2$ and tangent plane.

        At $(1 / 2, 1 / 2, 0)$
        \[
        \nabla f = (3 + 3 + 1, 1 + 3 + 1, 1 + 2) = (7, 5, 3).
        \]
        Plane with normal $(7, 5, 3)$ has equation
        \[
        7x + 5y + 3z = C
        \]
        $(1 / 2, 1 / 2, 0)$ lies on tangent plane
        \[
        7(1 / 2) + 5(1 / 2) + 3(0) = C = 6
        \]
        so equation of plane is
        \[
        7x + 5y + 3z = 6.
        \]
    \end{solution}
\end{problem}

\begin{problem}
    Find the Taylor series for the function $f(x, y) = \ln(1 + x)(1 + y) ^ {-1}$ around the point $(2, 3)$ to quadratic order.

    \begin{solution}
        \[
        f(x, y) = \frac{\ln(1 + x)}{(1 + y)}.
        \]

        At $(2, 3)$,
        \begin{align*}
            f(2, 3) &= \frac{\ln{3}}{4}, \\
            f_x &= \frac{1}{(1 + x)(1 + y)}, \\
            f_y &= -\frac{\ln(1 + x)}{(1 + y) ^ 2} = -\frac{\ln{3}}{16}, \\
            f_{xx} &= \frac{-1}{(1 + x) ^ 2(1 + y)} = -\frac{1}{3 ^ 2 \cdot 4} = -\frac{1}{36}, \\
            f_{xy} &= -\frac{1}{(1 + x)(1 + y) ^ 2} = \frac{-1}{3 \cdot 4 ^ 2} = -\frac{1}{48}, \\
            f_{yy} &= \frac{2\ln(1 + x)}{(1 + y) ^ 3} = \frac{\ln{3}}{32}.
        \end{align*}
        \begin{align*}
            f(x, y) &= f(x_0, y_0) + f_x(x - x_0) + f_y(y - y_0) + \frac{1}{2}(f_{xx}(x - x_0) ^ 2 \\
            &+ 2f_{xy}(x - x_0)(y - y_0) + f_{yy}(y - y_0) ^ 2) + \dotsc \\
            \frac{\ln(1 + x)}{1 + y} &= \frac{\ln{3}}{4} + \frac{1}{12}(x - 2) - \frac{\ln{3}}{16}(y - 3) \\
            &+ \frac{1}{2}\left(-\frac{1}{36}(x - 2) ^ 2 - \frac{1}{24}(x - 2)(y - 3) + \frac{\ln{3}}{32}(y - 3) ^ 2\right) + \dotsc.
        \end{align*}
    \end{solution}
\end{problem}

\begin{problem}
    Find and classify the stationary points of the function
    \[
    f(x, y) = x ^ 3 - x ^ 2y + xy ^ 2 - 9x.
    \]

    \begin{solution}
        \begin{align*}
            0 &= f_x = 3x ^ 2 - 2xy + y ^ 2 - 9 = 0 \\
            0 &= f_y = -x ^ 2 + 2xy = 0
        \end{align*}
        \[
        f_y = x(2y - x) = 0 \implies x = 0\text{ or } x = 2y.
        \]
        If $x = 0$ $f_x = 0 \implies y ^ 2 - 9 = 0 \implies y = \pm 3$
        if $x = 2y$ $f_x = 0 \implies 12y ^ 2 - 4y ^ 2 + y ^ 2 - 9 = 0 \implies 9y ^ 2 - 9 = 0 \implies y = \pm 1$.

        Stationary points at
        \[
        (0, 3), (0, -3), (2, 1), (-2, -1).
        \]
        Classify
        \[
        H = \begin{pmatrix}
            f_{xx} & f_{xy} \\
            f_{yx} & f_{yy}
        \end{pmatrix} = \begin{pmatrix}
            6x - 2y & 2y - 2x \\
            2y - 2x & 2x
        \end{pmatrix}
        \]
        \begin{align*}
            \det{H} &= (6x - 2y)2x - (2y - 2x) ^ 2 \\
            &= 12x ^ 2 - 4yx - 4x ^ 2 - 4y ^ 2 + 8yx \\
            &= 8x ^ 2 + 4yx - 4y ^ 2.
        \end{align*}
        At $(0, \pm 3)$ $\det{H} = -4y ^ 2 = -36 < 0$,
        so both saddle points.

        At $(2, 1)$ $\det{H} = 32 + 8 - 4 = 36 > 0$,
        $f_{xx} = 6x - 2y = 12 - 2 = 10 > 0$ so it is a minimum.
        
        At $(-2, -1)$ $\det{H} = 32 + 8 - 4 = 36 > 0$,
        $f_{xx} = 6x - 2y = 2 - 12 = -10 < 0$ so it is a maximum.
    \end{solution}
\end{problem}

\begin{problem}
    Find and classify the stationary points of $f(x, y, z) = -x ^ 4 - y ^ 4 - z ^ 4 + 4x + 4y + 4z$.

    \begin{solution}
        \[
        H = \begin{pmatrix}
            f_{xx} & f_{xy} & f_{xz} \\
            f_{yx} & f_{yy} & f_{yz} \\
            f_{zx} & f_{zy} & f_{zz}
        \end{pmatrix}
        \]
    \end{solution}
\end{problem}

\begin{problem}
    Using the method of Lagrange multipliers to find the stationary points of
    \[
    f(x, y) = \ln(5 + x) - \ln(5 - y)
    \]
    subject to the constraint $g(x, y) \equiv x ^ 2 + y ^ 2 - 1 = 0$ and hence show that
    \[
    -\ln\left(\frac{4}{3}\right) \leq f(x, y) \leq \ln\left(\frac{4}{3}\right)
    \]
    for $(x, y)$ satisfying the constraint.

    \begin{solution}
        \begin{align*}
            \phi &= f - \lambda g \\
            &= \ln(5 + x) - \ln(5 - y) - \lambda(x ^ 2 + y ^ 2 - 1)
        \end{align*}
        \begin{align*}
            0 = \phi_x &= \frac{1}{5 + x} - 2\lambda x \\
            0 = \phi_y &= \frac{1}{5 - y} - 2\lambda y \\
            0 &= x ^ 2 + y ^ 2 - 1
        \end{align*}
        \begin{align*}
            y\phi_x - x\phi_y = \frac{y}{5 + x} - \frac{x}{5 - y} = 0 &\implies y(5 - y) = x(5 + x) \\
            &\implies 5y - y ^ 2 = 5x + x ^ 2 \\
            &\implies 5y - 5x = x ^ 2 + y ^ 2 = 1 \\
            &\implies y = x + \frac{1}{5}.
        \end{align*}
        Substitute into the constraint
        \begin{align*}
            x ^ 2 + \left(x + \frac{1}{5}\right) ^ 2 = 1 &\iff 2x ^ 2 + \frac{2}{5}x + \frac{1}{25} = 1 \\
            &\iff 50x ^ 2 + 10x - 24 = 0 \\
            &\iff 25x ^ 2 + 5x - 12 = 0 \\
            &\iff (5x + 4)(5x - 3) = 0
        \end{align*}
        so $x = -\frac{4}{5}, y = -\frac{3}{5}$ or $x = \frac{3}{5}, y = \frac{4}{5}$.
        \[
        f\left(-\frac{4}{5}, -\frac{3}{5}\right) = \ln\left(\frac{5 + x}{5 - y}\right) = \ln\left(\frac{5 + \frac{4}{5}}{5 - \frac{3}{5}}\right) = \ln\left(\frac{\frac{28}{25}}{\frac{28}{25}}\right) = -\ln\frac{4}{3}
        \]
        \[
        f(3 / 5, 4 / 5) = \ln(4 / 3).
        \]
        Implies
        \[
        -\ln\frac{4}{5} \leq f(x, y) \leq \ln{\frac{4}{5}}.
        \]
    \end{solution}
\end{problem}

\newpage

\section{Problems Class \texorpdfstring{$2$}{}}

\begin{problem}[Resit $2022$]
    Consider the differential equation
    \[
    (x ^ 2 - 1)\frac{d ^ 2y}{dx ^ 2} + \lambda y = 0.
    \]
    \begin{enumerate}[label = (\alph*)]
        \item Which values of $x$ are singular points of the equation?

        \item The solution $y(x)$ to the equation is expressed as the following series
        \[
        y(x) = \infsum[n = 0]a_nx ^ n.
        \]
        Find a recurrence relation for the coefficients $a_n$.

        \item Use the recurrence relation to find a value of $\lambda$ for which the equation has a fourth order polynomial solution $P_4(x)$.
        If the normalisation of $P_4(x)$ is chosen so that $P_4(0) = 1$,
        find $P_4(x)$ and show that it can be written in the form
        \[
        P_r(x) = (1 - x ^ 2)Q(x)
        \]
        where $Q(x)$ is a quadratic polynomial you should find explicitly.
    \end{enumerate}

    \begin{solution}
        \begin{enumerate}[label = (\alph*)]
            \item 
            Putting it into canonical form
            \begin{align*}
                \frac{d ^ 2y}{dx ^ 2} + \frac{\lambda}{x ^ 2 - 1}y &= 0 \\
                y'' + py' + qy &= 0 \implies p = 0, q = \frac{\lambda}{x ^ 2 - 1}
            \end{align*}
            $q$ not analytic at $x = \pm 1$,
            so these are singular points.

            \item
            \[
            y = \infsum[n = 0]a_nx ^ n
            \]
            \begin{align*}
                0 &= (x ^ 2 - 1)\infsum[n = 0]a_nn(n - 1)x ^ {n - 2} + \lambda\infsum[n = 0]a_nx ^ n \\
                &= \infsum[n = 0]a_nn(n - 1)x ^ n - \infsum[n = 0]a_nn(n - 1)x ^ {n - 2} + \lambda\infsum[n = 0]a_nx ^ n \\
                \intertext{Putting $n = n + 2$ into the second term}
                &= \infsum[n = 0]a_nx ^ n\left[n(n - 1) + \lambda\right] - \infsum[n = -2]a_{n + 2}(n + 2)(n + 1)x ^ n \\
                &= \infsum[n = 0]a_nx ^ n\left[n(n - 1) + \lambda\right] - \infsum[n = 0]a_{n + 2}(n + 2)(n + 1)x ^ n \\
                &= 0
            \end{align*}
            the coefficient of $x ^ n$ should vanish so
            \[
            a_n\left[n(n - 1) + \lambda\right] = a_{n + 2}(n + 2)(n + 1) \implies a_{n + 2} = \frac{n(n - 1) + \lambda}{(n + 1)(n + 2)}a_n.
            \]

            \item
            Since $a_{n + 2}$ related to $a_n$ by recurrence relation all even $a_n$'s are related to $a_0$ and all the odd $a_n$'s are related to $a_1$.

            The even series will look as follows
            \[
            y = a_0 + a_2x ^ 2 + a_4x ^ 4 + a_6x ^ 6 + \dotsc
            \]
            with $a_6 = 0$ and $a_4 \neq 0$.
            For fourth order polynomial we need $a_6 = 0$ and $a_4 \neq 0$
            \[
            0 = a_6 = \frac{12 + \lambda}{30}a_4 \implies \lambda = -12
            \]
            \begin{align*}
                P_4(x) &= a_0 + a_2x ^ 2 + a_4x ^ 4 \\
                P_4(0) &= 1 \\
                &= a_0 \\
                a_2 &= \frac{\lambda}{1 \cdot 2}a_0 \\
                &= \frac{\lambda}{2} \\
                &= -6 \\
                a_4 &= \frac{2 + \lambda}{3 \cdot 4}a_2 \\
                &= \frac{-6(-10)}{12} \\
                &= 5
                \intertext{hence}
                P_4(x) &= 1 - 6x ^ 2 + 5x ^ 4 \\
                &= (1 - x ^ 2)(1 - 5x ^ 2) \implies Q(x) = 1 - 5x ^ 2.
            \end{align*}
        \end{enumerate}
    \end{solution}
\end{problem}

\begin{problem}[Resit $2022$]
    A differential operator $\mathcal{L}$ is defined to be such that
    \[
    \mathcal{L}y = \left(\frac{d ^ 2}{dx ^ 2} + (1 - x)\frac{d}{dx}\right)y = y'' + (1 - x)y'.
    \]
    \begin{enumerate}[label = (\alph*)]
        \item By considering the action of $\mathcal{L}$ on functions of the form $y = x ^ n$,
        show that $\mathcal{L}$ maps the space of $m$ order polynomials $\R[x]_m$ to itself.

        \item Find explicitly a $4 \times 4$ matrix $M$ which represents the action of $\mathcal{L}$ on cubic polynomials.

        \item Using your result for part (b) or otherwise,
        find a cubic polynomial $P_3(x)$ such that
        \[
        \mathcal{L}P_3(x) = -3P_3(x).
        \]
        
    \end{enumerate}

    \begin{solution}
        \begin{enumerate}[label = (\alph*)]
            \item 
            \begin{align*}
                \mathcal{L}x ^ n &= \frac{d ^ 2}{dx ^ 2}(x ^ n) + (1 - x)\frac{d}{dx}(x ^ n) \\
                &= n(n - 1)x ^ {n - 2} + (1 - x)nx ^ {n - 1} \\
                &= -nx ^ n + nx ^ {n - 1} + n(n - 1)x ^ {n - 2}
            \end{align*}
            $\mathcal{L}$ doesn't increase power of $x$;
            hence $\mathcal{L}$ is a map from $\R[x]_m \rightarrow \R[x]_m$,
            i.e. $\mathcal{L} : \R[x]_m \rightarrow \R[x]_m$.

            \item $y(x) = a_0 + a_1x + a_2x ^ 2 + a_3x ^ 3$
            with
            \[
            \begin{pmatrix}
                a_0 \\ a_1 \\ a_2 \\ a_3
            \end{pmatrix}
            \]
            \begin{align*}
               \mathcal{L}x ^ 0 &= 0 \implies M\begin{pmatrix}
                    1 \\ 0 \\ 0 \\ 0
                \end{pmatrix} = \begin{pmatrix}
                    0 \\ 0 \\ 0 \\ 0
                \end{pmatrix} \\
               \mathcal{L}x ^ 1 &= -x + 1 \implies M\begin{pmatrix}
                    0 \\ 1 \\ 0 \\ 0
                \end{pmatrix} = \begin{pmatrix}
                    1 \\ -1 \\ 0 \\ 0
                \end{pmatrix} \\
               \mathcal{L}x ^ 2 &= -2x ^ 2 + 2x + 2 \implies M\begin{pmatrix}
                    0 \\ 0 \\ 1 \\ 0
                \end{pmatrix} = \begin{pmatrix}
                    2 \\ -2 \\ 2 \\ 0
                \end{pmatrix} \\
               \mathcal{L}x ^ 3 &= -3x ^ 3 + 3x ^ 2 + 6x \implies M\begin{pmatrix}
                    0 \\ 0 \\ 0 \\ 1
                \end{pmatrix} = \begin{pmatrix}
                    0 \\ 6 \\ 3 \\ -3
                \end{pmatrix}.
            \end{align*}
            Thus
            \[
            M = \begin{pmatrix}
                0 & 1 & 2 & 0 \\
                0 & -1 & 2 & 6 \\
                0 & 0 & -2 & 3 \\
                0 & 0 & 0 & -3
            \end{pmatrix}
            \]

            \item Find eigenvector of $M$ with eigenvalue $-3$.
            \[
            (M - \lambda I)\mbf{v} = 0
            \]
            \[
            M = \begin{pmatrix}
                3 & 1 & 2 & 0 \\
                0 & 2 & 2 & 6 \\
                0 & 0 & 1 & 3 \\
                0 & 0 & 0 & 0
            \end{pmatrix}\begin{pmatrix}
                a_0 \\ a_1 \\ a_2 \\ a_3
            \end{pmatrix}
            \]
            from this we get:
            \begin{align*}
                a_2 + 3a_3 &= 0 \implies a_2 = -3a_3 \\
                2a_1 + 2a_2 + 6a_3 &= 0 \implies a_1 = 0 \\
                3a_0 + a_1 + 2a_2 &= 0 \implies a_0 = 2a_3.
            \end{align*}
            Giving us the eigenvector
            \[
            \begin{pmatrix}
                2a_3 \\ 0 \\ -3a_3 \\ a_3
            \end{pmatrix} = a_3\begin{pmatrix}
                2 \\ 0 \\ -3 \\ 1
            \end{pmatrix} \implies P_3(x) = a_3(x ^ 3 - 3x ^ 2 + 2).
            \]
        \end{enumerate}
    \end{solution}
\end{problem}

\begin{problem}[Resit $2021$]
    The function $y(x)$ satisfies the differential equation
    \[
    x ^ 2(x + 5)y'' - \frac{x}{5}(4x - 5)y' - y = 0.
    \]
    Explain briefly why in this case one should look for a series solution in the form $y(x) = \infsum[n = 0]a_nx ^ {n + r}$ where $r$ is not necessarily an integer.
    Find the possible values of $r$ and show that in the case that $r \notin \Z$,
    the solution is of the form
    \[
    y(x) = a_0x ^ rP(x)
    \]
    where $P(x)$ is a polynomial you should determine.

    \begin{solution}
        \[
        x ^ 2(x + 5)y'' - \frac{x}{5}(4x - 5)y' - y = 0.
        \]
        In canonical form:
        \[
        y'' - \frac{(4x + 5)}{5x(x + 5)}y' - \frac{1}{x ^ 2(x + 5)}y = 0
        \]
        $p = -\frac{4x + 5}{5x(x + 5)}$,
        $q = -\frac{1}{x ^ 2(x + 5)}$.
        $p, q$ not analytic at $x_0 = 0$ implies $x_0 = 0$ is singular.
        $px, qx ^ 2$:
        \begin{align*}
            px &= -\frac{4x + 5}{5(x + 5)} \\
            qx ^ 2 &= -\frac{1}{x + 5}
        \end{align*}
        are analytic at $x_0 = 0$ which is regular singular therefore can use Frobenius' method.

        Substitute $y = \infsum[n = 0]a_nx ^ {n + r}$ into the equation.
        \begin{align*}
            0 &= (x ^ 3 + 5x ^ 2)\infsum[n = 0]a_n(n + r)(n + r - 1)x ^ {n + r - 2} + \left(x - \frac{4}{5}x ^ 2\right)\infsum[n = 0]a_n(n + r)x ^ {n + r - 1} - \infsum[n = 0]a_nx ^ {n + r} \\
            &= \infsum[n = 0]a_n(n + r)(n + r - 1)x ^ {n + r + 1} + \infsum[n = 0]5(n + r)(n + r - 1)x ^ {n + r} + \infsum[n = 0]a_n(n + r)x ^ {n + r} \\
            &- \infsum[n = 0]\frac{4}{5}a_n(n + r)x ^ {n + r + 1} - \infsum[n = 0]a_nx ^ {n + r} \\
            &= \infsum[n = 0]a_n(n + r)\left(n + r - \frac{9}{5}\right)x ^ {n + r + 1} + \infsum[n = 0]a_nx ^ {n + r}\left[(n + r)(5n + 5r - 4) - 1\right] \\
            &= \infsum[n = 0]a_n(n + r)\left(n + r - \frac{9}{5}\right)x ^ {n + r + 1} + \infsum[n = -1]a_{n + 1}x ^ {n + r + 1}\left[(n + r + 1)(5n + 5r + 1) - 1\right].
        \end{align*}
        Looking at the lowest power of $x$:
        $x ^ r$,
        at $n = -1$ we get
        \[
        a_0\left[r(5r - 4) - 1\right] = 0 \implies 0 = 5r ^ 2 - 4r - 1 = (5r + 1)(r - 1) = 0 \implies r = 0, -\frac{1}{5}.
        \]
        Since $r \notin \Z$ so take $r = -\frac{1}{5}$.
        Setting coefficient of $x ^ {n + r + 1}$ to zero.
        \begin{align*}
            0 &= a_n(n + r)\left(n + r - \frac{9}{5}\right) + a_{n + 1}\left[(n + r + 1)(5n + 5 - 1) - 1\right] \\
            &= a_n\left(n - \frac{1}{5}\right)(n - 2) + a_{n + 1}\left[\left(n + \frac{4}{5}\right)5n - 1\right] \\
            &= \frac{a_n}{5}(5n - 1)(n - 2) + a_{n + 1}((5n + 4)n - 1) \\
            &= \frac{a_n}{5}(5n - 1)(n - 2) + a_{n + 1}(5n ^ 2 + 4n - 1) \\
            &= \frac{a_n}{5}(5n - 1)(n - 2) + a_{n + 1}(5n - 1)(n + 1) \\
            &\implies a_{n + 1} = \frac{2 - n}{5(n + 1)}a_n.
        \end{align*}
        \begin{align*}
            a_1 &= \frac{-a_0(-2)}{5 \cdot 1} = \frac{2}{5}a_0 \\
            a_2 &= \frac{-a_1}{5}\frac{(-1)}{2} = \frac{a_1}{10} = \frac{a_0}{25} \\
            a_3 &= -\frac{a_2}{5}\frac{0}{3} = 0.
        \end{align*}
        Hence
        \begin{align*}
            y(x) &= a_0x ^ r + a_1x ^ {r + 1} + a_2x ^ {r + 2} + a_3x ^ {r + 3} + \dotsc \\
            &= a_0x ^ {-\frac{1}{5}}\left(1 + \frac{2}{5}x + \frac{1}{25}x ^ 2\right)
        \end{align*}
    \end{solution}
\end{problem}

\newpage

\section{Problems Class \texorpdfstring{$3$}{}}

\begin{problem}[Resit $2020$]
    Find the Fourier transform $\tilde{f}(p)$ of the function.
    The Fourier transform $\tilde{f}(p)$ of a function $f(x)$ is defined to be
    \[
    \tilde{f}(p) = \int_{-\infty}^{\infty}f(x)e ^ {-ipx}\,dx.
    \]

    \begin{enumerate}[label = (\alph*)]
        \item 
        Find the Fourier transform $\tilde{\Pi}(p)$ of the function $\Pi(x)$ defined by
        \[
        \Pi(x) = \begin{cases}
            1 & |x| < 1 \\
            0 & |x| \geq 1.
        \end{cases}
        \]

        \item Show that if $g(x) = f(ax + b)$ for constants $a > 0$,
        $b$ then
        \[
        \tilde{g}(p) = \frac{e ^ {-ipb / a}}{a}\tilde{f}(p / a).
        \]

        \item Use the relation $\Pi(x / 2) = \Pi(x - 1) + \Pi(x + 1)$ and the result of part (b) to show that $\Pi(p)$ satisfies the equation
        \[
        \tilde{\Pi}(2p) = \cos(p)\tilde{\Pi}(p)
        \]
        and show explicitly that the answer to part (a) obeys this relation.
    \end{enumerate}

    \begin{solution}
        \begin{enumerate}[label = (\alph*)]
            \item
            \begin{align*}
                \tilde{\Pi}(p) &= \int_{-\infty}^{\infty}\Pi(x)e ^ {-ipx}\,dx \\
                &= \int_{-1}^{1}e ^ {-ipx}\,dx \\
                &= \left[\frac{e ^ {-ipx}}{-ip}\right]_{-1}^{1} \\
                &= -\frac{e ^ {-ip}}{-ip} + \frac{e ^ {ip}}{ip} \\
                &= \frac{2}{p}\left(\frac{e ^ {ip} - e ^ {-ip}}{2i}\right) \\
                &= \frac{2\sin{p}}{p}.
            \end{align*}

            \item
            \[
            \tilde{g}(p) = \int_{-\infty}^{\infty}g(x)e ^ {-ipx}\,dx = \int_{-\infty}^{\infty}f(ax + b)e ^ {-ipx}\,dx
            \]
            $x' = ax + b$
            \[
            x = \frac{x' - b}{a}\quad dx = \frac{dx'}{a}.
            \]
            \begin{align*}
                \tilde{g}(p) &= \int_{-\infty}^{\infty}g(x)e ^ {-ipx}\,dx \\
                &= \int_{-\infty}^{\infty}f(ax + b)e ^ {-ipx}\,dx \\
                &= \int_{-\infty}^{\infty}f(x')e ^ {-ip\frac{x' - b}{a}}\,\frac{dx'}{a} \\
                &= \frac{e ^ {ipb / a}}{a}\int_{-\infty}^{\infty}f(x')e ^ {-\frac{ipx'}{a}}\,dx' \\
                &= \frac{e ^ {ipb / a}}{a}\tilde{f}\left(\frac{p}{a}\right).
            \end{align*}

            \item

            Fourier transform on left hand side
            \begin{align*}
                g(x) &= \Pi(x / 2)\quad(a = 1 / 2, b = 0) \\
                \tilde{g}(p) &= 2\tilde{\Pi}(2p).
            \end{align*}

            Fourier transform on right hand side
            \begin{align*}
                g(x) &= \Pi(x - 1) + \Pi(x + 1) \\
                \tilde{g}(p) &= e ^ {-ip}\tilde{\Pi}(p) + e ^ {ip}\tilde{\Pi}(p) \\
                &= \tilde{\Pi}(p)\left[e ^ {ip} + e ^ {-ip}\right] \\
                &= 2\tilde{\Pi}(p)\cos(p)
            \end{align*}
            so $\tilde{\Pi}(2p) = \cos(p)\tilde{\Pi}(p)$.

            \[
            \tilde{\Pi}(2p) = \frac{2\sin(2p)}{2p} = \frac{2\sin{p}\cos{p}}{p} = \cos(p)\tilde{\Pi}(p).
            \]
        \end{enumerate}
    \end{solution}
\end{problem}

\begin{problem}[$2024$]
    \begin{enumerate}[label = (\alph*)]
        \item Define the Fourier transform $\tilde{f}(p)$ of the function $f(x)$.

        \item A series of function $f_n(x)$ are defined in terms of $f(x)$ as
        \[
        f_n(x) = \frac{1}{b ^ n}f(x - n),
        \]
        where $b$ is a constant greater than one.
        Use the Shift theorem to write the Fourier transform of $f_n(x)$ in terms of $\tilde{f}(p)$,
        the Fourier transform of $f(x)$.

        \item In the case that
        \[
        f(x) = \begin{cases}
            1 & 0 < x \leq 1 \\
            0 & \text{otherwise}
        \end{cases}
        \]
        find its Fourier transform $\tilde{f}(p)$.

        \item A function $F(x)$ is defined to be
        \[
        F(x) = \begin{cases}
            b ^ {-n} & \text{for $n < x \leq n + 1$ for each integer $n$ greater than or equal to zero} \\
            0 & x \leq 0
        \end{cases}
        \]
        where $b$ is a constant greater than one.
        Find a closed expression for its Fourier transform $\tilde{F}(p)$.
    \end{enumerate}

    \begin{solution}
        \begin{enumerate}[label = (\alph*)]
            \item
            \[
            \tilde{f}(p) = \int_{-\infty}^{\infty}f(x)e ^ {-ipx}\,dx.
            \]

            \item
            Shift theorem $f(x + a)$ has Fourier transform $\tilde{f}(p)e ^ {ipa}$.
            Fourier transform $\frac{1}{b ^ n}f(x - n) = \frac{1}{b ^ n}\tilde{f}(p)e ^ {-ipn}$.

            \item
            \begin{align*}
                \tilde{f}(p) &= \int_{-\infty}^{\infty}f(x)e ^ {-ipx}\,dx \\
                &= \int_{0}^{1}e ^ {-ipx}\,dx \\
                &= \left[\frac{e ^ {-ipx}}{-ip}\right]_{0}^{1} \\
                &= \frac{e ^ {-ip}}{-ip} - \frac{1}{-ip} \\
                &= \frac{1}{ip}(1 - e ^ {-ip}).
            \end{align*}

            \item
            \begin{align*}
                F(x) &= \infsum[n = 0]f_n(x) \\
                \tilde{F}(p) &= \infsum[n = 0]\tilde{f}_n(p) \\
                &= \infsum[n = 0]\frac{1}{b ^ n}\tilde{f}(p)e ^ {-ipn} \\
                &= \infsum[n = 0]\left(\frac{e ^ {-ip}}{b}\right)\tilde{f}(p) \\
                &= \infsum[n = 0]z ^ n\tilde{f}(p) \\
                &= \frac{1}{1 - z} \\
                &= \frac{1}{1 - \left(\frac{e ^ {-ip}}{b}\right)}\frac{1}{ip}(1 - e ^ {-ip}) \\
                &= \frac{be ^ {ip}}{(be ^ {ip} - 1)ip}(1 - e ^ {-ip}) \\
                &= \frac{b(e ^ {ip} - 1)}{(be ^ {ip} - 1)ip}.
            \end{align*}
        \end{enumerate}
    \end{solution}
\end{problem}

\begin{problem}[Resit $2024$]
    The one-sided exponential $f_a(x)$ is defined to be
    \[
    f_a(x) = \begin{cases}
        e ^ {-ax} & x \geq 0 \\
        0 & x < 0
    \end{cases}
    \]
    where $a$ is a positive constant.

    \begin{enumerate}[label = (\alph*)]
        \item Find the Fourier transform $\tilde{f}_a(p)$ of the function $f_a(x)$.

        \item The function $u(x)$ satisfies the differential equation $u_x + 3u = f_a(x)$ where $a > 0$ and $a \neq 3$.
        By taking the Fourier transform of this equation,
        find an expression for the Fourier transform $\tilde{u}(p)$.

        \item By using partial fractions on $\tilde{u}(p)$ or otherwise,
        find the corresponding solution $u(x)$ to the differential equaion,
        and explicitly check that it satisfies the equation.
    \end{enumerate}

    \begin{solution}
        \begin{enumerate}[label = (\alph*)]
            \item
            \begin{align*}
                \tilde{f}_a(p) &= \int_{-\infty}^{\infty}f_a(x)e ^ {-ipx}\,dx \\
                &= \int_{0}^{\infty}e ^ {-ax}e ^ {-ipx}\,dx \\
                &= \int_{-\infty}^{\infty}e ^ {-(ip + a)x}\,dx \\
                &= \left[\frac{e ^ {-(ip + a)x}}{-(ip + a)}\right]_{0}^{\infty} \\
                &= \frac{1}{ip + a}.
            \end{align*}

            \item
            \begin{align*}
                \tilde{u}(ip) + 3\tilde{u} &= \tilde{f}_a = \frac{1}{ip + a} \\
                \tilde{u}(ip + 3) &= \frac{1}{ip + a} \\
                &\implies \\
                \tilde{u}(p) &= \frac{1}{(ip + 3)(ip + a)}
            \end{align*}

            \item
            \begin{align*}
                \tilde{u}(p) &= \frac{1}{(ip + 3)(ip + a)} \\
                &= \frac{1}{3 - a}\left(\frac{1}{ip + a} - \frac{1}{ip + 3}\right) \\
                &= \frac{1}{3 - a}\left(\tilde{f}_a(p) - \tilde{f}_3(p)\right)
            \end{align*}
            \[
            u(x) = \frac{1}{3 - a}(f_a(x) - f_3(x)) = \begin{cases}
                \frac{1}{3 - a}\left(e ^ {-ax} - e ^ {-3x}\right) & x \geq 0 \\
                0 & x < 0.
            \end{cases}
            \]
            $x < 0$,
            $u = 0$
            \[
            u_x + 3u = 0 = f_a(x) \quad x < 0
            \]
            $x \geq 0$,
            $u = \frac{1}{3 - a}(e ^ {-ax} - e ^ {-3x})$
            \[
            u_x + 3u = \frac{1}{3 - a}\left(-ae ^ {-ax} + 3e ^ {-3x}\right) + \frac{3}{3 - a}(e ^ {-ax} - e ^ {-3x}) = e ^ {-ax} = f_a(x)\quad x \geq 0.
            \]
        \end{enumerate}
    \end{solution}
\end{problem}

\begin{problem}[$2020$]
    The function $\theta(x, t)$ satisfies
    \[
    \pd[\theta]{t} = \frac{\partial ^ 2\theta}{\partial x ^ 2} + \pd[\theta]{x} - \theta.
    \]
    The Fourier transform of $\theta$ is defined by
    \[
    \tilde{\theta}(p, t) = \int_{-\infty}^{\infty}e ^ {-ipx}\theta(x, t)\,dx.
    \]

    \begin{enumerate}[label = (\alph*)]
        \item Use the differential equation to obtain $\tilde{\theta}(p, t)$ in terms of its initial value $\tilde{\theta}(p, 0)$.

        \item If $\theta(x, 0) = e ^ {-x ^ 2 / 2}$ use the result of part (a) to show that
        \[
        \theta(x, t) = \frac{1}{\sqrt{2t + 1}}\exp\left(-\frac{(x + t) ^ 2}{4t + 2} - t\right).
        \]
        Fourier transform of $F(x) = e ^ {-bx ^ 2}$ is given by
        \[
        \tilde{F}(p) = \sqrt{\frac{\pi}{b}}e ^ {-\frac{p ^ 2}{4b ^ 2}}.
        \]
    \end{enumerate}

    \begin{solution}
        \begin{enumerate}[label = (\alph*)]
            \item Take Fourier transform of the equation
            \[
            \tilde{\theta_t} = (ip) ^ 2\tilde{\theta} + (ip)\tilde{\theta} - \tilde{\theta} = [-(p ^ 2 + 1) + ip]\tilde{\theta}
            \]
            \[
            \tilde{\theta}(p, t) = A(p)e ^ {[-(p ^ 2 + 1) + ip]t}.
            \]
            At $t = 0$,
            \[
            \tilde{\theta}(p, 0) = A(p)e ^ 0 = A(p) \implies \tilde{\theta}(p, t) = \tilde{\theta}(p, 0)e ^ {-(p ^ 2 + 1)t}e ^ {ipt}.
            \]

            \item
            $\theta(x, 0) = e ^ {-x ^ 2 / 2}$.
            \[
            \tilde{\theta}(p, 0) = \sqrt{2\pi}e ^ {-\frac{p ^ 2}{2}}
            \]
            \begin{align*}
                \tilde{\theta}(p, t) &= \sqrt{2\pi}e ^ {-p ^ 2 / 2}e ^ {-(p ^ 2 + 1)t}e ^ {ipt} \\
                &= \sqrt{2\pi}e ^ {-t}\frac{e ^ {-p ^ 2(t + 1 / 2)}e ^ {ipt}}{e ^ {-bx ^ 2}\sqrt{\frac{\pi}{b}}e ^ {-\frac{p ^ 2}{4b}}} \\
            \end{align*}

            \begin{align*}
                \frac{1}{4b} &= t + 1 / 2 \\
                b &= \frac{1}{4t + 2}
            \end{align*}

            \[
            e ^ {-\frac{x ^ 2}{4t + 2}}\text{ has Fourier transform } \sqrt{(4t + 2)\pi}e ^ {-p ^ 2(t + 1 / 2)}
            \]
            \[
            \frac{\sqrt{2\pi}e ^ {-t}}{\sqrt{2\pi(2t + 1)}}e ^ {-\frac{x ^ 2}{4t + 2}}
            \]
            \[
            \sqrt{2\pi}e ^ {-t}e ^ {-p ^ 2(t + 1 / 2)}.
            \]
            Shift theorem
            \[
            \frac{e ^ {-t}}{\sqrt{2t + 1}}e ^ {-\frac{(x + t) ^ 2}{4t + 2}}
            \]
            has Fourier transform
            \[
            \sqrt{2\pi}e ^ {-t}e ^ {-p ^ 2(t + 1 / 2)}e ^ {ipt}.
            \]
            So
            \[
            \theta(x, t) = \frac{e ^ {-t}}{\sqrt{2t + 1}}.
            \]
            
        \end{enumerate}
    \end{solution}
\end{problem}
















\end{document}