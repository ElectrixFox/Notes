\documentclass[10pt, a4paper]{article}
\usepackage{preamble}

\title{Calculus I \\
    \large Problem Class}
\author{Luke Phillips}
\date{Feburary 2025}

\begin{document}

\maketitle

\newpage

\tableofcontents

\newpage

\section{Problems Class \texorpdfstring{$1$}{}}

\begin{problem}
    Find the directional derivative of $f = x ^ 2y ^ 2 / 4 - 2y$ in the direction $\hat{\mbf{n}} = (3 / 5, -4 / 5)$ at $(x, y) = (2, 1)$.

    \begin{solution}
        \[
        \nabla f \cdot \hat{\mbf{n}}
        \]
        \[
        \hat{\mbf{n}} = (\frac{3}{5}, -\frac{4}{5}).
        \]
        \[
        \nabla f = (f_x, f_y) = (\frac{2xy ^ 2}{4}, \frac{2x ^ 2y}{4} - 2).
        \]
        At $(2, 1)$,
        $\nabla f = (1, 0)$.
        $\nabla f \cdot \hat{\mbf{n}} = (1, 0) \cdot (\frac{3}{5}, -\frac{4}{5}) = \frac{3}{5}$.
    \end{solution}
\end{problem}

\begin{problem}
    A change of coordinates $(x, y) \rightarrow (u, v)$ is given by the relations
    \[
    u = \frac{1}{2}(x ^ 2 + y ^ 2), \quad v = \frac{1}{2}(x ^ 2 - y ^ 2)
    \]
    where $x, y > 0$ and $u > v$.
    For an arbitrary differentiable function $f$,
    find expressions for $f_x$,
    $f_y$ and $f_{xx}$ expressed entirely in terms of $u$ and $v$.

    \begin{solution}
        \begin{align*}
            u + v &= x ^ 2 \\
            u - v &= y ^ 2
        \end{align*}
        \[
        \pd[f]{x} = \pd[u]{x}\pd[f]{u} + \pd[v]{x}\pd[f]{v} = x\pd[f]{u} + x\pd[f]{v} = x(f_u + f_v) = \sqrt{u + v}(f_u + f_v).
        \]
        \[
        \pd[f]{y} = \pd[u]{y}\pd[f]{u} + \pd[v]{y}\pd[f]{v} = y\pd[f]{u} - y\pd[f]{v} = y(f_u - f_v) = \sqrt{u - v}(f_u - f_v).
        \]
        \begin{align*}
            f_{xx} &= x\pd{u}f_x + x\pd{v}f_x \\
            &= x\left(\pd{u} + \pd{v}\right)f_x \\
            &= \sqrt{u + v}\left(\pd{u} + \pd{v}\right)[(u + v)(f_u + f_v)] \\
            &= \sqrt{u + v}\left(\frac{1}{2\sqrt{u + v}}(f_u + f_v) + \sqrt{u + v}(f_{uu} + f_{vu})\right. \\
            &+ \left.\frac{1}{2\sqrt{u + v}}(f_u + f_v) + \sqrt{u + v}(f_{uv} + f_{vv})\right) \\
            &= f_u + f_v + (u + v)(f_{uu} + 2f_{uv} + f_{vv}).
        \end{align*}
    \end{solution}
\end{problem}

\begin{problem}
    Define the gradient $\nabla f$ of the function $f(x, y, z)$.
    Calculate $\nabla f$ in the case that
    \[
    f(x, y, z) = 8x ^ 3y + 6xy + 4y ^ 2x + \ln(x + y + 2z),
    \]
    and use your answer to find the tangent plane to the surface $f(x, y, z) = 2$ at the point $(x, y, z) = (1 / 2, 1 / 2, 0)$.

    \begin{solution}
        $\nabla f = (f_x, f_y, f_z)$.
        \[
        \nabla f = \left(24x ^ 2y + 6y + \frac{1}{x + y + 2z}, 8x ^ 3 + 6x + 8yz + \frac{1}{x + y + 2z}, 4y ^ 2 + \frac{2}{x + y + 2z}\right).
        \]
        $\nabla f$ is normal to surface $f = 2$ and tangent plane.

        At $(1 / 2, 1 / 2, 0)$
        \[
        \nabla f = (3 + 3 + 1, 1 + 3 + 1, 1 + 2) = (7, 5, 3).
        \]
        Plane with normal $(7, 5, 3)$ has equation
        \[
        7x + 5y + 3z = C
        \]
        $(1 / 2, 1 / 2, 0)$ lies on tangent plane
        \[
        7(1 / 2) + 5(1 / 2) + 3(0) = C = 6
        \]
        so equation of plane is
        \[
        7x + 5y + 3z = 6.
        \]
    \end{solution}
\end{problem}

\begin{problem}
    Find the Taylor series for the function $f(x, y) = \ln(1 + x)(1 + y) ^ {-1}$ around the point $(2, 3)$ to quadratic order.

    \begin{solution}
        \[
        f(x, y) = \frac{\ln(1 + x)}{(1 + y)}.
        \]

        At $(2, 3)$,
        \begin{align*}
            f(2, 3) &= \frac{\ln{3}}{4}, \\
            f_x &= \frac{1}{(1 + x)(1 + y)}, \\
            f_y &= -\frac{\ln(1 + x)}{(1 + y) ^ 2} = -\frac{\ln{3}}{16}, \\
            f_{xx} &= \frac{-1}{(1 + x) ^ 2(1 + y)} = -\frac{1}{3 ^ 2 \cdot 4} = -\frac{1}{36}, \\
            f_{xy} &= -\frac{1}{(1 + x)(1 + y) ^ 2} = \frac{-1}{3 \cdot 4 ^ 2} = -\frac{1}{48}, \\
            f_{yy} &= \frac{2\ln(1 + x)}{(1 + y) ^ 3} = \frac{\ln{3}}{32}.
        \end{align*}
        \begin{align*}
            f(x, y) &= f(x_0, y_0) + f_x(x - x_0) + f_y(y - y_0) + \frac{1}{2}(f_{xx}(x - x_0) ^ 2 \\
            &+ 2f_{xy}(x - x_0)(y - y_0) + f_{yy}(y - y_0) ^ 2) + \dotsc \\
            \frac{\ln(1 + x)}{1 + y} &= \frac{\ln{3}}{4} + \frac{1}{12}(x - 2) - \frac{\ln{3}}{16}(y - 3) \\
            &+ \frac{1}{2}\left(-\frac{1}{36}(x - 2) ^ 2 - \frac{1}{24}(x - 2)(y - 3) + \frac{\ln{3}}{32}(y - 3) ^ 2\right) + \dotsc.
        \end{align*}
    \end{solution}
\end{problem}

\begin{problem}
    Find and classify the stationary points of the function
    \[
    f(x, y) = x ^ 3 - x ^ 2y + xy ^ 2 - 9x.
    \]

    \begin{solution}
        \begin{align*}
            0 &= f_x = 3x ^ 2 - 2xy + y ^ 2 - 9 = 0 \\
            0 &= f_y = -x ^ 2 + 2xy = 0
        \end{align*}
        \[
        f_y = x(2y - x) = 0 \implies x = 0\text{ or } x = 2y.
        \]
        If $x = 0$ $f_x = 0 \implies y ^ 2 - 9 = 0 \implies y = \pm 3$
        if $x = 2y$ $f_x = 0 \implies 12y ^ 2 - 4y ^ 2 + y ^ 2 - 9 = 0 \implies 9y ^ 2 - 9 = 0 \implies y = \pm 1$.

        Stationary points at
        \[
        (0, 3), (0, -3), (2, 1), (-2, -1).
        \]
        Classify
        \[
        H = \begin{pmatrix}
            f_{xx} & f_{xy} \\
            f_{yx} & f_{yy}
        \end{pmatrix} = \begin{pmatrix}
            6x - 2y & 2y - 2x \\
            2y - 2x & 2x
        \end{pmatrix}
        \]
        \begin{align*}
            \det{H} &= (6x - 2y)2x - (2y - 2x) ^ 2 \\
            &= 12x ^ 2 - 4yx - 4x ^ 2 - 4y ^ 2 + 8yx \\
            &= 8x ^ 2 + 4yx - 4y ^ 2.
        \end{align*}
        At $(0, \pm 3)$ $\det{H} = -4y ^ 2 = -36 < 0$,
        so both saddle points.

        At $(2, 1)$ $\det{H} = 32 + 8 - 4 = 36 > 0$,
        $f_{xx} = 6x - 2y = 12 - 2 = 10 > 0$ so it is a minimum.
        
        At $(-2, -1)$ $\det{H} = 32 + 8 - 4 = 36 > 0$,
        $f_{xx} = 6x - 2y = 2 - 12 = -10 < 0$ so it is a maximum.
    \end{solution}
\end{problem}

\begin{problem}
    Find and classify the stationary points of $f(x, y, z) = -x ^ 4 - y ^ 4 - z ^ 4 + 4x + 4y + 4z$.

    \begin{solution}
        \[
        H = \begin{pmatrix}
            f_{xx} & f_{xy} & f_{xz} \\
            f_{yx} & f_{yy} & f_{yz} \\
            f_{zx} & f_{zy} & f_{zz}
        \end{pmatrix}
        \]
    \end{solution}
\end{problem}

\begin{problem}
    Using the method of Lagrange multipliers to find the stationary points of
    \[
    f(x, y) = \ln(5 + x) - \ln(5 - y)
    \]
    subject to the constraint $g(x, y) \equiv x ^ 2 + y ^ 2 - 1 = 0$ and hence show that
    \[
    -\ln\left(\frac{4}{3}\right) \leq f(x, y) \leq \ln\left(\frac{4}{3}\right)
    \]
    for $(x, y)$ satisfying the constraint.

    \begin{solution}
        \begin{align*}
            \phi &= f - \lambda g \\
            &= \ln(5 + x) - \ln(5 - y) - \lambda(x ^ 2 + y ^ 2 - 1)
        \end{align*}
        \begin{align*}
            0 = \phi_x &= \frac{1}{5 + x} - 2\lambda x \\
            0 = \phi_y &= \frac{1}{5 - y} - 2\lambda y \\
            0 &= x ^ 2 + y ^ 2 - 1
        \end{align*}
        \begin{align*}
            y\phi_x - x\phi_y = \frac{y}{5 + x} - \frac{x}{5 - y} = 0 &\implies y(5 - y) = x(5 + x) \\
            &\implies 5y - y ^ 2 = 5x + x ^ 2 \\
            &\implies 5y - 5x = x ^ 2 + y ^ 2 = 1 \\
            &\implies y = x + \frac{1}{5}.
        \end{align*}
        Substitute into the constraint
        \begin{align*}
            x ^ 2 + \left(x + \frac{1}{5}\right) ^ 2 = 1 &\iff 2x ^ 2 + \frac{2}{5}x + \frac{1}{25} = 1 \\
            &\iff 50x ^ 2 + 10x - 24 = 0 \\
            &\iff 25x ^ 2 + 5x - 12 = 0 \\
            &\iff (5x + 4)(5x - 3) = 0
        \end{align*}
        so $x = -\frac{4}{5}, y = -\frac{3}{5}$ or $x = \frac{3}{5}, y = \frac{4}{5}$.
        \[
        f\left(-\frac{4}{5}, -\frac{3}{5}\right) = \ln\left(\frac{5 + x}{5 - y}\right) = \ln\left(\frac{5 + \frac{4}{5}}{5 - \frac{3}{5}}\right) = \ln\left(\frac{\frac{28}{25}}{\frac{28}{25}}\right) = -\ln\frac{4}{3}
        \]
        \[
        f(3 / 5, 4 / 5) = \ln(4 / 3).
        \]
        Implies
        \[
        -\ln\frac{4}{5} \leq f(x, y) \leq \ln{\frac{4}{5}}.
        \]
    \end{solution}
\end{problem}


\end{document}