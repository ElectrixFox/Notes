\documentclass[10pt, a4paper]{article}
\usepackage{preamble}

\newcommand{\limas}[3][n]{#2 \rightarrow #3 \text{ as } #1 \rightarrow \infty}

\title{Calculus I \\
    \large Important Results}
\author{Luke Phillips}
\date{January 2025}

\begin{document}

\maketitle

\newpage

\tableofcontents

\newpage

\section{Functions}

\begin{definition}[Even function]
    A function $f$ is even if $\forall \pm x \in \Dom{f}$
    \[
    f(x) = f(-x).
    \]
\end{definition}

\begin{definition}[Odd function]
    A function $f$ is odd if $\forall \pm x \in \Dom{f}$
    \[
    f(x) = -f(-x).
    \]
\end{definition}

\textbf{Horizontal line test}

If no horizontal line intersects the graph of $f$ more than once then $f$ is injective,
otherwise it is not.

\newpage

\section{Limits and continuity}

\begin{definition}[Continuity at a point]
    A function $f(x)$ is continuous at the point $x = a$ if the following properties all hold:
    \begin{enumerate}[label = (\roman*)]
        \item $f(a)$ exists.
        
        \item $\lim_{x \rightarrow a}f(x)$ exists.
        
        \item $\lim_{x \rightarrow a}f(x) = f(a)$.
    \end{enumerate}
\end{definition}

\begin{definition}[Continuity]
    A function $f(x)$ is continuous if it is continuous at every point in its domain.
\end{definition}

\begin{proposition}[Two trigonometric limits]
    \[
    \lim_{x \rightarrow 0}\frac{\sin{x}}{x} = 1\qquad\text{and}\qquad\lim_{x \rightarrow 0}\frac{1 - \cos{x}}{x} = 0.
    \]
\end{proposition}

For a function $f(x)$ with $L = \lim_{x \rightarrow a}f(x)$,
we have the following discontinuities:

\begin{definition}[Removable discontinuity]
    $L$ exists but $f(a) \neq L$.
    The discontinuity can be removed to make the continuous function
    \[
    g(x) = \begin{cases}
        f(x) & \text{if } x \neq a \\
        L & \text{if } x = a.
    \end{cases}
    \]
\end{definition}

\begin{definition}[Jump discontinuity]
    Both $L ^ {+}$ and $L ^ {-}$ exist but $L ^ {+} \neq L ^ {-}$.
\end{definition}

\begin{definition}[Infinite discontinuity]
    In this case at least one of $L ^ {+}$ or $L ^ {-}$ does not exist.
\end{definition}

\begin{theorem}[Intermediate Value Theorem]
    If $f(x)$ is continuous on $[a, b]$ and $u$ is any number between $f(a)$ and $f(b)$ then $\exists c \in (a, b)$ such that $f(c) = u$.
\end{theorem}

\newpage

\section{Differentiation}

\begin{definition}[The derivative]
    Given a function $f(x)$.
    $f(x)$ is differentiable at $x = a$ if
    \[
    f'(a) = \lim_{h \rightarrow 0}\frac{f(a + h) - f(a)}{h}
    \]
    exists.
\end{definition}

\begin{proposition}[The Leibniz rule]
    \[
    D ^ n(fg) = \sum_{k = 0}^{n}\binom{n}{k}(D ^ kf)(D ^ {n - k}g).
    \]
\end{proposition}

\begin{theorem}[Chain rule theorem]
    If $g(x)$ is differentiable at $x$ and $f(x)$ is differentiable at $g(x)$ then the composition $(f \circ g)(x)$ is differentiable at $x$ with
    \[
    (f \circ g)'(x) = f'(g(x))g'(x).
    \]
    Using the Leibniz notation we have
    \[
    \frac{d}{dx}f(g(x)) = \frac{df}{dg}\frac{dg}{dx}.
    \]
\end{theorem}

\begin{theorem}[L'H\^opital's rule]
    Let $f(x)$ and $g(x)$ be differentiable on $I = (a - h, a) \cup (a, a + h)$ for some $h > 0$,
    with $\lim_{x \rightarrow a}f(x) = \lim_{x \rightarrow a}g(x) = 0$.

    If $\lim_{x \rightarrow a}\frac{f'(x)}{g(x)}$ exists and $g'(x) \neq 0\ \forall x \in I$ then
    \[
    \lim_{x \rightarrow a}\frac{f(x)}{g(x)} = \lim_{x \rightarrow a}\frac{f'(x)}{g'(x)}.
    \]
\end{theorem}

For a function $f(x)$ defined in some interval $I$.
\begin{definition}[Global maximum]
    If there exists a constant $k_1$such that $f(x) \leq k_1\ \forall x \in I$ we say that $f(x)$ is bounded above in $I$ and we call $k_1$ an upper bound of $f(x)$ in $I$.

    If there exists a point $x_1$ in $I$ such that $f(x_1) = k_1$ we say that the upper bound of $k_1$ is attained and we call $k_1$ the global maximum value of $f(x)$ in $I$.
    
\end{definition}

\begin{definition}[The Jacobian]
    The Jacobian of the transformation from the variables $x, y$ to $u, v$ is
    \[
    J = \frac{\partial(x, y)}{\partial(u, v)} = \begin{vmatrix}
        \frac{\partial x}{\partial u} & \frac{\partial x}{\partial v} \\
        \frac{\partial y}{\partial u} & \frac{\partial y}{\partial v}
    \end{vmatrix}.
    \]
\end{definition}

To obtain the area element $dxdy$ in terms of the new variables we first compute the Jacobian $J$ and then use the result that
\[
dxdy = |J|dudv.
\]

\begin{definition}[Gaussian integral]
    \[
    \int_{-\infty}^{\infty}e ^ {-ax ^ 2}\,dx = \sqrt{\frac{\pi}{a}}
    \]
    where $a$ is a positive constant.
\end{definition}

\[
M(x, y)\,dx + N(x, y)\,dy = 0.
\]
Rearranging gives
\[
\frac{dy}{dx} = -\frac{M(x, y)}{N(x, y)}.
\]
For any function $g(x, y)$ the total differential $dg$ is defined to be
\[
dg = \frac{\partial g}{\partial x}\,dx + \frac{\partial g}{\partial y}\,dy.
\]
i.e.
\[
M = \frac{\partial g}{\partial x}\quad\text{and}\quad N = \frac{\partial g}{\partial y}.
\]

\textbf{Test for exactness}

The ODE $M(x, y)\,dx + N(x, y)\,dy = 0$ is exact if and only if
\[
\frac{\partial M}{\partial y} = \frac{\partial N}{\partial x}.
\]

A \textbf{Bernoulli equation} is a non-linear ODE of the form
\[
y' + p(x)y = q(x)y ^ n.
\]
To solve a Bernoulli equation use the substitution $v = y ^ {1 - n}$,
this converts the equation to a linear ODE for $v(x)$.

\begin{definition}[The Wronskian]
    Given two differentiable functions,
    $y_1(x), y_2(x)$,
    we define the Wronskian to be
    \[
    W(y_1, y_2) = \begin{vmatrix}
        y_1 & y_2 \\ y'_1 & y'_2
    \end{vmatrix}
    = y_1y'_2 - y_2y'_1.
    \]
\end{definition}

Solving the ODE
\[
\alpha_2y'' + \alpha_2y' + \alpha_0y = \phi
\]
\[
y = y_{CF} + y_{PI}
\]
with $y_{CF} = Ay_1 + By_2$.

Finding the particular solution found using the following
\[
y_{PI} = u_1y_1 + u_2y_2
\]
where
\[
u_1 = -\int\frac{y_2 \phi / \alpha_2}{W(y_1, y_2)}\,dx\quad\text{and}\quad
u_2 = \int\frac{y_1 \phi / \alpha_2}{W(y_1, y_2)}\,dx.
\]


\begin{definition}[Lagrange form for the remainder]
    \[
    R_n(x) = \frac{f ^ {(n + 1)}(c)}{(n + 1)!}(x - a) ^ {n + 1},\quad\text{for some $c \in (a, x)$}.
    \]
\end{definition}

\begin{definition}
    Let $n$ be a positive integer.
    We say that $f(x) = o(x ^ n)$ (as $x \rightarrow 0$)
    if
    \[
    \lim_{x \rightarrow 0}\frac{f(x)}{x ^ n} = 0.
    \]
\end{definition}

\begin{definition}[Fourier definition of a function]
    \[
    f(x) = \frac{a_0}{2} + \infsum\left(a_n\cos\frac{n\pi x}{L} + b_n\sin\frac{n\pi x}{L}\right).
    \]
    with
    \[
    a_0 = \frac{1}{L}\int_{-L}^{L}f(x)\,dx.
    \]
    \[
    a_n = \frac{1}{L}\int_{-L}^{L}f(x)\cos\frac{n\pi x}{L}\,dx.
    \]
    \[
    b_n = \frac{1}{L}\int_{-L}^{L}f(x)\sin\frac{n\pi x}{L}\,dx.
    \]
\end{definition}

\begin{theorem}[Dirichlet's theorem]
    Let $f(x)$ be a periodic function,
    with period $2L$,
    such that on the interval $(-L, L)$ it has a finite number of extreme values,
    a finite number of jump discontinuities and $|f(x)|$ is integrable on $(-L, L)$.
    Then its Fourier series converges for all values $x$.
    Furthermore,
    it converges to $f(x)$ at all points where $f(x)$ is continuous and if $x = a$ is a jump discontinuity then it converges to $\frac{1}{2}\lim_{x \rightarrow a ^ {-}}f(x) + \frac{1}{2}\lim{x \rightarrow a ^ {+}}f(x)$.
\end{theorem}

\begin{theorem}[Parseval’s theorem]
    If $f(x)$ is a function of period $2L$ with Fourier coefficients $a_n, b_n$ then
    \[
    \frac{1}{2L}\int_{-L}^{L}((f(x))) ^ 2\,dx = \frac{1}{4}a_0 ^ 2 + \frac{1}{2}\infsum[n = 1](a_n ^ 2 + b_n ^ 2).
    \]
\end{theorem}

\begin{proposition}[Complex extension of Fourier series]
    A Fourier series can be written in complex form with
    \[
    c_n = \frac{1}{2L}\int_{-L}^{L}f(x)e ^ {\frac{-in\pi x}{L}}\,dx,
    \]
    with the Fourier series
    \[
    f(x) = \infsum[n = -\infty]c_ne ^ {\frac{in\pi x}{L}}.
    \]
\end{proposition}


\end{document}