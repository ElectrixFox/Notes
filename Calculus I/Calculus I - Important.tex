\documentclass[10pt, a4paper]{article}
\usepackage{preamble}

\newcommand{\limas}[3][n]{#2 \rightarrow #3 \text{ as } #1 \rightarrow \infty}

\title{Calculus I \\
    \large Important Results}
\author{Luke Phillips}
\date{January 2025}

\begin{document}

\maketitle

\newpage

\tableofcontents

\newpage

\section{Functions}

\begin{definition}[Even function]
    A function $f$ is even if $\forall \pm x \in \Dom{f}$
    \[
    f(x) = f(-x).
    \]
\end{definition}

\begin{definition}[Odd function]
    A function $f$ is odd if $\forall \pm x \in \Dom{f}$
    \[
    f(x) = -f(-x).
    \]
\end{definition}

\textbf{Horizontal line test}

If no horizontal line intersects the graph of $f$ more than once then $f$ is injective,
otherwise it is not.

\newpage

\section{Limits and continuity}

\begin{definition}[Continuity at a point]
    A function $f(x)$ is continuous at the point $x = a$ if the following properties all hold:
    \begin{enumerate}[label = (\roman*)]
        \item $f(a)$ exists.
        
        \item $\lim_{x \rightarrow a}f(x)$ exists.
        
        \item $\lim_{x \rightarrow a}f(x) = f(a)$.
    \end{enumerate}
\end{definition}

\begin{definition}[Continuity]
    A function $f(x)$ is continuous if it is continuous at every point in its domain.
\end{definition}

\begin{proposition}[Two trigonometric limits]
    \[
    \lim_{x \rightarrow 0}\frac{\sin{x}}{x} = 1\qquad\text{and}\qquad\lim_{x \rightarrow 0}\frac{1 - \cos{x}}{x} = 0.
    \]
\end{proposition}

For a function $f(x)$ with $L = \lim_{x \rightarrow a}f(x)$,
we have the following discontinuities:

\begin{definition}[Removable discontinuity]
    $L$ exists but $f(a) \neq L$.
    The discontinuity can be removed to make the continuous function
    \[
    g(x) = \begin{cases}
        f(x) & \text{if } x \neq a \\
        L & \text{if } x = a.
    \end{cases}
    \]
\end{definition}

\begin{definition}[Jump discontinuity]
    Both $L ^ {+}$ and $L ^ {-}$ exist but $L ^ {+} \neq L ^ {-}$.
\end{definition}

\begin{definition}[Infinite discontinuity]
    In this case at least one of $L ^ {+}$ or $L ^ {-}$ does not exist.
\end{definition}

\begin{theorem}[Intermediate Value Theorem]
    If $f(x)$ is continuous on $[a, b]$ and $u$ is any number between $f(a)$ and $f(b)$ then $\exists c \in (a, b)$ such that $f(c) = u$.
\end{theorem}

\newpage

\section{Differentiation}

\begin{definition}[The derivative]
    Given a function $f(x)$.
    $f(x)$ is differentiable at $x = a$ if
    \[
    f'(a) = \lim_{h \rightarrow 0}\frac{f(a + h) - f(a)}{h}
    \]
    exists.
\end{definition}

\begin{proposition}[The Leibniz rule]
    \[
    D ^ n(fg) = \sum_{k = 0}^{n}\binom{n}{k}(D ^ kf)(D ^ {n - k}g).
    \]
\end{proposition}

\begin{theorem}[Chain rule theorem]
    If $g(x)$ is differentiable at $x$ and $f(x)$ is differentiable at $g(x)$ then the composition $(f \circ g)(x)$ is differentiable at $x$ with
    \[
    (f \circ g)'(x) = f'(g(x))g'(x).
    \]
    Using the Leibniz notation we have
    \[
    \frac{d}{dx}f(g(x)) = \frac{df}{dg}\frac{dg}{dx}.
    \]
\end{theorem}

\begin{theorem}[L'H\^opital's rule]
    Let $f(x)$ and $g(x)$ be differentiable on $I = (a - h, a) \cup (a, a + h)$ for some $h > 0$,
    with $\lim_{x \rightarrow a}f(x) = \lim_{x \rightarrow a}g(x) = 0$.

    If $\lim_{x \rightarrow a}\frac{f'(x)}{g(x)}$ exists and $g'(x) \neq 0\ \forall x \in I$ then
    \[
    \lim_{x \rightarrow a}\frac{f(x)}{g(x)} = \lim_{x \rightarrow a}\frac{f'(x)}{g'(x)}.
    \]
\end{theorem}

For a function $f(x)$ defined in some interval $I$.
\begin{definition}[Global maximum]
    If there exists a constant $k_1$such that $f(x) \leq k_1\ \forall x \in I$ we say that $f(x)$ is bounded above in $I$ and we call $k_1$ an upper bound of $f(x)$ in $I$.

    If there exists a point $x_1$ in $I$ such that $f(x_1) = k_1$ we say that the upper bound of $k_1$ is attained and we call $k_1$ the global maximum value of $f(x)$ in $I$.
    
\end{definition}









\end{document}