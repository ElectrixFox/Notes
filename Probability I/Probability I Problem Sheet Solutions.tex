\documentclass[10pt, a4paper]{article}
\usepackage{preamble}

\declaretheorem[style = avgstyle, name = Problem]{problem}

\title{Probability I Problem Sheet Solutions}
\author{Luke Phillips}
\date{October 2024}

\begin{document}

\maketitle

\newpage

\section{Introduction}

\begin{problem}[35]
    Find the probability that in a family of six children, the second child is a girl but they are not all girls (assuming all outcomes are equally likely)
    
    \begin{proof}[Solution]\renewcommand{\qedsymbol}{}
        Let $\P(A)$ be the probability that all children are girls and let $\P(B)$ be the probability that the second child is a girl.

        Here we can see that there are $2 ^ 6$ different possibilities hence we can apply classical probability and the definition of equally likely outcomes to see that
        \[
        \P{B} = \frac{1}{2 ^ 6} \cdot 1 = \frac{1}{2 ^ 6}\quad\text{since there is only one possibility that they are all girls.}
        \]
        Similarly
        \[
        \P{A} = \frac{1}{2 ^ 6} \cdot (2 ^ 6 - 2 ^ 5) = \frac{2 ^ 6 - 2 ^  5}{2 ^ 6}\quad\text{since there are $2 ^ 6 - 2 ^ 5$ number of possibilities where the second child is a girl.}
        \]
        Using these facts we can then apply C1 to obtain an expression for the probability of the event that the second child is a girl but they are not all girls
        \begin{align*}
            \P(A \setminus B) &= \P(A) - \P(A \cap B) \\
            &= \frac{2 ^ 6 - 2 ^ 5}{2 ^ 6} - \frac{1}{2 ^ 6} \cdot \frac{2 ^ 6 - 2 ^ 5}{2 ^ 6} \\
            &= \frac{1}{2}\left(1 - \frac{1}{2 ^ 6}\right) \\
            &= \frac{2 ^ 6 - 1}{2 ^ 7}.
        \end{align*}
        This gives us the final probability of $\dfrac{2 ^ 6 - 1}{2 ^ 7}$.
    \end{proof}
\end{problem}

\begin{problem}[68]
    Show that if $A$ and $B$ are independent,
    then $A$ and $B ^ c$ are also independent.
    \begin{proof}
        We know that
        \begin{equation}
            \P(A \cap B) = \P(A)\P(B).
        \end{equation}

        We have to show that $\P(A \cap B ^ c) = \P(A)\P(B ^ c)$.

        Observe that $A = (A \cap B) \cup (A \cap B ^ c)$,
        so $\P(A) \overset{A3}{=} \P(A \cap B) + \P(A \cap B ^ c)$,
        since $(A \cap B) \cap (A \cap B ^ c) = \emptyset$

        Using (1), we get
        \[
        \P(A) = \P(A)\P(B) + \P(A \cap B ^ c)
        \]
        or,
        \begin{align*}
            \P(A \cap B ^ c) &= \P(A) - \P(A)\P(B) \\
            &= \P(A)\underbrace{\left[1 - \P(B)\right]}_{= \P(B ^ c)} \\
            &= \P(A)\P(B ^ c).
        \end{align*}
    \end{proof}
\end{problem}

\begin{problem}[93]
    Let $X \sim \Bin(n, p)$,
    and let $Y = n - X$
    (i.e. if $X$ counts the number of successes in a binomial scenario,
    then Y counts the number of failures in that same scenario).
    Show that $Y \sim \Bin(n, 1 - p)$.
    \begin{proof}
        \[
        p_X(x) = \begin{cases}
            \binom{n}{x}p ^ x (1 - p) ^ {n - x}, & x = 0, 1, \dotsc, n, \\
            0,&\text{elsewhere}.
        \end{cases}
        \]
        Intuition tells us that $Y \sim \Bin(n, 1 - p)$.
        Let $p_Y$ denote the pmf for $Y$.
        Then
        \begin{align*}
            p_Y(y) &= \P(Y = y) \\
            &= \P(n - X = y) \\
            &= \P(X = n - y) \\
            &= p_X(n - y) \\
            &= \binom{n}{n - y}p ^ {n - y}(1 - p) ^ y,\text{ if } n - y \in \{0, 1, \dotsc, n\}
        \end{align*}
        Recall the identity that we discussed in chapter two
        \[
        \binom{n}{x} = \binom{n}{n - x},
        \]
        using this,
        we get
        \[
        p_Y(y) =
        \begin{cases}
            \binom{n}{y}(1 - p) ^ y p ^ {n - y},& y \in \{0, 1, \dotsc, n\}, \\
            0, &\text{elsewhere}
        \end{cases}
        \]
        which shows that $Y \sim \Bin(n, 1 - p)$.
    \end{proof}
\end{problem}

\begin{problem}[97]
    A boxer,
    at the start of his career,
    decides that he will retire after his first defeat.
    In each fight he has
    probability q of being defeated,
    with successive fights being assumed independent.
    Let $X$ be the total number of fights in the boxer’s career.
    \begin{enumerate}[label = (\alph*)]
        \item What is the distribution of $X$?
        \item Show that $\P(X > x) = (1 - q) ^ x$ for $x \in \{0, 1, 2, 3, \dotsc\}$.
        Interpret this result.
        \item Find a formula for the probability that $X$ is even
        (simplify your formula as much as possible).
        Evaluate this probability when $q = 0.4$.
    \end{enumerate}

    \begin{proof}[Solution]\renewcommand{\qedsymbol}{}\phantom{}
    \begin{enumerate}[label = (\alph*)]
        \item
        Let $X := \text{number of fights in boxer's career}$.
        Find the distribution for $X$.
        \[
        \Geo(q).
        \]
        \[
        \P(X = x) = \P(\text{the boxer wins the first $x - 1$ fights and losses the last,
        that is,
        $x$-th fight})
        \]
        \[
        \P(X = x) = \begin{cases}
            (1 - q) ^ {x - 1}q, &x = 1, 2, \dotsc \\
            0, &\text{elsewhere}
        \end{cases}.
        \]
        Hence $X \sim \Geo(q)$.
        \item Find $\P(X > x)$
        \begin{align*}
            \P(X > x) &= \sum_{t = x + 1}^{\infty}\P(X = t) \\
            &= \sum_{t = x + 1}^{\infty}q(q - 1) ^ {t - 1} \\
            &= q\sum_{t = x + 1}^{\infty}(1 - q) ^ {t - 1}.
        \end{align*}
        Apply a change of variables,
        $t - 1 = y$,
        \begin{align*}
            &= q \sum_{y = x}^{\infty}(1 - q) ^ y \\
            &= q(1 - q) ^ x \sum_{m = 0}^{\infty}(1 - q) ^ m \\
            &= q(1 - q) ^ x \frac{1}{1 - (1 - q)} \\
            &= (1 - q) ^ x.
        \end{align*}
    \end{enumerate}
    \end{proof}
\end{problem}

\begin{problem}[100]
    Let $X \sum \Po(\lambda)$ where $\lambda > 1$.
    Show that $p(x)$,
    as a function of $x$,
    increases monotonically,
    and then decreases.
    For what $x \in \Z_+$ is $p(x)$ maximal?
    \begin{proof}
        Recall : $X \sim \Po(\lambda)$,
        \[
        p(x) = \frac{e ^ {-\lambda}\lambda ^ x}{x!},\quad x \in \{0, 1, \dotsc\}
        \]
        \begin{align*}
            \frac{p(x + 1)}{p(x)} &= \frac{\lambda ^ {x + 1} e ^ {-\lambda}}{(x + 1)!} \frac{x!}{e ^ {-\lambda} \lambda ^ x} \\
            &= \frac{\lambda}{(x + 1)} \\
            &> 1 \iff \lambda > (x + 1)
        \end{align*}
        or $x < \lambda - 1$ and
        \[
        \frac{p(x + 1)}{p(x)} < 1 \iff \frac{\lambda}{(x + 1)} < 1
        \]
        that is,
        $x > (\lambda - 1)$
        Because $p(x)$ is monotonically increasing
        \[
        \frac{p(x + 1)}{p(x)} > 1
        \]
        $p(x)$ is increasing when $x < \lambda - 1$,
        and $p(x)$ is decreasing when $x > \lambda - 1$.
        This can be rephrased to say that $p(x)$ is increasing up to $x = \lfloor\lambda\rfloor$,
        and then $p(x)$ is decreasing,
        with $p(x)$ achieving the maximum value at $x = \lfloor\lambda\rfloor$
    \end{proof}
\end{problem}

\begin{problem}[125]
    Suppose that $X$ and $Y$ has joint probability density function
    \[
    f(x, y) = \begin{cases}
        6e ^ {-(2x + 3y)} & \text{if } x \geq 0 \text{ and } y \geq 0 \\
        0 & \text{otherwise}.
    \end{cases}
    \]
    By integrating the joint probability density function,
    calculate:
    \begin{enumerate}[label = (\roman*)]
        \item $\P\left(X < \frac{1}{2}, Y > \frac{1}{2}\right)$, and
        \item $\P(X > Y)$.
    \end{enumerate}
\end{problem}

\begin{problem}[137]
    Continuing from Exercise $130$,
    find $\E\left(X ^ 2\right)$ and $\E\left(\frac{1}{X}\right)$.
    \begin{proof}[Solution]\renewcommand{\qedsymbol}{}
        \begin{align*}
            \E(X ^ 2) &= \int_{-\infty}^{\infty}x ^ 2f(x)\,dx \\
            &= \int_{0}^{1}x ^ 2f(x)\,dx \\
            &= 3\int_{0}^{1}x ^ 4\,dx \\
            &= 3\left[\frac{x ^ 5}{5}\right]_{x = 0}^{1} \\
            &= \frac{3}{5}.
        \end{align*}
        \begin{align*}
            \E\left(\frac{1}{X}\right) &= \int_{0}^{1}\frac{1}{x}f(x)\,dx \\
            &= \int_{0}^{1}\frac{3x ^ 2}{x}\,dx \\
            &= \int_{0}^{1}3x\,dx \\
            &= \left[\frac{3}{2}x ^ 2\right]_{x = 0}^{1} \\
            &= \frac{3}{2}. \\
        \end{align*}
    \end{proof}
\end{problem}

\begin{problem}
    Let $X \sim \Geo(q)$ for $0 < q \leq 1$.
    Show that $\E(X(X - 1)) = 2\frac{(1 - q)}{q ^ 2}$.

    \begin{proof}[Solution]\renewcommand{\qedsymbol}{}
        \[
        p(x) = (1 - q) ^ xq,\,x = 1, 2, \dotsc.
        \]
        $X$ is the number of trials before the first success.
        \begin{align*}
            \E(X(X - 1)) &= \infsum[x = 1]x(x - 1)p(x) \\
            &= \infsum[x = 2]x(x - 1)(1 - q) ^ {x - 1}q \\
            &= q(q - 1)\infsum[x = 2]x(x - 1)(1 - q) ^ {x - 2}
        \end{align*}
        Let us for simplicity of notation write $(1 - q) = r$,
        then,
        we have to find
        \[
        \infsum[x = 2]x(x - 1)r ^ {x - 2} = \frac{d ^ 2}{dr ^ 2}[(1 - r) ^ {-1}].
        \]
        From Taylor's expansion
        \[
        (1 - r) ^ {-1} = \infsum[x = 0]r ^ x
        \]
        \[
        \frac{d}{dx}(1 - r) ^ {-1} = \infsum[x = 0]\frac{d}{dx}r ^ x = \infsum[x = 1]xr ^ {x - 1}
        \]
        or,
        \[
        \frac{d ^ 2}{dx ^ 2} = \infsum[x = 1]\frac{d}{dx}xr ^ {x - 1} = \infsum[x = 2]x(x - 1)r ^ {x - 2}.
        \]
    \end{proof}
\end{problem}

\newpage

$\nearrow$: Monotonically increasing.

\end{document}