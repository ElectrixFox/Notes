\documentclass[10pt, a4paper]{article}
\usepackage{preamble}

\title{Probability I}
\author{Luke Phillips}
\date{October 2024}

\begin{document}

\maketitle

\newpage

\section{Introduction}

Probability is based on axioms


Events are based on sets

\begin{definition}
    A set $S$ is a\footnote{Unordered} collection of elements or objects.    
\end{definition}
\begin{example}
    Let $S$ be the set of all even integers from $1$ to $9$.
    \[
    S =  \{2, 4, 6, 8\} = \{4, 6, 2, 8\}.
    \]
\end{example}

$\Z$ is the set of all integers.

$\N$ is the set of all natural numbers.\footnote{Positive integers}

$\N$ can also be defined as follows
\[
\N = \{x \in \Z^+: x > 0\}.
\]

$\varnothing$ denotes the empty set.

\begin{definition}
    Let $A, B$ be sets. $A \subseteq B \iff x \in A \implies x \in B$.

    A strict subset $A \subset B$ means that there is an $x \in B$ such that $x \notin A$ 
\end{definition}

\begin{definition}
    Let $A, B$ be sets. $A = B \iff A\subseteq B \And B \subseteq A$
\end{definition}

\subsection{Basic set operations}

\begin{definition}
    Let $A, B$ be sets. The union of $A, B$ is denoted by $A \cup B$. The union of two sets is the set of elements in $A$ or $B$ or both.
    \[
    A \cup B = \{x: x \in A \text{ or } x \in B \text{ or } x \in A\And B\}.
    \]
\end{definition}

\begin{definition}
    Let $A, B$ be sets. The intersection of $A, B$ is denoted by $A \cap B$. The intersection of two sets is the collection of elements that are in both $A$ and $B$.
    \[
    A \cap B = \{x: x \in A \And x \in B\}.
    \]
\end{definition}

\begin{definition}
    Let $A$ be a set. The complement of a set $A$ is denoted by $A ^ c$. A complement is the collection of elements not in $A$.
    \[
    A ^ c = \{x: x \notin A\}.
    \]
\end{definition}

\begin{example}
    Let $S :=$ collection of integers from $1$ to $9$ 
    $S = \{x \in \N: 1 \leq x \leq 9\}$
    
    Let $A :=$ collection of even integers from $1$ to $9$ 
    $A = \{x \in \N, x \text{ is even}: 1 \leq x \leq 9\}$

    $A \subseteq S\quad A \subseteq \N\quad S\subseteq\N$

    Let $B :=$ collection of odd integers from $1$ to $9$ 
    $B = \{x \in \N, x \text{ is odd}: 1 \leq x \leq 9\}$

    $A \cap B = \varnothing$,\qquad $A ^ c = \{x \in \N: 1\leq x \leq 9, x \notin A\}$

    \begin{align*}
    D &:= \text{All integers in } S \text{ that are divisible by 3} \\
    &= \{3, 6, 9\}
    \end{align*}
    $D \cap B = \{3, 9\}\quad D\cap A = \{6\}$.
\end{example}

\subsection{Cardinality}
\begin{definition}[Cardinality]
    Let $A$ be a finite set. Cardinality or the size of $A$ denoted by
    \[
    |A|
    \]
    is the number of elements it contains.
\end{definition}

If $|A| = m$, $A = \{x_1, x_2, \dots, x_n\}$\footnote{List or sequence or length $m$.}.

\textbf{Types of infinity}:

There are two types of infinity. Countable infinities implies that we can write it as a sequence of possibly infinite length. Uncountable infinities, for example $\R$ cannot be written as a sequence.

$\N = \{1, 2, 3, \dots\}$ can be written as a list.

$[a, b] = \{x \in \R : a \leq x \leq b\}$ for example take $a = 1$ and $b = 2$ and you can keep taking the midpoint between any two numbers infinitely and not find all numbers.

When you have a finite number of sets
\[
A_1, \dots A_n, \quad\text{abbreviation: } \bigcup_{i = 1}^{n}A_i = A_1 \cup A_2 \cup \dotsi \cup A_n\footnote{Finite union.}
\]
Similarly
\[
\bigcap_{i = 1}^{n}A_i = A_1 \cap A_2 \cap \dotsi \cap A_n\footnote{Finite intersection.}
\]

\begin{definition}[Power set]
    For a set $A$, its power set denoted by $2 ^ A$\footnote{or $\mathcal{P}(A)$} is the set containing all subsets of $A$.
    \[
    2 ^ A = \{B : B \subseteq A\}.
    \]
    $\emptyset,\,A \in 2 ^ A$.
\end{definition}


\subsection{Axiomatic Probability}
\begin{enumerate}[label = (\roman*)]
    \item Sample, space \& events

    Throw a  six-sided die, and note the score, that can be a number from $1$ to $6$.

    Outcomes are the score, which we can write as $1, 2, 3, 4, 5, 6$.

    The outcomes are collected in the sample space
    \begin{definition}[Sample space]
        The collection of all outcomes is called the sample space, denoted by $\Omega$.
        \[
        \Omega := \text{ collection of all possible outcomes}.
        \]
        and $\omega \in \Omega$, we say $\omega$ is an outcome.
    \end{definition}
    \begin{example}\phantom{}
    
        Throw a die, $\Omega = \{1, 2, 3, 4, 5, 6\}$.\footnote{$|\Omega| = 6$.}

        $6$ is an outcome.
    \end{example}

    In many examples, $\Omega$ will be finite or countably infinite

    \begin{definition}[Event]
        An event $A$ is a subset of the sample space $\Omega$.
    \end{definition}
    \begin{example} \phantom{}
    
        $A := \text{Even score} = \{2, 4, 6\} \subseteq \Omega$.

        $\emptyset := \text{Impossible event}.$ 
    \end{example}

    \begin{definition}[Probability]
        Associated to $\Omega$, there is a collection denoted by $\mathcal{F}$, which is the collection of all possible events.

        \[
        \mathcal{F} = \{A : A \subseteq \Omega\}.
        \]
        Discrete (for finite $\Omega$) we will take $\mathcal{F} = 2 ^ \Omega$.

        Probability is a measure of randomness that is associated that we assign to events $A \in \mathcal{F}$.
    \end{definition}
\end{enumerate}

\end{document}