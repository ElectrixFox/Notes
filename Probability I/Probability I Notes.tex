\documentclass[10pt, a4paper]{article}
\usepackage{preamble}

\declaretheorem[style = avgstyle, name = Counting Principle]{countprinc}

\title{Probability I}
\author{Luke Phillips}
\date{October 2024}

\begin{document}

\maketitle

\newpage

\section{Introduction}

Probability is based on axioms


Events are based on sets

\begin{definition}
    A set $S$ is a\footnote{Unordered} collection of elements or objects.    
\end{definition}
\begin{example}
    Let $S$ be the set of all even integers from $1$ to $9$.
    \[
    S =  \{2, 4, 6, 8\} = \{4, 6, 2, 8\}.
    \]
\end{example}

$\Z$ is the set of all integers.

$\N$ is the set of all natural numbers.\footnote{Positive integers}

$\N$ can also be defined as follows
\[
\N = \{x \in \Z^+: x > 0\}.
\]

$\varnothing$ denotes the empty set.

\begin{definition}
    Let $A, B$ be sets. $A \subseteq B \iff x \in A \implies x \in B$.

    A strict subset $A \subset B$ means that there is an $x \in B$ such that $x \notin A$ 
\end{definition}

\begin{definition}
    Let $A, B$ be sets. $A = B \iff A\subseteq B \And B \subseteq A$
\end{definition}

\subsection{Basic set operations}

\begin{definition}
    Let $A, B$ be sets. The union of $A, B$ is denoted by $A \cup B$. The union of two sets is the set of elements in $A$ or $B$ or both.
    \[
    A \cup B = \{x: x \in A \text{ or } x \in B \text{ or } x \in A\And B\}.
    \]
\end{definition}

\begin{definition}
    Let $A, B$ be sets. The intersection of $A, B$ is denoted by $A \cap B$. The intersection of two sets is the collection of elements that are in both $A$ and $B$.
    \[
    A \cap B = \{x: x \in A \And x \in B\}.
    \]
\end{definition}

\begin{definition}
    Let $A$ be a set. The complement of a set $A$ is denoted by $A ^ c$. A complement is the collection of elements not in $A$.
    \[
    A ^ c = \{x: x \notin A\}.
    \]
\end{definition}

\begin{example}
    Let $S :=$ collection of integers from $1$ to $9$ 
    $S = \{x \in \N: 1 \leq x \leq 9\}$
    
    Let $A :=$ collection of even integers from $1$ to $9$ 
    $A = \{x \in \N, x \text{ is even}: 1 \leq x \leq 9\}$

    $A \subseteq S\quad A \subseteq \N\quad S\subseteq\N$

    Let $B :=$ collection of odd integers from $1$ to $9$ 
    $B = \{x \in \N, x \text{ is odd}: 1 \leq x \leq 9\}$

    $A \cap B = \varnothing$,\qquad $A ^ c = \{x \in \N: 1\leq x \leq 9, x \notin A\}$

    \begin{align*}
    D &:= \text{All integers in } S \text{ that are divisible by 3} \\
    &= \{3, 6, 9\}
    \end{align*}
    $D \cap B = \{3, 9\}\quad D\cap A = \{6\}$.
\end{example}

\subsection{Cardinality}
\begin{definition}[Cardinality]
    Let $A$ be a finite set. Cardinality or the size of $A$ denoted by
    \[
    |A|
    \]
    is the number of elements it contains.
\end{definition}

If $|A| = m$, $A = \{x_1, x_2, \dots, x_n\}$\footnote{List or sequence or length $m$.}.

\textbf{Types of infinity}:

There are two types of infinity. Countable infinities implies that we can write it as a sequence of possibly infinite length. Uncountable infinities, for example $\R$ cannot be written as a sequence.

$\N = \{1, 2, 3, \dots\}$ can be written as a list.

$[a, b] = \{x \in \R : a \leq x \leq b\}$ for example take $a = 1$ and $b = 2$ and you can keep taking the midpoint between any two numbers infinitely and not find all numbers.

When you have a finite number of sets
\[
A_1, \dots A_n, \quad\text{abbreviation: } \bigcup_{i = 1}^{n}A_i = A_1 \cup A_2 \cup \dotsi \cup A_n\footnote{Finite union.}
\]
Similarly
\[
\bigcap_{i = 1}^{n}A_i = A_1 \cap A_2 \cap \dotsi \cap A_n\footnote{Finite intersection.}
\]

\begin{definition}[Power set]
    For a set $A$, its power set denoted by $2 ^ A$\footnote{or $\mathcal{P}(A)$} is the set containing all subsets of $A$.
    \[
    2 ^ A = \{B : B \subseteq A\}.
    \]
    $\emptyset,\,A \in 2 ^ A$.
\end{definition}


\subsection{Axiomatic Probability}
\begin{enumerate}[label = (\roman*)]
    \item Sample, space \& events

    Throw a  six-sided die, and note the score, that can be a number from $1$ to $6$.

    Outcomes are the score, which we can write as $1, 2, 3, 4, 5, 6$.

    The outcomes are collected in the sample space
    \begin{definition}[Sample space]
        The collection of all outcomes is called the sample space, denoted by $\Omega$.
        \[
        \Omega := \text{ collection of all possible outcomes}.
        \]
        and $\omega \in \Omega$, we say $\omega$\footnote{Also known as a sample point.} is an outcome.
    \end{definition}
    \begin{example}\phantom{}
    
        Throw a die, $\Omega = \{1, 2, 3, 4, 5, 6\}$.\footnote{$|\Omega| = 6$.}

        $6$ is an outcome.
    \end{example}

    In many examples, $\Omega$ will be finite or countably infinite

    \begin{definition}[Event]
        An event $A$ is a subset of the sample space $\Omega$.
    \end{definition}
    \begin{example} \phantom{}
    
        $A := \text{Even score} = \{2, 4, 6\} \subseteq \Omega$.

        $\emptyset := \text{Impossible event}.$ 
    \end{example}

    \begin{definition}[Probability]
        Associated to $\Omega$, there is a collection denoted by $\mathcal{F}$, which is the collection of all possible events.
        \[
        \mathcal{F} = \{A : A \subseteq \Omega\}.
        \]
        Discrete (for finite $\Omega$) we will take $\mathcal{F} = 2 ^ \Omega$.

        Probability is a measure of randomness that is associated that we assign to events $A \in \mathcal{F}$.
    \end{definition}
\end{enumerate}

Set operation is the same as event calculus.

\newpage

\begin{table}[h!]
    \centering
    \begin{tabular}{|c|c|c|c|}
        \hline
        Notations & Set theory language & Prob analogue & Meaning as events \\ [0.1em]
        \hline
        $A \cup B$ & $A$ union $B$ & $A$ or $B$ & $A$ occurs or $B$ occurs or both occur \\
        \hline
        $A \cap B$ & $A$ intersect $B$ & $A$ and $B$ & both $A$ and $B$ occur \\
        \hline
        $A^c$ & $A$ complement & not $A$ & $A$ does not occur \\
        $A \setminus B$ & $A$ set minus $B$ & $A$ but not $B$ & $A$ occurs but $B$ does not \\
        \hline
        $A \subseteq B$ & $A$ subset $B$ & $A$ implies $B$ & if $A$ occurs then $B$ also occurs \\
        \hline
        $\emptyset$ & null set & impossible event & \phantom{1} \\
        \hline
    \end{tabular}
    \caption{Set operations in event calculus}
\end{table}

$A \cap B = \emptyset$, $A$ and $B$ are mutually exclusive.

In order of the event calculus table here are the event notations for each thing. 

\begin{itemize}
    \item $A \cup B = \{\omega : \omega \in A \text{ or } \omega \in B\}$
    \item $A \cap B = \{\omega : \omega \in A \text{ and } \omega \in B\}$
    \item $A \setminus B = \{\omega : \omega \in A \text{ but } \omega \notin B\}$
    \item $A \subseteq B \text{ if } \omega \in A \implies \omega \in B$
\end{itemize}

\newpage

\textbf{De-Morgan's Laws}

\begin{enumerate}[label = (\roman*)]
    \item
    \[
    \left(\bigcup_{i = 1}^{n} A_i\right) ^ c = \bigcap_{i = 1}^{n} A_i ^ c
    \]
    \item
    \[
    \left(\bigcap_{i = 1}^{n} A_i\right) ^ c = \bigcup_{i = 1}^{n} A_i ^ c
    \]
\end{enumerate}

Non-examined
\begin{enumerate}[label = (\roman*)]
    \item
    \[
    \left(\bigcup_{n = 1}^{\infty} A_n\right) ^ c = \bigcap_{n = 1}^{\infty} A_n ^ c
    \]
    \item
    \[
    \left(\bigcap_{n = 1}^{\infty} A_n\right) ^ c = \bigcup_{n = 1}^{\infty} A_n ^ c
    \]
\end{enumerate}

\begin{definition}
    A probability "$\mathbb{P}$" on a sample space $\Omega$, with $\mathcal{F}$ (collection of events) is a mapping such that for every $A$ in $\mathcal{F}$, $\mathbb{P}(A)$ is a real number satisfying.
    \begin{enumerate}[label = A\arabic*]
        \item $\mathbb{P}(A) \geq 0$ (Prob is always non-negative)
        \item $\mathbb{P}(\Omega) = 1$ (Sample space has full probability)
        \item If $A, B \in \mathcal{F}$, and $A$ and $B$ are disjoint\footnote{$A$ and $B$ are mutually exclusive.}, that is $A \cap B = \emptyset$, then \[\mathbb{P}(A \cup B) = \mathbb{P}(A) + \mathbb{P}(B)\]
        (finite additivity)
        \item (Countable additivity). $A_1, A_2, \dots$ are pairwise disjoint, i.e. $A_i \cap A_j = \emptyset$, for all $i \neq j$, then
        \[
        \mathbb{P}\left(\bigcup_{n = 1}^{\infty}A_n\right) = \sum_{n = 1}^{\infty}\mathbb{P}(A_n).
        \]
    \end{enumerate}
\end{definition}

$(\Omega, \mathcal{F}, \mathbb{P})$, this triplet is known as the probability space.

\begin{example}
    $\Omega = \{\omega_1, \omega_2, \dots, \omega_m\}$, let $\omega_i$'s $1 \leq i \leq m$ be distinct. $A\subseteq \Omega,\ \mathbb{P}(A)$. $A = \{\omega_1, \omega_2, \omega_3\}$.
    \[
    \{\omega_1\} \subseteq \Omega, \{\omega_2\} \subseteq \Omega, \{\omega_3\} \subseteq \Omega.
    \]
    Therefore by (A3)
    \[
    \mathbb{P}(A) = \mathbb{P}(\{\omega_1\} \cup \{\omega_2\} \cup \{\omega_3\}) = \sum_{i = 1}^{3}\mathbb{P}(\{\omega_i\}).
    \]
\end{example}

\textbf{Equally likely outcomes}

\[
\mathbb{P}(\{\omega_i\}) = p_i \geq 0,\quad 1 \leq i \leq m,
\]
such that $\sum_{i = 1}^{m}p_i = 1$. $\Omega = \bigcup_{i = 1}^{m}\{\omega_i\}$, therefore $\mathbb{P}(\Omega) =\footnote{From A3.} \sum_{i = 1}^{m}\mathbb{P}(\{\omega_i\}) = 1\footnote{From A2.}$

Then $A \subseteq \Omega$,
\[
\mathbb{P}(A) = \sum_{i : \omega_i \in A}p_i
\]

Equally likely outcomes it means that $p_i$ are all equal and in this example $p_1 = p_2 = \dotsi = p_m = \frac{1}{m}$

\begin{example}
    Throw a fair die

    $\Omega = \{1, 2, \dots, 6\}$

    Fair = equally likely
    \[
    \mathbb{P}(\{1\}) = \mathbb{P}(\{2\}) = \dotsi = \mathbb{P}(\{6\}) = \frac{1}{6}
    \]
    $A = \{\text{even score}\} = \{2, 4, 6\}$
    \[
    \mathbb{P}(A) = \sum_{i : \omega_i \in A}p_i = \frac{1}{6}|A| = \frac{3}{6} = \frac{1}{2}.
    \]
\end{example}


\begin{definition}[Partition]
    $E_1, E_2,\dotsc,E_k \in \mathcal{F}$ for a finite partition of the sample space $\Omega$, if
    \begin{enumerate}[label = (\roman*)]
        \item $\mathbb{P}(E_i) > 0$ for all $1 \leq i \leq k$,
        \item $E_i\,\&\,E_j$ are pairwise disjoint, i.e. $\forall i \neq j,\,E_i \cap E_j = \emptyset$,
        \item $\displaystyle\Omega = \bigcup_{i = 1}^{k}E_i$.
    \end{enumerate}
\end{definition}
Easily, we can see that
\[
\sum_{i = 1}^{k} \mathbb{P}(E_i) = 1.
\]

\[
\mathbb{P}(\Omega) = \bigcup_{i = 1}^{k}(E_i) \implies 1 = \sum_{i = 1}^{k}\mathbb{P}(E_i)\qquad\text{by A3}
\]


\subsubsection{Sigma algebra}
\begin{definition}
    $\mathcal{F}$, a collection of subsets of $\Omega$ is called a $\sigma$-algebra, if it satisfies
    \begin{enumerate}[label = (S\arabic*)]
        \item $\Omega \in \mathcal{F}$
        \item $A \in \mathcal{F}$, then $A ^ c \in \mathcal{F}$.
        \item if $\underbrace{A_1,\,A_2,\,\dotsc}_{\text{countably infinite}} \in \mathcal{F}$, then $\bigcup_{n = 1}^{\infty}A_n \in \mathcal{F}$.
    \end{enumerate}
\end{definition}

\subsection{Consequences of (A1-A3)}

\begin{proposition} \phantom{} \\
    \begin{enumerate}[label = (C\arabic*)]
        \item $A, B \in \mathcal{F}$, then $\mathbb{P}(B \setminus A) = \mathbb{P}(B) - \mathbb{P}(A \cap B)$.
        \item $A \in \mathcal{F}$, $\mathbb{P}(A ^ c) = 1 - \mathbb{P}(A)$ (A2 + A3)
        \item $\mathbb{P}(\emptyset) = 0$
        \item $\mathbb{P}(A) \leq 1,\quad\text{for } A \in \mathcal{F}$ (Hint: use C2)
        \item Monotonicity: $A \subseteq B$
        \[
        \mathbb{P}(A) \leq \mathbb{P}(B)
        \]
        \item $\mathbb{P}(A \cup B) = \mathbb{P}(A) + \mathbb{P}(B) - \mathbb{P}(A \cap B)$
        \item Finite additivity: $A_1, A_2,\dotsc, A_k$ pairwise mutually exclusive
        \[
        \mathbb{P}\left(\bigcup_{i = 1}^{k}\right) = \sum_{i = 1}^{k}\mathbb{P}(A)
        \]
        \item Boole's inequality (if they are not mutually exclusive):
        \[
        \mathbb{P}\left(\bigcup_{i = 1}^{n}A_i\right) \leq \sum_{i = 1}^{n}\mathbb{P}(A_i)
        \]
    \end{enumerate}
    \begin{proof}
        C2:
        $A \cup A ^ c = \Omega$\quad$A \cap A ^ C = \emptyset$

        $\mathbb{P}(A \cup A ^ c) = 1$ (from A2)
        (from A3)
        $\mathbb{P}(A) + \mathbb{P}(A ^ c) = 1$

        C3:
        $\mathbb{P}(\emptyset) = 1 - \mathbb{P}(\Omega) = 1 - 1 = 0$ (by A3)

        C4:
        From 1, $\mathbb{P}(A) + \mathbb{P}(A ^ c) = 1$
        
        $\mathbb{P}(A) \geq 0,\,\mathbb{P}(A ^ c) \geq 0$, hence $\mathbb{P}(A) \leq 1$
    \end{proof}
\end{proposition}

\section{Counting}

\subsection{Equally likely outcomes}
Finite sample space $\Omega = \{\omega_1,\omega_2,\dotsc, \omega_m\}$, $\mathbb{P}(\omega_1) = \mathbb{P}(\omega_2) = \mathbb{P}(\omega_m) = \frac{1}{m}$\footnote{$\mathbb{P}(\{\omega_i\}) = \mathbb{P}(\omega_i)$}, $A$, $\mathbb{P}(A) = \frac{|A|}{m} = \frac{|A|}{|\Omega|}$

\begin{countprinc}[Multiplication principle]
    $k$ choices in succession, where
    \begin{enumerate}[label = (\roman*)]
        \item $m_1$, possibility of first choice
        \item $m_2$, possibility of second choice \\ \vdots
        \item $m_k$, possibility of $k$th choice
    \end{enumerate}
    and at each stage, our choices are not affected by the past.

    The total number of distinct choices = $\underbrace{m_1 \times m_2 \times \dotsi \times m_k}_{k \text{ times}} = \prod_{i = 1}^{k}m_i$.
\end{countprinc}

\begin{example}
    $3$ choices for breakfast, $4$ for lunch and $5$ for dinner. The total number of choices is
    \[
    3 \times 4 \times 5.
    \]
\end{example}

\begin{example}
    Debit Card Pins
    \begin{enumerate}[label = (\roman*)]
        \item Cannot be the same digit at all $4$ places.
        \item Cannot be an increasing or decreasing sequence of consecutive.
    \end{enumerate}

    $A := \text{ Number of possible PINs}$
    
    $A^c := \text{ Number of not possible PINs}$

    The number of sequences satisfying (i) is $10$.

    The number of (ii) increasing consecutive sequences can start with $0,\, 1,\, 2,\,\dotsc,\, 6$ implies $7$ options. Similarly, decreasing consecutive sequences can start with $9,\,8,\,7,\,\dotsc,\,3$.

    The total number of choices satisfying (ii) is $14$.

    $|A ^ c| = 10 + 14 = 24$.

    The number of unrestricted choices $= 10 ^ 4$.

    $|A| = 10 ^ 4 - 24$.
\end{example}

\begin{countprinc}[Ordered choice of distinct objects with replacement]
    $m$ distinct objects, choose $r$ of them with replacement, then the number of different ordered lists (i.e. $r$-tuples) is $\underbrace{m \times m \times \dotsi \times m}_{r \text{ choices}} = m ^ r$.
\end{countprinc}

\begin{countprinc}[Ordered choices of distinct objects without replacement]
    $m$ distinct objects, choose $r$ ($r \leq m$) of them without replacement, then the number of such choices is denoted by $(m)_r$, where $(m)_r = m \times (m - 1) \times \dotsc \times (m - r + 1) = \frac{m!}{(m - r)!}$.
\end{countprinc}

\begin{example}[Birthday Problem]
    There are $n$ people, a year is $365$ days,
    \[
    B = \{\text{At least two people have the same birthday}\}
    \]
    Find $\mathbb{P}(B)$.

    $B ^ c = \{\text{No two people have the same birth day}\}$
\end{example}

\end{document}