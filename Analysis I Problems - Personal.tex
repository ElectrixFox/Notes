\documentclass[10pt, a4paper]{article}
\usepackage{preamble}

\newcommand{\limas}[3][n]{#2 \rightarrow #3 \text{ as } #1 \rightarrow \infty}
\newcommand{\infsum}[1][n = 1]{\sum_{#1}^{\infty}}
\newcommand{\seq}[2][x]{(#1_{#2})_{#2 \in \N}}

\title{Analysis Problems}
\author{Luke Phillips}
\date{September 2024}

\begin{document}

\section{Introduction}
The problems in this sheet are from the course given by Dirk Schuetz.

\section{Sequences}

\subsection{Basics about sequences and limits}

\begin{example}
    Use the definition of convergence to show that the following sequences converge to $0$.
    \begin{enumerate}[label = \alph*.]
        \item $x_n = \dfrac{1}{n ^ 2}$.
        \item $y_n = \dfrac{1}{\sqrt{n}}$.
    \end{enumerate}
    \begin{proof}
    Considering the sequences separately.
    \begin{enumerate}[label = \alph*.]
    \item
        Let $\epsilon > 0$ be given, we now must find a value $N$ such that for all $n > N$,
        \[
        \left|\frac{1}{n ^ 2} - 0\right| < \epsilon.
        \]
        Now we need an $N$ such that for any $n > N$, $\frac{1}{n ^ 2} < \epsilon$, i.e. $n > \frac{1}{\sqrt{\epsilon}}$. This solves the problem, $n > N = \frac{1}{\sqrt{\epsilon}}$.
    \item
        Let $\epsilon > 0$ be given, we now must find a value $N$ such that for all $n > N$,
        \[
        \left|\frac{1}{\sqrt{n}} - 0\right| < \epsilon.
        \]
        Now we need an $N$ such that for any $n > N$, $\frac{1}{\sqrt{n}} < \epsilon$, i.e. $n > \frac{1}{\epsilon ^ 2}$. This solves the problem, $n > N = \frac{1}{\epsilon ^ 2}$.
    \end{enumerate}
    \end{proof}
\end{example}

\begin{example}
    Let $(x_n)_{n \in \N}$ be a sequence of real numbers with $x_n \geq 0$ and $\limas{x_n}{0}$. Using the definition of convergence, show that $\limas{\sqrt{x_n}}{0}$.
    \begin{proof}
        Let $\epsilon > 0$ be given. To show that $\limas{\sqrt{x_n}}{0}$ we must find an $N$ such that for all $n \geq N$,
        \[
        |\sqrt{x_n} - 0| < \epsilon.
        \]
        Given that $\limas{x_n}{0}$ there exists an $N$ such that for all $n \geq N$,
        \[
        |x_n - 0| < \epsilon ^ 2.
        \]
        $x_n \geq 0$ implies that $|x_n| = x_n$ which means that $x_n < \epsilon$ and then $\sqrt{x_n} < \epsilon$ by simple deduction.
    \end{proof}
\end{example}

\begin{example}
    Let $X \subset \R$ and assume that $C \in \R$ is an upper bound for $X$. Show that $C = \sup X$ if and only if there exists a sequence $(x_n)_{n\in\N}$ with $x_n\in X$ for all $n$ and $\lim_{n\rightarrow\infty} x_n = C$.
    \begin{proof}
        $C = \sup X$ if $C$ is an upper bound of $X$ (which is assumed). Hence, we need to show that whenever $B \in \R$ is an upper bound of $X$, then $C \leq B$. Using the fact that $\lim_{n\rightarrow\infty} x_n = C$ this implies that given any $\epsilon > 0$ there exists an $N$ such that for all $n > N$,
        \[
        |x_n - C| < \epsilon.
        \]
        This implies that for all $n > N$, $C - \epsilon < x_n < C + \epsilon$ which since for all $n,\, x_n \in X$, $X \leq C + \epsilon$ we can take $B = C + \epsilon$ then it is trivial to show that our $B$ satisfies the condition for $C$ to be the supremum.
    \end{proof}
\end{example}

\subsection{Convergence criteria}

\begin{example}
    Calculate $\lim_{n \rightarrow \infty} x_n$ (or show that no limit exists) for 
    \[
    x_n = \frac{(-1) ^ n}{\sqrt{n ^ 2 + n}}.
    \]
    \begin{proof}
        For this we can apply the sandwich theorem and previous observations. \\
        Since 
        \[
        \limas{\frac{1}{\sqrt{n}}}{0}
        \]
        and
        \[
        \left|\frac{(-1) ^ n}{\sqrt{n ^ 2 + n}}\right| \leq \frac{1}{\sqrt{n}}
        \]
        then by the sandwich theorem we can state that $x_n$ converges to $0$.
    \end{proof}
\end{example}

\begin{example}
    Let $(x_n)_{n \in \N}$ and $(y_n)_{n \in \N}$ be two convergent sequences with limits $x$ and $y$, respectively. Let $a, b \in \R$. Show that $(ax_n + by_n)_{n \in \N}$ is also convergent with limit $ax + by$.
    \begin{proof}
        Let $\epsilon > 0$ be given, then obviously $\dfrac{\epsilon}{|a| + |b|} > 0$. Since $x_n$ and $y_n$ are convergent, we will assume that $ax_n$ and $by_n$ are convergent too. Working from our assumption we will say that there exists an $N$ such that for all $n \geq N$,
        \[
        |ax_n - ax| < \frac{\epsilon}{|a| + |b|}\text{and}\qquad|by_n - by| < \frac{\epsilon}{|a| + |b|}.
        \]
        Then we have
        \begin{align*}
            |(ax_n + by_n) - (ax + by)| &= |(ax_n - ax) + (by_n - by)| \\
            &\leq |ax_n - ax| + |by_n - by| \\
            &= |a||x_n - x| + |b||y_n - y| \\
            &< |a|\frac{\epsilon}{|a| + |b|} + |b|\frac{\epsilon}{|a| + |b|} \\
            &= \frac{\epsilon}{|a| + |b|}\left(|a| + |b|\right) = \epsilon.
        \end{align*}
        This concludes our proof since we have shown that as $\epsilon > 0$ becomes small we can find an $N$ such that for all $n \geq N$,
        \[
        |(ax_n + by_n) - (ax + by)| < \epsilon
        \]
        hence $(ax_n + by_n)_{n \in \N}$ converges to $ax + by$.
    \end{proof}
\end{example}

\subsection{The Bolzano-Weierstrass Theorem}

\begin{example}
    Let $x_n = \sqrt{n}(\sqrt{n}+(-1)^n\sqrt{n - 1})$. Show that this sequence has a convergent subsequence, and calculate its limit.
    \begin{proof}
        Take $x_{2n + 1}$ then
        \begin{align*}
        x_{2n + 1} &= \sqrt{2n + 1}(\sqrt{2n + 1} - \sqrt{2n}) \\
        &= \frac{\sqrt{2n + 1}}{\sqrt{2n + 1} + \sqrt{2n}}.
        \end{align*}
        Hence we can do some manipulation to see that,
        \begin{align*}
        \frac{\sqrt{2n + 1}}{\sqrt{2n + 1} + \sqrt{2n}} &= \frac{\sqrt{2n + 1}}{\sqrt{2n + 1}\left(1 + \sqrt{\frac{2n}{2n + 1}}\right)} \\
        &= \frac{1}{1 + \sqrt{\frac{2n}{2n + 1}}} \\
        &= \frac{1}{1 + \sqrt{\frac{2n + 1}{2n + 1} - \frac{1}{2n + 1}}} \\
        &= \frac{1}{1 + \sqrt{1 - \frac{1}{2n + 1}}}
        \end{align*}
        which after applying the sandwich theorem to $\frac{1}{2n + 1}$ to obtain $\limas{\frac{1}{2n + 1}}{0}$ we can clearly see the following,
        \[
        \lim_{n \rightarrow \infty}\frac{1}{1 + \sqrt{1 - \frac{1}{2n + 1}}} = \frac{1}{1 + \sqrt{1 - 0}} = \frac{1}{2}.
        \]
        Concluding that $\limas{x_{2n + 1}}{\dfrac{1}{2}}$.
    \end{proof}
\end{example}

\begin{example}
    Let $(a_n)_{n \in \N}$ and $(b_n)_{n \in \N}$ be bounded sequences, such that $(a_n \cdot b_n)_{n \in \N}$ converges to 0. Show that $(a_n)_{n \in \N}$ or $(b_n)_{n \in \N}$ has a subsequence which converges to $0$.
    \begin{proof}
        By the Bolzano-Weierstrass Theorem there is a convergent subsequence $(a_{n_i})_{i \in \N}$ which is convergent with limit $l$. If $l = 0$ then we have proven the statement. Otherwise, we can take $l \neq 0$ so $a_{n_i} \neq 0$ for all $i \in \N$ which means we can use $(b_{n_i})$ for our subsequence. But $\limas{a_n \cdot b_n}{0}$ therefore we can use COLT\footnote{Calculus of limit theorems} to see that,
        \[
        \lim_{n \rightarrow \infty}\frac{a_n \cdot b_n}{a_n} = \frac{l \cdot b_n}{l} = b_n.
        \]
        So $(b_{n_i})$ is the required subsequence
    \end{proof}
\end{example}

\subsection{Lim sup and lim inf}

\begin{example}
    Let $(x_n)_{n \in \N}$ be a bounded sequence. Show that $(x_n)_{n \in \N}$ is convergent if and only if $\liminf_{n \rightarrow \infty} x_n = \limsup_{n \rightarrow \infty} x_n.$
\end{example}


\subsection{Cauchy sequences}

\begin{example}
    Let $f : \R \rightarrow \R$ be a function with
    \[
    |f(x) - f(y)| \leq \frac{1}{2}|x - y|
    \]
    for all $x, y \in \R$.
    \begin{enumerate}[label = \alph*.]
        \item Pick any $x_1 \in \R$ and for $n \geq 2$ let $x_n = f(x_{n - 1})$. Show that $\seq{n}$ is a Cauchy sequence.
        \item Let $x = \lim_{n \rightarrow \infty} x_n$ with the sequence as in (a). Show that $f(x) = x$.
        \item Show that there is exactly one $a \in \R$ with $f(a) = a$.
    \end{enumerate}

    \begin{proof}
    \begin{enumerate}[label = \alph*.]
        \item 
        To start observe that we can have the following inequality for the function
        \begin{align*}
            |x_{n + 1} - x_n| &= |f(x_{n}) - f(x_{n - 1})| \\
            &\leq \frac{1}{2}|x_{n} - x_{n - 1}| \\
            &= \frac{1}{2}|f(x_{n - 1}) - f(x_{n - 2})| \\
            &\leq \frac{1}{4}|x_{n - 1} - x_{n - 2}| \\
            \vdots \\
            &\leq \frac{1}{2 ^ {n - 1}}|x_2 - x_1|.
        \end{align*}
        Now we have an inequality we can then say that for all $n, m$ where $m > n$ the following is true,
        \begin{align*}
            |x_m - x_n| &= |(x_m - x_{m - 1}) + (x_{m - 1} - x_{m - 2}) + \cdots + (x_{n - 1} - x_{n})| \\
            &\leq |x_m - x_{m - 1}| + |x_{m - 1} - x_{m - 2}| + \cdots + |x_{n - 1} - x_{n}| \\
            &\leq |x_2 - x_1|\frac{1}{2 ^ {m - 1}} + |x_2 - x_1|\frac{1}{2 ^ {m - 2}} + \cdots + |x_2 - x_1|\frac{1}{2 ^ {n - 1}} \\
            &= |x_2 - x_1|\frac{1}{2 ^ {n - 1}}\left(1 + \frac{1}{2} + \frac{1}{4} + \cdots + \frac{1}{2 ^ {m - n - 2}}\right) \\
            &= |x_2 - x_1|\frac{1}{2 ^ {n - 1}}\left(\frac{1 - \left(\frac{1}{2}\right) ^ {m - n - 2}}{1 - \frac{1}{2}}\right) \\
            &< |x_2 - x_1|\frac{1}{2 ^ {n - 1}} \cdot \frac{1}{1 - \frac{1}{2}} \\
            &= |x_2 - x_1|\frac{1}{2 ^ {n - 2}}.
        \end{align*}
        Let $\epsilon > 0$ be given. Therefore we can say that we can find an $N \in \N$ such that $\frac{|x_2 - x_1|}{2 ^ {N - 2}} < \epsilon$ for any $n, m \geq N$,
        \[
        |x_m - x_n| < \epsilon
        \]
        hence, $x_n$ is a Cauchy sequence.
    \end{enumerate}
    \end{proof}
\end{example}

\begin{example}
    Let $(y_n)_{n \in \N}$ be the Fibonacci sequence, that is, $y_2 = y_1 = 1$ and $y_n = y_{n - 1} + y_{n - 2}$ for $n \geq 3$. Let $x_n = \frac{y_{n + 1}}{y_n}$.
    \begin{enumerate}[label = \alph*.]
        \item Show that
        \[
        \frac{1}{y_n y_{n + 1}} \leq \frac{1}{y_n} - \frac{1}{y_{n + 1}}
        \]
        for all $n \geq 2$.
        \item Show that $y_{n + 2}y_n - y_{n + 1}^2 = (-1)^{n + 1}$ for all $n \in \N$ by induction.
        \item Show that $(x_n)_{n \in \N}$ is a Cauchy sequence.
        \item Calculate $\lim_{n \rightarrow \infty}x_n$.
    \end{enumerate}

    \begin{proof}
        \begin{enumerate}[label = \alph*.]
        \item
            \[
            \frac{1}{y_n y_{n + 1}} \leq \frac{1}{y_n} - \frac{1}{y_{n + 1}} \implies 1 \leq y_{n + 1} - y_n.
            \]
            Now, using $y_n = y_{n - 1} + y_{n - 2}$,
            \begin{align*}
                1 &\leq y_{n + 1} - y_n \\
                &= y_n - y_{n - 1} - y_n = y_{n - 1}.
            \end{align*}
            $1 \leq y_{n - 1}$ completes the proof since $y_n$ starts at $y_1$ therefore $n - 1 \geq 1$ which is equivalent to $n \geq 2$. Hence the inequality holds.
            \item
                For $n = 1$
                \begin{align*} 
                    y_3 y_1 - y_2 ^ 2 &= 2(1) - 1 ^ 2 \\
                    &= 2 - 1 \\
                    &= 1 = (-1) ^ 2
                \end{align*}

                Assume true for $n = k$
                \[
                y_{k + 2}y_k - y_{k + 1} ^ 2 = (-1)^{k + 1}.
                \]

                Then for $n = k + 1$,
                \begin{align*}
                    y_{k + 3}y_{k + 1} - y_{k + 2} ^ 2 &= (y_{k + 2} + y_{k + 1})(y_{k + 1}) - y_{k + 2} ^ 2 \\
                    &= y_{k + 2}y_{k + 1} + y_{k + 1} ^ 2 - y_{k + 2} ^ 2 \\
                    &= y_{k + 2}y_{k + 1} + (y_{k + 1} - y_{k + 2})(y_{k + 1} + y_{k + 2}) \\
                    &= y_{k + 2}y_{k + 1} + (-y_k)(y_{k + 1} + y_{k + 2}) \\
                    &= y_{k + 2}y_{k + 1} - y_ky_{k + 1} + y_ky_{k + 2} \\
                    &= y_{k + 1}(y_{k + 2} - y_k) + y_ky_{k + 2} \\
                    &= y_{k + 1}y_{k + 1} - y_ky_{k + 2} \\
                    &= -1(y_ky_{k + 2} - y_{k + 1}^2) \\
                    &= -1(-1)^{k + 1} \\
                    &= (-1)^{k + 2}
                \end{align*}
                We have shown true for $n = 1$ then assumed true for $n = k$, using this we have then proven this true for $n = k + 1$ for all $n \in \N$.
        \end{enumerate}
    \end{proof}
\end{example}

\begin{example}
    Let $0 < b \leq a$ and define a sequence $(x_n)_{n \in \N}$ by
    \[
    x_1 = a,\qquad x_2 = b,
    \]
    and for $n \geq 1$ let
    \begin{align*}
        x_{2n + 1} &= \frac{x_{2n} + x_{2n - 1}}{2}, \\
        x_{2n + 2} &= \sqrt{x_{2n + 1}x_{2n}}.
    \end{align*}
    For $n \geq 1$ let $A(n)$ be the statement that
    \[
    x_{2n} \leq x_m \leq x_{2n + 1}\quad\text{for all } m \geq 2n.
    \]
    \begin{enumerate}[label = \alph*.]
        \item Show that $A(n)$ is true for all $n \in \N$.
        \item Show that $(x_n)_{n \in \N}$ is a Cauchy sequence.
    \end{enumerate}
\end{example}

\newpage

\section{Series}

\subsection{Fundamental notions and properties of series}

\begin{example}
    Let $(x_n){n \in \N}$ be a sequence. Show that the series
    $(x_1 - x_2) + (x_2 - x_3) + (x_3 - x_4) + \cdots$ converges if and only if the sequence $(x_n){n \in \N}$ converges.
    \begin{proof}
        Assume that the series $(x_1 - x_2) + (x_2 - x_3) + (x_3 - x_4) + \cdots$ converges. Let $\seq{s}$ be the sequence of partial sums,
        \[
        s_n = \sum_{i = 1}^{n}(x_i - x_{i + 1}).
        \]
        Then we can see that $x_n = x_1 - s_{n - 1}$. Since $s$ converges so does $s_{n - 1}$, this implies that $\limas{x_n = x_1 - s_{n - 1}}{x_1 - s}$ which means that $x_n$ converges (by COLT). \\

        Next, assume that $\seq{x}$ converges. Using the partial sums we can get $s_n = x_1 - x_{n + 1}$. $\limas{x_n}{l}$ implies that $\limas{x_{n + 1}}{l}$ which means that $\limas{s_n = x_1 - x_{n + 1}}{x_1 - l}$, hence $\seq{x}$ converges (by COLT).
    \end{proof}
\end{example}

\begin{example}
    Let $q \in (-1, 1)$ and $N \in \N$. Find a formula for
    \[
    \infsum[k = N]{q ^ k}.
    \]
    \begin{proof}
        Using the general geometric series formula
        \[
        \infsum[k = N] q ^ k = q ^ N \infsum[k = 0] q ^ k = \frac{q ^ N}{1 - q}.
        \]
    \end{proof}
\end{example}

\begin{example}
    Calculate the limit of the following series.
    \begin{enumerate}[label = \alph*.]
        \item $\displaystyle \infsum[k = 0] \left(\frac{1}{2 ^ k} + \frac{(-1) ^ k}{3 ^ k}\right).$
        \item $\displaystyle \infsum[k = 1] \frac{1}{k(k + 2)}.$
        \item $\displaystyle \infsum[k = 1] \frac{1}{k(k + 1)(k + 2)}.$
        \item $\displaystyle \infsum[k = 1] \frac{4k + 1}{(2k - 1)(2k)(2k + 1)(2k + 2)}.$
    \end{enumerate}
    \begin{proof}
        \begin{enumerate}[label = \alph*.]
        \item 
        Using the geometric series sum to infinity and COLT we can obtain 
        \[
        \infsum[k = 0] \left(\frac{1}{2 ^ k} + \frac{(-1) ^ k}{3 ^ k}\right) = \frac{1}{1 - \frac{1}{2}} + \frac{1}{1 -- \frac{1}{3}} = \frac{11}{4}.
        \]
        \item
        In order to simplify our expression we will write the sum as a sum of partial fractions,
        \[
        \sum_{k = 1}^{n} \frac{1}{k(k + 2)} = \frac{1}{2}\sum_{k = 1}^{n} \left(\frac{1}{k} - \frac{1}{k + 2}\right).
        \]
        Now we can determine a formula for this sum
        \[
        \frac{1}{2}\sum_{k = 1}^{n} \left(\frac{1}{k} - \frac{1}{k + 2}\right) = \frac{3}{4} - \frac{1}{2}\left(\frac{1}{n + 1} + \frac{1}{n + 2}\right).
        \]
        Finally, we can see that using some standard limits and COLT that
        \[
        \sum_{k = 1}^{n} \frac{1}{k(k + 2)} = \frac{3}{4}.
        \]
    \end{enumerate}
    \end{proof}
\end{example}

\begin{example}
    Show that the series
    \[
    \sum_{n = 2}^{\infty}\frac{1}{n\log n}
    \]
    is divergent, using an argument similar to the one that was used for the harmonic series.
\end{example}

\begin{example}
    Let $\seq[a]{n}$ be a monotonically increasing bounded sequence of positive real numbers. Show that the series
    \[
    \sum_{n = 1}^{\infty}\left(\frac{a_{n + 1}}{a_n} - 1\right)
    \]
    is convergent.
\end{example}

\subsection{Convergence criteria}

\begin{example}
    Decide which of the following series is convergent or not.
    \begin{enumerate}[label = \alph*.]
        \item $\displaystyle \sum_{k = 1}^{\infty}\frac{k}{\sqrt{1 + k ^ 6}}.$
        \item $\displaystyle \sum_{k = 1}^{\infty}\frac{1}{k + \sqrt{k}}.$
        \item $\displaystyle \sum_{k = 1}^{\infty}\frac{(-1) ^ k}{\log{(k ^ 2 + 1)}}.$
        \item $\displaystyle \sum_{k = 1}^{\infty}\frac{(\log k) ^ 4}{k ^ 2}.$
        \item $\displaystyle \sum_{k = 1}^{\infty}(-1)^{k + 1}\frac{k - 1}{\sqrt[3]{8k ^ 3 + k ^ 2 + 12}}.$
        \item $\displaystyle \sum_{k = 1}^{\infty}\left(\sqrt{1 + k ^ 2} - k\right).$
    \end{enumerate}
\end{example}

\begin{example}
    Show that the series
    \[
    \sum_{k = 1}^{\infty} (-1)^{k + 1} \left|\frac{1}{k} + \frac{(-1) ^ {k + 1}}{\sqrt{k}}\right|
    \]
    is divergent.
\end{example}

\begin{example}
    Decide whether the following series converges.
    \[
    \sum_{n = 2}^{\infty} \frac{(-1)^{n + 1}}{n + (-1) ^ n}.
    \]
    is divergent.
\end{example}

\subsection{Absolute Convergence}

\begin{example}
    Assume that $\infsum x_n$ converges, and $\infsum y_n$ converges absolutely. Show that $\infsum x_ny_n$ converges absolutely. If we knew only that $\infsum x_n$ and $\infsum y_n$ converge, would it follow that $\infsum x_ny_n$ converged as well?
\end{example}

\begin{example}
    Determine whether or not the following series converge.
    \begin{enumerate}[label = \alph*.]
        \item $\displaystyle \infsum[k = 0]\frac{(2k)!}{5 ^ k (k!) ^ 2}.$
        \item $\displaystyle \infsum[k = 0]\frac{2 ^ k (2k)!}{9 ^ k (k!) ^ 2}.$
        \item $\displaystyle \infsum[k = 0]\frac{k - 1}{(k ^ 2 + 2)\sqrt[4]{(k ^ 2 + 1)}}.$
        \item $\displaystyle \infsum[k = 0]\frac{(k!) ^ 2}{(2k)!}.$
    \end{enumerate}
\end{example}

\begin{example}
    Determine for which values of $\alpha \in \R$ the following series converge.
    \begin{enumerate}[label = \alph*.]
        \item $\displaystyle \infsum\alpha ^ n n ^ \alpha.$
        \item $\displaystyle \infsum\frac{\alpha ^ {n - 1}}{n 3 ^ n}.$
        \item $\displaystyle \infsum[n = 0] n! \alpha ^ n.$
        \item $\displaystyle \infsum\frac{n \alpha ^ n}{2 ^ n (3n - 1)}.$
    \end{enumerate}
\end{example}

\begin{example}
    Test the following series for convergence.
    \[
    \infsum \left(n ^ 4\left(\frac{2n}{3n ^ 3 - 2n ^ 2 + 5}\right) ^ 2\right) ^ n.
    \]
\end{example}

\begin{example}
    Test the following series for convergence.
    \[
    \infsum \frac{(3n - 1)! - 4 ^ {n + 1}}{(3n)!}.
    \]
\end{example}

\subsection{Rearrangements of series}

\begin{example}
    Consider the series
    \[
    \infsum (-1)^{n + 1} \frac{1}{\sqrt{n}}
    \]
    \begin{enumerate}[label = \alph*.]
        \item Show that this series is conditionally convergent.
        \item Consider the rearrangement given by
        \[
        1 + \frac{1}{\sqrt{3}} - \frac{1}{\sqrt{2}} + \frac{1}{\sqrt{5}} + \frac{1}{\sqrt{7}} - \frac{1}{\sqrt{4}} + \frac{1}{\sqrt{9}} + \frac{1}{\sqrt{11}} - \frac{1}{\sqrt{6}} + \dotsc
        \]
        Decide whether this converges or not.
    \end{enumerate}
\end{example}

\begin{example}
    For $n \in \N_0$ let $a_n = \frac{1}{2 ^ n} = b_n$.
    \begin{enumerate}[label = \alph*.]
        \item Work out a simple formula for
        \[
        c_n = \sum_{k = 0}^n a_k b_{n - k}.
        \]
        \item Show that
        \[
        \infsum \frac{n(n + 1)}{2 ^ n} = 8.
        \]
        \item Show that
        \[
        \infsum \frac{n(n + 1)(n + 2)}{2 ^ n} = 48.
        \]
    \end{enumerate}
\end{example}

\begin{example}
    Show that the Cauchy product is not necessarily convergent if both series $\infsum[k = 0] a_k$ and $\infsum[k = 0] b_k$ are only conditionally convergent by discussing the choice
    \[
    a_k = b_k = \frac{(-1) ^ k}{\sqrt{k + 1}}.
    \]
\end{example}


\end{document}