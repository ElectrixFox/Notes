\documentclass[10pt, a4paper]{article}
\usepackage{preamble}
\usepackage{tikz-cd}

\title{Linear Algebra I \\
\large Problem Class}
\author{Luke Phillips}
\date{January 2025}

\begin{document}

\maketitle

\newpage

\tableofcontents

\newpage

\section{Class \texorpdfstring{$1$}{}}

\begin{problem}
    Let $A = \frac{1}{2}\begin{pmatrix}
        9 & -7 \\
        -7 & 9
    \end{pmatrix}$

    \begin{enumerate}[label = (\alph*)]
        \item Find an invertible $M$ such that $M ^ {-1}AM = \text{diagonal}$.

        \item Find $B$ such that $B ^ 3 = A$.
    \end{enumerate}

    \begin{solution}\phantom{}
        \begin{enumerate}[label = (\alph*)]
            \item
            \begin{align*}
                p_A(t) &= \det\begin{pmatrix}
                    \frac{9}{2} - t & -\frac{7}{2} \\
                    -\frac{7}{2} & \frac{9}{2} - t
                \end{pmatrix} \\
                &= \left(\frac{9}{2} - t\right) ^ 2 - \left(-\frac{7}{2}\right) ^ 2 \\
                &= \left(\frac{9}{2} - t\right)\left(\frac{9}{2} - t + \frac{7}{2}\right) \\
                &= (1 - t)(8 - t)
            \end{align*}
            so the eigenvalues are $\lambda = 1, 8$
        
            \[
            \lambda = 1
            \]
            \begin{align*}
                (A - I)\mbf{v} &= \mbf{0} \\
                \begin{pmatrix}
                    \frac{7}{2} & -\frac{7}{2} \\
                    -\frac{7}{2} & \frac{7}{2}
                 \end{pmatrix}
                 \begin{pmatrix}
                     a \\ b
                 \end{pmatrix} = \begin{pmatrix}
                     0 \\ 0
                 \end{pmatrix} \\
                 \implies a = b
            \end{align*}
            \[
            \mbf{v}_{\lambda = 1} = \begin{pmatrix}
                1 \\ 1
            \end{pmatrix}.
            \]
    
            \[
            \lambda = 8
            \]
            $\mbf{v}_{\lambda = 8} = \begin{pmatrix}
                1 \\ -1
            \end{pmatrix}$.
            
            \begin{align*}
                V_{\lambda = 1} &= \mathrm{span}\left\{\begin{pmatrix}
                    1 \\ 1
                \end{pmatrix}\right\} \\
                V_{\lambda = 8} &= \mathrm{span}\left\{\begin{pmatrix}
                    1 \\ -1
                \end{pmatrix}\right\} \\
                \R ^ 2 &= V_{\lambda = 1} \oplus V_{\lambda = 8}.
            \end{align*}
            Hence
            $M = [\mbf{v}_{\lambda = 1}, \mbf{v}_{\lambda = 8}] = \begin{pmatrix}
                1 & 1 \\ 1 & -1
            \end{pmatrix}$.

            \item We want
            \begin{align*}
                B ^ 3 = A = MDM ^ {-1} &\iff M ^ {-1}B ^ 3M = D \\
                &\iff M ^ {-1}BBBM = D = M ^ {-1}BMM ^ {-1}BMM ^ {-1}BM = D \\
                &\iff (M ^ {-1}BM) ^ 3 = D.
            \end{align*}
            Need $\tilde{D}$ such that $\tilde{D} ^ 3 = \begin{pmatrix}
                1 & 0 \\ 0 & 8
            \end{pmatrix} \implies \tilde{D} = \begin{pmatrix}
                1 & 0 \\ 0 & 2
            \end{pmatrix}$
            so $M ^ {-1}BM = \tilde{D} \implies B = M\tilde{D}M ^ {-1}$.
        \end{enumerate}
    \end{solution}
\end{problem}

\begin{problem}\phantom{}
    \begin{enumerate}[label = (\alph*)]
        \item Find $D$ such that $P ^ {-1}AP$ is diagonal when $A = \begin{pmatrix}
            1 & -1 \\ 5 & -1
        \end{pmatrix}$

        \item Find the solution $x(t), y(t)$ to the ODE system
        \begin{align*}
            \dot{x} &= x - y \\
            \dot{y} = 5x - y
        \end{align*}
        with
        \[
        \begin{pmatrix}
            x(0) \\ y(0)
        \end{pmatrix} = \begin{pmatrix}
            0 \\ 4
        \end{pmatrix}.
        \]
    \end{enumerate}

    \begin{solution}\phantom{}
        \begin{enumerate}[label = (\alph*)]
            \item 
            \begin{align*}
                p_A(t) &= \det\begin{pmatrix}
                    1 - t & -1 \\ 5 & -1 - t
                \end{pmatrix} \\
                &= t ^ 2 + 4 \\
                &= (t + 2i)(t - 2i).
            \end{align*}
            Even though $A$ is real the eigenvalues are complex.
            Hence $\lambda = -2i, 2i$.
    
            \[
            \lambda = -2i
            \]
            \begin{align*}
                (A - (-2i)I)\mbf{v} &= \begin{pmatrix}
                    1 + 2i & -1 \\ 5 & -1 + 2i
                \end{pmatrix}\begin{pmatrix}
                    a \\ b
                \end{pmatrix} = \begin{pmatrix}
                     0 \\ 0
                \end{pmatrix} \\
                &\implies
                (1 + 2i)a - b = 0 \\
                5a + (-1 + 2i)b = 0 \\
                \implies b = (1 + 2i)a
            \end{align*}
    
            \[
            \mbf{v}_{\lambda = -2i} = \begin{pmatrix}
                1 \\ 1 + 2i
            \end{pmatrix}.
            \]
            Likewise,
            $\mbf{v}_{\lambda = 2i} = \begin{pmatrix}
                1 \\ 1 - 2i
            \end{pmatrix}$.
    
            So $P = [\mbf{v}_{\lambda = -2i}, \mbf{v}_{\lambda = 2i}] = \begin{pmatrix}
                1 & 1 \\ 1 + 2i & 1 - 2i
            \end{pmatrix} \implies P\begin{pmatrix}
                2 + i & -i \\
                2 - i & i
            \end{pmatrix}$
            \[
            P ^ {-1}AP = \begin{pmatrix}
                -2i & 0 \\
                0 & 2i
            \end{pmatrix} = D
            \]
            \[
            \implies A = PDP ^ {-1}.
            \]

            \item
            \[
            \begin{pmatrix}
                \dot{x} \\ \dot{y}
            \end{pmatrix} = A\begin{pmatrix}
                x \\ y
            \end{pmatrix}
            \]
            with $A$ as above.

            \[
            \dot{\mbf{x}} = A\mbf{x} = PDP ^ {-1}\mbf{x} \implies (P ^ {-1}\dot{\mbf{x}}) = DP ^ {-1}\mbf{x}
            \]
            i.e.
            if $\mbf{w} = P ^ {-1}\mbf{x}$,
            $\dot{\mbf{w}} = D\mbf{w}$.

            If $\mbf{w} = \begin{pmatrix}
                w_1 \\ w_2
            \end{pmatrix},
            \begin{pmatrix}
                \dot{w}_1 \\ \dot{w}_2
            \end{pmatrix} = \begin{pmatrix}
                -2i & 0 \\ 0 & 2i
            \end{pmatrix}\begin{pmatrix}
                w_1 \\ w_2
            \end{pmatrix} \implies$
            \begin{align*}
                \dot{w}_1 &= -2iw_1 \\
                \dot{w}_2 &= 2iw_2
            \end{align*}
            \[
            \implies
            \begin{pmatrix}
                w_1 \\ w_2
            \end{pmatrix} = \begin{pmatrix}
                \alpha e ^ {-2it} \\
                \beta e ^ {2it}
            \end{pmatrix}
            \]
            \[
            \implies \begin{pmatrix}
                x(t) \\ y(t)
            \end{pmatrix} = P\begin{pmatrix}
                w_1(t) \\ w_2(t)
            \end{pmatrix} = \begin{pmatrix}
                \alpha e ^ {-2it} + \beta e ^ {2it} \\
                (1 + 2i)\alpha e ^ {-2it} + (1 - 2i)\beta e ^ {wit}
            \end{pmatrix}.
            \]
            Fix $\alpha$ and $\beta$:
            \[
            \begin{pmatrix}
                x(0) \\ y(0)
            \end{pmatrix} = \begin{pmatrix}
                0 \\ 4
            \end{pmatrix} \\
            \begin{pmatrix}
                \alpha - \beta \\
                (1 + 2i)\alpha + (1 - 2i)\beta
            \end{pmatrix}
            \]
            solve
            \[
            \alpha + \beta = 0 \implies \alpha = -\beta
            \]
            etc.
            $\implies \alpha = -i$.
            \[
            \begin{pmatrix}
                x(t) \\ y(t)
            \end{pmatrix} = \begin{pmatrix}
                -2\sin(2t) \\
                4\cos(2t)
            \end{pmatrix}
            \]
            real.
        \end{enumerate}
    \end{solution}
\end{problem}

\newpage

\section{Class \texorpdfstring{$2$}{}}

\begin{problem}
    $a\ddot{x} + b\dot{x} + cx = 0$ with $x(0) = \alpha$ and $\dot{x}(0) = \beta$.

    Introduce $y = \dot{x}$ so we get
    \[
    a\dot{y} + by + cx = 0
    \]
    with $x(0) = \alpha$ and $y(0) = \beta$.

    So we can rewrite a first order system of ODEs:
    $\dot{x} = y$ and $\dot{y} = -\frac{c}{a}x - \frac{b}{a}y$.

    This can be solved with matrices
    \[
    \begin{pmatrix}
        \dot{x} \\ \dot{y}
    \end{pmatrix} = \begin{pmatrix}
        0 & 1 \\
        -\frac{c}{a} & -\frac{b}{a}
    \end{pmatrix}\begin{pmatrix}
        x \\ y
    \end{pmatrix}
    \]
    so
    $\dot{\mbf{x}} = B\mbf{x}$.
    Then we would diagonalise $B$ where $B = \begin{pmatrix}
        0 & 1 \\
        -\frac{c}{a} & -\frac{b}{a}
    \end{pmatrix}$
    \[
    p_B(t) = \det\begin{pmatrix}
        -t & 1 \\ -\frac{c}{a} & -\frac{b}{a} - t
    \end{pmatrix} = t ^ 2 + \frac{b}{a}t + \frac{c}{a} = 0 \iff at ^ 2 + bt + c = 0.
    \]
    The problem case is
    \[
    \begin{pmatrix}
        \lambda & ? \\ 0 & \lambda
    \end{pmatrix}.
    \]
\end{problem}

\begin{problem}[Fibonacci sequence]
    \[
    F_{n + 1} = F_n + F_{n - 1}
    \]
    with $F_0 = 0, F_1 = 1$.
    Define
    \[
    \mbf{F}_n = \begin{pmatrix}
        F_{n + 1} \\ F_n
    \end{pmatrix}
    \]
    \[
    \mbf{F}_0 = \begin{pmatrix}
        F_1 \\ F_0
    \end{pmatrix} = \begin{pmatrix}
        1 \\ 0
    \end{pmatrix}
    \]
    \[
    \mbf{F}_n = \begin{pmatrix}
        F_{n + 1} \\ F_n
    \end{pmatrix} = \begin{pmatrix}
        F_n + F_{n - 1} \\ F_n
    \end{pmatrix} = \begin{pmatrix}
        1 & 1 \\ 1 & 0
    \end{pmatrix}\begin{pmatrix}
        F_n \\ F_{n - 1}
    \end{pmatrix} = A\mbf{F}_{n - 1}
    \]
    \[
    A = \begin{pmatrix}
        1 & 1 \\ 1 & 0
    \end{pmatrix}.
    \]
    \[
    \mbf{F}_n = A\mbf{F}_{n - 1} = A ^ 2\mbf{F}_{n - 2} = \dotsi = A ^ n\mbf{F}_0.
    \]
    We would just diagonalise $A$.
    \[
    p_A(t) = \det\begin{pmatrix}
        1 - t & 1 \\ 1 & -t
    \end{pmatrix} = (1 - t)(-t) - 1 = t ^ 2 - t - 1
    \]
    so $t = \frac{1 \pm \sqrt{5}}{2}$.
\end{problem}

\begin{problem}
    $V = \R ^ 3$,
    \[
    (\mbf{x}, \mbf{y}) = 2x_1y_1 + x_2y_2 - 2x_2y_3 + ax_3y_2 + kx_3y_3
    \]
    $a, k \in \R$.
    For which values of $a$ and $k$ does this define an inner product?

    \begin{solution}
        Check that this is linear and homogeneous,
        the quickest way to do this is to show that $(\mbf{x}, \mbf{y}) = \mbf{y} ^ TB\mbf{x}$.

        \[
        B = \begin{pmatrix}
            2 & 0 & 0 \\
            0 & 1 & a \\
            0 & -2 & k
        \end{pmatrix}.
        \]
        Is it symmetric?
        \[
        B \overset{?}{=} B ^ T = \begin{pmatrix}
            2 & 0 & 0 \\
            0 & 1 & -2 \\
            0 & a & k
        \end{pmatrix} \implies a = -2.
        \]

        Is it positive definite?
        \[
        B = \begin{pmatrix}
            2 & 0 & 0 \\
            0 & 1 & -2 \\
            0 & -2 & k
        \end{pmatrix}
        \]
        using Sylvester's:
        \[
        B_1 = (2),\qquad B_2 = \begin{pmatrix}
            2 & 0 \\ 0 & 1
        \end{pmatrix},\qquad B_3 = B
        \]
        these must be strictly positive
        \begin{align*}
            \det(B_1) &= 2 > 0 \\
            \det(B_2) &= 2 > 0 \\
            \det(B_3) &= 2(k - 4)
        \end{align*}
        so for $B$ to be positive definite we need $k > 4$
        ($k = 4$ not allowed).
        
        Or
        $a = -2$
        \begin{align*}
            (\mbf{x}, \mbf{x}) &= 2x_1 ^ 2 + x_2 ^ 2 - 4x_2x_3 + kx_3 ^ 2 \\
            &= 2x_1 ^ 2 + (x_2 - 2x_3 ^ 2) + (k - 4)x_3 ^ 2
        \end{align*}
        clearly $\geq 0$ for $k \geq 4$.
        Does $(\mbf{x}, \mbf{x}) = 0 \iff \mbf{x} = \mbf{0}$.
        For $k > 4$,
        we must have $x_1 = 0$,
        $x_2 - 2x_3 = 0$ and $x_3 = 0$ hence $\mbf{x} = \mbf{0}$.
        But for $k = 4$ we only need $x_1 = 0$ and $x_2 = 2x_3$,
        for example
        \[
        \begin{pmatrix}
            0 \\ 2 \\ 1
        \end{pmatrix}
        \begin{pmatrix}
            0 \\ 2 \\ 1
        \end{pmatrix} = 0
        \]
        so we cannot have $k = 4$.
    \end{solution}
\end{problem}

\begin{problem}
    $V = \R[x]_1 = \{p_0 + p_1x,\ p_0, p_1 \in \R\}$ and
    \[
    (p(x), q(x)) = \int_{0}^{1}p(x)q(x)\,dx.
    \]
    Find the angle between $p(x) = 1$ and $q(x) = x$.
    
    \begin{solution}
        \[
        (p, q) = \|p\|\|q\|\cos{\theta}.
        \]
        \begin{align*}
            \|p\| ^ 2 &= \|1\| ^ 2 = \int_{0}^{1}1\cdot 1\,dx = 1 \\
            \|q\| ^ 2 &= \|x\| ^ 2 = \int_{0}^{1}x\cdot x\,dx = \frac{1}{\sqrt{3}}
        \end{align*}
        \[
        (p, q) = \int_{0}^{1}1\cdot x\,dx = \frac{1}{2}
        \]
        so
        \[
        \cos{\theta} = \frac{(p, q)}{\|p\|\|q\|} = \frac{\sqrt{3}}{2} \implies \theta = \frac{\pi}{6}.
        \]
    \end{solution}
\end{problem}

\begin{problem}
    $V = \C ^ 2$ standard basis,
    \[
    \langle\mbf{z}, \mbf{w}\rangle = z_1\Bar{w}_1 + iz_1\Bar{w}_2 - iz_2\Bar{w}_1 + kz_2\Bar{w}_2.
    \]
    For which $k \in \C$ does this define a complex inner product?

    \begin{solution}
        To be sesquilinear one of the conditions is that we must have that $\langle\mbf{z}, \mbf{w}\rangle = \mbf{w} ^ {*}B\mbf{z}$ for some $B$.

        Here
        \[
        B = \begin{pmatrix}
            1 & -i \\
            i & k
        \end{pmatrix}
        \]
        does the job.
        It is Hermitian if and only if $B ^ {*} = (\Bar{B}) ^ T \overset{?}{=} B$.
        \[
        (\Bar{B}) ^ T = \begin{pmatrix}
            1 & -i \\ i & \bar{k}
        \end{pmatrix} \overset{?}{=} \begin{pmatrix}
            1 & -i \\ i & k
        \end{pmatrix}
        \]
        so we need $k = \Bar{k}$,
        i.e. $k \in \R$.
        Positive definite
        \begin{align*}
            B_1 &= (1)&\det(B_1) = 1 > 0 \\
            B_2 &= B&\det(B_2) = \det\begin{pmatrix}
                1 & -i \\ i & k
            \end{pmatrix} = k - (i)(-i) = k - 1
        \end{align*}
        so we need $k > 1$.
        \[
        \]
        
    \end{solution}
\end{problem}
































\end{document}