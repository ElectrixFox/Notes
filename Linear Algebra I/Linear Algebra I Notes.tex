\documentclass[10pt, a4paper]{article}
\usepackage{preamble}

\newcommand{\mbf}[1]{\mathbf{#1}}

\title{Linear Algebra I}
\author{Luke Phillips}
\date{October 2024}

\begin{document}

\maketitle

\newpage

\section{Vectors and vector spaces}

\subsection{Vector space $\R ^ n$}

\begin{definition}
    $n$-dimensional real space $\R ^ n$ a set of elements as such\[
    \R ^ n = \left\{\underline{x} = \begin{pmatrix}
        x_1 \\
        x_2 \\
        \vdots \\
        x_n
    \end{pmatrix}
    \quad:\ x_i \in \R
    \right\}\footnote{Vectors are $\underline{x}$ or $\mbf{x}$.}\]
\end{definition}

\begin{definition}
    The zero vector is defined as
    \[
    \mbf{0} = \begin{pmatrix}
        0 \\
        0 \\
        \vdots \\
        0
    \end{pmatrix}
    \in \R ^ n
    \]
\end{definition}

There are two operations on $\R ^ n$

\textbf{Vector addition}:

Vector addition is a function defined as such
\[
\R ^ n \times \R ^ n \mapsto \R ^ n
\]
This function takes two vectors $\mbf{v, w}$ and obtains
\[
(\mbf{v, w})\mapsto \mbf{v + w}
\]
For example
\[
\mbf{v + w} = \begin{pmatrix}
    v_1 \\
    v_2 \\
    \vdots \\
    v_n
\end{pmatrix} + \begin{pmatrix}
    w_1 \\
    w_2 \\
    \vdots \\
    w_n
\end{pmatrix}
=
\begin{pmatrix}
    v_1 + w_1 \\
    v_2 + w_2 \\
    \vdots \\
    v_n + w_n
\end{pmatrix}
=
\mbf{v + w}
\]

\textbf{Scalar multiplication}:

Scalar multiplication is defined as a function
\[
\R \times \R ^ n \mapsto \R ^ n.
\]
This is written as follows
\[
(\lambda, \mbf{v}) \mapsto \lambda\mbf{v}
\]
The operation is defined as such
\[
\lambda\mbf{v} = \lambda\begin{pmatrix}
    v_1 \\
    v_2 \\
    \vdots \\
    v_n
\end{pmatrix}
=
\begin{pmatrix}
    \lambda v_1 \\
    \lambda v_2 \\
    \vdots \\
    \lambda v_n
\end{pmatrix}
\]

\textbf{Intuition}

Vectors can be thought of a point in $n$-dimensional space

\textbf{Gr1}

As a direction to a point

\textbf{Gr2}

As up to translation

\textbf{Gr3}


Vector addition

\textbf{Gr4}

Scalar multiplication

\textbf{Gr5}


These operations satisfy the \textbf{axioms} of a real vector space.
\begin{enumerate}[label = (\roman*)]
    \item There exists an additive identity $\mbf{0} \in \R ^ n$ such that
    \[
    \mbf{0 + v = v + 0 = v} \ \forall \mbf{v} \in \R ^ n.
    \]
    \item Commutativity
    \[
    \forall\mbf{w, v} \in \R ^ n\quad\mbf{w + v =  v + w}
    \]
    \item Existence of additive inverses
    \[
    \forall\mbf{v} \in \R ^ n\ \exists\mbf{-v} \in \R ^ n \text{ s.t. } \mbf{v + (-v) = (-v) + v = 0}
    \]
    \item Associativity
    $\forall\mbf{u, v, w} \in \R ^ n$
    \[
    (\mbf{u + v}) + \mbf{w} = \mbf{u} + (\mbf{v + w})
    \]
\end{enumerate}
(i), (iii), (iv) $\iff (\R ^ n, +)$ is a group

(i), (ii), (iii), (iv) $\iff (\R ^ n, +)$ is an abelian group \\


\textbf{Axioms} for scalar multiplication
\begin{enumerate}[label = (\roman*)]
    \item $0\mbf{v} = \mbf{0}\quad\forall\mbf{v}\in \R ^ n$.
    \item $1\mbf{v} = \mbf{v}\quad\forall\mbf{v}\in \R ^ n$.
    \item Associativity
    \[
    \lambda (\mu\mbf{v}) = (\lambda\mu)\mbf{v}\quad\forall\lambda,\mu \in \R\quad\forall\mbf{v}\in\R ^ n
    \]
    \item Distributivity, $\forall\lambda,\mu\in\R\quad\forall\mbf{v, w} \in \R ^ n$
    \begin{align*}
    (\lambda + \mu)\mbf{v} &= \lambda\mbf{v} + \mu\mbf{v} \\
    \lambda(\mbf{v + w}) &= \lambda\mbf{v} + \lambda\mbf{w}
    \end{align*}
\end{enumerate}

$\R ^ n$ clearly satisfies these axioms (if not obvious check) because they can be checked component-wise once we set 
\[
-\mbf{v} = \begin{pmatrix}
    -v_1 \\
    -v_2 \\
    \vdots \\
    -v_n
\end{pmatrix}
\]

\begin{definition}
    Standard basis vectors
    
    For $1 \leq i \leq n$ we define $\mbf{e}_i \in \R ^ n$
    \[
    \mbf{e}_i = \begin{pmatrix}
        0 \\
        0 \\
        \vdots \\
        0 \\
        1\footnote{$i$th position from the top.} \\
        0 \\
        \vdots \\
        0
    \end{pmatrix}
    \]    
\end{definition}

\begin{example}
    \[
    \mbf{e}_1 = \begin{pmatrix}
        1 \\
        0 \\
        \vdots \\
        0
    \end{pmatrix}
    \]
    \[
    \mbf{e}_2 = \begin{pmatrix}
        0 \\
        1 \\
        \vdots \\
        0
    \end{pmatrix}
    \]
\end{example}

We can express any vector $\mbf{x} \in \R ^ n$ uniquely as a linear combination of the standard basis vectors.

\begin{example}
    \begin{align*}
    \mbf{x} &= \begin{pmatrix}
        x_1 \\
        x_2 \\
        \vdots \\
        x_n
    \end{pmatrix}
    =
    x_1 \begin{pmatrix}
        1 \\
        0 \\
        \vdots \\
        0
    \end{pmatrix}
    +
    x_2 \begin{pmatrix}
        0 \\
        1 \\
        \vdots \\
        0
    \end{pmatrix}
    +
    \dots
    +
    x_n \begin{pmatrix}
        0 \\
        0 \\
        \vdots \\
        0 \\
        1
    \end{pmatrix} \\
    &= x_1 \mbf{e}_1 + x_2 \mbf{e}_2 + \dots + x_n \mbf{e}_n
    \end{align*}
\end{example}
The $x_i$'s are sometimes called the Cartesian coordinates of the vector $\mbf{x}$.

\subsection{The scalar (or dot) product in $\R ^ n$}

\begin{definition}
    The scalar product is defined as $\R ^ n \times \R ^ n \mapsto \R$, $(\mbf{u, v}) \mapsto \mbf{u \cdot v}$
    \begin{align*}
    \mbf{u \cdot v} &= \begin{pmatrix}
        u_1 \\
        u_2 \\
        \vdots \\
        u_n
    \end{pmatrix} \cdot
    \begin{pmatrix}
        v_1 \\
        v_2 \\
        \vdots \\
        v_n
    \end{pmatrix} \\
    &= u_1 v_1 + u_2 v_2 + \dots + u_n v_n  \\
    &= \sum_{i = 1}^{n}u_iv_i \in \R
    \end{align*}
\end{definition}

\begin{example}
    \[
    \begin{pmatrix}
        4 \\
        1 \\
        2
    \end{pmatrix}
    \cdot
    \begin{pmatrix}
        -3 \\
        -2 \\
        1
    \end{pmatrix}
    =
    -12 - 2 + 2 = -12.
    \]
\end{example}


\textbf{Axioms} of the scalar product
% To-Do add in the foralls here
\begin{enumerate}[label = (\roman*)]
    \item Symmetry
    \[
    \mbf{u \cdot v = v \cdot u}\qquad\forall\mbf{u, v} \in \R ^ n
    \]
    \item Linearity (1)
    \begin{align*}
    (\mbf{u + v)\cdot w}) &= \mbf{u \cdot w + v \cdot w}\qquad\forall \lambda \in \R\\
    \lambda\mbf{u})\cdot \mbf{w} &= \lambda (\mbf{u \cdot w})\qquad\forall\mbf{u, v, w} \in \R ^ n
    \end{align*}
    \item Linearity (2)
    \begin{align*}
        \mbf{u} \cdot (\mbf{v + w}) &= \mbf{u} \cdot \mbf{v} + \mbf{u} \cdot \mbf{w}\qquad \forall \lambda \in \R \\
        \mbf{u} \cdot \lambda\mbf{v} &= \lambda(\mbf{u \cdot v})\qquad \forall\mbf{u, v, w} \in \R ^ n
    \end{align*}
    \item Positivity
    \begin{align*}
        \mbf{v \cdot v} \geq 0\quad \forall \mbf{v} \in \R ^ n \\
        \text{and } \mbf{v \cdot v} = 0 \iff \mbf{v = 0}
    \end{align*}
\end{enumerate}

\begin{definition}
    Given a vector $\mbf{v} \in \R ^ n$ we define its magnitude (or length) to be
    \[
    |\mbf{v}| = \sqrt{\mbf{v \cdot v}}
    \]
\end{definition}
Note $\mbf{v \cdot v} \geq 0$ so positive square roots exist. Likewise note

$|\mbf{v}| \geq 0 \text{ and } |\mbf{v}| = 0 \iff \mbf{v = 0}$

\begin{example}
    \[
    \left|\begin{pmatrix}
        -3 \\
        4
    \end{pmatrix}\right|
    = \sqrt{(-3) ^ 2 + 4 ^ 2} = \sqrt{25} = 5
    \]
\end{example}

Thinking about a vector $\mbf{v} \in \R ^ 2$ where $\mbf{v \neq 0}$

\textbf{Gr9}

\[
\mbf{v}  = \begin{pmatrix}
    r\cos\theta \\
    r\sin\theta
\end{pmatrix}
\]
where
$r = |\mbf{v}|$ and $0 \leq \theta < 2\pi$, $\theta$ is the angle made by $\mbf{v}$ in an anticlockwise direction with the positive real axis.

We define $(r, \theta) \in (0, \infty) \times [0, 2\pi)$
to be the unique numbers such that
$\mbf{v}  = \begin{pmatrix}
    r\cos\theta \\
    r\sin\theta
\end{pmatrix}$
we call $(r, \theta)$ the polar coordinates of $\mbf{v}$

\begin{example}
    Suppose $\mbf{v} = \begin{pmatrix}
        2 \\
        3
    \end{pmatrix}$,
    what are its polar coordinates?

    $r = |\mbf{v}| = \sqrt{2 ^ 2 + 3 ^ 2} = \sqrt{13}$

    $\theta = \arcsin \left({\dfrac{3}{\sqrt{13}}}\right)$
\end{example}

\begin{example}
    Suppose $\mbf{v} = \begin{pmatrix}
        2 \\
        -2
    \end{pmatrix}$,
    what are its polar coordinates?

    $r = |\mbf{v}| = \sqrt{2 ^ 2 + (-2) ^ 2} = \sqrt{8} = 2\sqrt{2}$

    $\theta = \dfrac{7\pi}{4}$
\end{example}


\end{document}