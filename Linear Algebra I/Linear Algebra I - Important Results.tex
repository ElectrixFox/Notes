\documentclass[10pt, a4paper]{article}
\usepackage{preamble}


\title{Linear Algebra I \\
    \large Important Results}
\author{Luke Phillips}
\date{January 2025}

\begin{document}

\maketitle

\newpage

\tableofcontents

\newpage

\section{Vectors in \texorpdfstring{$\R ^ n$}{}}

\begin{proposition}[Properties of the scalar product]
    \phantom{}
    \begin{enumerate}[label = (\roman*)]
        \item Symmetry
        \[
        \mbf{u \cdot v = v \cdot u}\qquad\text{for all }\mbf{u, v}\text{ in } \R ^ n
        \]
        \item Linearity in first factor
        \begin{enumerate}[label = (\alph*)]
            \item $(\mbf{u} + \mbf{v}) \cdot \mbf{w} = \mbf{u} \cdot \mbf{w} + \mbf{v} \cdot \mbf{w}$\qquad for all $\mbf{u,\ v,\ w}$ in $\R ^ n$
            \item $(\lambda\mbf{u}) \cdot \mbf{v} = \lambda(\mbf{u} \cdot \mbf{v})$\qquad for all $\lambda \in \R$ and all $\mbf{u,\ v}$ in $\R ^ n$
        \end{enumerate}
        \item Linearity in the second factor
        \begin{enumerate}[label = (\alph*)]
            \item $\mbf{u} \cdot (\mbf{v} + \mbf{w}) = \mbf{u} \cdot \mbf{w} + \mbf{u} \cdot \mbf{v}$\qquad for all $\mbf{u,\ v,\ w}$ in $\R ^ n$
            \item $\mbf{u} \cdot (\lambda\mbf{v}) = \lambda(\mbf{u} \cdot \mbf{v})$\qquad for all $\lambda \in \R$ and all $\mbf{u,\ v}$ in $\R ^ n$
        \end{enumerate}
        \item Positivity
        
        $\mbf{v \cdot v} \geq 0$ for all $v$, and $\mbf{v \cdot v} = 0$ if and only if $\mbf{v = 0}$.
    \end{enumerate}
\end{proposition}

\begin{definition}[Magnitude]
    We define the magnitude o $\mbf{v}$,
    denoted $|\mbf{v}|$,
    by
    \[
    |\mbf{v}| = \sqrt{\mbf{v} \cdot \mbf{v}}.
    \]

    A vector is \textbf{non-trivial} if it has positive length,
    the only non-trivial vector in $\R ^ n$ is the zero vector $\mbf{0}$.
\end{definition}

\begin{lemma}
    Suppose that $\mbf{v}$ and $\mbf{w}$ are vectors in $\R ^ 2$ of length $r > 0 $ and $s > 0$ respectively. Let $\theta$ be the angle between them. Then
    \[
    \mbf{v \cdot w} = rs\cos(\theta).
    \]
    In particular, we have
    \[
    |\mbf{v \cdot w}| \leq |\mbf{v}||\mbf{w}|
    \]
    with equality if and only if $\mbf{v}$ and $\mbf{w}$ are collinear, i.e. are scalar multiples of each other.
\end{lemma}

\begin{remark}
    Note that using this lemma we can obtain an expression for the angle between two vectors
    \[
    \cos(\theta) = \frac{\mbf{v \cdot w}}{|\mbf{v}||\mbf{w}|}.
    \]
\end{remark}

\begin{theorem}[Cauchy-Schwarz inequality]
    Suppose that $\mbf{v}$ and $\mbf{w}$ are vectors in $\R ^ n$. Then
    \[
    |\mbf{v \cdot w}| \leq |\mbf{v}||\mbf{w}|
    \]
    with equality if and only if $\mbf{v}$ and $\mbf{w}$ are collinear.
\end{theorem}

\begin{corollary}
    The non-trivial vectors $\mbf{v}$ and $\mbf{w}$ in $\R ^ n$ are orthogonal if an only if we have $\mbf{v} \cdot \mbf{w} = 0$.
\end{corollary}

\begin{definition}[Vector product in $\R ^ 3$]
    The vector (or cross) product is a function
    \[
    \R ^ 3 \times \R ^ 3 \rightarrow \R ^ 3
    \]
    defined as follows. If $\mbf{v, w} \in \R ^ 3$, their cross product is given by
    \[
    \mbf{v \times w} =
    \begin{pmatrix} v_1 \\ v_2 \\ v_3 \end{pmatrix}
    \times
    \begin{pmatrix} w_1 \\ w_2 \\ w_3 \end{pmatrix}
    =
    \begin{pmatrix} v_2 w_3 - v_3 w_2 \\ v_3 w_1 - v_1 w_3 \\ v_1 w_2 - w_1 v_2 \end{pmatrix}
    \]
\end{definition}

\begin{proposition}[Properties of the cross product]\phantom{}
    \begin{enumerate}[label = (\roman*)]
        \item Anti-Symmetry:
        \[
        \mbf{u \times v = -v \times u}\quad\text{for all }\mbf{u, v} \in \R ^ 3
        \]
        \item Linearity in first factor
        \begin{enumerate}[label = (\arabic*)]
            \item $(\mbf{u + v}) \times \mbf{w} = \mbf{u \times w} + \mbf{v \times w}\quad\text{for all }\mbf{u, v, w} \in \R ^ 3$
            \item $(\lambda\mbf{u}) \times \mbf{v} = \lambda(\mbf{u \times v})\quad\text{for all } \lambda \in \R \text{ and all }\mbf{u, v} \in \R ^ 3$
        \end{enumerate}
        \item Linearity in the second factor
        \begin{enumerate}[label = (\arabic*)]
            \item $\mbf{u} \times (\mbf{v + w}) = \mbf{u \times w} + \mbf{u \times v}\quad\text{for all }\mbf{u, v, w} \in \R ^ 3$
            \item $\mbf{u} \times (\lambda\mbf{v}) = \lambda(\mbf{u \times v})\quad\text{for all } \lambda \in \R \text{ and all }\mbf{u, v} \in \R ^ 3$
        \end{enumerate}
        \item Orthogonality to the input
        \[
        \mbf{v} \cdot (\mbf{v \times w}) = \mbf{w} \cdot (\mbf{v \times w}) = 0\quad\text{for all vectors } \mbf{v} \text{ and } \mbf{w} \text{ in } \R ^ 3.
        \]
    \end{enumerate}
\end{proposition}

\begin{lemma}
    Suppose that $\mbf{v}$ and $\mbf{w}$ are vectors of length $r > 0$ and $s > 0$ respectively. Let $\theta \in [0, \pi]$ be the angle between them. Then
    \[
    |\mbf{v \times w}| = rs\sin(\theta).
    \]
    In particular, we have
    \[
    \mbf{v \times w = 0}\quad\iff\quad\mbf{v}\text{ and }\mbf{w}\text{ are collinear}.
    \]
\end{lemma}

\begin{definition}[Parametric form of a plane]
    A plane $\Pi$ in $\R ^ 3$ can be described in parametric form as the collection of all points corresponding to vectors of the form
    \[
    \mbf{a} + \lambda\mbf{d}_1 + \mu\mbf{d}_2
    \]
    where $\lambda, \mu \in \R$,
    $\mbf{a}$ is a point in $\Pi$ and $\mbf{d}_1$ and $\mbf{d}_2$ are two direction vectors for $\Pi$.
\end{definition}

\begin{definition}[Normal vector form of a plane]
    A plane $\Pi$ in $\R ^ 3$ can be described as follows
    \[
    \mbf{x} \cdot (\mbf{d}_1 \times \mbf{d}_2) = 0
    \]
    where $\mbf{x}$ a point in $\Pi$,
    and $\mbf{d}_1, \mbf{d}_2$ direction vectors for $\Pi$.

    We can write this as
    \[
    \mbf{n} \cdot \mbf{x} = \ell
    \]
    where $\mbf{n}$ is the normal vector to $\Pi$ and $\mbf{x}$ is a point on $\Pi$.
\end{definition}

\begin{definition}[Cartesian form of a plane]
    We have the Cartesian form of a plane from the normal vector form of a plane
    \[
    \mbf{n} \cdot \mbf{x} = \begin{pmatrix}
        a \\ b \\ c
    \end{pmatrix} \cdot \begin{pmatrix}
        x \\ y \\ z
    \end{pmatrix} = ax + by + cz = \ell.
    \]
    Thus we have
    \[
    \Pi = \left\{\{\mbf{x} \in \R ^ 3\,:\,\mbf{n} \cdot \mbf{x} = \ell\right\} = \left\{\begin{pmatrix}
        x \\ y \\ z
    \end{pmatrix}\,:\, ax + by + cz = \ell\right\}.
    \]
\end{definition}

\begin{definition}[Scalar triple product]
    The scalar triple product is a function
    \[
    \R ^ 3 \times \R ^ 3 \times \R ^ 3 \rightarrow \R.
    \]
    Given three vectors $\mbf{a, b, c}$ in $\R ^ 3$ we define their scalar triple product $[\mbf{a, b, c}]$ by
    \[
    [\mbf{a, b, c}] = \mbf{a} \cdot(\mbf{b} \times \mbf{c}).
    \]
\end{definition}

\begin{lemma}[Invariance under cyclic permutation]
    For any three vectors $\mbf{a, b, c}$ in $\R ^ 3$ we have
    \[
    [\mbf{a, b, c}] = [\mbf{b, c, a}] = [\mbf{c, a, b}] = -[\mbf{a, b, c}] = -[\mbf{b, c, a}] = -[\mbf{c, a, b}].
    \]
\end{lemma}

\newpage

\section{Matrix algebra}

\begin{definition}[Transpose]
    Given an $m \times n$ matrix $W$ we define $W ^ t$, the transpose of $W$, to be the $n \times m$ matrix whose $(i,\,j)$th entry is the $(j,\,i)$th entry of $W$; that is the matrix for which
    \[
    (W ^ t)_{ij} = W_{ji}.
    \]
\end{definition}

\begin{proposition}[Properties of matrix addition and scalar multiplication]
    Matrix addition and scalar multiplication for $M_{m, n}(\R)$ satisfy exactly the same properties as vector addition and scalar multiplication for $\R ^ n$. That is, the operations satisfy the axioms of a (real) vector space.

    \textbf{Axioms for addition}
    \begin{enumerate}[label = (\roman*)]
        \item existence of additive identity: there is a matrix $O_{m \times n}$ in $M_{m, n}(\R)$ so that for all $X$ in $M_{m, n}(\R)$ we have $X + O_{m \times n} = X = O_{m \times n} + X$ 
        \item commutativity: $X + Y = Y + x$ for all $X$ and $Y$ in $M_{m, n}(\R)$
        \item existence of additive inverses: for each $X$ in $M_{m, n}(\R)$ there is an element $-X$ in $M_{m, n}(\R)$ such that $X + (-X) = O_{m \times n} = (-X) + X$
        \item associativity: $(X + Y) + Z = X + (Y + Z)$ for all $X, Y, Z \in M_{m, n}(\R)$
    \end{enumerate}

    \textbf{Axioms for Scalar Multiplication}
    \begin{enumerate}[label = (\roman*)]
        \item $0X = O_{m \times n}$ for all $X$ in $M_{m, n}(\R)$
        \item $1X = X$ for all $X$ in $M_{m, n}(\R)$
        \item associativity: $(\lambda\mu)X = \lambda(\mu X)$ for all $\lambda, \mu \in \R$ and for all $X \in M_{m, n}(\R)$
        \item distributivity: $(\lambda + \mu)X = \lambda X + \mu X$ and $\lambda (X + Y) = \lambda X + \lambda Y$ for all real $\lambda, \mu$ and for all $X$ and $Y$ in $M_{m, n}(\R)$.
    \end{enumerate}
\end{proposition}

\begin{definition}[Matrix Multiplication]
    For any $X \in M_{m \times n}(\R)$ and $Y \in M_{n \times p}(\R)$ we define matrix multiplication as follows
    \[
    (XY)_{ij} = x_{i\,1}y_{1\,j} + x_{i\,2}y_{2\,j} + \dotsi + x_{i\,n}x_{n\,j} = \sum_{k = 1}^{n}{x_{ik}y_{kj}},
    \]
    where we get a matrix of the form $M_{m \times p}(\R)$.
\end{definition}

\begin{proposition}[Multiplication rules]
    Let $X$ and $X'$ be $m \times n$, $Y$ and $Y'$ be $n \times p$ and $Z$ be $p \times q$ matrices. Let $\lambda$ and $\lambda'$ be scalars. Then
    \begin{enumerate}[label = (\roman*)]
        \item $O_{p \times m}X = O_{p \times n}$ and $XO_{n \times q} = O_{m \times q}$.
        \item $I_{m}X = X = XI_n$.
        \item $\lambda(XY) = (\lambda X)Y = X(\lambda Y)$.
        \item associativity: $(XY)Z = X(YZ)$
        \item distributive rules:
        \begin{enumerate}[label = (\alph*)]
            \item $(X + X')Y = XY + X'Y$ and $X(Y + Y') = XY + XY'$.
            \item $\lambda(X + X') = \lambda X + \lambda X'$ and $(\lambda + \lambda')X = \lambda X + \lambda' X$.
        \end{enumerate}
    \end{enumerate}
\end{proposition}

\begin{definition}[Invertible Matrix]
    Let $A$ be a square $(n\times n)$ matrix. We say $A$ is invertible (or non-singular) if there exists an $n \times n$ matrix $B$ such that
    \[
    AB = BA = I_n.
    \]
    If such a matrix $B$ exists we call $B$ the inverse of $A$ and we write $A ^ {-1}$ for this inverse. (there can only exist at most one inverse).
    If no such inverse $B$ exists for a matrix $A$, we say $A$ is non-invertible (or singular).
\end{definition}

\begin{proposition}[Properties of the inverse and transpose]\phantom{}
    \begin{enumerate}[label = (\roman*)]
        \item A matrix has at most one inverse.
        \item Assume $A$ and $B$ are invertible $n \times n$ matrices. Then the product $AB$ is invertible and
        \[
        (AB) ^ {-1} = B ^ {-1}A ^ {-1}.
        \]
        \item We have $(AB) ^ t = B ^ t A ^ t$ and $(\lambda A) ^ t = \lambda A ^ t$ for any matrix $A$ and any real constant $\lambda$.
        \item Assume $A$ is invertible. then the transpose $A ^ t$ is invertible and
        \[
        (A ^ t) ^ {-1} = (A ^ {-1}) ^ t.
        \]
        \item Let $A \in M_{n, n}(\R)$ and assume $BA = I_n$ for some $B \in M_{n, n}(\R)$. Then $AB = I_n$ as well as vice versa. So the "left inverse" is automatically the "right inverse".
    \end{enumerate}
\end{proposition}

\begin{remark}
    We also have
    \[
    (X + Y) ^ t = X ^ t + Y ^ t.
    \]
\end{remark}

\newpage

\section{Gauss-Jordan elimination}

\begin{lemma}\phantom{}
    \begin{enumerate}[label = (\roman*)]
        \item Let $A \in M_{n,\,n}(\R)$ be a square matrix and let $\mbf{v} \in \R ^ n$ be a non-zero vector such that $A\mbf{v} = \mbf{0}$. Then $A$ is singular.
        \item Let $A \in M_{n,\,n}(\R)$ be an invertible square matrix. Then the inhomogeneous system $A\mbf{x} = \mbf{b}$ has a single unique solution. In particular, the only solution to the homogeneous system $A\mbf{x} = \mbf{0}$ is $\mbf{x} = \mbf{0}$.
    \end{enumerate}
\end{lemma}

\begin{definition}
    A matrix of numbers is said to be in row reduced echelon form (RREF) if
    \begin{enumerate}[label = (\roman*)]
        \item The first (leftmost) non-zero entry in any non-zero row is a $1$ (called its leading $1$).
        \item If a row has its leading $1$ in the $j$th column then
        \begin{enumerate}[label = (\alph*)]
            \item all the other entries in the $j$th column are $0$; and
            \item the leading $1$s of subsequent rows are in columns to the right of the $j$th column.
        \end{enumerate}
    \item Any row(s) of zeros come after all the rows with non-zero entries.
    \end{enumerate}
\end{definition}

\begin{remark}
    If we can express variables in terms of the other ones,
    we call these free variables.
\end{remark}

\begin{remark}
    The ERO's are simple operations you can do to a matrix to turn it into another matrix.
    There are three fundamental operations, which are as follows,
    \begin{enumerate}[label = (\alph*)]
        \item $P_{rs}$: switch row $r$ with row $s$;
        \item $M_{r}(\pmb{\lambda})$: multiple (all the entries in) row $r$ by the number $\lambda \neq 0$;
        \item $A_{rs}$: add (each entry in) row $r$ to (the corresponding entry in) row $s$.
        \item[(c')] $A_{rs}(\pmb{\lambda})$: add $\lambda$ times (each entry in) row $r$ to (the corresponding entry in) row $s$.
    \end{enumerate}
\end{remark}

\begin{remark}
    When an augmented matrix is in RREF,
    a variable $x_j$ being a free variable corresponds to column $j$ of the coefficient matrix note containing a leading $1$.
\end{remark}

\begin{lemma}
    Let $A \in M_{n \times n}(\R)$ be a square matrix.
    Then the following are (pairwise) equivalent.
    \begin{enumerate}[label = (\roman*)]
        \item In each column of RREF of $A$ there exists a leading $1$.
        \item The RREF of $A$ is the identity matrix $I_n = \begin{pmatrix}
            1 & \phantom{} & \phantom{} \\
            \phantom{} & \ddots & \phantom{} \\
            \phantom{} & \phantom{} & 1
        \end{pmatrix}$.
        \item The only solution to the system $A\mbf{x} = \mbf{0}$ is $\mbf{x} = \mbf{0}$.
    \end{enumerate}
\end{lemma}

\begin{theorem}[Fundamental Theorem of Invertible Matrices]
    A square matrix $A \in M_{n \times n}(\R)$ is invertible if and only if any one of the conditions (i), (ii), (iii) of the previous lemma holds.
\end{theorem}

\newpage

\section{Determinants}

\begin{remark}
    For integers $r$ and $s$ between $1$ and $n$ write $A_{r, s}$ for the $(n - 1) \times (n - 1)$ matrix obtained from $A$ by deleting its $r$th row and its $s$th column.
    $A_{r, s}$ is called the $(r, s)$th minor of $A$.
\end{remark}

\begin{remark}
    The value $\det(A_{r, s})$ is called the $(r, s)$th unsigned cofactor of $A$ and the value $(-1) ^ {r + s}\det(A_{r, s})$ is called the $(r, s)$th (signed) cofactor of $A$.
\end{remark}

\begin{definition}[The determinant]
    For any matrix $A \in M_{m \times n}(\R)$,
    the determinant of $A$ is defined by the following
    \[
    \det{A} = \sum_{k = 1}^{n}(-1) ^ {1 + k}a_{1k}\det(A_{1, k}).
    \]
\end{definition}

\begin{lemma}
    If $A$ is an $n \times n$ matrix then for any $1 \leq i \leq n$ we have
    \[
    \det(A) = \sum_{k = 1}^{n}(-1) ^ {i + k}a_{ik}\det(A_i, k).
    \]
    That is,
    the determinant formula remains valid via expansion along the $i$th row
    (note however the signs).
\end{lemma}

\begin{proposition}[Fundamental properties of the determinant]
    For notational convenience,
    we write in terms of the row vector $\mbf{a}_r$ of an $n \times n$ matrix $A = (a_{ij})$ that is $\mbf{a}_r = (a_{r1}\; a_{r2}\; \dotsi \; a_{rn})$
    \begin{enumerate}[label = (\roman*)]
        \item $\det I_n = 1$.
        \item If one multiplies
        (all the elements in)
        the $r$th row by some scalar $\lambda$
        (that is, if we apply the ERO $M_r(\lambda)$),
        then the determinant changes by precisely this factor:
        \[
        \det(M_r(\lambda)A) = \det\begin{pmatrix}
            \mbf{a} \\ \vdots \\ \lambda\mbf{a}_r \\ \vdots \\ \mbf{a}_n
        \end{pmatrix} = \lambda\det A.
        \]
        \item For any row vector $\mbf{b}_r$ we have
        \[
        \det\begin{pmatrix}
            \mbf{a}_1 \\ \vdots \\ \mbf{a}_r + \mbf{b}_r \\ \vdots \\ \mbf{a}_n
        \end{pmatrix} =
        \det \begin{pmatrix}
            \mbf{a}_1 \\ \vdots \\ \mbf{a}_r \\ \vdots \\ \mbf{a}_n
        \end{pmatrix}
        +
        \det\begin{pmatrix}
            \mbf{a}_1 \\ \vdots \\ \mbf{b}_r \\ \vdots \\ \mbf{a}_n
        \end{pmatrix}.
        \]
        Note that (ii) and (iii) can be summarised as linearity in each row.
        \item Anti-symmetry in rows.
        The determinant switches signs if one swaps a pair of rows (i.e. if one applies the ERO $P_{rs}$);
        that is,
        \[
        \det(A) = \det\begin{pmatrix}
            \mbf{a}_1 \\ \vdots \\ \mbf{a}_r \\ \vdots \\ \mbf{a}_s \\ \vdots \\ \mbf{a}_n
        \end{pmatrix}
        = -\det\begin{pmatrix}
            \mbf{a}_1 \\ \vdots \\ \mbf{a}_s \\ \vdots \\ \mbf{a}_r \\ \vdots \\ \mbf{a}_n
        \end{pmatrix}
        =
        -\det(P_{rs}A).
        \]
        \item $\det A = 0$ if $A$ has two rows that are equal
        (or even two rows that are collinear - follows simply by property (ii)).
        \item $\det A = 0$ if $A$ has a row of zeros.
        \item The determinant does not change under the ERO $A_{rs}(\lambda)$; that is,
        \[
        \det(A_{rs}(\lambda)A) = \det\begin{pmatrix}
            \mbf{a}_1 \\ \vdots \\ \mbf{a}_r \\ \vdots \\ \mbf{a}_s + \lambda\mbf{a}_r \\ \vdots \\ \mbf{a}_n
        \end{pmatrix}
        =
        \det\begin{pmatrix}
            \mbf{a}_1 \\ \vdots \\ \mbf{a}_r \\ \vdots \\ \mbf{a}_s \\ \vdots \\ \mbf{a}_n
        \end{pmatrix}
        = \det A.
        \]
        \end{enumerate}
\end{proposition}

\begin{remark}
    Given any matrix $A \in M_n(\R)$,
    the operation $A_{r, s}$ doesn't change the value of the determinant,
    $M_{r}(\lambda)$ changes the determinant by $\lambda\det{A}$,
    finally $P_{rs}$ changes the determinant to $-\det{A}$.
\end{remark}

\begin{theorem}
    The $n \times n$ matrix $A$ has an inverse if and only if $\det(A) \neq 0$.
\end{theorem}

\begin{theorem}
    Let $A$ and $B$ be $n \times n$ matrices.
    Then
    \[
    \det(AB) = \det(A)\det(B).
    \]
\end{theorem}

\begin{theorem}
    We have
    \[
    \det(A ^ t) = \det(A).
    \]
\end{theorem}

\begin{definition}[The adjoint]
    Given a square matrix $A$,
    we define the adjoint matrix $\Adj(A)$ by
    \[
    (\Adj(A))_{ij} = (-1) ^ {i + j}\det(A_{j, i}).
    \]
\end{definition}

\begin{remark}
    The adjoint is the transpose of the matrix of signed cofactors.
\end{remark}

\begin{theorem}[Inverse by adjoint matrix]
    Let $A \in M_n(\R)$ be non-singular.
    Then the inverse $A ^ {-1}$ is given by
    \[
    A ^ {-1} = \frac{1}{\det A}\Adj(A).
    \]
\end{theorem}

\begin{theorem}[Cramer's rule]
    Let $A\mbf{x} = \mbf{b}$ be a linear system with a non-singular matrix $A \in M_n(\R)$.
    The unique solution $\begin{pmatrix}
        x_1 \\ \vdots \\ x_n
    \end{pmatrix}$ is given by
    \[
    x_i = \frac{\det(\mbf{a}_1, \dotsc, \mbf{a}_{i - 1}, \mbf{b}, \mbf{a}_{i + 1}, \dotsc, \mbf{a}_n)}{\det A}.
    \]
    Here $\mbf{a}_\ell$ denotes (for a change) the $\ell$-th column vector of $A$.
\end{theorem}

\begin{theorem}[Area of parallelogram]
    For $\mbf{v, w} \in \R ^ 2$ two vectors in the plane,
    form the parallelogram $P$ with vertices $\mbf{0, v, w}$ and $\mbf{v + w}$.
    Then
    \[
    \mathrm{area}(P) = |\det(\mbf{v, w})|.
    \]
\end{theorem}

\begin{proposition}[Area of triangle]
    The triangle with vertices $\mbf{a, b}$ and $\mbf{c}$ has area
    \[
    \frac{1}{2}|\det(\mbf{b - a, c - a})|.
    \]
\end{proposition}

\begin{proposition}[Volume of parallelepiped]
    The volume of the parallelepiped $P$ in $\R ^ 3$ spanned by the vectors $\mbf{u, v, w}$ is given by
    \[
    \mathrm{vol}(P) = |\det(\mbf{u, v, w})|.
    \]
\end{proposition}

\begin{remark}
    The volume of the $n$-dimensional parallelepiped $P$ spanned by the vectors $\mbf{v}_1, \dotsc, \mbf{v}_n$ is given by
    \[
    \mathrm{vol}(P) = |\det(\mbf{v}_1, \dotsc, \mbf{v}_n)|.
    \]
\end{remark}

\newpage

\section{Spanning sets and linear independence in \texorpdfstring{$\R ^ n$}{}}

\begin{definition}[Vector subspace of $\R ^ n$]
    A (non-empty) subset $U$ of $\R ^ n$ is called a vector subspace (or linear subspace) of $\R ^ n$ if it satisfies the following three conditions:
    \begin{enumerate}[label = (\roman*)]
        \item closure under addition:
        if $\mbf{u}_1$ and $\mbf{u}_2$ are in $U$,
        then $\mbf{u}_1 + \mbf{u}_2$ is also in $U$;
        \item closure under scalar multiplication:
        if $\mbf{u}$ is in $U$ and $\lambda$ is in $\R$,
        then $\lambda\mbf{u}$ is in $U$;
        \item existence of an origin:
        the vector $\mbf{0}$ is in $U$.
    \end{enumerate}
\end{definition}

\begin{proposition}
    The solution set of an arbitrary $m \times n$ homogeneous linear system $A\mbf{x} = \mbf{0}$ is a vector subspace of $\R ^ n$.
\end{proposition}

\begin{definition}[Linear combination]
    Let $\mbf{u}_1, \dotsc, \mbf{u}_k$ be vectors in $\R ^ n$.
    A linear combination of these vectors is a vector $\mbf{u}$ of the form
    \[
    \mbf{u} = \lambda_1\mbf{u}_1 + \dotsi + \lambda_k\mbf{u}_k
    \]
    for some (real) scalars $\lambda_1, \dotsc, \lambda_k$. 
\end{definition}

\begin{definition}[Linear span]
    Let $\mbf{u}_1, \dotsc, \mbf{u}_k$ be $k$ vectors in $\R ^ n$.
    Their (linear) span is the subset $U$ of $\R ^ n$ consisting of all linear combinations of these vectors.
    So
    \[
    U = \{\lambda_1\mbf{u}_1 + \dotsi + \lambda_k\mbf{u}_k\quad\lambda_1, \dotsc, \lambda_k \in \R\}.
    \]
    Conversely,
    we say that such a $U$ is spanned by $\mbf{u}_1, \dotsc, \mbf{u}_k$ and write $U = \mathrm{span}(\mbf{u}_1, \dotsc, \mbf{u}_k)$.
    Equivalently,
    $\{\mbf{u}_1, \dotsc, \mbf{u}_k\}$ is called a spanning set of $U$.
\end{definition}

\begin{proposition}
    Let $\mbf{u}_1, \dotsc, \mbf{u}_k \in \R ^ n$ be $k$ vectors,
    and let $A = (\mbf{u}_1, \dotsc, \mbf{u}_k) \in M_{n, k}(\R)$ be the associated matrix of column vectors.
    Then we have
    \[
    \lambda_1\mbf{u}_1 + \dotsi + \lambda_k\mbf{u}_k = A\pmb{\lambda},
    \]
    where $\pmb{\lambda} = \begin{pmatrix}
        \lambda_1 \\ \vdots \\ \lambda_k
    \end{pmatrix} \in \R ^ k$ is the column vector of the $\lambda_j$.
    Moreover,
    we have
    {
    \small
    \[
    span(\mbf{u}_1, \dotsc, \mbf{u}_k) = \{\mbf{u} \in \R ^ n\,:\,\text{The system } A\pmb{\lambda} = \mbf{u}\text{ has a solution}\} = \{A\pmb{\lambda}:\pmb{\lambda} \in \R ^ k\}.
    \]
    }
\end{proposition}

\begin{lemma}
    The linear span $U$ of any collection $\mbf{u}_1, \dotsc, \mbf{u}_k$ of vectors is a vector subspace of $\R ^ n$.
\end{lemma}

\begin{lemma}
    Let $\mbf{u}_1, \dotsc, \mbf{u}_k \in \R ^ n$ and assume that one of the vectors,
    say $\mbf{u}_1$,
    can be expressed as a linear combination of the others.
    Then
    \[
    \mathrm{span}(\mbf{u}_1, \mbf{u}_2, \dotsc, \mbf{u}_k) = \mathrm{span}(\mbf{u}_2, \dotsc, \mbf{u}_k).
    \]
\end{lemma}

\begin{definition}[Linear independence]
    The vectors $\mbf{u}_1, \dotsc, \mbf{u}_k$ are said to be linearly independent if the only solution to the equation
    \[
    \lambda_1\mbf{u}_1 + \dotsi + \lambda_k\mbf{u}_k = \mbf{0}\quad\text{if}\quad\lambda_1 = \lambda_2 = \dotsi = \lambda_k = 0.
    \]
\end{definition}

\begin{corollary}
    Let $\mbf{u}_1, \dotsc, \mbf{u}_k$ be $k$ vectors in $\R ^ n$ and let $A = (\mbf{u}_1, \dotsc, \mbf{u}_k) \in M_{n, k}(\R)$ be the associated matrix of column vectors.
    Then
    {
    \small
    \[
    \mbf{u}_1, \dotsc, \mbf{u}_k\text{ are linear independent} \iff \text{The system } A\pmb{\lambda} = \mbf{0}\text{ has no non-trivial solution}.
    \]
    }
\end{corollary}

\begin{lemma}
    A pair of (non-zero) vectors $\mbf{u}, \mbf{v}$ in $\R ^ n$ are linearly dependent if and only if there is $\lambda \in \R$ so that $\mbf{v} = \lambda\mbf{u}$.
    In particular,
    any two (non-zero) non-collinear vectors are always linear independent.
\end{lemma}

\begin{proposition}
    Let $\mbf{u}_1, \dotsc, \mbf{u}_k$ be $k$ linear independent vectors in $\R ^ n$ and let $\mbf{u} \in \R ^ n$ such that
    \[
    \mbf{u} \notin \mathrm{span}(\mbf{u}_1, \dotsc, \mbf{u}_k).
    \]
    Then
    \[
    \mbf{u}_1, \dotsc, \mbf{u}_k, \mbf{u}\text{ are linear independent}.
    \]
\end{proposition}

\newpage

\section{Real vector spaces}

\begin{definition}[A (real) vector space]
    A (real) vector space is a non-empty set $V$ together with two operations:
    \begin{enumerate}[label = (\roman*)]
        \item Addition:
        This is a function
        \[
        V \times V \rightarrow V
        \]
        (Given any $\mbf{u}$ and $\mbf{v}$ in $V$,
        we write their sum as $\mbf{u + v}$);
        \item Scalar multiplication:
        This is a function
        \[
        \R \times V \rightarrow V
        \]
        (Given $\mbf{u} \in V$ and $\lambda \in \R$ we write the multiplication of $\mbf{u}$ by $\lambda$ as $\lambda\mbf{u}$).
    \end{enumerate}
\end{definition}

\begin{proposition}[Axioms of addition and scalar multiplication for a real vector space]\phantom{}

    \textbf{Axioms for addition:}
    \begin{enumerate}[label = (\roman*)]
        \item Existence of additive identity:
        there is a vector $\mbf{0}$ in $V$ so that $\mbf{v + 0} = \mbf{v} = \mbf{0 + v}$ for all $\mbf{v}$ in $V$;
        \item Commutativity:
        $\mbf{v + w} = \mbf{w + v}$ for all $\mbf{v}$ and $\mbf{w}$ in $V$;
        \item Existence of additive inverses:
        for each $\mbf{v}$ in $V$ there is an element $-\mbf{v}$ in $V$ such that $\mbf{v} + (-\mbf{v}) = \mbf{0} = (-\mbf{v}) + \mbf{v}$;
        \item Associativity:
        $\mbf{u} + (\mbf{v + w}) = (\mbf{u + v}) + \mbf{w}$ for all $\mbf{u, v, w} \in V$.
    \end{enumerate}

    \textbf{Axioms for scalar multiplication:}
    \begin{enumerate}[label = (\roman*)]
        \item $0\mbf{v} = \mbf{0}$ for all $\mbf{v}$ in $V$;
        \item $1\mbf{v} = \mbf{v}$ for all $\mbf{v}$ in $V$;
        \item Associativity:
        $(\lambda\mu)\mbf{v} = \lambda(\mu\mbf{v})$ for all $\lambda, \mu$ in $\R$ and for all $\mbf{v}$ in $V$;
        \item Distributivity:
        For all $\lambda, \mu \in \R$ and for all $\mbf{v, w} \in V$ we have
        \begin{enumerate}[label = \alph*)]
            \item $(\lambda + \mu)\mbf{v} = \lambda\mbf{v} + \mu\mbf{v}$;
            \item $\lambda(\mbf{v + w}) = \lambda\mbf{v} + \lambda\mbf{w}$.
        \end{enumerate}
    \end{enumerate}
\end{proposition}

\begin{definition}[Vector subspace]
    A vector subspace of a vector space $V$ is a non-empty subset that is itself a vector space with respect to the same addition and multiplication operations of $V$.
\end{definition}

\begin{proposition}[Subspace criterion]
    A non-empty subset $U$ of a vector space $V$ is a vector subspace of $V$ if and only if it satisfies the following three conditions:
    \begin{enumerate}[label = (\roman*)]
        \item closure under addition:
        if $\mbf{u}_1$ and $\mbf{u}_2$ are in $U$,
        then $\mbf{u}_1 + \mbf{u}_2$ is also in $U$;
        \item closure under scalar multiplication:
        if $\mbf{u}$ is in $U$ and $\lambda$ is in $\R$,
        then $\lambda\mbf{u}$ is in $U$;
        \item existence of an origin:
        the vector $\mbf{0}$ is in $U$.
    \end{enumerate}
\end{proposition}

\begin{definition}[Basis]
    A (finite) basis for a vector space $V$ is a finite subset of vectors $\{\mbf{v}_1, \mbf{v}_2, \dotsc, \mbf{v}_k\}$ in $V$ such that
    \begin{enumerate}[label = (\roman*)]
        \item the vectors $\mbf{v}_1, \mbf{v}_2, \dotsc, \mbf{v}_k$ are linearly independent,
        and
        \item $V = \mathrm{span}(\mbf{v}_1, \mbf{v}_2, \dotsc, \mbf{v}_k)$.
    \end{enumerate}
\end{definition}

\begin{lemma}
    Let $\mbf{u}_1, \dotsc, \mbf{u}_k$ be $k$ vectors in a vector space $V$.
    Then $\mathrm{span}(\mbf{u}_1, \dotsc, \mbf{u}_k)$ is a vector subspace of $V$.
\end{lemma}

\begin{lemma}
    Let $\mbf{u}_1, \dotsc, \mbf{u}_k$ be vectors in a vector space $V$ and assume that one of the vectors,
    say $\mbf{u}_1$,
    can be expressed as a linear combination of the others.
    Then
    \[
    \mathrm{span}(\mbf{u}_1, \mbf{u}_2, \dotsc, \mbf{u}_k) = \mathrm{span}(\mbf{u}_2, \dotsc, \mbf{u}_k).
    \]
\end{lemma}

\begin{lemma}
    Suppose that the set $\{\mbf{u}_1, \dotsc, \mbf{u}_k\}$ is not linearly independent.
    Then for some $1 \leq i \leq k$,
    we have that
    \[
    \mbf{u}_i \in \mathrm{span}(\mbf{u}_1, \dotsc, \mbf{u}_{i - 1}, \mbf{u}_{i + 1}, \dotsc, \mbf{u}_k).
    \]
    Then in particular
    \[
    \mathrm{span}(\mbf{u}_1, \mbf{u}_2, \dotsc, \mbf{u}_k) = \mathrm{span}(\mbf{u}_1, \dotsc, \mbf{u}_{i - 1}, \mbf{u}_{i + 1}, \dotsc, \mbf{u}_k).
    \]
\end{lemma}

\begin{lemma}
    Let $\mbf{u}_1, \dotsc, \mbf{u}_k$ be $k$ linearly independent vectors in a vector space $V$ and let $\mbf{u} \in V$ such that
    \[
    \mbf{u} \notin \mathrm{span}(\mbf{u}_1, \dotsc, \mbf{u}_k).
    \]
    Then
    \[
    \mbf{u}_1, \dotsc, \mbf{u}_k, \mbf{u}\text{ are linearly independent}.
    \]
\end{lemma}

\begin{theorem}[Existence of finite basis]
    Every vector space that is spanned by finitely many vectors has a basis.
\end{theorem}

\begin{theorem}["What is a basis?"]
    Let $\mbf{u}_1, \dotsc, \mbf{u}_k$ be vectors in a vector space $V$.
    Then the following statements are
    (pairwise)
    equivalent:
    \begin{enumerate}[label = (\roman*)]
        \item The vectors $\mbf{u}_1, \dotsc, \mbf{u}_k$ form a basis of $V$.
        \item The vectors $\mbf{u}_1, \dotsc, \mbf{u}_k$ form a maximal linear independent set in $V$,
        i.e., one cannot add another vector in $V$ to the $\mbf{u}_i$,
        and still obtain a linearly independent set.
        \item Any vector $\mbf{u}$ in $V$ can be written uniquely as
        \[
        \mbf{u} = \lambda_1\mbf{u}_1 + \lambda_2\mbf{u}_2 + \dotsi + \lambda_k\mbf{u}_k.
        \]
        \item The vectors $\mbf{u}_1, \dotsc, \mbf{u}_j$ form a minimal spanning set of $V$;
        i.e., one cannot remove any of the $\mbf{u}_i$ and still have a spanning set of $V$.
    \end{enumerate}
\end{theorem}

\begin{theorem}[Steinitz Exchange Theorem]
    Let $S = \{\mbf{v}_1, \dotsc, \mbf{v}_k\}$ be a spanning set of a vector space $V = \mathrm{span}(S)$.
    Let $\{\mbf{u}_1, \dotsc, \mbf{u}_{\ell}\} \subseteq V$ be a linearly independent set.

    Then there exist $\ell$ distinct vectors $\mbf{v}_{i_1}, \mbf{v}_{i_2}, \dotsc, \mbf{v}_{i_{\ell}} \in S$ such that for all $j$ with $0 \leq j \leq \ell$ we have
    \[
    (S \setminus \{\mbf{v}_{i_1}, \dotsc, \mbf{v}_{i_{\ell}}\}) \cup \{\mbf{u}_1, \dotsc, \mbf{u}_j\}
    \]
    is a spanning set for $V$.
\end{theorem}

\begin{corollary}
    In the situation above $k \geq \ell$.
    In other words,
    a spanning set for a vector space is always at least as big as any linearly independent set within the vector space.
\end{corollary}

\begin{corollary}
    Let $V$ be a finite-dimensional vector space.
    Then every basis for $V$ has the same number of elements.
\end{corollary}

\begin{definition}[Dimension]
    Let $V$ be a vector space that is generated by finitely many vectors.
    The dimension,
    $\dim{(V)}$,
    of $V$ is defined to be the cardinality of any basis for $V$.
\end{definition}

\begin{theorem}
    Let $V$ be a finite dimensional vector space and $U$ a vector subspace of $V$.
    Then
    \begin{enumerate}[label = (\roman*)]
        \item $U$ is also finite-dimensional;
        \item $\dim{(U)} \leq \dim{(V)}$;
        \item $\dim{(U)} = \dim{(V)} \iff U = V$.
    \end{enumerate}
\end{theorem}

\begin{theorem}
    Let $V$ be a vector space of dimension $k$ and let $\mbf{v}_1, \dotsc, \mbf{v}_{\ell}$ be $\ell$ vectors in $V$.
    Then
    \begin{enumerate}[label = (\roman*)]
        \item If $\ell > k$,
        then the $\mbf{v}_1, \dotsc, \mbf{v}_{\ell}$ are linearly dependent.
        \item If $\ell < k$,
        then the $\mbf{v}_1, \dotsc, \mbf{v}_{\ell}$ do not span $V$.
        \item If $k = \ell$,
        then the following statements are (pairwise) equivalent
        \begin{enumerate}[label = (\alph*)]
            \item The vectors $\mbf{v}_1, \dotsc, \mbf{v}_k$ form a basis of $V$.
            \item The vectors $\mbf{v}_1, \dotsc, \mbf{v}_k$ are linear independent.
            \item The vectors $\mbf{v}_1, \dotsc, \mbf{v}_k$ span $V$.
        \end{enumerate}
    \end{enumerate}
\end{theorem}

\begin{definition}[Standard basis of $\R ^ n$]
    \[
    \mbf{e}_1 = \begin{pmatrix}
        1 \\ 0 \\ \vdots \\ 0
    \end{pmatrix},
    \quad
    \mbf{e}_2 = \begin{pmatrix}
        0 \\ 1 \\ \vdots \\ 0
    \end{pmatrix},
    \quad
    \dotsc,
    \quad
    \mbf{e}_n = \begin{pmatrix}
        0 \\ 0 \\ \vdots \\ n
    \end{pmatrix}.
    \]
\end{definition}

\begin{theorem}
    Let $\mbf{u}_1, \dotsc, \mbf{u}_k$ be $k$ vectors in $\R ^ n$.
    Then
    \begin{enumerate}[label = (\roman*)]
        \item If $k > n$,
        then the $\mbf{u}_1, \dotsc, \mbf{u}_k$ are linearly dependent.
        \item If $k < n$,
        then the $\mbf{u}_1, \dotsc, \mbf{u}_k$ do not span $\R ^ n$.
        \item If $k = n$,
        then
        \begin{align*}
            \mbf{u}_1, \dotsc, \mbf{u}_n\text{ form a basis of } \R ^ n &\iff\text{The matrix $(\mbf{u}_1, \dotsc, \mbf{u}_n)$ is non singular} \\
            &\iff \det(\mbf{u}_1, \dotsc, \mbf{u}_n) \neq 0.
        \end{align*}
        So,
        the vectors $\mbf{u}_1, \dotsc, \mbf{u}_k$ form a basis for $\R ^ n$ if and only if $k = n$ and the square matrix $(\mbf{u}_1, \dotsc, \mbf{u}_k)$ is non-singular.
    \end{enumerate}
\end{theorem}

\begin{definition}
    Let $A \in M_{n \times k}(\R)$ be a $n \times k$ matrix.
    \begin{enumerate}[label = (\roman*)]
        \item The column space of $A$ is the span of the columns of $A$.
        The column rank of $A$ is the dimension of the column space of $A$.

        The column space is the subspace of all $\mbf{b} \in \R ^ n$ such that $A\mbf{x} = \mbf{b}$ has a solution.

        \item The row space of $A$ is the span of the rows of $A$.
        The row rank of $A$ is the dimension of the row space of $A$.

        The row space is the subspace of all $c \in \R ^ k$ such that $A ^ t\mbf{y} = \mbf{c}$ has a solution.

        \item The nullspace
        (or kernal)
        of $A$ is the solution space to the homogeneous system $A\mbf{x} = \mbf{0}$.
        The nullity of $A$ is the dimension of the nullspace.
    \end{enumerate}
\end{definition}

\begin{lemma}\label{pre_linalg_lem_linindepvecmulmatrisindep}
    Let $\mbf{v}_1, \dotsc, \mbf{v}_k$ be linear independent vectors in $\R ^ n$ and let $P \in M_n(\R)$ be an invertible matrix.
    Then the vectors $P\mbf{v}_1, \dotsc,P\mbf{v}_k$ are linear independent.
\end{lemma}

\begin{proposition}\label{pre_linalg_prop_invmatrsameranksandnul}
    Let $A \in M_{n \times k}(\R)$ and let $P \in M_{n}(\R)$ be an invertible matrix.
    Then $A$ and $PA$ have the same column rank,
    row rank,
    and nullity.
    In particular,
    elementary row operations do not change column rank,
    row rank,
    and nullity.
    More precisely,
    \begin{enumerate}
        \item $A$ and $PA$ have the same row space.
        In particular,
        ERO's do not change the row space.
        \item $PA$ may have a different column space than $A$,
        but the dimension stays the same.
        In particular,
        ERO's do not change the dimension of the column space.
        \item $A$ and $PA$ have the same nullsapce.
        In particular,
        ERO's do not change the nullspace.
    \end{enumerate}
\end{proposition}

\begin{theorem}[Rank Theorem]
    The row rank of a matrix is equal to its column rank.
\end{theorem}

\begin{definition}[Rank]
    We define the rank of a matrix $A$ to be its $\text{row rank} = \text{column rank}$.
    We write $\mathrm{rank}(A) = \mathrm{rk}(A)$.
\end{definition}

\begin{theorem}[Rank-Nullity Theorem]
    Let $A \in M_{n \times k}(\R)$ be an $n \times k$ matrix.
    Then
    \[
    \mathrm{rk}(A) + \mathrm{nullity}(A) = k.
    \]
\end{theorem}

\begin{corollary}
    For any matrix $A \in M_{n \times k}(\R)$ we have
    \begin{enumerate}[label = (\roman*)]
        \item The rank of $A$ is the number of leading $1$'s in the RREF of $A$.
        \item The nullity of $A$ is the number columns in the RREF of $A$ without a leading $1$;
        i.e.,
        the number of free variables in the solution set to $A\mbf{x} = \mbf{0}$.
    \end{enumerate}
\end{corollary}

\begin{definition}
    Let $V$ be a vector space and let $U, W$ be subspaces.
    Then define
    \begin{align*}
        U + W &:= \{\mbf{v} \in V\,:\, \mbf{v} = \mbf{u + w}\text{ for some }\mbf{u} \in U, \mbf{w} \in W\}, \\
        U \cap W &:= \{\mbf{v} \in V\,:\, \mbf{v} \in U\text{ and } \mbf{v} \in W\}.
    \end{align*}
    Both $U + W$ and $U \cap W$ are vector subspaces of $V$.
\end{definition}

\begin{proposition}
    Let $U$ and $W$ be finite-dimensional vector subspaces of a vector space $V$.
    Then
    \[
    \dim(U) + \dim(W) = \dim(U + W) + \dim(U \cap W).
    \]
\end{proposition}

\begin{definition}[Direct Sum]
    Suppose that $U$ and $W$ are vector subspaces of $V$ with
    \begin{enumerate}[label = (\roman*)]
        \item $U + W = V$;
        \item $U \cap W = \{\mbf{0}\}$.
    \end{enumerate}
    Then,
    we say $V$ is the direct sum of $U$ and $W$ and write $V = U \oplus W$.
\end{definition}

\begin{proposition}
    Let $U$ and $W$ be two subspaces of a vector space $V$.
    Then the following statements are pairwise equivalent:
    \begin{enumerate}[label = (\roman*)]
        \item $V = U \oplus W$;
        \item If $\{\mbf{u}_1, \dotsc, \mbf{u}_s\}$ is a basis for $U$ and $\{\mbf{w}_1, \dotsc, \mbf{w}_t\}$ is a basis for $W$,
        then $\{\mbf{u}_1, \dotsc, \mbf{u}_s, \mbf{w}_1, \dotsc, \mbf{w}_t\}$ is a basis for $V$;
        \item $V = U + W$ and $\dim{V} = \dim{U} + \dim{W}$;
        \item Every $\mbf{v} \in V$ can be written uniquely as $\mbf{v} = \mbf{u} + \mbf{w}$ where $\mbf{u} \in U$ and $\mbf{w} \in W$.
    \end{enumerate}
\end{proposition}

\begin{lemma}
    Consider a basis $\mbf{v}_1, \mbf{v}_2, \dotsc, \mbf{v}_n$ of $\R ^ n$,
    and let $P = (\mbf{v}_1, \mbf{v}_2, \dotsc, \mbf{v}_n)$ be the associated matrix of column vectors.
    Then the coordinates $(x_1, \dotsc, x_n)$ of $\mbf{u}$ with respect to $\mbf{v}_1, \mbf{v}_2, \dotsc, \mbf{v}_n$ are given by
    \[
    \begin{pmatrix}
        x_1 \\ \vdots \\ x_n
    \end{pmatrix} = P ^ {-1}\mbf{u}.
    \]
    Note that if $\{\mbf{v}_1, \mbf{v}_2, \dotsc, \mbf{v}_n\} = \{\mbf{e}_1, \dotsc, \mbf{e}_n\}$ then $P = I_n$.
\end{lemma}

\newpage

\section{Linear maps}

\begin{definition}[Linear map]
    Let $V$ and $W$ be two vector spaces,
    with addition denoted by $+_{V}$ and $+_{W}$ respectively.
    A linear map $T : V \rightarrow W$ is a function satisfying the following properties
    \begin{enumerate}[label = (\roman*)]
        \item for all $\mbf{u}$ and $\mbf{v}$ in $V$ we have $T(\mbf{u} +_{V} \mbf{v}) = T(\mbf{u}) +_{W} T(\mbf{v})$;
        \item for all $\mbf{v}$ in $V$ and all $\lambda \in \R$ we have $T(\lambda\mbf{v}) = \lambda T(\mbf{v})$.
    \end{enumerate}
\end{definition}

\begin{lemma}
    Let $\{\mbf{v}_1, \dotsc, \mbf{v}_n\}$ be a basis of a vector space $V$.
    Then
    \begin{enumerate}[label = (\roman*)]
        \item Any linear map $T : V \rightarrow W$ of vector spaces is determined by its values $T(\mbf{v}_1), \dotsc, T(\mbf{v}_n)$ on the basis vectors of $V$.
        \item Conversely,
        given arbitrary vectors $\mbf{w}_1, \dotsc, \mbf{w}_n$ in $W$,
        then there exists a linear map $T : V \rightarrow W$ such that
        \[
        T(\mbf{v}_1) = \mbf{w}_1, \dotsc, T(\mbf{v}_n) = \mbf{w}_n.
        \]
    \end{enumerate}
\end{lemma}

\begin{lemma}
    A map $T : \R ^ n \rightarrow \R ^ m$ is linear if and only if there is a $m \times n$ matrix $A$ such that
    \[
    T(\mbf{x}) = A\mbf{x}.
    \]
    Moreover,
    $A$ is given by
    \[
    A = (T(\mbf{e}_1), T(\mbf{e}_2), \dotsi, T(\mbf{e}_n)).
    \]
    So the $i$th column is the image of the standard basis vector $\mbf{e}_i$.
\end{lemma}

\begin{lemma}
    Let $T : V \rightarrow W$ and $S : W \rightarrow U$ be two linear maps of vector spaces.
    Then the composition $S \circ T : V \rightarrow U$ is a linear map.
    In particular,
    if $T : \R ^ n \rightarrow \R ^ m$ and $S : \R ^ m \rightarrow \R ^ k$ are linear maps given by $T(\mbf{x}) = A\mbf{x}$ and $S(\mbf{y}) = B\mbf{y}$,
    then the composition $S \circ T : \R ^ n \rightarrow \R ^ k$ is again linear,
    and moreover we have
    \[
    (S \circ T)(\mbf{x}) = (BA)\mbf{x}.
    \]
    So the composition of linear maps corresponds to matrix multiplication.
\end{lemma}

\begin{definition}[Isomorphism]
    Let $T : V \rightarrow W$ be a bijective linear map of vector spaces
    (that is,
    a linear map which is one-to-one and onto).
    Then we call $T$ an isomorphism and call $V$ and $W$ isomorphic.
    We write $V \cong W$.
\end{definition}

\begin{lemma}
    Let $T : V \rightarrow W$ be an isomorphism of vector spaces.
    Then the inverse map $T ^ {-1} : W \rightarrow V$ is also a linear map,
    in fact an isomorphism.
    For the linear map $T : \R ^ n \rightarrow \R ^ n$ given by the matrix $A \in M_n(\R)$,
    we have that $T$ is an isomorphism if and only if $A$ is invertible.
    In that case,
    we have
    \[
    T ^ {-1}(\mbf{y}) = A ^ {-1}\mbf{y}.
    \]
\end{lemma}

\begin{lemma}
    Let $V$ b a vector space of dimension $n$.
    Then the coordinate map $\Phi : V \rightarrow \R ^ n$ is an isomorphism.
    In particular,
    every finite dimensional vector space is isomorphic to the Euclidean space of the same dimension.
\end{lemma}

\begin{theorem}
    Let $V$ be a vector space of dimension $n$ and let $W$ be a vector space of dimension $m$.
    Let $\phi : V \rightarrow \R ^ n$ and $\psi : W \rightarrow \R ^ m$ be two coordinate maps with respect to the choice of some bases $V$ and $W$ respectively.
    Let $T : V \rightarrow W$ be a linear map.
    Then there exists a $m \times n$ matrix $A$ so that for all $\mbf{v} \in V$ we have
    \[
    \phi(T(\mbf{v})) = A\phi(\mbf{v}).
    \]
    We say $T$ is represented by $A$.
    (The matrix $A$ depends on the choice of the bases made.)
\end{theorem}

\begin{definition}[Image and kernel]
    Let $V$ and $W$ be vector spaces and $T : V \rightarrow W$ be a linear map.
    We define the image of $T$ by
    \[
    \mathrm{im}(T) = \{\mbf{w} \in W : \text{there exists $\mbf{v} \in V$ so that } \mbf{w} = T(\mbf{v})\}.
    \]
    We define the kernel of $T$ by
    \[
    \ker(T) = \{\mbf{v} \in V : T(\mbf{v}) = \mbf{0}\}.
    \]
\end{definition}

\begin{lemma}
    Let $T : V \rightarrow W$ be a linear map of vector spaces.
    Then $\mathrm{im}(T)$ is a vector subspace of $W$ and $\ker(T)$ is a vector subspace of $V$.
\end{lemma}

\begin{lemma}
    Let $T : V \rightarrow W$ be a linear map of vector spaces,
    and let $\{\mbf{v}_1, \dotsc, \mbf{v}_n\}$ be a basis for $V$.
    Then the image $\mathrm{im}(T)$ is spanned by $T(\mbf{v}_1), \dotsc, T(\mbf{v}_n)$.
\end{lemma}

\begin{proposition}
    Let $T : V \rightarrow W$ be a linear map of vector spaces.
    Then the following statements are
    (pairwise)
    equivalent:
    \begin{enumerate}[label = (\roman*)]
        \item $\ker(T) = \{\mbf{0}\}$;
        \item $T$ is one-to-one
        (that is,
        it is injective);
        \item If $\mbf{v}_1, \dotsc, \mbf{v}_k$ are linear independent vectors in $V$,
        then the images $T(\mbf{v}_1), \dotsc, T(\mbf{v}_k)$ are linearly independent in $W$.
    \end{enumerate}
\end{proposition}

\begin{corollary}
    Let $T : V \rightarrow W$ be a linear map of vector spaces with $\ker(T) = \{\mbf{0}\}$.
    Assume $\dim{V} = n$ with a basis $\{\mbf{v}_1, \dotsc, \mbf{v}_n\}$.
    Then the images $\{T(\mbf{v}_1), \dotsc, T(\mbf{v}_n)\}$ are a basis for $\mathrm{im}(T)$.
    In particular,
    we have $\dim(\mathrm{im}(T)) = \dim(V)$.
\end{corollary}

\begin{definition}[Rank and nullity of a linear map]
    Let $V$ and $W$ be vector spaces and $T : V \rightarrow W$ be a linear map.
    We define the rank of $T$ to be
    \[
    \mathrm{rk}(T) = \dim(\mathrm{im}(T)).
    \]
    We define the nullity of $T$ to be
    \[
    n(T) = \dim(\ker(T)).
    \]
\end{definition}

\begin{proposition}
    Let $T : V \rightarrow W$ be a linear map of finite-dimensional vector spaces and let $\phi : V' \rightarrow V$ and $\psi : W \rightarrow W'$ be two isomorphisms.
    Then
    \[
    \mathrm{rk}(\psi \circ T \circ \phi) = \mathrm{rk}(T)\quad\text{and}\quad n(\psi \circ T \circ \phi) = n(T).
    \]
\end{proposition}

\begin{theorem}[Rank-Nullity Theorem for linear maps]
    Let $V$ and $W$ be two finite-dimensional vector spaces and $T : V \rightarrow W$ be a linear map.
    Then
    \[
    \dim(V) = \mathrm{rk}(T) + n(T).
    \]
\end{theorem}

\begin{theorem}
    For two finite-dimensional vector spaces $V$ and $W$ we have
    \[
    V \cong W \iff \dim{V} = \dim{W}.
    \]
\end{theorem}

\begin{theorem}
    Let $T : \R ^ n \rightarrow \R ^ m$ be a linear map.
    Suppose that with respect to the standard bases $T$ is represented by the matrix $A$.
    Let $S = \{\mbf{v}_1, \dotsc, \mbf{v}_n\}$ be any basis of $\R ^ n$ and let $P = (\mbf{v}_1, \dotsi, \mbf{v}_n)$.
    Let $S' = \{\mbf{w}_1, \dotsc, \mbf{w}_m\}$ be any basis of $\R ^ m$ and let $Q = (\mbf{w}_1, \dotsc, \mbf{w}_m)$.
    Then with respect to the bases $S$ of $\R ^ n$ and $S'$ of $\R ^ m$ the transformation $T$ is represented by the matrix $B = Q ^ {-1}AP$.
\end{theorem}

\begin{corollary}
    Suppose that $\{\mbf{v}_1, \dotsc, \mbf{v}_n\}$ and $\{\mbf{v}'_1, \dotsc, \mbf{v}'_n\}$ are any bases of $\R ^ n$.
    Let $P$ be a matrix so that $\mbf{v}'_j = P\mbf{v}_j$.
    Suppose that $\{\mbf{w}_1, \dotsc, \mbf{w}_m\}$ and $\{\mbf{w}'_1, \dotsc, \mbf{w}'_m\}$ are any bases of $\R ^ m$.
    Let $Q$ be a matrix so that $\mbf{w}'_j = Q\mbf{w}_j$.

    Let $T : \R ^ n \rightarrow \R ^ m$ be a linear map.
    Suppose that with respect to the bases $\{\mbf{v}_1, \dotsc, \mbf{v}_n\}$ and $\{\mbf{w}_1, \dotsc, \mbf{w}_m\}$ the map $T$ is represented by the matrix $A$.
    Then with respect to the bases $\{\mbf{v}'_1, \dotsc, \mbf{v}'_n\}$ and $\{\mbf{w}'_1, \dotsc, \mbf{w}'_m\}$ the map $T$ is represented by the matrix $B = Q ^ {-1}AP$.
\end{corollary}

\newpage

\section{Methods}

\subsection{Finding the kernel,
rank and nullity of a linear map/matrix}

Let $A \in M_{n \times k}(\R)$
\begin{definition}[Column space]
    The column space of $A$ is the span of the columns of $A$.
    \textit{Note:
    the column space is the subspace of all $\mbf{b} \in \R ^ n$ such that $A\mbf{x} = \mbf{b}$ has a solution}.
\end{definition}

\begin{definition}[Row space]
    The row space of $A$ is the span of the rows of $A$.
    \textit{Note:
    the row space is the subspace of all $\mbf{c} \in \R ^ k$ such that $A ^ t\mbf{y} = \mbf{c}$ has a solution}.
\end{definition}

\begin{definition}[Nullspace]
    The nullspace
    (or kernel)
    of $A$ is the solution space to the homogeneous system $A\mbf{x} = \mbf{0}$.
\end{definition}

\begin{definition}[Nullity]
    The nullity of $A$ is the dimension of the nullspace
    (kernel).
\end{definition}

\subsubsection{Finding the column space
(and row space)}

To find the column space,
simply find the span of the columns of $A$
\begin{example}
    for a $2 \times 2$ matrix $\begin{pmatrix}
        1 & 2 \\ 3 & 1
    \end{pmatrix}$ we can find the column space
    \[
    \begin{pmatrix}
        1 & 2 \\ 3 & 1
    \end{pmatrix}\begin{pmatrix}
        x \\ y
    \end{pmatrix} = \begin{pmatrix}
        a \\ b
    \end{pmatrix} \implies \begin{pmatrix}
        x \\ y
    \end{pmatrix} = \begin{pmatrix}
        \frac{11}{5} \\ -\frac{3}{5}
    \end{pmatrix}a + \begin{pmatrix}
        -\frac{2}{5} \\ \frac{1}{5}
    \end{pmatrix}b
    \]
    meaning that we can the column space
    \[
    \mathrm{colspace}(A) = \mathrm{span}\left\{\begin{pmatrix}
        \frac{11}{5} \\ -\frac{3}{5}
    \end{pmatrix}, \begin{pmatrix}
        -\frac{2}{5} \\ \frac{1}{5}
    \end{pmatrix}\right\}
    \]
    giving us $\mathrm{colrank}(A) = 2$.
\end{example}

We can do the same for row rank by using $A ^ t\mbf{x} = \mbf{a}$.

\subsubsection{Finding the kernel and image of a linear map}

To find the kernel of a linear map we get the following from the definition of the kernel
\[
\mathrm{ker}(T) = \{\mbf{v} \in V\,:\,T(\mbf{v} = \mbf{0}\} \implies T(\mbf{v}) = \mbf{0}
\]
the definition implies we just have to find the vectors $\mbf{v}$ which map to $\mbf{0}$.

For the image we would use the definition
\[
\mathrm{im}(T) = \{\mbf{w} \in W\,:\,\exists\mbf{v} \in V,\,\mbf{w} = T(\mbf{v})\} \implies \mbf{w} = T(\mbf{v})
\]
which means we would need to find a $\mbf{v}$ such that for all $\mbf{w} \in W$ we get $\mbf{w} = T(\mbf{v})$.

\subsection{Diagonalisation and powers of matrices}

To diagonalise a matrix $A$ we do the following
\begin{enumerate}[label = (\arabic*)]
    \item
    Find the characteristic polynomial of the matrix,
    i.e. $p_A(t) = \det(A - tI)$.

    \item
    Solve this polynomial for when $p_A(t) = 0$ to get the eigenvalues $\lambda_1, \dotsc, \lambda_n$.

    \item
    Now go through all the eigenvalues $\lambda_i$ to solve $(A - \lambda_iI)\mbf{x} = \mbf{0}$ which gets us the eigenvectors of the matrix $A$.

    \item
    Finally we get a diagonal matrix $D = \begin{pmatrix}
        \lambda_1 & 0 & \dotsi & 0 \\
        0 & \lambda_2 & \dotsi & 0 \\
        \vdots & \vdots & \ddots & \vdots \\
        0 & 0 & \dotsi & \lambda_n
    \end{pmatrix}$
    and a matrix $P$ whose columns are the eigenvectors.

    \item
    We get the final relation $A = P ^ {-1}DP$.
\end{enumerate}

To find powers of our newly diagonalised matrix $A$,
after manipulation we see
\[
A ^ k = P ^ {-1}D ^ kP.
\]

Also
\begin{theorem}[Cayley-Hamilton theorem]
    Let $A$ be $n \times n$ with characteristic polynomial $p_A(t)$.
    Then $p_A(A) = 0$
    (as a matrix equation).
\end{theorem}

\textit{Note:
we can show equality of matrices by showing that if $A = B$ then $p_A(A) = 0 = p_A(B)$.}

\subsection{Solving a system of differential equations with diagonalisation}

\begin{example}
    Solve the system of ODEs
    \begin{align*}
        \dot{x} &= -x + 4y \\
        \dot{y} &= -2x + 5y
    \end{align*}
    where $x = x(t)$,
    $y = y(t)$,
    $\dot{x} = \frac{dx}{dt}$,
    $\dot{y} = \frac{dy}{dt}$.

    \begin{solution}
        Let $M = \begin{pmatrix}
            -1 & 4 \\ -2 & 5
        \end{pmatrix}$
        (the system of equations).
        
        Denote $\mbf{v}(t) = \begin{pmatrix}
            x(t) \\ y(t)
        \end{pmatrix}$.

        Put $\mbf{w}(t) = M ^ {-1}\mbf{v}(t)$ which is equivalent to $\mbf{v}(t) = M\mbf{w}(t)$.
        Have $\mbf{w}(t)$ be the vector of solutions.
        
        Then we have that $\mbf{w}(t)$ satisfies the system $\dot{\mbf{w}}(t) = D\mbf{w}(t)$,
        with $D = \begin{pmatrix}
            1 & 0 \\ 0 & 3
        \end{pmatrix}$.
        \textit{Note:
        the intermediate step
        \[
        \dot{\mbf{w}}(t) = D\mbf{w}(t) \iff \begin{pmatrix}
            \dot{w}_1(t) \\ \dot{w}_2(t)
        \end{pmatrix} = \begin{pmatrix}
            1 & 0 \\ 0 & 3
        \end{pmatrix}\begin{pmatrix}
            w_1(t) \\ w_2(t)
        \end{pmatrix} \implies \dot{w}_1(t) = w_1(t) \text{ and } \dot{w}_2(t) = 3w_2(t)
        \]
        then solve the two independent differential equations.}
        Giving us the solution $w_1(t) = c_1e ^ t$,
        $w_2(t) = c_2e ^ {3t}$.
        Finally $\mbf{v}(t) = M\mbf{w}(t)$ gives
        \begin{align*}
            x(t) &= 2c_1e ^ t + c_2e ^ {3t} \\
            y(t) &= c_1e ^ t + c_2e ^ {3t}.
        \end{align*}
    \end{solution}
\end{example}









\end{document}