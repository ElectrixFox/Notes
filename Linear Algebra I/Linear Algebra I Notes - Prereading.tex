\documentclass[10pt, a4paper]{article}
\usepackage{preamble}

\newcommand{\mbf}[1]{\mathbf{#1}}

\title{Linear Algebra I \\
    \large Prereading}
\author{Luke Phillips}
\date{October 2024}

\begin{document}

\maketitle

\newpage

\tableofcontents

\newpage

\section{The vector space \(\R ^ n\)}
\begin{definition}[$n$-dimensional real space]
    The space $\R ^ n$ ($n$-dimensional real space) is the collection of all (column) vectors of the form
    \[
    \mathbf{x} =
    \begin{pmatrix}
        x_1 \\
        \vdots \\
        x_n
    \end{pmatrix}.
    \]
\end{definition}
Vectors are typically denoted by bold letters such as $\mathbf{x}$ and sometimes using underlined letters $\underline{x}$ to differentiate them from ordinary real numbers.

We write $\mathbf{0}$ for the zero-vector
\[
\mathbf{0} = \begin{pmatrix}
    0 \\
    \vdots \\
    0
\end{pmatrix}
\]
this is the point of origin of $\R ^ n$.

There are two important operations which can be performed on vectors.

\textbf{Addition}: for any vectors $\mathbf{v}$ and $\mathbf{w}$, we define
\[
\mathbf{v + w} =
\begin{pmatrix}
    v_1 \\
    \vdots \\
    v_n
\end{pmatrix}
+
\begin{pmatrix}
    w_1 \\
    \vdots \\
    w_n
\end{pmatrix}
=
\begin{pmatrix}
    v_1 + w_1 \\
    \vdots \\
    v_n + w_n
\end{pmatrix}
\]
this is component-wise addition.


\textbf{Scalar multiplication}: for any vector $\mathbf{v} \in \R ^ n$ and real number $\lambda \in \R$ we define
\[
\lambda\mathbf{v} =
\lambda\begin{pmatrix}
    v_1 \\
    \vdots \\
    v_n
\end{pmatrix}
=
\begin{pmatrix}
    \lambda v_1 \\
    \vdots \\
    \lambda v_n
\end{pmatrix}
\]
this is component-wise multiplication by $\lambda$.

\begin{proposition}
    These operations satisfy the axioms of a real vector space
\end{proposition}

{\large \textbf{Axioms for Addition}}
\begin{enumerate}[label = (\roman*)]
    \item existence of additive identity: there is a vector $\mathbf{0}$ in $\R ^ n$ so that $\mathbf{v} + \mathbf{0} = \mathbf{v} = \mathbf{0} + \mathbf{v}$ for all $\mathbf{v}$ in $\R ^ n$.
    \item commutativity: $\mbf{v + w = w + v}$ for all $\mathbf{v}$ and $\mathbf{w}$ in $\R ^ n$
    \item existence of additive inverses: for each $\mbf{v}$ in $\R ^ n$ there is an element $-\mbf{v}$ in $\R ^ n$ such that $\mbf{v + (-v) = 0 = (-v) + v}$
    \item associativity: $\mbf{u + (v + w) = (u + v) + w}$ for all $\mbf{u, v, w}$ in $\R ^ n$
\end{enumerate}
these axioms can be summarised by saying that $\R ^ n$ is an abelian group under addition.


{\large\textbf{Axioms for Scalar Multiplication}}
\begin{enumerate}[label = (\roman*)]
    \item multiplication by 0: $0\mbf{v = 0}$ for all $\mbf{v}$ in $\R ^ n$
    \item multiplication by 1: $1\mbf{v = v}$ for all $\mbf{v}$ in $\R ^ n$
    \item associativity: $(\lambda\mu)\mbf{v} = \lambda (\mu\mbf{v})\ \forall \lambda, \mu \in \R$ and $\forall \mbf{v} \in \R ^ n$
    \item distributivity: $(\lambda + \mu)\mbf{v} = \lambda\mbf{v} + \mu\mbf{v}$ and $\lambda (\mbf{v + w}) = \lambda\mbf{v} + \lambda\mbf{w}$ $\forall \lambda, \mu \in \R$ and $\forall \mbf{v, w} \in \R ^ n$
\end{enumerate}

\begin{proof}
    The above axioms ate trivially satisfied by $\R$ (the case $n = 1$), many of the proofs follow from this observation. Once we set
    \[
    \mbf{0} = \begin{pmatrix}
        0 \\
        \vdots \\
        0
    \end{pmatrix},
    \qquad
    -\mbf{v} = -\begin{pmatrix}
        v_1 \\
        \vdots \\
        v_n
    \end{pmatrix}
    =
    \begin{pmatrix}
        -v_1 \\
        \vdots \\
        -v_n
    \end{pmatrix}
    =
    (-1)\begin{pmatrix}
        v_1 \\
        \vdots \\
        v_n
    \end{pmatrix}
    \]
    then the proofs are very straight forward
\end{proof}

\begin{definition}[Standard basis vectors]
    We set
    \[
    \mbf{e_1} = \begin{pmatrix}
        1 \\
        0 \\
        \vdots \\
        0
    \end{pmatrix},
    \ 
    \mbf{e_2} = \begin{pmatrix}
        0 \\
        1 \\
        \vdots \\
        0
    \end{pmatrix},
    \dotsc,\ 
    \mbf{e_n} = \begin{pmatrix}
        0 \\
        0 \\
        \vdots \\
        1
    \end{pmatrix}
    \]
    and call these vectors the standard basis vectors of $\R ^ n$.
\end{definition}
Sometimes the entries of a vector are referred to as the Cartesian coordinates of a point in space.

Addition can be thought of as an operation that inputs a pair of vectors, each from $\R ^ n$, and outputs a single vector in $\R ^ n$. In other words, addition is a function $\R ^ n \times \R ^ n \mapsto \R ^ n$. Similarly, scalar multiplication is a function $\R \times \R ^ n \mapsto \R ^ n$. The domains of each of these functions are called Cartesian products.

\section{The scalar product in $\R ^ n$}

\begin{definition}[Scalar product]
    The scalar (or dot) product is a function
    \[
    \R ^ n \times \R ^ n \mapsto \R
    \]
    defined as follows. For $\mbf{v, w}$ in $\R ^ n$ set
    \[
    \mbf{v \cdot w} =
    \begin{pmatrix}
        v_1 \\
        \vdots \\
        v_n
    \end{pmatrix}
    \cdot
    \begin{pmatrix}
        w_1 \\
        \vdots \\
        w_n
    \end{pmatrix}
    =
    v_1 w_1 + \cdots + v_n w_n.
    \]
    This is a real number, a scalar.
\end{definition}

The transpose $\mbf{v} ^ t$ of $\mbf{v}$ is defined to be the row vector with the same entries as the column vector $\mbf{v}$. So if
\[
\mbf{v} = \begin{pmatrix}
    v_1 \\
    \vdots \\
    v_n
\end{pmatrix}
\qquad\text{then}
\qquad\mbf{v} ^ t = (v_1,\ \dots,\ v_n).
\]

The dot product of $\mbf{v}$ and $\mbf{w}$ is then obtained by matrix multiplying $\mbf{v} ^ t$ and $\mbf{w}$.

\begin{proposition}[Properties of the scalar product]
\phantom{}
\begin{enumerate}[label = (\roman*)]
    \item Symmetry
    \[
    \mbf{u \cdot v = v \cdot u}\qquad\text{for all }\mbf{u, v}\text{ in } \R ^ n
    \]
    \item Linearity in first factor
    \begin{enumerate}[label = (\alph*)]
        \item $(\mbf{u} + \mbf{v}) \cdot \mbf{w} = \mbf{u} \cdot \mbf{w} + \mbf{v} \cdot \mbf{w}$\qquad for all $\mbf{u,\ v,\ w}$ in $\R ^ n$
        \item $(\lambda\mbf{u}) \cdot \mbf{v} = \lambda(\mbf{u} \cdot \mbf{v})$\qquad for all $\lambda \in \R$ and all $\mbf{u,\ v}$ in $\R ^ n$
    \end{enumerate}
    \item Linearity in the second factor
    \begin{enumerate}[label = (\alph*)]
        \item $\mbf{u} \cdot (\mbf{v} + \mbf{w}) = \mbf{u} \cdot \mbf{w} + \mbf{u} \cdot \mbf{v}$\qquad for all $\mbf{u,\ v,\ w}$ in $\R ^ n$
        \item $\mbf{u} \cdot (\lambda\mbf{v}) = \lambda(\mbf{u} \cdot \mbf{v})$\qquad for all $\lambda \in \R$ and all $\mbf{u,\ v}$ in $\R ^ n$
    \end{enumerate}
    \item Positivity
    
    $\mbf{v \cdot v} \geq 0$ for all $v$, and $\mbf{v \cdot v} = 0$ if and only if $\mbf{v = 0}$.
\end{enumerate}

\begin{proof}
    \phantom{}
    \begin{enumerate}[label = (\roman*)]
        \item
        First we can determine some expressions for $\mbf{u\cdot v}$ and $\mbf{v\cdot u}$
        
        $\mbf{u\cdot v} = u_1 v_1 + u_2 v_2 + \dotsc + u_n v_n$ and

        $\mbf{v\cdot u} = v_1 u_1 + v_2 u_2 + \dotsc + v_n u_n$.

        Then by the commutativity of multiplication we have
        \begin{align*}
            \mbf{u\cdot v} &= u_1 v_1 + u_2 v_2 + \dotsc + u_n v_n \\
            &= v_1 u_1 + v_2 u_2 + \dotsc + v_n u_n \\
            &= \mbf{v \cdot u}
        \end{align*}
        which proves our goal as required
        
    \end{enumerate}
\end{proof}
\end{proposition}

\begin{definition}[Magnitude]
    Suppose
    \[
    \mbf{v} = \begin{pmatrix}
        v_1 \\
        \vdots \\
        v_n
    \end{pmatrix}
    \]
    then $\mbf{v \cdot v} = v_1 ^ 2 + \dotsc + v_n ^ 2$ which is a non-negative real number. We define the magnitude (or length) of $\mbf{v}$, denoted $|\mbf{v}|$, by
    \[
    |\mbf{v}| = \sqrt{\mbf{v \cdot v}}.
    \]
    So long as the $n$ coordinate directions are mutually orthogonal, the magnitude
    \[
    |\mbf{v}| = \sqrt{\mbf{v \cdot v}} = \sqrt{x_1 ^ 2 + \cdots + v_n ^ 2}
    \]
    is the euclidean distance from the origin to the point with coordinates given by $\mbf{v}$. We say that a vector is non-trivial if it has positive length. It is easy to see that the only trivial vector in $\R ^ n$ is the zero vector $\mbf{0}$.
\end{definition}


We can describe the position of a point by what direction it lies and what distance away from us it is, using polar coordinates. \\


Using the length of a vector in $\R ^ 2$ we define polar co-ordinates on $\R ^ 2$ as follows. Let $\mbf{v} = \left(\frac{x}{y}\right)$ be a non-trivial vector with length $|\mbf{v}| = r > 0$. Let $\theta \in [0, 2\pi)$ be the angle at origin $\mbf{0}$ measured in an anticlockwise direction from the $x$-axis to $\mbf{v}$. By basic trigonometry
\[
\cos(\theta) = \frac{x}{r},\qquad\sin(\theta) = \frac{y}{r}.
\]
We define the polar coordinates of $\mbf{v \neq 0}$ to be the unique $(r, \theta)$ in $\R_+ \times [0, 2\pi)$ so that
\[
\mbf{v} = \begin{pmatrix}
    r\cos(\theta) \\
    r\sin(\theta)
\end{pmatrix}
=
r\begin{pmatrix}
    \cos(\theta)
    \sin(\theta)
\end{pmatrix}
\]

\begin{lemma}
    Suppose that $\mbf{v}$ and $\mbf{w}$ are vectors in $\R ^ 2$ of length $r > 0 $ and $s > 0$ respectively. Let $\theta$ be the angle between them. Then
    \[
    \mbf{v \cdot w} = rs\cos(\theta).
    \]
    In particular, we have
    \[
    |\mbf{v \cdot w}| \leq |\mbf{v}||\mbf{w}|
    \]
    with equality if and only if $\mbf{v}$ and $\mbf{w}$ are collinear, i.e. are scalar multiples of each other.

    \begin{proof}
        Suppose that $\mbf{v}$ has polar coordinates $(r, \gamma)$ and $\mbf{w}$ has polar coordinates $(s, \rho)$. The angle between $\mbf{v}$ and $\mbf{w}$ is $\theta = |\gamma - \rho|$. Then
        \[
        \mbf{v} = \begin{pmatrix}
            r\cos\gamma \\
            r\sin\gamma
        \end{pmatrix},
        \qquad
        \mbf{w} = \begin{pmatrix}
            s\cos\rho \\
            s\sin\rho
        \end{pmatrix}.
        \]
        Applying the dot product using these definitions of $\mbf{v}$ and $\mbf{w}$ we obtain the following, applying trigonometric addition formula and the fact that cosine is even we get,
        \[
        \mbf{v \cdot w} =  rs\cos\gamma\cos\rho + rs\sin\gamma\sin\rho = rs(\cos(\gamma - \rho)) = rs\cos(|\gamma - \rho|) = rs\cos\theta.
        \]
    \end{proof}
\end{lemma}
Note that using this lemma we can obtain an expression for the angle between two vectors
\[
\cos(\theta) = \frac{\mbf{v \cdot w}}{|\mbf{v}||\mbf{w}|}
\]

\begin{theorem}[Cauchy-Schwarz inequality]
    Suppose that $\mbf{v}$ and $\mbf{w}$ are vectors in $\R ^ n$. Then
    \[
    |\mbf{v \cdot w}| \leq |\mbf{v}||\mbf{w}|
    \]
    with equality if and only if $\mbf{v}$ and $\mbf{w}$ are collinear

    \begin{proof}
        
    \end{proof}
\end{theorem}

\begin{definition}[Angle between two vectors]\label{def_linalg_vec_angbetvecs}
    Suppose that $\mbf{v}$ and $\mbf{w}$ are vectors in $\R ^ n$ of length $r > 0$ and $s > 0$ respectively. Then we define the angle $\theta \in [0, \pi]$ between them by
    \[
    \cos(\theta) = \frac{\mbf{v \cdot w}}{rs}.
    \]
\end{definition}

We say that the vectors $\mbf{v}$ and $\mbf{w}$ are orthogonal (or perpendicular) if the angle between them is $\frac{\pi}{2}$. \\

\begin{corollary}
    The non-trivial vectors $\mbf{v}$ and $\mbf{w}$ in $\R ^ n$ are orthogonal if and only if we have $\mbf{v \cdot w} = 0$.

    \begin{proof}
        By \autoref{def_linalg_vec_angbetvecs} we have that\footnote{Since $\cos\theta$ can only be $0$ when $\theta = \frac{\pi}{2}$ then they have to be orthogonal and the converse is true too.}
        \[
        \mbf{v \cdot w} = 0 \iff \cos\theta = 0 \iff \theta = \frac{\pi}{2}.
        \]
    \end{proof}
\end{corollary}



\end{document}