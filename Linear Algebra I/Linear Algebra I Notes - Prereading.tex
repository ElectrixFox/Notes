\documentclass[10pt, a4paper]{article}
\usepackage{preamble}

\newcommand{\mbf}[1]{\mathbf{#1}}

\title{Linear Algebra I \\
    \large Prereading}
\author{Luke Phillips}
\date{October 2024}

\begin{document}

\maketitle

\newpage

\tableofcontents

\newpage

\section{The vector space \(\R ^ n\)}
\begin{definition}[$n$-dimensional real space]
    The space $\R ^ n$ ($n$-dimensional real space) is the collection of all (column) vectors of the form
    \[
    \mathbf{x} =
    \begin{pmatrix}
        x_1 \\
        \vdots \\
        x_n
    \end{pmatrix}.
    \]
\end{definition}
Vectors are typically denoted by bold letters such as $\mathbf{x}$ and sometimes using underlined letters $\underline{x}$ to differentiate them from ordinary real numbers.

We write $\mathbf{0}$ for the zero-vector
\[
\mathbf{0} = \begin{pmatrix}
    0 \\
    \vdots \\
    0
\end{pmatrix}
\]
this is the point of origin of $\R ^ n$.

There are two important operations which can be performed on vectors.

\textbf{Addition}: for any vectors $\mathbf{v}$ and $\mathbf{w}$, we define
\[
\mathbf{v + w} =
\begin{pmatrix}
    v_1 \\
    \vdots \\
    v_n
\end{pmatrix}
+
\begin{pmatrix}
    w_1 \\
    \vdots \\
    w_n
\end{pmatrix}
=
\begin{pmatrix}
    v_1 + w_1 \\
    \vdots \\
    v_n + w_n
\end{pmatrix}
\]
this is component-wise addition.


\textbf{Scalar multiplication}: for any vector $\mathbf{v} \in \R ^ n$ and real number $\lambda \in \R$ we define
\[
\lambda\mathbf{v} =
\lambda\begin{pmatrix}
    v_1 \\
    \vdots \\
    v_n
\end{pmatrix}
=
\begin{pmatrix}
    \lambda v_1 \\
    \vdots \\
    \lambda v_n
\end{pmatrix}
\]
this is component-wise multiplication by $\lambda$.

\begin{proposition}
    These operations satisfy the axioms of a real vector space
\end{proposition}

{\large \textbf{Axioms for Addition}}
\begin{enumerate}[label = (\roman*)]
    \item existence of additive identity: there is a vector $\mathbf{0}$ in $\R ^ n$ so that $\mathbf{v} + \mathbf{0} = \mathbf{v} = \mathbf{0} + \mathbf{v}$ for all $\mathbf{v}$ in $\R ^ n$.
    \item commutativity: $\mbf{v + w = w + v}$ for all $\mathbf{v}$ and $\mathbf{w}$ in $\R ^ n$
    \item existence of additive inverses: for each $\mbf{v}$ in $\R ^ n$ there is an element $-\mbf{v}$ in $\R ^ n$ such that $\mbf{v + (-v) = 0 = (-v) + v}$
    \item associativity: $\mbf{u + (v + w) = (u + v) + w}$ for all $\mbf{u, v, w}$ in $\R ^ n$
\end{enumerate}
these axioms can be summarised by saying that $\R ^ n$ is an abelian group under addition.


{\large\textbf{Axioms for Scalar Multiplication}}
\begin{enumerate}[label = (\roman*)]
    \item multiplication by 0: $0\mbf{v = 0}$ for all $\mbf{v}$ in $\R ^ n$
    \item multiplication by 1: $1\mbf{v = v}$ for all $\mbf{v}$ in $\R ^ n$
    \item associativity: $(\lambda\mu)\mbf{v} = \lambda (\mu\mbf{v})\ \forall \lambda, \mu \in \R$ and $\forall \mbf{v} \in \R ^ n$
    \item distributivity: $(\lambda + \mu)\mbf{v} = \lambda\mbf{v} + \mu\mbf{v}$ and $\lambda (\mbf{v + w}) = \lambda\mbf{v} + \lambda\mbf{w}$ $\forall \lambda, \mu \in \R$ and $\forall \mbf{v, w} \in \R ^ n$
\end{enumerate}

\begin{proof}
    The above axioms ate trivially satisfied by $\R$ (the case $n = 1$), many of the proofs follow from this observation. Once we set
    \[
    \mbf{0} = \begin{pmatrix}
        0 \\
        \vdots \\
        0
    \end{pmatrix},
    \qquad
    -\mbf{v} = -\begin{pmatrix}
        v_1 \\
        \vdots \\
        v_n
    \end{pmatrix}
    =
    \begin{pmatrix}
        -v_1 \\
        \vdots \\
        -v_n
    \end{pmatrix}
    =
    (-1)\begin{pmatrix}
        v_1 \\
        \vdots \\
        v_n
    \end{pmatrix}
    \]
    then the proofs are very straight forward
\end{proof}

\begin{definition}[Standard basis vectors]
    We set
    \[
    \mbf{e_1} = \begin{pmatrix}
        1 \\
        0 \\
        \vdots \\
        0
    \end{pmatrix},
    \ 
    \mbf{e_2} = \begin{pmatrix}
        0 \\
        1 \\
        \vdots \\
        0
    \end{pmatrix},
    \dotsc,\ 
    \mbf{e_n} = \begin{pmatrix}
        0 \\
        0 \\
        \vdots \\
        1
    \end{pmatrix}
    \]
    and call these vectors the standard basis vectors of $\R ^ n$.
\end{definition}

\end{document}