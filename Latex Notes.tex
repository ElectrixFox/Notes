\documentclass[10pt, a4paper]{article}
\usepackage{preamble}

\title{Latex Notes}
\author{Luke Phillips}
\date{October 2024}

\begin{document}

\maketitle

\newpage

\section{Introduction}
Here I will go over some basics of \LaTeX\ and then move onto some more advanced \LaTeX\ stuff.

\subsection{Code layout}
In the end \LaTeX\ is a programming language,
therefore we need to format our code as though it is any other language.
By format I mean,
we should use comments for setup and actual \LaTeX\ syntax.

Some of the things that I do in my documents to keep my code clean are:
I generally insert a new line after every part of punctuation,
this doesn't matter too much it is more so that all of the code isn't on one line,
this can become hard to read when there is a lot to say on one line
\footnote{It also means that I have to use punctuation!}.

\newpage

\section{Random useful stuff}

You can write text under braces as follows
\begin{verbatim}
    f(x) = \underbrace{2k}_{\text{for $k > 0$}} > 2k - 1.
\end{verbatim}
\[
f(x) = \underbrace{2k}_{\text{for $k > 0$}} > 2k - 1.
\]
However this causes there to be big spaces,
the \verb|\mathclap{}| command can be used to solve this:
\[
f(x) = \underbrace{2k}_{\mathclap{\text{for $k > 0$}}} > 2k - 1.
\]
By simply changing \verb|_{\text{...}}| to \verb|_{\mathclap{}}|.

\verb|\underset{above}{below}| provides text underneath the the argument as follows
\[
\underset{above}{below}
\]
This is sort of an example of how to produce an augmented matrix
\begin{verbatim}
\left(\hspace{-5pt}
    \begin{array}{cc|c}
        1 & \alpha & b_1 \\
        0 & 0 & b_2
    \end{array}
\hspace{-5pt}\right)
\end{verbatim}

Using the following increases the maximum number of matrix columns to $n$,
\verb|\setcounter{MaxMatrixCols}{n}|.

In order to make a symbol e.g. $\lambda$ bold use the command \verb|\pmb{\lambda}|$ = \pmb{\lambda}$

Use \verb|\intertext{}| to add text inside of an align environment.

In order to remove the warning sign created when adding Maths equations in the section use the following setup
\begin{verbatim}
    \section[text]{todisplay}.
\end{verbatim}
This will allow for the Toc file to find the correct bit of the file to actually render the Maths text.
Here is an example,
\begin{verbatim}
    \section[Limits as x approaches infinity]{Limits as $x \rightarrow \infty$}.
\end{verbatim}

If you want to have a piecewise function such that the right is automatically words,
you can use the "dcases*" environment as such
\begin{verbatim}
    f(n) = \begin{dcases*}
        1 & if $n$ is even \\
        0 & if $n$ is odd
    \end{dcases*}
\end{verbatim}
which gets
\[
f(n) = \begin{dcases*}
    1 & if $n$ is even \\
    0 & if $n$ is odd
\end{dcases*}
\]

\end{document}