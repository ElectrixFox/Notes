\documentclass[10pt, a4paper]{article}
\usepackage{preamble}

\title{Dynamics I}
\author{Luke Phillips}
\date{January 2025}

\begin{document}

\maketitle

\newpage

\tableofcontents

\newpage

\section{Particle on a line}

\subsection{The Basics}

\begin{definition}[Position]
    The position $x(t)$ where $t$ is time.
\end{definition}

\begin{definition}[Velocity]
    The velocity $\dot{x}(t)$ where $t$ is time.
\end{definition}

\textit{Note that $\dot{x}$ means $\frac{dx}{dt}$.}

\begin{definition}[Acceleration]
    The acceleration $a(t) = \dot{v}(t) = \ddot{x}(t)$.
\end{definition}

\begin{definition}[Speed]
    The speed is $|v(t)|$,
    the magnitude of the velocity.
\end{definition}

\begin{definition}[Mass]
    A particle has mass $m > 0$.
\end{definition}

\begin{definition}[Momentum]
    A particle's momentum,
    $p$ is given by
    \[
    p = mv = m\dot{x}.
    \]
\end{definition}

\begin{definition}[Equation of motion]
    In an applied force $F$,
    then the equation of motion is $\dot{p} = F$.
\end{definition}

\begin{remark}\phantom{}
    \begin{enumerate}[label = (\roman*)]
        \item Could have $m$ depending on time
        (e.g. rocket).

        Unless otherwise stated,
        we take $m$ to be a \textbf{constant}.

        Then the EoM\footnote{Equation of motion} is $m\dot{v} = m\ddot{x} = F$.

        \item In general,
        $F = F(t, x, \dot{x})$ and $m\ddot{x} = F(t, x, \dot{x})$ is a second order differential equation which cannot be solved
        (explicitly).
    \end{enumerate}
\end{remark}

We focus on special cases which can be solved.

The simplest example is $F$ is constant.
For example,
vertical motion under gravity close to the surface of a planet:
then the magnitude of $F$ is $|F| = mg$
(the weight)
where the constant $g$ depends on the planet.

\begin{example}
    An object is shot up vertically from the ground with some initial speed $u$.
    What maximum height $H$ does it reach?
    \begin{solution}
        Could use $x$ upwards or $x$ downwards,
        both work,
        choose a convenient one.
        Here having $x$ upwards seems more natural.
        Then $m\ddot{x} = -mg$ with $x = 0$ at the ground.
        Then
        \begin{align*}
            \ddot{x} = -g &\xrightarrow{\int\,dt} \dot{x} = -gt + u \\
            \intertext{where $u$ is the constant of integration.} \\
            &\xrightarrow{\int\,dt} x(t) = -\frac{1}{2}gt ^ 2 + ut + 0.
        \end{align*}
        Max $\dot{x} = 0 \implies t = u / g$.
        So $H = x\left(\frac{u}{g}\right) = -\frac{1}{2}g\left(\frac{u}{g}\right) ^ 2 + u\left(\frac{u}{g}\right) = \frac{u ^ 2}{2g}$.
    \end{solution}
\end{example}

\subsection{Case \texorpdfstring{$F = F(t)$}{}}
The same method works,
do $\int\,dt$ twice.

\begin{example}
    A particle of mass $m = 1$ is at rest at $x = 0$.
    A force $F = e ^ {-t}$ is switched on at time $t = 0$.
    Find $x(t)$ for $t > 0$.

    \textit{Note:
    expect $\dot{x} > 0$ and $x > 0$ for $t > 0$
    (since $F > 0$).}

    \begin{solution}
        \[
        \ddot{x} = e ^ {-t} \xrightarrow{\int\,dt} \dot{x} = -e ^ {-t} + 1 \xrightarrow{\int\,dt} x = e ^ {-t} + t - 1.
        \]
    \end{solution}
\end{example}

\subsection{Case \texorpdfstring{$F = F(\dot{x})$}{}}

\begin{remark}
    Here $m\ddot{x} = F(\dot{x})$ second-order non linear differential equation.
\end{remark}

\begin{example}
    A particle of mass $m$ is slowed by a frictional force of magnitude $be ^ {av}$
    ($a$ and $b$ are constant).
    Initial speed $u$.
    What time $T$ does the particle travel for before coming to rest?
    
    \begin{solution}
        The particle either moves to the right or the left,
        the force moves in the opposite direction.

        Choosing the first situation.
        So
        \begin{align*}
            m\frac{dv}{dt} = -be ^ {av} &\implies \int e ^ {av}\,dv = -\frac{b}{m}\int\,dt \\
            \intertext{could use a constant of integration,
            fixed by $v(0) = u$.} \\
            \int e ^ {av}\,dv = -\frac{b}{m}\int\,dt &\implies \int_{u}^{0} e ^ {av}\,dv = -\frac{b}{m}\int_{0}^{T}\,dt \\
            &\implies \left[-\frac{1}{a}e ^ {-av}\right]_{u}^{0} = -\frac{bT}{m} \\
            &\implies T = \frac{m}{ab}\left(1 - e ^ {-au}\right).
        \end{align*}
    \end{solution}
\end{example}

\begin{example}
    A mass $M$ falls downwards under gravity,
    a resistive force $bw$,
    where $w$ is its speed.
    Initially $w(0) = 0$,
    find $w(t)$.
    What is the "least upper bound" of this $w$,
    the terminal speed.

    \begin{solution}
        We could use the $x$ axis pointing up or down.
        Use $x$ as downwards since the motion is downwards.

        Then
        \[
        m\dot{v} = mg - bv
        \]
        where $v$ is the velocity,
        here the velocity $v = w$.
        So
        \[
        m\dot{w} = mg - bw \implies \frac{dw}{dt} = g - \frac{b}{m}w \implies \int\frac{1}{g - \frac{b}{m}w}\,dw = \int\,dt
        \]
        which results in $w(t) = \frac{mg}{b}\left(1 - e ^ {-\frac{bt}{m}}\right)$,
        the terminal speed is
        \[
        \liminfty[t]\frac{mg}{b}\left(1 - e ^ {-\frac{bt}{m}}\right) = \frac{mg}{b}.
        \]
    \end{solution}
\end{example}

\subsection{Case \texorpdfstring{$F = F(x)$}{}}
Here $m\ddot{x} = F(x)$.
If $F(x) = bx + c$
($b$ and $c$ are constant),
then $m\ddot{x} - bx = c$ linear,
the solution is the complementary function $+$ the particular integral.
This doesn't work for $F(x)$ non-linear.
For that think of $v = v(x)$.
Then
\[
m\frac{dv}{dt} = m\frac{d}{dt}v(x(t)) = m\frac{dv}{dx}\cdot\frac{dx}{dt} = m\frac{dv}{dx}v = F(x).
\]
This is a first order separable ODE,
solve to get $v(x)$.
To get $x(t)$,
use $\frac{dx}{dt} = v(x)$ again first order separable.

\begin{example}
    A mass with $m = 2$ moves in a force with $F = 3x ^ 2$,
    initial conditions $x = 1$ and $v = 1$ at $t = 0$.
    What time $T$ does it take to reach $x = 9$?

    \begin{solution}
        \begin{align*}
            2v\frac{dv}{dx} = 3x ^ 2 &\implies \int 2v\,dv = \int 3x ^ 2\,dx \\
            &\implies v ^ 2 = x ^ 3 + c &\text{($c = 0$ from initial conditions)} \\
            &\implies v ^ 2 = x ^ 3 \\
            &\implies v = x ^ {\frac{3}{2}} &\text{(positive since $x = v = 1$ at $t = 0$)}
        \end{align*}
        Now
        \begin{align*}
            \frac{dx}{dt} = x ^ {\frac{3}{2}} &\implies \int_{1}^{9} x ^ {-\frac{3}{2}}\,dx = \int_{0}^{T}\,dt \\
            &\implies T = \left[-2x ^ {-\frac{1}{2}}\right]_{2}^{9} = -2\left(\frac{1}{3} - 1\right) = \frac{4}{3}.
        \end{align*}
    \end{solution}
\end{example}

\subsection{More Examples}

\begin{example}
    $F(t, v) = -v + e ^ {-t}$
    (with $v(0) = 0$)
    \begin{solution}
        This is linear,
        so solve with an integrating factor and apply the initial conditions to get
        \[
        v(t) = te ^ {-t}.
        \]
    \end{solution}
\end{example}

\begin{example}
    A rocket of mass $m(t)$ burns fuel at a constant rate $\dot{m} = -c$
    (with $c > 0$).
    Thus produces a constant force $F = kc$.
    Start with $x(0) = 0$ and $\dot{x}(0) = 0$,
    and initial mass $m_0$.
    Find $\dot{x}(t)$,
    hence $x(t)$.
    \begin{solution}
        Here the equation of motion $\frac{d}{dt}(mv) = F$ where $v = \dot{x}$.
        Do $\int\,dt$ giving $mv = kct + 0$
        (with $0$ being the constant of integration).
        Also $\dot{m} = -c \implies m(t) = -ct + m_0$.
        Thus $v(t) = \frac{kct}{m_0 - ct}$ for $0 \leq t < \frac{m_0}{c}$.

        Finally,
        $x(t) = \int v(t)\,dt$.
    \end{solution}
\end{example}

Two methods for $F = F(\dot{x})$.

\textbf{Method $1$}
is $m\frac{dv}{dt} = F(v)$,
get $v(t)$,
then $\int\,dt$ gives $x(t)$.

\textbf{Method $2$}
is $mv\frac{dv}{dx} = F(v)$ which is separable,
get $v(x)$,
then $\frac{dx}{dt} = v(x)$.

\begin{example}
    A unit mass
    (meaning $m = 1$)
    experiences a frictional force of magnitude $v + v ^ 2$
    ($v > 0$ is velocity).
    Find the general solution for $x(t)$.

    \begin{solution}
        Here method $2$ says
        \begin{align*}
            v\frac{dv}{dx} = -(v + v ^ 2) &\implies \frac{dv}{dx} = -(1 + v) \\
            &\implies \int\frac{1}{1 + v}\,dv = -\int\,dx \\
            &\implies \log(1 + v) = -x + C \\
            &\implies v(x) = e ^ {c - x} - 1.
        \end{align*}

        Then
        \begin{align*}
            \frac{dx}{dt} = v = e ^ {c - x} - 1 &\implies \int\frac{1}{e ^ {c - x} - 1}\,dx = \int\,dt = t + b \\
            &\implies \int\frac{e ^ x}{e ^ c - e ^ x}\,dx = -\log(e ^ c - e ^ x) = t + b \\
            &\implies -\log(e ^ c - e ^ x) = t + b \\
            &\implies e ^ c - e ^ x = e ^ {-b - b} \\
            &\implies x = \log(e ^ c - e ^ {-b - t}).
        \end{align*}
    \end{solution}
\end{example}

\begin{example}
    A unit mass is acted on by a force $F = \frac{1}{1 + x ^ 2}$,
    "starts" at $x = -\infty$ with $\dot{x} = u_0 > 0$.
    What is $\dot{x}$ as $x \rightarrow \infty$?

    \textit{[Note:
    $F > 0$ so we expect an answer $> u_0$.]}
    \begin{solution}
        We really want $v(x)$.
        The equation is
        \begin{align*}
            v\frac{dv}{dx} = \frac{1}{1 + x ^ 2} &\implies \int_{u_0}^{u_1} v\,dv = \int_{-\infty}^{\infty}\frac{1}{1 + x ^ 2}\,dx \\
            &\implies \frac{1}{2}u_1 ^ 2 - \frac{1}{2}u_0 ^ 2 = \left[\arctan{x}\right]_{-\infty}^{\infty} = \frac{\pi}{2} - \left(-\frac{\pi}{2}\right) = \pi \\
            &\implies u_1 ^ 2 = u_0 ^ 2 + 2\pi \\
            &\implies u_1 = \sqrt{u_0 ^ 2 + 2\pi}\qquad(> u_0\text{ as expected}).
        \end{align*}
    \end{solution}
\end{example}

Define $V(x) = -\int F(x)\,dx$.
Then
\[
\frac{d}{dx}\left(\frac{1}{2}mv ^ 2\right) = m\frac{dv}{dx}v = F = -\frac{d}{dx}V
\]
so $\frac{d}{dx}\left(\frac{1}{2}mv ^ 2 + V\right) = 0 \implies \frac{1}{2}mv ^ 2 + V(x) = E$.
This constant $E$ is the
(total)
energy,
while $\frac{1}{2}mv ^ 2$ is kinetic energy and $V(x)$ is potential energy.
Also note $\frac{d}{dt}E = \frac{dE}{dx}\frac{dx}{dt} = 0$,
therefore $E$ is constant in time.
\begin{remark}
    \begin{enumerate}[label = (\roman*)]
        \item We say that $E$ is conserved,
        and the force is conservative.

        \item Can add a constant
        (of integration)
        to $V$,
        and hence to $E$.

        \item Example:
        gravity close to the surface of a planet,
        take $x$ upwards,
        then force is
        \[
        F = -mg \implies V(x) = mgx\quad(+\text{optional constant}).
        \]
        
        \item In cases where one cannot solve for $x(t)$,
        energy arguments can still tell us something.
        See next example.
    \end{enumerate}
\end{remark}

\begin{example}
    Take $m = 2$ and $V(x) = x ^ 6$.
    So
    \[
    E = \frac{1}{2}mv ^ 2 + V(x) = \dot{x} ^ 2 + x ^ 6.
    \]
    Try to find $x(t)$ by
    \[
    \frac{dx}{dt} = \sqrt{E - x ^ 6} \implies \int\frac{1}{\sqrt{E - x ^ 6}}\,dx = \int\,dt.
    \]
    Note $V(x) \leq E$ restricting $x$ to say initial conditions $x(0) = 0, \dot{x}(0) = 1$.
    So $E = 1 ^ 6 + 0 ^ 6 = 1$ at $t = 0$,
    therefore for all $t$.
    Thus $-1 \leq x(t) \leq 1$.
    Think of a bowl,
    gravity acting downwards,
    particle rolls in it
    (just an analogy).

    Now $x$ increases to $1$,
    turns around and decreases,
    passes $x = 0$ and "rolls up" the other side,
    reaches $x = -1$,
    turns round,
    etc.
    Motion is periodic.

    Can we compute the period $P$
    (time to complete one cycle)?
    No,
    however
    \[
    \int\frac{1}{\sqrt{1 - x ^ 6}}\,dx,
    \]
    can't do that.
\end{example}

\begin{definition}
    An equilibrium point is where force is $0$,
    i.e. $\frac{dV}{dx} = 0$.
\end{definition}

In the above example,
one equilibrium point at $x = 0$.
(Particle at rest at such a point will stay there.)

\subsection{Motion in a potential}

\begin{example}
    Particle of mass $m = 3$ in a potential $V(x) = \frac{1}{4}x ^ 4 - \frac{1}{3}x ^ 3 - x ^ 2$.
    Describe the motion.

    Draw a graph of $V$.
    Compute $\frac{dV}{dx} = x ^ 3 - x ^ 2 - 2x = x(x ^ 2 - x - 2) = x(x - 2)(x + 1)$.
    So there are three equilibrium points $x = 0, 2, -1$.

    Need enough energy to get over the "hill" at $x = 0$.
    Initial direction doesn't matter.
    The equation is something like
    \[
    E = \frac{1}{2}mu ^ 2 + V(1) = \text{tiny bit} + V(0).
    \]
    Thus $\frac{1}{2}mu ^ 2 + V(1) > V(0)$.
    \[
    \frac{3}{2}u ^ 2 - \frac{13}{12} > 0 \iff u > \sqrt{\frac{13}{18}}.
    \]
\end{example}

\subsection{Simple Harmonic Motion}
Usually a mass-spring system,
restoring force $F = -kx$ where $k > 0$ is the spring constant.
This $F$ is a conservative,
and $V(x) = -\int F\,dx = \frac{1}{2}kx ^ 2$.
Equation of motion is $m\ddot{x} = -kx \implies m\ddot{x} + kx = 0$,
general solution $x(t) = A\cos(\omega t) + B\sin(\omega t)$,
where $\omega = \sqrt{\frac{k}{m}}$ and $A$ and $B$ are arbitrary constants.
Here $\omega$ is the angular frequency,
and the period is $p = \frac{2\pi}{\omega}$.

Variant:
spring hangs downwards,
under gravity.
Now
\[
m\ddot{x} = -kx - mg,
\]
and
\[
x(t) = A\cos(\omega t) + B\sin(\omega t) - \frac{mg}{k}.
\]
The effect is to shift the equilibrium by an amount $\frac{mg}{k}$.

\subsection{Damping}
Car suspension,
each wheel has a spring,
and also a shock absorber,
which provides damping.
Has the form $-(\text{positive constant})\dot{x}$ so opposes velocity.
No longer have a conserved energy.
The equation is
\[
m\ddot{x} = -kx - 2mb\dot{x}
\]
with $b > 0$ constant\footnote{The $2mb$ is just for convenience,
it can be any constant.}.

Auxiliary equation
\begin{align*}
    m\lambda ^2 + 2mb\lambda + k = 0 &\implies \lambda ^ 2 + 2b\lambda + \frac{k}{m} = 0 \\
    &\implies \lambda ^ 2 + 2b\lambda + \omega ^ 2 = 0.
\end{align*}
Roots
\[
\lambda_{\pm} = -b \pm \sqrt{b ^ 2 - \omega ^ 2}
\]
so
\[
x(t) = Ae ^ {\lambda_{+}t} + Be ^ {\lambda_{-}t}.
\]
Case when $b > \omega \implies \lambda_{\pm}$ real,
both negative.
Any initial $x$ dies out exponentially fast.
This is called over damping.
No oscillation at all.

\subsection{Small oscillations}
If $V'(x_0) = 0$ then $x_0$ is an equilibrium point,
where $F(x_0) = 0$.
It is a stable equilibrium if it is a local minimum,
i.e. $V''(x_0) > 0$.
Otherwise unstable,
$V''(x_0) < 0$.
Then the period of small oscillations about a stable equilibrium is the period of the best fitting simple harmonic oscillator.
Computing $P$.
Let $x_0$ be a local minimum of $V(x)$.
Write $x(t) = x_0 + \epsilon(t)$ where $\epsilon(t)$ is a small function.
Substitute into equation of motion
\[
m\ddot{x} = F(x) = -V'(x) \implies m\ddot{\epsilon} = F(x_0 + \epsilon) = \underbrace{F(x_0)}_{=0} + \epsilon F'(x_0) + \dotsc
\]
\[
\implies m\ddot{\epsilon} \approx \epsilon F'(x_0) = -V''(x_0)\epsilon.
\]
This is the equation for simple harmonic motion,
where $V''(x_0)$ is the "spring constant" $k$.
So the period of small oscillations is $P = 2\pi\sqrt{\frac{m}{V''(x)}}$.
Makes sense for local minimum where $V''(x_0) > 0$.

\begin{example}
    $m = 3$,
    $V(x) = \frac{1}{4}x ^ 4 - \frac{1}{3}x ^ 3 - x ^ 2$.
    Then $V'(x) = x(x + 1)(x - 2)$,
    $x = 0$ is unstable,
    $x = -1, 2$ are stable.
    Look for period $P$ about $x = -1$.
    $V''(x) = \frac{d}{dx}\left[x(x + 1)(x - 2)\right]$,
    so $V''(-1) = \left. x(x - 2)\right|_{x = -1} = 3$,
    hence $P = 2\pi\sqrt{\frac{m}{V''(x_0)}} = 2\pi\sqrt{\frac{3}{3}} = 2\pi$.
\end{example}

\subsection{Conservation of Momentum}
Here the external force is $0$.

Fact:
the total momentum is conserved by collisions.
Total momentum is the sum of the momenta of each particle.
The total momentum after the collision is equal to the total momentum before the collision.
The total energy
(which is all kinetic energy)
may be conserved
(then the collision was elastic)
or not conserved
(the collision was inelastic).

\newpage

\section{Dynamics in space}

\subsection{Charged particle in electromagnetic field}

\begin{definition}
    An electric field is a vector $\mbf{E}$
    (in general function of $x, y, z, t$ but we will take it constant).
\end{definition}

Similarly magnetic field $\mbf{B}$.

A particle of charge $q$ experiences the Lorentz force.
\[
\mbf{F} = q\mbf{E} + q\mbf{v} \times \mbf{B}
\]
where $\mbf{v}$ is velocity of particle.

Convention:
$\mbf{e}_1 \times \mbf{e}_2 = \mbf{e}_3$,
$\mbf{e}_2 \times \mbf{e}_3 = \mbf{e}_1$,
$\mbf{e}_3 \times \mbf{e}_1 = \mbf{e}_2$,
$\mbf{e}_2 \times \mbf{e}_1 = -\mbf{e}_3$ etc.

\begin{example}
    Take $\mbf{E} = \mbf{0}$ and $\mbf{B} = B\mbf{e}_3$ with $B$ constant.
    Initial conditions $\mbf{r}(0) = \mbf{0}$ and $\dot{\mbf{r}} = u\mbf{e}_1$.
    Find $\mbf{r}(t)$.

    \begin{solution}
        Equation of motion $m\ddot{\mbf{r}} = \mbf{F} = q\dot{\mbf{r}} \times \mbf{B} = qB \dot{\mbf{r}} \times \mbf{e}_3$.

        Do $\int\,dt$ to get $m\dot{\mbf{r}} = qB\mbf{r} \times \mbf{e}_3 + \underbrace{mu\mbf{e}_1}_{\text{constant of integration}}$.
        Write $\mbf{r} = x\mbf{e}_1 + y\mbf{e}_2 + z\mbf{e}_3$,
        then
        \begin{align*}
            m\dot{x} &= mu + qBy \\
            m\dot{y} &= -qBx \\
            m\dot{z} &= 0
        \end{align*}
        so $z(t) = 0$ solves for $z$.
        Left with two coupled equations for $x$ and $y$.
        Eliminate $x$ or $y$
        (either):
        say $y$.
        Namely $m\ddot{x} = qB\dot{y} = -\left(\frac{qB}{m}\right)x(qB) \implies \ddot{x} + \left(\frac{qB}{m}\right) ^ 2x = 0$.

        General solution is $x(t) = \alpha\cos(\omega t) + \beta\sin(\omega t)$,
        $\alpha, \beta$ constants,
        $\omega = \frac{qB}{m}$.
        And $x(0) = 0 \implies \alpha = 0$ so $x(t) = \beta\sin(\omega t)$.
        And $\dot(0) = u \implies \beta\omega = u \implies \beta = \frac{\omega}{u}$.
        Finally the first equation implies $y = \frac{m}{qB}(\dot{x} - u) = \frac{mu}{qB}(1 - \cos(\omega t)) = \frac{u}{\omega}(1 - \cos(\omega t))$.
    \end{solution}
\end{example}

\begin{example}
    As before,
    with $m = q = 1$,
    and an electric field $\mbf{E} = E\mbf{e}_2$ with $E$ constant.

    \begin{solution}
        $\ddot{\mbf{r}} = B\dot{\mbf{r}} \times \mbf{e}_3 + E\mbf{e}_2$.

        \[
        \int\,dt \implies \dot{\mbf{r}} = B\mbf{r} \times \mbf{e}_3 + Et\mbf{e}_2 + u\mbf{e}_1.
        \]
        $z(t) = 0$ for all $t$ since no vector points in direction $\mbf{e}_3$.
        \begin{align*}
            \dot{x} &= u + By \\
            \dot{y} &= -Bx + Et.
        \end{align*}
        Eliminate $y$:
        $\ddot{x} = B\dot{y} = -B ^ 2x + EBt \implies \ddot{x} + B ^ 2x = EBt$.
        Complementary function is $x = \alpha\cos(Bt) + \beta\sin(Bt)$,
        particular integral $x_{PI} = \frac{E}{B}t$.
        Etc.
        
    \end{solution}
\end{example}

\subsection{Conservative forces and potential energy}
The kinetic energy of a particle of mass $m$ and speed $v$ is $\frac{1}{2}mv ^ 2$.
If the force $\mbf{F}$ depends only on position $\mbf{r}$,
and if there exists a real function $V(\mbf{r})$ such that
\[
\mbf{F} = \left(-\pd[V]{x}, -\pd[V]{y}, -\pd[V]{z}\right) = -\pd[V]{x}\mbf{e}_1 -\pd[V]{y}\mbf{e}_2 - \pd[V]{z}\mbf{e}_3,
\]
then the force is conservative,
and $V$ is called the potential energy.
The total energy is
\[
E(\mbf{r}, \mbf{v}) = \frac{1}{2}mv ^ 2 + V(\mbf{r}).
\]

\begin{proposition}
    In a conservative force field,
    the total energy is conserved.

    \begin{proof}
        Write $\mbf{v} = (v_x, v_y, v_z)$.
        Then
        \[
        \frac{dV}{dt} = \pd[V]{x}v_x + \pd[V]{y}v_y + \pd[V]{z}v_z = -\mbf{F} \cdot \mbf{v},
        \]
        by the chain rule.
        Secondly,
        \[
        \frac{d}{dt}\left(\frac{1}{2}mv ^ 2\right) = \frac{d}{dt}\left(\frac{1}{2}m(v_x ^ 2 + v_y ^ 2 + v_z ^ 2)\right) = m(v_x\dot{v}_x + v_y\dot{v}_y + v_z\dot{v}_z) = m\mbf{a} \cdot \mbf{v}.
        \]
        And so
        \[
        \frac{dE}{dt} = (m\mbf{a} - \mbf{F}) \cdot \mbf{v} = 0.
        \]
    \end{proof}
\end{proposition}

\begin{remark}
    The force $\mbf{F} = F_x\mbf{e}_1 + F_y\mbf{e}_2 + F_z\mbf{e}_3$ is conservative if and only if
    \[
    \pd[F_z]{y} = \pd[F_y]{z},\qquad\pd[F_x]{z} = \pd[F_z]{x},\qquad\pd[F_y]{x} = \pd[F_x]{y}.
    \]
\end{remark}

\begin{example}
    Take $\mbf{F} = (z - y)\mbf{e}_1 + (\alpha x - z)\mbf{e}_2 + (\beta x - y)\mbf{e}_3$,
    where $\alpha$ and $\beta$ are constants.
    For which value(s) of these constants,
    if any,
    is $\mbf{F}$ conservative?
    For these values,
    find $V$.

    \begin{solution}
        We can start by identifying that
        \[
        \pd[V]{x} = z - y \implies V(x, y, z) = (y - z)x + g(y, z)
        \]
        this is because $\mbf{F} = \left(-\pd[V]{x}, -\pd[V]{y}, -\pd[V]{z}\right)$ so $-\pd[V]{x} = (z - y)$.

        Now we can apply this $V$ since we know that
        \[
        \pd[V]{y} = z - \alpha x = (0 - 1)x + h(z) \iff h(z) - x
        \]
        so $h(z) = z$ and $\alpha = -1$.

        Similarly
        \[
        \pd[V]{z} = y - \beta x = -x + f(y) \iff f(y) - x
        \]
        so $f(y) = y$ and $\beta = 1$.

        Since $\pd{y}g(y, z) = h(z) = z$ and $\pd{z}g(y, z) = f(y) = y$ we have naturally that $g(y, z) = yz$.

        Using these we can see that
        \[
        V(x, y, z) = (y - z)x + yz = yx - xz + yz + C
        \]
        where $C$ is an arbitrary constant.
    \end{solution}
\end{example}

\subsection{The simple pendulum}
A point mass $m$ is attached to one end of a rigid light rod of length $L$.
The rod is pivoted at the other end,
and is free to swing in a vertical plane.
Let $\theta(t)$ denote the angle between the pendulum and the vertical.
The gravitational potential energy is $V = mgz$,
where $z$ is the height of the mass above its hanging-straight-down position.
So $V$ is effectively a function of $\theta$ given by
\[
V(\theta) = mgL(1 - \cos{\theta}).
\]
If the coordinate in the horizontal direction is $x$,
then $x = L\sin{\theta}$,
and so
\[
v ^ 2 = \dot{x} ^ 2 + \dot{z} ^ 2 = L ^ 2\dot{\theta} ^ 2.
\]
Here $\dot{\theta}$ is the angular velocity;
note $v = L|\dot{\theta}|$.
The total energy is conserved is
\[
E = \frac{1}{2}mL ^ 2\dot{\theta} ^ 2 + mgL(1 - \cos{\theta}).
\]
The equation of motion obtained from $\frac{dE}{dt} = 0$ is $\ddot{\theta} = -\left(\frac{g}{L}\right)\sin{\theta}$.
For small oscillations about the stable equilibrium,
the equation becomes $\ddot{\theta} = -\frac{g\theta}{L}$,
so we get simple harmonic motion with angular frequency $\omega = \sqrt{\frac{g}{L}}$.
This is just an approximation for small $\theta$.

\begin{example}
    Release the pendulum from rest in a horizontal position.
    What is the speed $u$ of the mass when it passes the vertical position?

    \begin{solution}
        By conservation of energy
        \[
        E = \frac{1}{2}mL ^ 2\dot{\theta} ^ 2 + mgL(1 - \cos{\theta}).
        \]
        So initial $E$ equals final $E$
        \[
        V\left(\frac{\pi}{2}\right) - \frac{1}{2}mL ^ 2\dot{\theta} ^ 2 = \frac{1}{2}m(L\dot{\theta}) ^ 2 + V(0) \iff V\left(\frac{\pi}{2}\right) = \frac{1}{2}mL ^ 2\dot{\theta} ^ 2 + 0 + V(0)
        \]
        \[
        V\left(\frac{\pi}{2}\right) = mgL = \frac{1}{2}m(L\dot{\theta}) ^ 2 + V(0) = \frac{1}{2}m(L\dot{\theta}) ^ 2
        \]
        \[
        mgL = \frac{1}{2}m(L\dot{\theta}) ^ 2 \iff \sqrt{2gL} = L|\dot{\theta}| = v
        \]
        hence $v = \sqrt{2gL}$.
    \end{solution}
\end{example}

\begin{example}
    If you start at $\theta(0) = 0$,
    with $\dot{\theta}(0) = \omega$,
    how large does $\omega$ have to be for the pendulum to 'loop the loop'?

    \begin{solution}
        We need $E = \frac{1}{2}mL ^ 2\dot{\theta} ^ 2 + V(0) > V(\pi)$ since the energy before the movement has to be larger than the energy at the top of the swing.
        \[
        \frac{1}{2}mL ^ 2\dot{\theta} ^ 2 > V(\pi) = 2mgL \iff \dot{\theta} > 2\sqrt{\frac{g}{L}}
        \]
        but $\dot{\theta}(0) = \omega$ so $\omega > 2\sqrt{\frac{g}{L}}$.
    \end{solution}
\end{example}

\subsection{Projectiles}
Setup.
You fire a projectile at an angle $\theta$ to the horizontal,
with initial speed $u$.
The subsequent motion is in a plane.
Take axes $z$ and $\mbf{e}_3$ upwards,
and $x$ and $\mbf{e}_1$ horizontal,
with no motion in the $y$-direction.
The initial velocity is
\[
\mbf{v}(0) = u\mbf{e}_1\cos{\theta} + u\mbf{e}_3\sin{\theta}.
\]
The forces are $-mg\mbf{e}_3$ for gravity,
and $-\lambda\mbf{v}$ for air resistance.
Take $\mbf{r}(0) = \mbf{0}$.
If there is no resistance,
then we get
\[
\mbf{r}(t) = ut\mbf{e}_1\cos{\theta} + \left(ut\sin{\theta} - \frac{1}{2}gt ^ 2\right)\mbf{e}_3.
\]

\begin{example}
    Suppose that the ground surface is horizontal,
    and that there is no resistance.
    Let $T$ be the time taken for the projectile to land.
    In other words,
    $z(T) = 0$.
    Find the maximum range of the projectile.

    \begin{solution}
        \[
        \begin{pmatrix}
            x(t) \\ 0 \\ 0
        \end{pmatrix} = \mbf{r}(t) = \begin{pmatrix}
            ut\cos{\theta} \\
            0 \\
            ut\sin{\theta} - \frac{1}{2}gt ^ 2
        \end{pmatrix}
        \]
        so $ut\sin{\theta} - \frac{1}{2}gt ^ 2 = 0 \iff t = \frac{2u}{g}\sin{\theta}$.

        Hence $T = \frac{2u}{g}\sin{\theta}$.
        To maximise the range we have
        \[
        x(t) = ut\cos{\theta} = \frac{2u ^ 2}{g}\sin{\theta}\cos{\theta} = \frac{u ^ 2}{g}\sin(2\theta)
        \]
        so $\dot{x} = \frac{2u ^ 2}{g}\cos(2\theta)$ which maximised would be $\dot{x} = 0 = \frac{2u ^ 2}{g}\cos(2\theta) \implies \theta = \frac{\pi}{4}$.

        So for a given $u$ we can maximise the range by $\theta = \frac{\pi}{4}$.
    \end{solution}
\end{example}


\subsection{Momentum and collisions}

\begin{definition}
    Momentum.
    A particle with mass $m$ and velocity $\mbf{v}$ has momentum $\mbf{p} = m\mbf{v}$.
    If a force $\mbf{F}$ is acting,
    then the equation of motion is $\dot{\mbf{p}} = \mbf{F}$.
    In general,
    momentum is not conserved.

    If
    (say)
    the $x$-component of the force is zero,
    then $p_x$ is conserved
    (constant in time).
\end{definition}

\textbf{Collisions}.
If there are no external forces,
then the total momentum is conserved.
If the collision is elastic,
then kinetic energy is conserved too.

\begin{example}
    A particle of uni mass is travelling with velocity $\mbf{v} = 3\mbf{e}_1 + 3\mbf{e}_2$ when it collides with a stationary particle of mass $2$.
    If the unit-mass particle comes away from the collision with velocity $\mbf{u} = -\mbf{e}_1 + \mbf{e}_2$,
    calculate the velocity $\mbf{w}$ of the other particle after the collision,
    and calculate the fractional loss in kinetic energy.

    \begin{solution}
        Total momentum before is $\mbf{p} = 3\mbf{e}_1 + 3\mbf{e}_2$ and total momentum after is $(-\mbf{e}_1 + \mbf{e}_2) + 2\mbf{w}$.
        Conservation says
        $\begin{pmatrix}
            3 \\ 3
        \end{pmatrix} = \begin{pmatrix}
            -1 \\ 1
        \end{pmatrix} + 2\mbf{w} \iff \mbf{w} = \begin{pmatrix}
            2 \\ 1
        \end{pmatrix}$.

        The total kinetic energy before is $K = \frac{1}{2}\sqrt{3 ^ 2 + 3 ^ 2} ^ 2 = 9$ and after $K' = \frac{1}{2}\sqrt{(-1) ^ 2 + 1 ^ 2} ^ 2 + \frac{1}{2}\cdot 2\sqrt{2 ^ 2 + 1 ^ 2} ^ 2 = 6$ so the fractional kinetic energy loss is
        \[
        \frac{K - K'}{K} = \frac{1}{3}.
        \]
    \end{solution}
\end{example}

\subsection{Angular Momentum}
\begin{definition}
    Particle mass $m$ at position $\mbf{r}$ with velocity $\mbf{v} = \dot{\mbf{r}}$ has angular momentum $\mbf{L} = m\mbf{r} \times \mbf{v}$.
\end{definition}

So
\begin{align*}
    \frac{d}{dt}\mbf{L} &= m\dot{\mbf{r}} \times \mbf{v} + m\mbf{r} \times \dot{\mbf{v}} \\
    &= \mbf{r} \times (m\dot{\mbf{v}}) \\
    &= \mbf{r} \times \mbf{F} \\
    &= \mbf{0} &(\mbf{v} = \dot{\mbf{r}}).
\end{align*}
$\mbf{F}$ is the torque.

This angular momentum and torque are about the origin $O$.
Note:
if $\mbf{F} = \alpha\mbf{r}$ for some real function $\alpha$,
then torque is zero,
and $\mbf{L}$ is conserved:
constant in time.

\subsection{Central Forces}
A force $\mbf{F}$ is a central force if it has the form
\[
\mbf{F} = f(r)\mbf{e}_r = r ^ {-1}f(r)\mbf{r}.
\]
Note $|\mbf{F}| = |f|$.
So $\mbf{L} = m\mbf{r} \times \dot{\mbf{r}}$ is conserved,
also $\mbf{L} \cdot \mbf{r} = 0$
(since $\mbf{r} \times \dot{\mbf{r}}$ is orthogonal to $\mbf{r}$).
Time $\mbf{L} \cdot \mbf{r} = 0$ defines a plane in $\R ^ 3$ through $O$.
Note:
it's a constant plane,
since $\mbf{L}$ is constant.
So the trajectory of the particle lies in this fixed plane,
which we can take take the $xy$-plane,
i.e. the $r\theta$-plane.
This means $\mbf{L} = L\mbf{e}_2$ pointing in the $z$-direction.

\begin{example}
    Gravity is a central force
    (planet moving around a fixed star)
    with $f(r) = -\frac{k}{r ^ 2}$ with $k > 0$ constant.
\end{example}

\subsection{Equation of motion and Energy}
Central force $\mbf{F} = f(r)\mbf{e}_r$.
Recall
\[
\mbf{r} = r\mbf{e}_r,\quad\mbf{v} = \dot{\mbf{r}} = \dot{r}\mbf{e}_r + r\dot{\theta}\mbf{e}_{\theta} \implies \mbf{L} = m\mbf{r} \times \mbf{v} = mr ^ 2\dot{\theta}\overbrace{\mbf{e}_r \times \mbf{e}_{\theta}} ^ {=\mbf{e}_3}.
\]
So
\[
L = mr ^ 2\dot{\theta}.
\]
Next
\[
\dot{\mbf{v}} = \ddot{\mbf{r}} = (\ddot{r} - r\dot{\theta} ^ 2)\mbf{e}_r + (2\dot{r}\dot{\theta} + r\ddot{\theta})\mbf{e}_{\theta},
\]
so the equation of motion $m\ddot{\mbf{r}} = \mbf{F} = f(r)\mbf{e}_n$ becomes $m(\ddot{r} - r\dot{\theta} ^ 2) = f(r)$,
$2\dot{r}\dot{\theta} + r\ddot{\theta} = 0$.

Note:
\[
\frac{dL}{dt} = \frac{d}{dt}(mr ^ 2\dot{\theta}) = m(2r\dot{r}\dot{\theta} + r ^ 2\ddot{\theta}) = mr(2\dot{r}\dot{\theta} + r\ddot{\theta}) = 0
\]
since $L$ is constant.
So we're left with $mr ^ 2\dot{\theta} = L$ constant,
and $m(\ddot{r} - r\dot{\theta} ^ 2) = f$.
Substituting $\dot{\theta} = \frac{L}{mr ^ 2}$ into the $\ddot{r}$ equation gives
\[
m\ddot{r} - \frac{L ^ 2}{mr ^ 3} = f(r).
\]

\subsection{Gravitational Trajectories}

Take $m = 1$,
$f(r) = -1 / r ^ 2$,
start at $r = 1$,
with speed $v_0$.

\begin{problem}
    Show that the particle is bounded,
    if $v_0 < \sqrt{2}$;
    if $v_0 < \sqrt{2}$ and the initial velocity $v_0$ is perpendicular to the position vector $\mbf{r}$,
    find $r_{\text{min}}$ and $r_{\text{max}}$.

    \begin{solution}
        Look at energy.
        Potential energy is $V(r) = -\int f(r)\,dr = -\frac{1}{r}$.
        So
        \[
        E = \frac{1}{2}v ^ 2 + \left(-\frac{1}{r}\right) \overset{t = 0}{=} \frac{1}{2}v_0 ^ 2 - 1.
        \]
        Thus
        \[
        v ^ 2 = (v_0 ^ 2 - 2) + \frac{2}{r} \geq 0 \implies \frac{2}{r} \geq 2 - v_0 ^ 2
        \]
        if $2 - v_0 ^ 2 > 0$
        \[
        r \leq \frac{2}{2 - v_0 ^ 2}
        \]
        so if $v_0 < \sqrt{2}$ then $r \leq \frac{2}{2 - v_0 ^ 2}$.


        For the second part,
        $\mbf{r}(0) = \mbf{e}_r$ and $\mbf{v}(0) = v_0\mbf{e}_{\theta}$.
        Then angular momentum is $\mbf{L} = \mbf{r} \times \mbf{v} = v_0\mbf{e}_r \times \mbf{e}_{\theta} = v_0\mbf{e}_3$ hence $L = v_0$.

        Now
        \[
        E = \frac{1}{2}(\dot{r} ^ 2 + r ^ 2\dot{\theta} ^ 2) - \frac{1}{r} = \frac{1}{2}v_0 ^ 2 - 1.
        \]
        $L = r ^ 2\dot{\theta}$.
        \[
        E = \frac{1}{2}\left(\dot{r} ^ 2 + \frac{v_0 ^ 2}{r ^ 2}\right) - \frac{1}{r} = \frac{1}{2}v_0 ^ 2 - 1.
        \]
        So
        \[
        \dot{r} ^ 2 + \frac{v_0 ^ 2}{r ^ 2} - \frac{2}{r} + (2 - v_0 ^ 2) = 0.
        \]
        \[
        \dot{r} ^ 2 + \frac{1}{r ^ 2}Q(r)
        \]
        where
        $Q(r) = (2 - r_0 ^ 2)r ^ 2 - 2r + r_0 ^ 2$.
        Since $\dot{r} ^ 2 \geq 0$ we need $Q(r) \leq 0$.

        If $v_0 = 1$ then $Q(r) = r ^ 2 - 2r + 1 = (r - 1) ^ 2$ so $r = 1$ for all $t$ and $\dot{r} = 0$ for all $t$.

        If $v_0 = \sqrt{2}$ then $Q(r) = -2r + 2 = 2(1 - r)$ so $r \geq 1$.

        If $v_0 > \sqrt{2}$ $r(t) \geq 1$ same as previous case.

        If $v_0 < \sqrt{2}$ then roots $r = 1, \frac{v_0 ^ 2}{2 - v_0 ^ 2}$,
        these two roots are $r_{\text{min}}$ and $r_{\text{max}}$
        (in some order).
        If $v_0 = 1 \implies r_{\text{max}} = r_{\text{min}} = 1$.

        Bounded orbit.
    \end{solution}
\end{problem}

\begin{example}
    For shooting a projectile from surface of the Earth,
    we have $f(r) = -\frac{GMm}{r ^ 2}$
    ($M$ mass of Earth,
    $m$ mass of projectile,
    $G$ is Newton's constant)
    Start at $r = R$ radius of Earth.

    \begin{solution}
        Now $E = \underbrace{\frac{1}{2}mv ^ 2}_{\geq 0} - \underbrace{\frac{GMm}{r}}_{\rightarrow 0\text{ for escape}} = \frac{1}{2}mv_0 ^ 2 - \frac{GMm}{R}$.
        So the escape speed has
        \[
        \frac{1}{2}mv_0 ^ 2 \geq \frac{GMm}{R} \implies v_0 \geq \sqrt{\frac{2GM}{R ^ 2}}
        \]
        independent of $m$ and of direction.
    \end{solution}
\end{example}

\subsection{Finding an orbit}

We have an equation for $r(t)$,
namely
\[
m\ddot{r} - \frac{L ^ 2}{mr ^ 3} = f(r).
\]
Want a curve
(orbit)
as $r = r(\theta)$.
Use the chain rule:
for any function $h$ we have $\frac{dh}{dt} = \frac{dh}{d\theta}\frac{d\theta}{dt}$.
Recall,
$L = mr ^ 2\dot{\theta}$.
Which takes us to
\[
\frac{dh}{dt} = \frac{dh}{d\theta}\frac{d\theta}{dt} = \frac{dh}{d\theta}\frac{L}{mr ^ 2}.
\]
Also,
it is convenient to change the variable from $r$ to $u = \frac{1}{r}$.
Then
\[
\frac{dr}{dt} = \frac{d}{dt}\underbrace{\left(\frac{1}{u}\right)}_{= r} = -\frac{1}{u ^ 2}\frac{du}{d\theta}\frac{L}{mr ^ 2} = -\frac{L}{m}\frac{du}{d\theta}.
\]
And to get $\ddot{r}$ we need
\[
\frac{d}{dt}\left(\frac{dr}{dt}\right) = \frac{d}{dt}\left(-\frac{L}{m}\frac{du}{d\theta}\right) = -\frac{L}{m}\frac{d}{dt}\left(\frac{du}{d\theta}\right) = -\frac{L}{m}\frac{d ^ 2u}{d\theta ^ 2}\frac{L}{m}u ^ 2 = -\frac{L ^ 2u ^ 2}{m ^ 2}\frac{d ^ 2u}{d\theta ^ 2}.
\]
So
\[
m\ddot{r} - \frac{L ^ 2}{mr ^ 3} = f(r)
\]
becomes
\[
-\frac{L ^ 2u ^ 2}{m}\frac{d ^ 2u}{d\theta ^ 2} - \frac{L ^ 2}{m}u ^ 3 = f\left(\frac{1}{u}\right)
\]
this can be rewritten as
\[
\frac{du ^ 2}{d\theta ^ 2} + u = -\frac{m}{L ^ 2u ^ 2}f\left(\frac{1}{u}\right).
\]

\begin{example}
    $f(r) = -\frac{1}{r ^ 2}$ with $m = 1$.

    This equation becomes
    \[
    \frac{d ^ 2u}{d\theta ^ 2} + u = -\frac{1}{L ^ 2u ^ 2}(-u ^ 2) = \frac{1}{L ^ 2}.
    \]
    The solution is the complementary function plus the particular integral.
    Impose $u'(0) = 0$ where $u' = \frac{du}{d\theta}$.
    Solution
    \[
    u(\theta) = \underbrace{A\cos{\theta}}_{CF} + \underbrace{\frac{1}{L ^ 2}}_{PI}.
    \]
    Can write as
    \[
    r(\theta) = \frac{L ^ 2}{1 + \varepsilon\cos{\theta}}
    \]
    where $\varepsilon = AL ^ 2$.

    Note:
    if $|\varepsilon| < 1$ then $r(\theta)$ is finite for all $\theta$,
    and $r_{\text{min}} \leq r(\theta) \leq r_{\text{max}}$.
    Also
    \[
    L ^ 2 = r + \varepsilon r\cos{\theta} = r + \varepsilon x
    \]
    which can be rewritten as
    \[
    r = L ^ 2 - \varepsilon x \implies r ^ 2 = (L ^ 2 - \varepsilon x) ^ 2
    \]
    this is the equation of an ellipse in the $xy$-plane.

    $\varepsilon$ is the eccentricity of the ellipse.
    The case $\varepsilon = 0$ is a circle $r = \text{constant}$.

    If $|\varepsilon| \geq 1$ it is an unbound orbit
    (trajectory).
\end{example}

\begin{example}
    $m = 1$,
    $f(r) = -\frac{1}{r ^ 3}$.
    Then the new equation is
    (with $u'(0) = 0$)
    \[
    u'' + u = -\frac{1}{L ^ 2u ^ 2}(-u ^ 3) = \frac{1}{L ^ 2}u
    \]
    this is equivalent to
    \[
    u'' + \left(1 - \frac{1}{L ^ 2}\right)u = 0.
    \]
    Case $L = 1$:
    this has solution $u = \text{constant} \implies r = \text{constant}$ which is a circle.

    If initial velocity $\mbf{v}$ is perpendicular to $\mbf{r}$,
    then $\dot{r}(0) = 0$,
    this implies $u'(0) = 0$
    (from $\dot{r} = -\frac{L}{m}u'$).

    Case $L > 1$:
    this implies $1 - \frac{1}{L ^ 2} > 0$,
    write $1 = \frac{1}{L ^ 2} = -\omega ^ 2$ and
    \[
    u(\theta) = A\cos(\omega\theta)
    \]
    which implies $r(\theta) = \frac{1}{A\cos(\omega\theta)}$ for $|\theta| < \frac{\pi}{2\omega}$,
    and the particle escapes to infinity.
\end{example}

\newpage

\section{Moments of Inertia}

\subsection{Rigid Bodies}

Centre of mass.
System of $N$ particles,
masses $m_1, m_2, \dotsc, m_N$ at positions $\mbf{r}_1, \mbf{r}_2, \dotsc, \mbf{r}_N$ has centre of mass at
(weighted average of $\mbf{r}_i$)
\[
\mbf{R} = \frac{m_1\mbf{r}_1 + m_2\mbf{r}_2 + \dotsc + m_N\mbf{r}_N}{m_1 + m_2 + \dotsc m_N}.
\]

A rigid body is a system of particles such that the position of any particle relative to any other particle is fixed.
Motion of a rigid body complicated in general.
Simplest is motion without any rotation:
then it behaves like a single particle at the centre of mass $\mbf{R}(t)$.

Next simplest:
allow rotation about a fixed axis.
Position of body given by $\theta(t)$:
all the particles have the same angular velocity $\dot{\theta}$.

\begin{example}
    Simple pendulum.

    Kinetic energy is
    \[
    K = \frac{1}{2}m_1v_1 ^ 2 + \dotsc + \frac{1}{2}m_Nv_N ^ 2
    \]
    Massses $m_i$ at distance $r_i$ from axis,
    ($0 \leq i \leq N$)
    speeds $v_i$,
    note $v_i = r_i|\dot{\theta}|$.
    \[
    K = \frac{1}{2}m_1(r_1\dot{\theta}) ^ 2 + \dotsc + \frac{1}{2}m_N(r_N\dot{\theta}) ^ 2 = \frac{1}{2}I\dot{\theta} ^ 2
    \]
    where $I = m_1r_1 ^ 2 + m_2r_2 ^ 2 + \dotsc + m_Nr_N ^ 2$.
    This $I$ is the moment of inertia of the system about the given axis.
\end{example}

\begin{example}[$0a$]
    Light rod and two masses.
    Distance $D$ from axis to centre of mass:
    $D = \frac{1}{2m}$.
    Axis at origin.
    \[
    D = \frac{1}{2m}\left(m\cdot\frac{L}{2} + m\cdot L\right) = \frac{3}{4}L
    \]
    
    Moment of inertia:
    \[
    I = m\left(\frac{L}{2}\right) ^ 2 + mL ^ 2 = \frac{5}{4}mL ^ 2.
    \]
    So
    \[
    K = \frac{5}{8}mL ^ 2\dot{\theta} ^ 2.
    \]
\end{example}

\begin{example}
    Light rod,
    length $L$,
    three masses $m$,
    as shown:

    One particle $\frac{L}{3}$ from the axis and the other two spaced $\frac{L}{3}$ apart,
    i.e. the far left particle is $\frac{L}{3}$ from the axis and the far right is $\frac{2L}{3}$ from the axis.

    Moment of inertia is
    \[
    I = m\left(\frac{L}{3}\right) ^ 2 + m\left(\frac{L}{3}\right) ^ 2 + m\left(\frac{2L}{3}\right) ^ 2 = \frac{2}{3}mL ^ 2.
    \]
    Distance $D$ from axis to centre of mass is
    \[
    D = \frac{1}{3m}\left[m\left(-\frac{L}{3}\right) + m\left(\frac{L}{3}\right) + m\frac{2L}{3}\right] = \frac{2}{9}L
    \]
\end{example}

\subsection{Energy and Dynamics of Compound Pendulum}

Revisiting example $0a$.
We have $I$,
hence kinetic $K = \frac{1}{2}I\dot{\theta} ^ 2$.
Compute
\[
V(\theta) = mgL(1 - \cos{\theta}) + mg\left(\frac{L}{2}\right)(1 - \cos{\theta}) = \frac{3}{2}mgL(1 - \cos{\theta}).
\]
Can also compute $V$ as $(\text{total mass}) \cdot g \cdot (\text{distance pivot to centre of mass})(1 - \cos{\theta})$,
we get $(2m)\cdot g\cdot \left(\frac{3}{4}L\right)(1 - \cos{\theta})$.

\begin{remark}
    This way of calculating $V(\theta)$ always works,
    because the calculation of centre of mass is "the same" as that of adding up $v(\theta)$ for each part of the pendulum.
\end{remark}

\begin{example}\label{dyn:exam:examp2}
    Pendulum consisting of a solid rigid bar of mass $M$,
    length $L$,
    pivoted at one end.
    Compute $E = K + V$.

    \begin{solution}
        First,
        the distance to the centre of mass is $\frac{1}{2}L$
        (confirmed below),
        so
        \[
        V(\theta) = M \cdot g \cdot \left(\frac{1}{2}L\right) \cdot (1 - \cos{\theta}).
        \]
        Calculation of centre of mass:
        Centre of mass at
        \[
        \frac{1}{M}\int_{0}^{L}\underbrace{\left(M\frac{dx}{L}\right)}_{\text{Mass of small bit}} \cdot \underbrace{x}_{\text{Location of bit}} = \frac{1}{L}\left[\frac{1}{2}x ^ 2\right]_{0}^{L} = \frac{1}{2}L.
        \]
        Need to compute
        \[
        I = \int_{0}^{L}\left(M\frac{dx}{L}\right)x ^ 2 = \frac{1}{3}ML ^ 2 \implies K = \frac{1}{6}ML ^ 2\dot{\theta} ^ 2.
        \]
        Hence
        \[
        E = K + V = \frac{1}{6}ML ^ 2\dot{\theta} ^ 2 + \frac{1}{2}MgL(1 - \cos{\theta})
        \]
    \end{solution}
\end{example}

$V(\theta) = MgD(1 - \cos{\theta})$ with $D$ the distance between the pivot and the centre of mass,
$I(\theta) = \int_{-c}^{L - c}\left(M\frac{dx}{L}\right)x ^ 2$,
with $c$ the left hand side of the bar,
keeping origin at the pivot.

\begin{example}[continues = dyn:exam:examp2]\phantom{}
    \begin{enumerate}[label = (\alph*)]
        \item What is the equation of motion for $\theta(t)$?

        \item Period $P$ of small oscillations about stable equilibrium.

        \item Release bar from rest horizontally,
        what is maximum value of $\dot{\theta}$?
    \end{enumerate}

    \begin{solution}
        \begin{enumerate}[label = (\alph*)]
            \item Use
            \[
            \frac{dE}{dt} = 0 \implies \frac{1}{3}ML ^ 2\dot{\theta}\ddot{\theta} + \frac{1}{2}MgL(\sin{\theta})\dot{\theta} = 0
            \]
            so equation of motion is
            \[
            \ddot{\theta} + \frac{3}{2}\cdot\frac{g}{L}\sin{\theta} = 0.
            \]

            \item $\theta$ small implies
            \[
            \ddot{\theta} + \frac{3}{2}\cdot\frac{g}{L}\theta = 0 \implies P = \frac{2\pi}{\sqrt{\frac{3g}{2L}}} = 2\pi\sqrt{\frac{2L}{3g}}.
            \]

            \item Start $\theta = \frac{\pi}{2}$,
            $\dot{\theta} = 0$,
            finish $\theta = 0$,
            $\dot{\theta}$ is maximum.
            Conservation of energy gives
            \[
            \underset{\text{(start)}}{\frac{1}{2}MgL} = \underset{\text{(finish)}}{\frac{1}{6}ML ^ 2\dot{\theta}_{\text{max}} ^ 2}.
            \]
            \[
            \dot{\theta}_{\text{max}} = \dotsc
            \]
        \end{enumerate}
    \end{solution}
\end{example}

\begin{example}
    A uniform round pulley of mass $M$ and radius $L$ can rotate without friction about a fixed horizontal axis through its centre.
    A mass $m$ is connected by a light string wound around the pulley,
    hanging downwards,
    with no slippage.
    Find the acceleration $a$ of the mass.

    \begin{solution}
        Compute the moment of inertia of the pulley about its axis.
        Easiest to use polar coordinates $(r, \theta)$.
        Then
        \[
        I = \iint M\left(\frac{r}{\pi L ^ 2}\right)\,dr\,d\theta r ^ 2 = \frac{1}{2}ML ^ 2.
        \]
        The speed $v$ of the mass is related to the angular speed by $v = L\dot{\theta}$.
        Taking $z$ to point downwards,
        the potential energy is $V = -mgz$.
        Putting everything together gives the total energy
        \[
        E = \frac{1}{2}I\dot{\theta} ^ 2 + \frac{1}{2}mv ^ 2 - mgz = \frac{1}{2}\left(\frac{1}{2}M + m\right)\dot{z} ^ 2 - mgz.
        \]
        Using $dE / dt = 0$ and dividing by $\dot{z}$ then leaves
        \[
        a = \ddot{z} = \frac{g}{1 + M / (2m)} < g.
        \]
    \end{solution}
\end{example}
















\end{document}