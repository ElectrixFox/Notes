\documentclass[10pt, a4paper]{article}
\usepackage{preamble}

\title{Dynamics I}
\author{Luke Phillips}
\date{January 2025}

\begin{document}

\maketitle

\newpage

\tableofcontents

\newpage

\section{Particle on a line}

\subsection{The Basics}

\begin{definition}[Position]
    The position $x(t)$ where $t$ is time.
\end{definition}

\begin{definition}[Velocity]
    The velocity $\dot{x}(t)$ where $t$ is time.
\end{definition}

\textit{Note that $\dot{x}$ means $\frac{dx}{dt}$.}

\begin{definition}[Acceleration]
    The acceleration $a(t) = \dot{v}(t) = \ddot{x}(t)$.
\end{definition}

\begin{definition}[Speed]
    The speed is $|v(t)|$,
    the magnitude of the velocity.
\end{definition}

\begin{definition}[Mass]
    A particle has mass $m > 0$.
\end{definition}

\begin{definition}[Momentum]
    A particle's momentum,
    $p$ is given by
    \[
    p = mv = m\dot{x}.
    \]
\end{definition}

\begin{definition}[Equation of motion]
    In an applied force $F$,
    then the equation of motion is $\dot{p} = F$.
\end{definition}

\begin{remark}\phantom{}
    \begin{enumerate}[label = (\roman*)]
        \item Could have $m$ depending on time
        (e.g. rocket).

        Unless otherwise stated,
        we take $m$ to be a \textbf{constant}.

        Then the EoM\footnote{Equation of motion} is $m\dot{v} = m\ddot{x} = F$.

        \item In general,
        $F = F(t, x, \dot{x})$ and $m\ddot{x} = F(t, x, \dot{x})$ is a second order differential equation which cannot be solved
        (explicitly).
    \end{enumerate}
\end{remark}

We focus on special cases which can be solved.

The simplest example is $F$ is constant.
For example,
vertical motion under gravity close to the surface of a planet:
then the magnitude of $F$ is $|F| = mg$
(the weight)
where the constant $g$ depends on the planet.

\begin{example}
    An object is shot up vertically from the ground with some initial speed $u$.
    What maximum height $H$ does it reach?
    \begin{solution}
        Could use $x$ upwards or $x$ downwards,
        both work,
        choose a convenient one.
        Here having $x$ upwards seems more natural.
        Then $m\ddot{x} = -mg$ with $x = 0$ at the ground.
        Then
        \begin{align*}
            \ddot{x} = -g &\xrightarrow{\int\,dt} \dot{x} = -gt + u \\
            \intertext{where $u$ is the constant of integration.} \\
            &\xrightarrow{\int\,dt} x(t) = -\frac{1}{2}gt ^ 2 + ut + 0.
        \end{align*}
        Max $\dot{x} = 0 \implies t = u / g$.
        So $H = x\left(\frac{u}{g}\right) = -\frac{1}{2}g\left(\frac{u}{g}\right) ^ 2 + u\left(\frac{u}{g}\right) = \frac{u ^ 2}{2g}$.
    \end{solution}
\end{example}

\subsection{Case \texorpdfstring{$F = F(t)$}{}}
The same method works,
do $\int\,dt$ twice.

\begin{example}
    A particle of mass $m = 1$ is at rest at $x = 0$.
    A force $F = e ^ {-t}$ is switched on at time $t = 0$.
    Find $x(t)$ for $t > 0$.

    \textit{Note:
    expect $\dot{x} > 0$ and $x > 0$ for $t > 0$
    (since $F > 0$).}

    \begin{solution}
        \[
        \ddot{x} = e ^ {-t} \xrightarrow{\int\,dt} \dot{x} = -e ^ {-t} + 1 \xrightarrow{\int\,dt} x = e ^ {-t} + t - 1.
        \]
    \end{solution}
\end{example}

\subsection{Case $F = F(\dot{x})$}

\begin{remark}
    Here $m\ddot{x} = F(\dot{x})$ second-order non linear differential equation.
\end{remark}







\end{document}