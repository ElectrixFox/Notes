\documentclass[10pt, a4paper]{article}
\usepackage{preamble}

\title{Dynamics I}
\author{Luke Phillips}
\date{January 2025}

\begin{document}

\maketitle

\newpage

\tableofcontents

\newpage

\section{Particle on a line}

\subsection{The Basics}

\begin{definition}[Position]
    The position $x(t)$ where $t$ is time.
\end{definition}

\begin{definition}[Velocity]
    The velocity $\dot{x}(t)$ where $t$ is time.
\end{definition}

\textit{Note that $\dot{x}$ means $\frac{dx}{dt}$.}

\begin{definition}[Acceleration]
    The acceleration $a(t) = \dot{v}(t) = \ddot{x}(t)$.
\end{definition}

\begin{definition}[Speed]
    The speed is $|v(t)|$,
    the magnitude of the velocity.
\end{definition}

\begin{definition}[Mass]
    A particle has mass $m > 0$.
\end{definition}

\begin{definition}[Momentum]
    A particle's momentum,
    $p$ is given by
    \[
    p = mv = m\dot{x}.
    \]
\end{definition}

\begin{definition}[Equation of motion]
    In an applied force $F$,
    then the equation of motion is $\dot{p} = F$.
\end{definition}

\begin{remark}\phantom{}
    \begin{enumerate}[label = (\roman*)]
        \item Could have $m$ depending on time
        (e.g. rocket).

        Unless otherwise stated,
        we take $m$ to be a \textbf{constant}.

        Then the EoM\footnote{Equation of motion} is $m\dot{v} = m\ddot{x} = F$.

        \item In general,
        $F = F(t, x, \dot{x})$ and $m\ddot{x} = F(t, x, \dot{x})$ is a second order differential equation which cannot be solved
        (explicitly).
    \end{enumerate}
\end{remark}

We focus on special cases which can be solved.

The simplest example is $F$ is constant.
For example,
vertical motion under gravity close to the surface of a planet:
then the magnitude of $F$ is $|F| = mg$
(the weight)
where the constant $g$ depends on the planet.

\begin{example}
    An object is shot up vertically from the ground with some initial speed $u$.
    What maximum height $H$ does it reach?
    \begin{solution}
        Could use $x$ upwards or $x$ downwards,
        both work,
        choose a convenient one.
        Here having $x$ upwards seems more natural.
        Then $m\ddot{x} = -mg$ with $x = 0$ at the ground.
        Then
        \begin{align*}
            \ddot{x} = -g &\xrightarrow{\int\,dt} \dot{x} = -gt + u \\
            \intertext{where $u$ is the constant of integration.} \\
            &\xrightarrow{\int\,dt} x(t) = -\frac{1}{2}gt ^ 2 + ut + 0.
        \end{align*}
        Max $\dot{x} = 0 \implies t = u / g$.
        So $H = x\left(\frac{u}{g}\right) = -\frac{1}{2}g\left(\frac{u}{g}\right) ^ 2 + u\left(\frac{u}{g}\right) = \frac{u ^ 2}{2g}$.
    \end{solution}
\end{example}

\subsection{Case \texorpdfstring{$F = F(t)$}{}}
The same method works,
do $\int\,dt$ twice.

\begin{example}
    A particle of mass $m = 1$ is at rest at $x = 0$.
    A force $F = e ^ {-t}$ is switched on at time $t = 0$.
    Find $x(t)$ for $t > 0$.

    \textit{Note:
    expect $\dot{x} > 0$ and $x > 0$ for $t > 0$
    (since $F > 0$).}

    \begin{solution}
        \[
        \ddot{x} = e ^ {-t} \xrightarrow{\int\,dt} \dot{x} = -e ^ {-t} + 1 \xrightarrow{\int\,dt} x = e ^ {-t} + t - 1.
        \]
    \end{solution}
\end{example}

\subsection{Case \texorpdfstring{$F = F(\dot{x})$}{}}

\begin{remark}
    Here $m\ddot{x} = F(\dot{x})$ second-order non linear differential equation.
\end{remark}

\begin{example}
    A particle of mass $m$ is slowed by a frictional force of magnitude $be ^ {av}$
    ($a$ and $b$ are constant).
    Initial speed $u$.
    What time $T$ does the particle travel for before coming to rest?
    
    \begin{solution}
        The particle either moves to the right or the left,
        the force moves in the opposite direction.

        Choosing the first situation.
        So
        \begin{align*}
            m\frac{dv}{dt} = -be ^ {av} &\implies \int e ^ {av}\,dv = -\frac{b}{m}\int\,dt \\
            \intertext{could use a constant of integration,
            fixed by $v(0) = u$.} \\
            \int e ^ {av}\,dv = -\frac{b}{m}\int\,dt &\implies \int_{u}^{0} e ^ {av}\,dv = -\frac{b}{m}\int_{0}^{T}\,dt \\
            &\implies \left[-\frac{1}{a}e ^ {-av}\right]_{u}^{0} = -\frac{bT}{m} \\
            &\implies T = \frac{m}{ab}\left(1 - e ^ {-au}\right).
        \end{align*}
    \end{solution}
\end{example}

\begin{example}
    A mass $M$ falls downwards under gravity,
    a resistive force $bw$,
    where $w$ is its speed.
    Initially $w(0) = 0$,
    find $w(t)$.
    What is the "least upper bound" of this $w$,
    the terminal speed.

    \begin{solution}
        We could use the $x$ axis pointing up or down.
        Use $x$ as downwards since the motion is downwards.

        Then
        \[
        m\dot{v} = mg - bv
        \]
        where $v$ is the velocity,
        here the velocity $v = w$.
        So
        \[
        m\dot{w} = mg - bw \implies \frac{dw}{dt} = g - \frac{b}{m}w \implies \int\frac{1}{g - \frac{b}{m}w}\,dw = \int\,dt
        \]
        which results in $w(t) = \frac{mg}{b}\left(1 - e ^ {-\frac{bt}{m}}\right)$,
        the terminal speed is
        \[
        \liminfty[t]\frac{mg}{b}\left(1 - e ^ {-\frac{bt}{m}}\right) = \frac{mg}{b}.
        \]
    \end{solution}
\end{example}

\subsection{Case \texorpdfstring{$F = F(x)$}{}}
Here $m\ddot{x} = F(x)$.
If $F(x) = bx + c$
($b$ and $c$ are constant),
then $m\ddot{x} - bx = c$ linear,
the solution is the complementary function $+$ the particular integral.
This doesn't work for $F(x)$ non-linear.
For that think of $v = v(x)$.
Then
\[
m\frac{dv}{dt} = m\frac{d}{dt}v(x(t)) = m\frac{dv}{dx}\cdot\frac{dx}{dt} = m\frac{dv}{dx}v = F(x).
\]
This is a first order separable ODE,
solve to get $v(x)$.
To get $x(t)$,
use $\frac{dx}{dt} = v(x)$ again first order separable.

\begin{example}
    A mass with $m = 2$ moves in a force with $F = 3x ^ 2$,
    initial conditions $x = 1$ and $v = 1$ at $t = 0$.
    What time $T$ does it take to reach $x = 9$?

    \begin{solution}
        \begin{align*}
            2v\frac{dv}{dx} = 3x ^ 2 &\implies \int 2v\,dv = \int 3x ^ 2\,dx \\
            &\implies v ^ 2 = x ^ 3 + c &\text{($c = 0$ from initial conditions)} \\
            &\implies v ^ 2 = x ^ 3 \\
            &\implies v = x ^ {\frac{3}{2}} &\text{(positive since $x = v = 1$ at $t = 0$)}
        \end{align*}
        Now
        \begin{align*}
            \frac{dx}{dt} = x ^ {\frac{3}{2}} &\implies \int_{1}^{9} x ^ {-\frac{3}{2}}\,dx = \int_{0}^{T}\,dt \\
            &\implies T = \left[-2x ^ {-\frac{1}{2}}\right]_{2}^{9} = -2\left(\frac{1}{3} - 1\right) = \frac{4}{3}.
        \end{align*}
    \end{solution}
\end{example}

\subsection{More Examples}

\begin{example}
    $F(t, v) = -v + e ^ {-t}$
    (with $v(0) = 0$)
    \begin{solution}
        This is linear,
        so solve with an integrating factor and apply the initial conditions to get
        \[
        v(t) = te ^ {-t}.
        \]
    \end{solution}
\end{example}

\begin{example}
    A rocket of mass $m(t)$ burns fuel at a constant rate $\dot{m} = -c$
    (with $c > 0$).
    Thus produces a constant force $F = kc$.
    Start with $x(0) = 0$ and $\dot{x}(0) = 0$,
    and initial mass $m_0$.
    Find $\dot{x}(t)$,
    hence $x(t)$.
    \begin{solution}
        Here the equation of motion $\frac{d}{dt}(mv) = F$ where $v = \dot{x}$.
        Do $\int\,dt$ giving $mv = kct + 0$
        (with $0$ being the constant of integration).
        Also $\dot{m} = -c \implies m(t) = -ct + m_0$.
        Thus $v(t) = \frac{kct}{m_0 - ct}$ for $0 \leq t < \frac{m_0}{c}$.

        Finally,
        $x(t) = \int v(t)\,dt$.
    \end{solution}
\end{example}

Two methods for $F = F(\dot{x})$.

\textbf{Method $1$}
is $m\frac{dv}{dt} = F(v)$,
get $v(t)$,
then $\int\,dt$ gives $x(t)$.

\textbf{Method $2$}
is $mv\frac{dv}{dx} = F(v)$ which is separable,
get $v(x)$,
then $\frac{dx}{dt} = v(x)$.

\begin{example}
    A unit mass
    (meaning $m = 1$)
    experiences a frictional force of magnitude $v + v ^ 2$
    ($v > 0$ is velocity).
    Find the general solution for $x(t)$.

    \begin{solution}
        Here method $2$ says
        \begin{align*}
            v\frac{dv}{dx} = -(v + v ^ 2) &\implies \frac{dv}{dx} = -(1 + v) \\
            &\implies \int\frac{1}{1 + v}\,dv = -\int\,dx \\
            &\implies \log(1 + v) = -x + C \\
            &\implies v(x) = e ^ {c - x} - 1.
        \end{align*}

        Then
        \begin{align*}
            \frac{dx}{dt} = v = e ^ {c - x} - 1 &\implies \int\frac{1}{e ^ {c - x} - 1}\,dx = \int\,dt = t + b \\
            &\implies \int\frac{e ^ x}{e ^ c - e ^ x}\,dx = -\log(e ^ c - e ^ x) = t + b \\
            &\implies -\log(e ^ c - e ^ x) = t + b \\
            &\implies e ^ c - e ^ x = e ^ {-b - b} \\
            &\implies x = \log(e ^ c - e ^ {-b - t}).
        \end{align*}
    \end{solution}
\end{example}

\begin{example}
    A unit mass is acted on by a force $F = \frac{1}{1 + x ^ 2}$,
    "starts" at $x = -\infty$ with $\dot{x} = u_0 > 0$.
    What is $\dot{x}$ as $x \rightarrow \infty$?

    \textit{[Note:
    $F > 0$ so we expect an answer $> u_0$.]}
    \begin{solution}
        We really want $v(x)$.
        The equation is
        \begin{align*}
            v\frac{dv}{dx} = \frac{1}{1 + x ^ 2} &\implies \int_{u_0}^{u_1} v\,dv = \int_{-\infty}^{\infty}\frac{1}{1 + x ^ 2}\,dx \\
            &\implies \frac{1}{2}u_1 ^ 2 - \frac{1}{2}u_0 ^ 2 = \left[\arctan{x}\right]_{-\infty}^{\infty} = \frac{\pi}{2} - \left(-\frac{\pi}{2}\right) = \pi \\
            &\implies u_1 ^ 2 = u_0 ^ 2 + 2\pi \\
            &\implies u_1 = \sqrt{u_0 ^ 2 + 2\pi}\qquad(> u_0\text{ as expected}).
        \end{align*}
    \end{solution}
\end{example}

Define $V(x) = -\int F(x)\,dx$.
Then
\[
\frac{d}{dx}\left(\frac{1}{2}mv ^ 2\right) = m\frac{dv}{dx}v = F = -\frac{d}{dx}V
\]
so $\frac{d}{dx}\left(\frac{1}{2}mv ^ 2 + V\right) = 0 \implies \frac{1}{2}mv ^ 2 + V(x) = E$.
This constant $E$ is the
(total)
energy,
while $\frac{1}{2}mv ^ 2$ is kinetic energy and $V(x)$ is potential energy.
Also note $\frac{d}{dt}E = \frac{dE}{dx}\frac{dx}{dt} = 0$,
therefore $E$ is constant in time.
\begin{remark}
    \begin{enumerate}[label = (\roman*)]
        \item We say that $E$ is conserved,
        and the force is conservative.

        \item Can add a constant
        (of integration)
        to $V$,
        and hence to $E$.

        \item Example:
        gravity close to the surface of a planet,
        take $x$ upwards,
        then force is
        \[
        F = -mg \implies V(x) = mgx\quad(+\text{optional constant}).
        \]
        
        \item In cases where one cannot solve for $x(t)$,
        energy arguments can still tell us something.
        See next example.
    \end{enumerate}
\end{remark}

\begin{example}
    Take $m = 2$ and $V(x) = x ^ 6$.
    So
    \[
    E = \frac{1}{2}mv ^ 2 + V(x) = \dot{x} ^ 2 + x ^ 6.
    \]
    Try to find $x(t)$ by
    \[
    \frac{dx}{dt} = \sqrt{E - x ^ 6} \implies \int\frac{1}{\sqrt{E - x ^ 6}}\,dx = \int\,dt.
    \]
    Note $V(x) \leq E$ restricting $x$ to say initial conditions $x(0) = 0, \dot{x}(0) = 1$.
    So $E = 1 ^ 6 + 0 ^ 6 = 1$ at $t = 0$,
    therefore for all $t$.
    Thus $-1 \leq x(t) \leq 1$.
    Think of a bowl,
    gravity acting downwards,
    particle rolls in it
    (just an analogy).

    Now $x$ increases to $1$,
    turns around and decreases,
    passes $x = 0$ and "rolls up" the other side,
    reaches $x = -1$,
    turns round,
    etc.
    Motion is periodic.

    Can we compute the period $P$
    (time to complete one cycle)?
    No,
    however
    \[
    \int\frac{1}{\sqrt{1 - x ^ 6}}\,dx,
    \]
    can't do that.
\end{example}

\begin{definition}
    An equilibrium point is where force is $0$,
    i.e. $\frac{dV}{dx} = 0$.
\end{definition}

In the above example,
one equilibrium point at $x = 0$.
(Particle at rest at such a point will stay there.)

\subsection{Motion in a potential}

\begin{example}
    Particle of mass $m = 3$ in a potential $V(x) = \frac{1}{4}x ^ 4 - \frac{1}{3}x ^ 3 - x ^ 2$.
    Describe the motion.

    Draw a graph of $V$.
    Compute $\frac{dV}{dx} = x ^ 3 - x ^ 2 - 2x = x(x ^ 2 - x - 2) = x(x - 2)(x + 1)$.
    So there are three equilibrium points $x = 0, 2, -1$.
\end{example}





































\end{document}