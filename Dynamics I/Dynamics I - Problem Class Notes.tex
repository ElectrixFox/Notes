\documentclass[10pt, a4paper]{article}
\usepackage{preamble}

\declaretheorem[style = avgstyle, name = Problem]{problem}

\title{Dynamics I \\
    \large Problem Class}
\author{Luke Phillips}
\date{Feburary 2025}

\begin{document}

\maketitle

\newpage

\tableofcontents

\newpage

\section{Questions from \texorpdfstring{$2020$}{} exam}

\begin{problem}
    Unit mass and charge,
    magnetic field upwards
    (constant)
    with magnitude $B$,
    gravity downwards
    (acceleration $g$).
    Throw mass with initial speed $u$,
    at angle $\alpha$ to the horizontal.
    Find condition on $S = Bu\sin{\alpha}$ such that the mass returns to its starting location.

    \begin{solution}
        (Expect answer involves $g$ and $n = 1, 2, 3, \dotsc$).
        Choose
        (say)
        $\mbf{e}_3$ upwards,
        $\mbf{e}_1$ and $\mbf{e}_2$ horizontal,
        (say)
        \[
        \dot{\mbf{r}}(0) = u(\cos{\alpha})\mbf{e}_1 + u(\sin{\alpha})\mbf{e}_3.
        \]
        And $\mbf{r}(0) = \mbf{0}$.
        The equation of motion is
        \[
        m\ddot{\mbf{r}} = \ddot{\mbf{r}} = -g\mbf{e}_3 + \dot{\mbf{r}} \times (B\mbf{e}_3).
        \]
        (Optional)
        Do $\int\,dt$ gives $\dot{\mbf{r}} = -gt\mbf{e}_3 + B\mbf{r} \times \mbf{e}_3 + u(\cos{\alpha})\mbf{e}_1 + u(\sin{\alpha})\mbf{e}_3$.

        This is
        \begin{align*}
            \dot{x} &= By + u\cos{\alpha} \\
            &\implies \ddot{x} = B\dot{y} = -B ^ 2x \\
            &\implies x(t) = C\sin(Bt) \\
            \intertext{$\dot{x}(0) = Bc = u\cos{\alpha} \implies \frac{u}{b}\cos{\alpha}$}
            &\implies x(t) = \frac{u}{b}(\cos{\alpha})\sin(Bt) \\
            \dot{y} &= -Bx \\
            &\implies y(t) = B ^ {-1}(\dot{x} - u\cos{\alpha}) \\
            &\implies y(t) = \frac{u}{B}(\cos{\alpha})(\cos(Bt) - 1) \\
            \dot{z} &= -gt + u\sin{\alpha} \\
            &\implies z(t) = -\frac{gt ^ 2}{2} + ut\sin{\alpha} + 0.
        \end{align*}

        We want $\mbf{r}(t) = \mbf{0} \implies t = \frac{2u}{g}\sin{\alpha}, \sin(Bt) = 0, \cos(Bt) = 1$,
        so $Bt = 2\pi n$,
        with $n \in \N$.
        Thus $\frac{2uB}{g}\sin{\alpha} = 2\pi n \implies S = \pi n g$ for $n = 1, 2, 3, \dotsc$.
    \end{solution}
\end{problem}

\begin{problem}
    Mass $m$ on $x$-axis
    ($x > 0$),
    repelled from $x = 0$ by a force $\frac{k}{\sqrt{x}}$
    ($k$ constant).
    Initial conditions $x(0) = 1$,
    $\dot{x}(0) = 2\sqrt{\frac{k}{m}}$.
    Find $x(t)$.

    \begin{solution}
        Two equivalent methods:
        \[
        F = mv\frac{dv}{dx} = \frac{k}{\sqrt{x}}
        \]
        get $v(x)$ etc.
        Use conserved energy $E$.

        Potential energy $V(x) = -\int F(x)\,dx = -\int\frac{k}{\sqrt{x}}\,dx = -2k\sqrt{x}+ C$
        ($C$ works out to be $0$).

        So
        \[
        E = \frac{1}{2}m\dot{x} ^ 2 - 2k\sqrt{x} \overset{t = 0}{=} 2k - 2k = 0.
        \]
        So
        \[
        \frac{dx}{dt} = \sqrt{\frac{4k}{m}}x ^ {\frac{1}{4}} \implies \int x ^ {-\frac{1}{4}}\,dx = \int\sqrt{\frac{4k}{m}}\,dt \implies \frac{4}{3}x ^ {\frac{3}{4}} = \sqrt{\frac{4k}{m}}t + C
        \]
        etc.
    \end{solution}
\end{problem}


\end{document}