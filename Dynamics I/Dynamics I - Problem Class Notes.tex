\documentclass[10pt, a4paper]{article}
\usepackage{preamble}

\title{Dynamics I \\
    \large Problem Class}
\author{Luke Phillips}
\date{Feburary 2025}

\begin{document}

\maketitle

\newpage

\tableofcontents

\newpage

\section{Questions from problem class \texorpdfstring{$1$}{}}

\begin{problem}[$2020$ exam]
    Unit mass and charge,
    magnetic field upwards
    (constant)
    with magnitude $B$,
    gravity downwards
    (acceleration $g$).
    Throw mass with initial speed $u$,
    at angle $\alpha$ to the horizontal.
    Find condition on $S = Bu\sin{\alpha}$ such that the mass returns to its starting location.

    \begin{solution}
        (Expect answer involves $g$ and $n = 1, 2, 3, \dotsc$).
        Choose
        (say)
        $\mbf{e}_3$ upwards,
        $\mbf{e}_1$ and $\mbf{e}_2$ horizontal,
        (say)
        \[
        \dot{\mbf{r}}(0) = u(\cos{\alpha})\mbf{e}_1 + u(\sin{\alpha})\mbf{e}_3.
        \]
        And $\mbf{r}(0) = \mbf{0}$.
        The equation of motion is
        \[
        m\ddot{\mbf{r}} = \ddot{\mbf{r}} = -g\mbf{e}_3 + \dot{\mbf{r}} \times (B\mbf{e}_3).
        \]
        (Optional)
        Do $\int\,dt$ gives $\dot{\mbf{r}} = -gt\mbf{e}_3 + B\mbf{r} \times \mbf{e}_3 + u(\cos{\alpha})\mbf{e}_1 + u(\sin{\alpha})\mbf{e}_3$.

        This is
        \begin{align*}
            \dot{x} &= By + u\cos{\alpha} \\
            &\implies \ddot{x} = B\dot{y} = -B ^ 2x \\
            &\implies x(t) = C\sin(Bt) \\
            \intertext{$\dot{x}(0) = Bc = u\cos{\alpha} \implies \frac{u}{b}\cos{\alpha}$}
            &\implies x(t) = \frac{u}{b}(\cos{\alpha})\sin(Bt) \\
            \dot{y} &= -Bx \\
            &\implies y(t) = B ^ {-1}(\dot{x} - u\cos{\alpha}) \\
            &\implies y(t) = \frac{u}{B}(\cos{\alpha})(\cos(Bt) - 1) \\
            \dot{z} &= -gt + u\sin{\alpha} \\
            &\implies z(t) = -\frac{gt ^ 2}{2} + ut\sin{\alpha} + 0.
        \end{align*}

        We want $\mbf{r}(t) = \mbf{0} \implies t = \frac{2u}{g}\sin{\alpha}, \sin(Bt) = 0, \cos(Bt) = 1$,
        so $Bt = 2\pi n$,
        with $n \in \N$.
        Thus $\frac{2uB}{g}\sin{\alpha} = 2\pi n \implies S = \pi n g$ for $n = 1, 2, 3, \dotsc$.
    \end{solution}
\end{problem}

\begin{problem}[$2020$ exam]
    Mass $m$ on $x$-axis
    ($x > 0$),
    repelled from $x = 0$ by a force $\frac{k}{\sqrt{x}}$
    ($k$ constant).
    Initial conditions $x(0) = 1$,
    $\dot{x}(0) = 2\sqrt{\frac{k}{m}}$.
    Find $x(t)$.

    \begin{solution}
        Two equivalent methods:
        \[
        F = mv\frac{dv}{dx} = \frac{k}{\sqrt{x}}
        \]
        get $v(x)$ etc.
        Use conserved energy $E$.

        Potential energy $V(x) = -\int F(x)\,dx = -\int\frac{k}{\sqrt{x}}\,dx = -2k\sqrt{x}+ C$
        ($C$ works out to be $0$).

        So
        \[
        E = \frac{1}{2}m\dot{x} ^ 2 - 2k\sqrt{x} \overset{t = 0}{=} 2k - 2k = 0.
        \]
        So
        \[
        \frac{dx}{dt} = \sqrt{\frac{4k}{m}}x ^ {\frac{1}{4}} \implies \int x ^ {-\frac{1}{4}}\,dx = \int\sqrt{\frac{4k}{m}}\,dt \implies \frac{4}{3}x ^ {\frac{3}{4}} = \sqrt{\frac{4k}{m}}t + C
        \]
        etc.
    \end{solution}
\end{problem}

\newpage

\section{Questions from problem class \texorpdfstring{$2$}{}}

\begin{problem}[$2019$ exam]
    Particle mass $m$ moves on the $xy$-plane,
    with velocity $u(\cos{\alpha})\mbf{e}_1 + u(\sin{\alpha})\mbf{e}_2$.
    Another particle,
    mass $2m$,
    has velocity $-U\mbf{e}_1$ where $U > 0$.
    After an elastic collision,
    the $2m$ particle is at rest.
    Find $U$ in terms of $u$ and $\alpha$.

    \begin{solution}
        Know $u$ and $\alpha$
        (and $m$),
        unknowns $U$ and $\mbf{w}$ velocity of $m$ afterwards:
        \[
        \mbf{w} = w_1\mbf{e}_1 + w_2\mbf{e}_2,
        \]
        so three unknown numbers.
        Need three equations:
        momentum gives two,
        plus energy
        ("elastic").

        \begin{align*}
            mu\cos{\alpha} - 2mU &= mw_1 \\
            mu\sin{\alpha} &= mw_2 \\
            \frac{1}{2}mu ^ 2 + \frac{1}{2}2mU ^ 2 &= \frac{1}{2}m(w_1 ^ 2 + w_2 ^ 2).
        \end{align*}
        Eliminate $w_1$ and $w_2$ using the two linear equations:
        substitute into the final equation.
        \begin{align*}
            u ^ 2 + 2U ^ 2 &= (u\cos{\alpha} - 2U) ^ 2 + (u\sin{\alpha}) ^ 2 \\
            &= u ^ 2 - 4uU\cos{\alpha} + 4U ^ 2 \\
            &\iff \\
            2U ^ 2 - 4uU\cos{\alpha} &= 0 \implies U = 2u\cos{\alpha} && (U > 0).
        \end{align*}
    \end{solution}
\end{problem}

\begin{problem}[$2020$ exam]
    Mass $m$ moves on $x$-axis,
    in potential $V(x) = me ^ x(x - 1) ^ 2$.
    \begin{enumerate}[label = (\alph*)]
        \item Find period $P$ of small oscillations about the stable equilibrium.

        \item Starting at $x = 0$ with speed $u$,
        how big does $u$ have to be for $x \rightarrow -\infty$.
    \end{enumerate}

    \begin{solution}
        \begin{enumerate}[label = (\alph*)]
            \item
            \[
            V'(x) = me ^ x(x - 1)((x - 1) + 2) = me ^ x(x - 1)(x + 1).
            \]
            Two equilibria at $x = \pm 1$.
            Note $V(x) \geq 0$,
            $V(x) = 0$ only at $x = 1$,
            $V(x) \rightarrow \infty$ as $x \rightarrow \infty$,
            $V(x) \rightarrow 0$ as $x \rightarrow -\infty$.

            Stable equilibrium at $x = 1$,
            period,
            with $\omega = \sqrt{\frac{V''(1)}{m}}$,
            $p = \frac{2\pi}{\omega} = 2\pi\sqrt{\frac{m}{V''(1)}}$.
            Now
            \[
            V''(1) = me ^ x(x + 1)|_{x = 1} = 2me.
            \]
            Hence $p = \frac{2\pi}{\sqrt{2e}}$.

            \item Energy:
            \begin{align*}
                \frac{1}{2}mu ^ 2 + V(0) &> V(-1) \\
                &\iff \\
                u ^ 2 &> \frac{2}{m}(V(-1) - V(0) \\
                &= 2(4e ^ {-1} - 1).
            \end{align*}
            $u > \sqrt{2(4e ^ {-1} - 1)}$.
        \end{enumerate}
    \end{solution}
\end{problem}

\begin{problem}[$2022$ exam]
    Unit mass moves on a line with a force $F = -2v + e ^ {-t}$
    ($v$ is velocity).
    At rest at $t = 0$.
    For what value of $t$ is the speed maximum?

    \begin{solution}
        \[
        \frac{dv}{dt} = -2v + e ^ {-t},
        \]
        linear.
        Integrating factor,
        get
        \[
        v(t) = e ^ {-t} - e ^ {-2t}.
        \]
        Then
        \[
        \frac{dv}{dt} = -e ^ {-t} + 2e ^ {-2t} = 0 \iff e ^ {-t} = \frac{1}{2} \iff t = \log(2).
        \]
        
    \end{solution}
\end{problem}

\newpage

\section{Questions from problem class \texorpdfstring{$3$}{}}

\begin{problem}[$2021$ exam]
    Pendulum:
    light rod length $L$,
    pivot at one end,
    mass $m$ other end,
    mass $3m$ distance $L / 3$ from pivot.

    \begin{enumerate}[label = (\roman*)]
        \item Find distance $D$ from pivot to centre of mass.

        \item Find the moment of inertia $I$.

        \item Begin at stable equilibrium,
        mass at end velocity $u$.
        What $u$ causes rod to reach horizontal but no further?

        \item Find period $P$ of small oscillation about stable equilibrium.
    \end{enumerate}

    \begin{solution}
        \begin{enumerate}[label = (\roman*)]
            \item
            \[
            D = \frac{1}{4m}\left(mL + (3m)\cdot\left(\frac{L}{3}\right)\right) = \frac{1}{2}L.
            \]

            \item
            \[
            I = mL ^ 2 + (3m)\cdot\left(\frac{L}{3}\right) ^ 2 = \frac{4}{3}mL ^ 2.
            \]

            \item Energy
            \[
            E = \frac{2}{3}mL ^ 2\dot{\theta} ^ 2 + (4m)\cdot g\cdot\left(\frac{L}{2}\right)(1 - \cos{\theta}).
            \]
            End at $\theta = \frac{\pi}{2}$,
            $\dot{\theta} = 0$ which implies $E = 2mgL$,
            start $\theta = 0$,
            $L\dot{\theta} = u$ implies $E = \frac{2}{3}mu ^ 2$
            which implies
            \[
            2mgL = \frac{2}{3}mu ^ 2 \implies gL = \frac{1}{3}u ^ 2 \implies u = \sqrt{3gL}.
            \]

            \item Equation of motion
            \[
            \frac{dE}{dt} = 0 \implies \frac{4}{3}mL ^ 2\ddot{\theta} + 2mgL\sin{\theta} = 0.
            \]
            Take $\theta$ small,
            so
            \[
            \ddot{\theta} + \frac{3g}{2L}\theta = 0 \implies P = 2\pi\sqrt{\frac{2L}{3g}}.
            \]
        \end{enumerate}
    \end{solution}
\end{problem}

\begin{problem}[$2022$ exam]
    Find $u(x, t)$ from $u_{tt} = 4u_{xx}$,
    $u(x, 0) = \sin ^ 2(2x)$ and $u_t(x, 0) = 8\sin(2x)\cos(2x)$.

    \begin{solution}
        Here $c = \sqrt{4} = 2$.
        Need
        \[
        \frac{1}{c}\int S(x)\,dx = \sin ^ 2(2x).
        \]
        (Note $\frac{d}{dx}\sin ^ 2(2x) = 4\sin(2x)\cos(2x)$)
        so
        \[
        u(x, t) = \sin ^ 2(2(x + 2t)).
        \]
    \end{solution}
\end{problem}

\begin{problem}[$2021$ exam]
    Water drop falls vertically from rest,
    only force is its weight $mg$.
    Mass increases like $m(t) = m_0e ^ {kt}$
    ($m_0, k$ are positive constants).
    What is its speed $w$ when the mass has doubled?

    \begin{solution}
        The equation
        \[
        \frac{d}{dt}(mv) = mg \implies \frac{d}{dt}(m_0e ^ {kt}v(t)) = gm_0e ^ {kt}
        \]
        \[
        \int\,dt \implies e ^ {kt}v(t) = \frac{g}{k}(e ^ {kt} - 1)
        \]
        using $v(0) = 0$ then
        \[
        v(t) = \frac{g}{k}(1 - e ^ {-kt}).
        \]
        $m(0) = m_0$ goes to $2m_0$ so $e ^ {kt} = 2$ so
        \[
        w = \frac{g}{k}\left(1 - \frac{1}{2}\right) = \frac{g}{2k}.
        \]
    \end{solution}
\end{problem}

\begin{problem}[$2021$ exam]
    Unit mass on $x$-axis,
    starts at $x = 0$ with velocity $u > 0$,
    slowed by force of magnitude $2vx$,
    ($v$ velocity)
    what is the maximum value $x$ reaches?

    \begin{solution}
        Equation
        \[
        v\frac{dv}{dx} = -2vx \implies \frac{dv}{dx} = -2x \implies v(x) = -x ^ 2 + u.
        \]
        So $x_{\text{max}}$ at $v = 0$ so $x_{\text{max}} = \sqrt{u}$.
    \end{solution}
\end{problem}

\newpage

\section{Questions from problem class \texorpdfstring{$4$}{}}

\begin{problem}[2024 4.2.]
    Unit mass $m$ attractive central force magnitude $\lambda ^ 2r$
    ($\lambda > 0$ constant).
    Initially $\mbf{r}(0) = \mbf{e}_r$,
    $\mbf{v}(0) = u\mbf{e}_{\theta}$,
    with $0 < u < \lambda$.
    \begin{enumerate}[label = (\roman*)]
        \item Find equation like $\ddot{r} ^ 2 = W(r)$.

        \item Find $r_{\text{min}}$ and $r_{\text{max}}$.
    \end{enumerate}

    \begin{solution}
        \begin{enumerate}[label = (\roman*)]
            \item 
            Angular momentum
            \[
            \mbf{L} = m\mbf{r} \times \mbf{v} \overset{t = 0}{=} \mbf{e}_r \times (u\mbf{e}_{\theta}) = u\mbf{e}_3  \implies L = u
            \]
            and $L = mr ^ 2\dot{\theta} = r ^ 2\dot{\theta} = u$,
            therefore $\dot{\theta} = u / r ^ 2$.
            Energy $E = K + V$ where $V = -\int f(r)\,dr$ with $f(r) = -\lambda ^ 2r$,
            which implies $V = \frac{1}{2}\lambda ^ 2r ^ 2$.
            So
            \[
            E \overset{t = 0}{=} \frac{1}{2}mv_0 ^ 2 + V_{t = 0} = \frac{1}{2}u ^ 2 + \frac{1}{2}\lambda ^ 2
            \]
            and $E = \frac{1}{2}m(\dot{r} + r ^ 2\dot{\theta} ^ 2) + V(r) = \frac{1}{2}\left(\dot{r} ^ 2 + \frac{u ^ 2}{r ^ 2}\right) + \frac{1}{2}\lambda ^ 2r ^ 2$,
            which implies $\dot{r} ^ 2 = W(r) = (u ^ 2 + \lambda ^ 2) - \frac{u ^ 2}{r ^ 2} - \lambda ^ 2r ^ 2$.

            \item
            Max/min when $\dot{r} = 0$ and $W(r) = 0$,
            $W(r) = (\lambda ^ 2 - u ^ 2 / r ^ 2)(1 - r ^ 2)$,
            zeros at $r = 1, r = u / \lambda$.
            $r_{\text{max}} = 1$,
            $r_{\text{min}} = u / \lambda$.
        \end{enumerate}
    \end{solution}
\end{problem}

\begin{problem}[2024 3.2.]
    Particle mass $M$ moves along $x$-axis collides with two particles both stationary with $m = 10$.
    After the collision $M$ is at rest and the other two particles is moving at $\phi$ to the $x$-axis.

    Find $R = \frac{M}{m}$ in terms of $\phi$.

    \begin{solution}
        Let the speed of the first particle be $v$ and the speeds of the other two particles after the collision $u$
        (they have the same speedy by symmetry).

        Momentum:
        \[
        Mv = 2mu\cos{\phi}.
        \]
        Energy:
        \[
        \frac{1}{2}Mv ^ 2 = mu ^ 2.
        \]
        So
        \[
        \frac{u ^ 2}{v ^ 2} = \frac{M}{2m} = \left(\frac{M}{2m\cos{\phi}}\right) ^ 2 = \frac{M ^ 2}{4m ^ 2\cos ^ 2{\phi}} \implies R = \frac{M}{m} = 2\cos ^ 2{\phi}.
        \]
    \end{solution}
\end{problem}

\begin{problem}[2024 5.2.]
    Rod mass $M$ length $L$,
    pivoted at one end,
    mass $2M$ at other end and mass $3M$ in the middle.

    \begin{enumerate}[label = (\roman*)]
        \item Distance $D$ from pivot to centre of mass
        (Regard rod as single particle).

        \item Find moment of inertia $I$.

        \item Find energy in terms of angle.

        \item Release rod from horizontal position,
        what is the speed $u$ of the end when vertical.
    \end{enumerate}

    \begin{solution}
        \begin{enumerate}[label = (\roman*)]
            \item 
            \[
            \frac{1}{6M}\left(M \cdot \frac{1}{2}L + 2M\cdot L + 3M \cdot \frac{1}{2}L\right) = \frac{2}{3}L.
            \]

            \item
            \[
            I = \int_{0}^{L}M\frac{1}{L}x ^ 2\,dx = \frac{1}{3}ML ^ 2
            \]
            so $I = \frac{1}{3}ML ^ 2 = 2ML ^ 2 + 3M\left(\frac{1}{2}L ^ 2\right) ^ 2 = \frac{37}{12}ML ^ 2$.

            \item
            \[
            E = \frac{1}{2}I\dot{\theta} ^ 2 + V(\theta) = \frac{37}{24}ML ^ 2\dot{\theta} ^ 2 + (6M)g\left(\frac{2}{3}L\right)(1 - \cos{\theta}).
            \]

            \item
            \[
            4MgL = \frac{37}{24}ML ^ 2\dot{\theta} ^ 2 = \frac{37}{24}u ^ 2.
            \]
            $u = L\dot{\theta}$
        \end{enumerate}
    \end{solution}
\end{problem}

\end{document}