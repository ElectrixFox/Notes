\documentclass[10pt, a4paper]{article}
\usepackage{preamble}

\title{Dynamics I}
\author{Luke Phillips}
\date{January 2025}

\begin{document}

\maketitle

\newpage

\tableofcontents

\newpage

\section{Particle on a line}

\subsection{The Basics}

\begin{definition}[Position]
    The position $x(t)$ where $t$ is time.
\end{definition}

\begin{definition}[Velocity]
    The velocity $\dot{x}(t)$ where $t$ is time.
\end{definition}

\textit{Note that $\dot{x}$ means $\frac{dx}{dt}$.}

\begin{definition}[Acceleration]
    The acceleration $a(t) = \dot{v}(t) = \ddot{x}(t)$.
\end{definition}

\begin{definition}[Speed]
    The speed is $|v(t)|$,
    the magnitude of the velocity.
\end{definition}

\begin{definition}[Mass]
    A particle has mass $m > 0$.
\end{definition}

\begin{definition}[Momentum]
    A particle's momentum,
    $p$ is given by
    \[
    p = mv = m\dot{x}.
    \]
\end{definition}

\begin{definition}[Equation of motion]
    In an applied force $F$,
    then the equation of motion is $\dot{p} = F$.
\end{definition}

\begin{remark}\phantom{}
    \begin{enumerate}[label = (\roman*)]
        \item Could have $m$ depending on time
        (e.g. rocket).

        Unless otherwise stated,
        we take $m$ to be a \textbf{constant}.

        Then the EoM\footnote{Equation of motion} is $m\dot{v} = m\ddot{x} = F$.

        \item In general,
        $F = F(t, x, \dot{x})$ and $m\ddot{x} = F(t, x, \dot{x})$ is a second order differential equation which cannot be solved
        (explicitly).
    \end{enumerate}
\end{remark}

We focus on special cases which can be solved.

The simplest example is $F$ is constant.
For example,
vertical motion under gravity close to the surface of a planet:
then the magnitude of $F$ is $|F| = mg$
(the weight)
where the constant $g$ depends on the planet.

\begin{example}
    An object is shot up vertically from the ground with some initial speed $u$.
    What maximum height $H$ does it reach?
    \begin{solution}
        Could use $x$ upwards or $x$ downwards,
        both work,
        choose a convenient one.
        Here having $x$ upwards seems more natural.
        Then $m\ddot{x} = -mg$ with $x = 0$ at the ground.
        Then
        \begin{align*}
            \ddot{x} = -g &\xrightarrow{\int\,dt} \dot{x} = -gt + u \\
            \intertext{where $u$ is the constant of integration.} \\
            &\xrightarrow{\int\,dt} x(t) = -\frac{1}{2}gt ^ 2 + ut + 0.
        \end{align*}
        Max $\dot{x} = 0 \implies t = u / g$.
        So $H = x\left(\frac{u}{g}\right) = -\frac{1}{2}g\left(\frac{u}{g}\right) ^ 2 + u\left(\frac{u}{g}\right) = \frac{u ^ 2}{2g}$.
    \end{solution}
\end{example}

\subsection{Case \texorpdfstring{$F = F(t)$}{} and \texorpdfstring{$F = F(\dot{x})$}{}}
The same method works,
do $\int\,dt$ twice.

\begin{example}
    A particle of mass $m = 1$ is at rest at $x = 0$.
    A force $F = e ^ {-t}$ is switched on at time $t = 0$.
    Find $x(t)$ for $t > 0$.

    \textit{Note:
    expect $\dot{x} > 0$ and $x > 0$ for $t > 0$
    (since $F > 0$).}

    \begin{solution}
        \[
        \ddot{x} = e ^ {-t} \xrightarrow{\int\,dt} \dot{x} = -e ^ {-t} + 1 \xrightarrow{\int\,dt} x = e ^ {-t} + t - 1.
        \]
    \end{solution}
\end{example}

\begin{remark}
    Here $m\ddot{x} = F(\dot{x})$ second-order non linear differential equation.
\end{remark}

\begin{example}
    A particle of mass $m$ is slowed by a frictional force of magnitude $be ^ {av}$
    ($a$ and $b$ are constant).
    Initial speed $u$.
    What time $T$ does the particle travel for before coming to rest?
    
    \begin{solution}
        The particle either moves to the right or the left,
        the force moves in the opposite direction.

        Choosing the first situation.
        So
        \begin{align*}
            m\frac{dv}{dt} = -be ^ {av} &\implies \int e ^ {av}\,dv = -\frac{b}{m}\int\,dt \\
            \intertext{could use a constant of integration,
            fixed by $v(0) = u$.} \\
            \int e ^ {av}\,dv = -\frac{b}{m}\int\,dt &\implies \int_{u}^{0} e ^ {av}\,dv = -\frac{b}{m}\int_{0}^{T}\,dt \\
            &\implies \left[-\frac{1}{a}e ^ {-av}\right]_{u}^{0} = -\frac{bT}{m} \\
            &\implies T = \frac{m}{ab}\left(1 - e ^ {-au}\right).
        \end{align*}
    \end{solution}
\end{example}

\begin{example}
    A mass $M$ falls downwards under gravity,
    a resistive force $bw$,
    where $w$ is its speed.
    Initially $w(0) = 0$,
    find $w(t)$.
    What is the "least upper bound" of this $w$,
    the terminal speed.

    \begin{solution}
        We could use the $x$ axis pointing up or down.
        Use $x$ as downwards since the motion is downwards.

        Then
        \[
        m\dot{v} = mg - bv
        \]
        where $v$ is the velocity,
        here the velocity $v = w$.
        So
        \[
        m\dot{w} = mg - bw \implies \frac{dw}{dt} = g - \frac{b}{m}w \implies \int\frac{1}{g - \frac{b}{m}w}\,dw = \int\,dt
        \]
        which results in $w(t) = \frac{mg}{b}\left(1 - e ^ {-\frac{bt}{m}}\right)$,
        the terminal speed is
        \[
        \liminfty[t]\frac{mg}{b}\left(1 - e ^ {-\frac{bt}{m}}\right) = \frac{mg}{b}.
        \]
    \end{solution}
\end{example}

\subsection{Case \texorpdfstring{$F = F(x)$}{}}
Here $m\ddot{x} = F(x)$.
If $F(x) = bx + c$
($b$ and $c$ are constant),
then $m\ddot{x} - bx = c$ linear,
the solution is the complementary function $+$ the particular integral.
This doesn't work for $F(x)$ non-linear.
For that think of $v = v(x)$.
Then
\[
m\frac{dv}{dt} = m\frac{d}{dt}v(x(t)) = m\frac{dv}{dx}\cdot\frac{dx}{dt} = m\frac{dv}{dx}v = F(x).
\]
This is a first order separable ODE,
solve to get $v(x)$.
Integrating gives
\[
\frac{1}{2}mv ^ 2 = \int F(x)\,dx,
\]
and hence you get $v(x)$.
To get $x(t)$,
use $\frac{dx}{dt} = v(x)$ again first order separable.

\begin{example}
    A mass with $m = 2$ moves in a force with $F = 3x ^ 2$,
    initial conditions $x = 1$ and $v = 1$ at $t = 0$.
    What time $T$ does it take to reach $x = 9$?

    \begin{solution}
        \begin{align*}
            2v\frac{dv}{dx} = 3x ^ 2 &\implies \int 2v\,dv = \int 3x ^ 2\,dx \\
            &\implies v ^ 2 = x ^ 3 + c &\text{($c = 0$ from initial conditions)} \\
            &\implies v ^ 2 = x ^ 3 \\
            &\implies v = x ^ {\frac{3}{2}} &\text{(positive since $x = v = 1$ at $t = 0$)}
        \end{align*}
        Now
        \begin{align*}
            \frac{dx}{dt} = x ^ {\frac{3}{2}} &\implies \int_{1}^{9} x ^ {-\frac{3}{2}}\,dx = \int_{0}^{T}\,dt \\
            &\implies T = \left[-2x ^ {-\frac{1}{2}}\right]_{2}^{9} = -2\left(\frac{1}{3} - 1\right) = \frac{4}{3}.
        \end{align*}
    \end{solution}
\end{example}

\begin{remark}
    For problems with $F = F(v)$ we can use $F(v) = m\frac{dv}{dt}$ as we saw before.
    Sometimes it may be more convenient to use $F(v) = mv\frac{dv}{dx}$,
    in situations where we want $v(x)$ rather than $v(t)$.
\end{remark}

\subsection{Two more general examples}

\begin{example}['Mixed' example]
    A unit mass moves along a line,
    acted on by a force $F(v, t) = -v + e ^ {-t}$ where $v$ is its velocity.
    Given that $v(0) = 0$,
    find $v(t)$.
    \begin{solution}
        The equation of motion is
        \[
        \dot{v} = -v + e ^ {-t},
        \]
        which is a linear ODE for $v(t)$.
        Solving it via an integrating factor $e ^ t$ gives
        \[
        v(t) = te ^ {-t}.
        \]
    \end{solution}
\end{example}

\begin{example}[Rocket: non-constant mass]
    A rocket of mass $m(t)$ burns fuel at a constant rate $\dot{m} = -c$,
    where $c > 0$.
    This produces a constant thrust $F = k|\dot{m}| = kc$.
    The rocket starts at rest at $x = 0$,
    with initial mass $m_0$.
    Find its subsequent position $x(t)$.
    \begin{solution}
        The constant force gives $mv = kct$.
        Now $m(t) = m_0 - ct$,
        so
        \[
        v(t) = \frac{kct}{m_0 - ct} = -k + \frac{km_0}{m_0 - ct}
        \]
        for $0 \leq t < \frac{m_0}{c}$.
        Integrating this gives
        \[
        x(t) = -kt + \frac{km_0}{c}\log\left(\frac{m_0}{m_0 - ct}\right)
        \]
        for $0 \leq t < \frac{m_0}{c}$.
        Note that for small $t$ we have
        \[
        v(t) \approx \frac{kct}{m_0};
        \]
        whereas when $t \rightarrow \frac{m_0}{c}$ we get $v \rightarrow \infty$ and $x \rightarrow \infty$.
    \end{solution}
\end{example}

\subsection{Energy}

\begin{definition}
    Consider the case $F = F(x)$,
    and write $V(x) = -\int F(x)\,dx$.
    Then
    \[
    \frac{d}{dx}\left(\frac{1}{2}mv ^ 2\right) = mv\frac{dv}{dx} = ma = F = -\frac{dV}{dx},
    \]
    and so
    \[
    E = \frac{1}{2}mv ^ 2 + V(x)
    \]
    is constant in time,
    since the chain rule gives
    \[
    \frac{dE}{dt} = \frac{dE}{dx}\dot{x} = 0.
    \]
    This quantity $E$ is the energy of the particle,
    we say that it is conserved.
    The term $K = \frac{1}{2}mv ^ 2$ is the kinetic energy,
    $V(x)$ is the potential energy;
    neither of these is conserved in general.
\end{definition}

\begin{remark}
    \begin{enumerate}[label = (\roman*)]
        \item A force of the form $F(x)$ is said to be conservative,
        since it conserves energy.

        \item The potential $V$,
        and hence the energy $E$,
        are only defined up to an arbitrary constant.

        \item The simplest example is where $F$ is constant,
        so $V$ is linear in $x$.

        \item Using conservation of energy can give us a good picture of the particle's motion,
        even when we cannot get $x(t)$ explicitly.

        \item A point $x$ where $F(x) = 0$,
        or equivalently $V'(x) = 0$,
        is called an equilibrium position.
    \end{enumerate}
\end{remark}

\begin{example}
    A unit-mass particle moves along the $x$-axis,
    subject to a force $F = -2x ^ 3$.
    Derive an expression for its energy $E$,
    such that $E = 0$ if the particle is stationary at $x = 0$.
    Then compute $E$ if its speed is $v = 1$ at $x = 1$.
    \begin{solution}
        Writing $V(x) = -\int F(x)\,dx$.
        \[
        V(x) = -\int -2x ^ 3\,dx = \frac{1}{2}x ^ 4 + C.
        \]
        Now we have
        \begin{align*}
            E &= \frac{1}{2}mv ^ 2 + V(x) \\
            &= \frac{1}{2}v ^ 2 + \frac{1}{2}x ^ 4 + C \\
            \intertext{But $E = 0, v = 0, x = 0$ so $0 = 0 + 0 + C$ hence $C = 0$} \\
            &= \frac{1}{2}\left(v ^ 2 + x ^ 4\right).
        \end{align*}
        Finally,
        we can see that $E$ with $v = 1$ and $x = 1$ is
        \[
        E = \frac{1}{2}\left(v ^ 2 + x ^ 4\right) = \frac{1}{2}(1 + 1) = 1.
        \]
    \end{solution}
\end{example}

\subsection{Motion in a potential}
Consider a particle moving along the $x$-axis,
in a potential $V(x)$.
Except in simple cases,
we won't know exactly what the trajectory $x(t)$ is.
But we can say something about what it looks like,

\begin{example}
    A double-well potential.
    A particle of mass $3$ moves in a potential
    \[
    V(x) = \frac{x ^ 4}{4} - \frac{x ^ 3}{3} - x ^ 2.
    \]
    At what points can the particle remain at rest
    (be in equilibrium)?
    If the particle starts at $x = 1$ with speed $u$,
    how large does $u$ have to be for it to reach $x = -1$?
    \begin{solution}
        We can see that the equilibrium points are at $V'(x) = 0$.
        Now
        \[
        0 = V'(x) = x(x - 2)(x + 1)
        \]
        so there are three equilibria,
        at $x = -1, 0, 2$.
        The potential $V(x)$ is a 'double well',
        with local maximum at $x = 0$,
        and local minima at $x = -1$ and $x = 2$.
        To get from $x = 1$ to $x = -1$,
        the particle needs enough initial kinetic energy to get over the 'hill' at $x = 0$.
        In other words,
        we need
        \[
        \frac{3}{2}u ^ 2 + V(1) > V(0),
        \]
        using $V(1) = -\frac{13}{12}$ and $V(0) = 0$ then gives $u > \sqrt{\frac{13}{18}}$.
    \end{solution}
\end{example}






























\end{document}