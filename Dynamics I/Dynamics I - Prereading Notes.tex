\documentclass[10pt, a4paper]{article}
\usepackage{preamble}

\title{Dynamics I}
\author{Luke Phillips}
\date{January 2025}

\begin{document}

\maketitle

\newpage

\tableofcontents

\newpage

\section{Particle on a line}

\subsection{The Basics}

\begin{definition}[Position]
    The position $x(t)$ where $t$ is time.
\end{definition}

\begin{definition}[Velocity]
    The velocity $\dot{x}(t)$ where $t$ is time.
\end{definition}

\textit{Note that $\dot{x}$ means $\frac{dx}{dt}$.}

\begin{definition}[Acceleration]
    The acceleration $a(t) = \dot{v}(t) = \ddot{x}(t)$.
\end{definition}

\begin{definition}[Speed]
    The speed is $|v(t)|$,
    the magnitude of the velocity.
\end{definition}

\begin{definition}[Mass]
    A particle has mass $m > 0$.
\end{definition}

\begin{definition}[Momentum]
    A particle's momentum,
    $p$ is given by
    \[
    p = mv = m\dot{x}.
    \]
\end{definition}

\begin{definition}[Equation of motion]
    In an applied force $F$,
    then the equation of motion is $\dot{p} = F$.
\end{definition}

\begin{remark}\phantom{}
    \begin{enumerate}[label = (\roman*)]
        \item Could have $m$ depending on time
        (e.g. rocket).

        Unless otherwise stated,
        we take $m$ to be a \textbf{constant}.

        Then the EoM\footnote{Equation of motion} is $m\dot{v} = m\ddot{x} = F$.

        \item In general,
        $F = F(t, x, \dot{x})$ and $m\ddot{x} = F(t, x, \dot{x})$ is a second order differential equation which cannot be solved
        (explicitly).
    \end{enumerate}
\end{remark}

We focus on special cases which can be solved.

The simplest example is $F$ is constant.
For example,
vertical motion under gravity close to the surface of a planet:
then the magnitude of $F$ is $|F| = mg$
(the weight)
where the constant $g$ depends on the planet.

\begin{example}
    An object is shot up vertically from the ground with some initial speed $u$.
    What maximum height $H$ does it reach?
    \begin{solution}
        Could use $x$ upwards or $x$ downwards,
        both work,
        choose a convenient one.
        Here having $x$ upwards seems more natural.
        Then $m\ddot{x} = -mg$ with $x = 0$ at the ground.
        Then
        \begin{align*}
            \ddot{x} = -g &\xrightarrow{\int\,dt} \dot{x} = -gt + u \\
            \intertext{where $u$ is the constant of integration.} \\
            &\xrightarrow{\int\,dt} x(t) = -\frac{1}{2}gt ^ 2 + ut + 0.
        \end{align*}
        Max $\dot{x} = 0 \implies t = u / g$.
        So $H = x\left(\frac{u}{g}\right) = -\frac{1}{2}g\left(\frac{u}{g}\right) ^ 2 + u\left(\frac{u}{g}\right) = \frac{u ^ 2}{2g}$.
    \end{solution}
\end{example}

\subsection{Case \texorpdfstring{$F = F(t)$}{} and \texorpdfstring{$F = F(\dot{x})$}{}}
The same method works,
do $\int\,dt$ twice.

\begin{example}
    A particle of mass $m = 1$ is at rest at $x = 0$.
    A force $F = e ^ {-t}$ is switched on at time $t = 0$.
    Find $x(t)$ for $t > 0$.

    \textit{Note:
    expect $\dot{x} > 0$ and $x > 0$ for $t > 0$
    (since $F > 0$).}

    \begin{solution}
        \[
        \ddot{x} = e ^ {-t} \xrightarrow{\int\,dt} \dot{x} = -e ^ {-t} + 1 \xrightarrow{\int\,dt} x = e ^ {-t} + t - 1.
        \]
    \end{solution}
\end{example}

\begin{remark}
    Here $m\ddot{x} = F(\dot{x})$ second-order non linear differential equation.
\end{remark}

\begin{example}
    A particle of mass $m$ is slowed by a frictional force of magnitude $be ^ {av}$
    ($a$ and $b$ are constant).
    Initial speed $u$.
    What time $T$ does the particle travel for before coming to rest?
    
    \begin{solution}
        The particle either moves to the right or the left,
        the force moves in the opposite direction.

        Choosing the first situation.
        So
        \begin{align*}
            m\frac{dv}{dt} = -be ^ {av} &\implies \int e ^ {av}\,dv = -\frac{b}{m}\int\,dt \\
            \intertext{could use a constant of integration,
            fixed by $v(0) = u$.} \\
            \int e ^ {av}\,dv = -\frac{b}{m}\int\,dt &\implies \int_{u}^{0} e ^ {av}\,dv = -\frac{b}{m}\int_{0}^{T}\,dt \\
            &\implies \left[-\frac{1}{a}e ^ {-av}\right]_{u}^{0} = -\frac{bT}{m} \\
            &\implies T = \frac{m}{ab}\left(1 - e ^ {-au}\right).
        \end{align*}
    \end{solution}
\end{example}

\begin{example}
    A mass $M$ falls downwards under gravity,
    a resistive force $bw$,
    where $w$ is its speed.
    Initially $w(0) = 0$,
    find $w(t)$.
    What is the "least upper bound" of this $w$,
    the terminal speed.

    \begin{solution}
        We could use the $x$ axis pointing up or down.
        Use $x$ as downwards since the motion is downwards.

        Then
        \[
        m\dot{v} = mg - bv
        \]
        where $v$ is the velocity,
        here the velocity $v = w$.
        So
        \[
        m\dot{w} = mg - bw \implies \frac{dw}{dt} = g - \frac{b}{m}w \implies \int\frac{1}{g - \frac{b}{m}w}\,dw = \int\,dt
        \]
        which results in $w(t) = \frac{mg}{b}\left(1 - e ^ {-\frac{bt}{m}}\right)$,
        the terminal speed is
        \[
        \liminfty[t]\frac{mg}{b}\left(1 - e ^ {-\frac{bt}{m}}\right) = \frac{mg}{b}.
        \]
    \end{solution}
\end{example}

\subsection{Case \texorpdfstring{$F = F(x)$}{}}
Here $m\ddot{x} = F(x)$.
If $F(x) = bx + c$
($b$ and $c$ are constant),
then $m\ddot{x} - bx = c$ linear,
the solution is the complementary function $+$ the particular integral.
This doesn't work for $F(x)$ non-linear.
For that think of $v = v(x)$.
Then
\[
m\frac{dv}{dt} = m\frac{d}{dt}v(x(t)) = m\frac{dv}{dx}\cdot\frac{dx}{dt} = m\frac{dv}{dx}v = F(x).
\]
This is a first order separable ODE,
solve to get $v(x)$.
Integrating gives
\[
\frac{1}{2}mv ^ 2 = \int F(x)\,dx,
\]
and hence you get $v(x)$.
To get $x(t)$,
use $\frac{dx}{dt} = v(x)$ again first order separable.

\begin{example}
    A mass with $m = 2$ moves in a force with $F = 3x ^ 2$,
    initial conditions $x = 1$ and $v = 1$ at $t = 0$.
    What time $T$ does it take to reach $x = 9$?

    \begin{solution}
        \begin{align*}
            2v\frac{dv}{dx} = 3x ^ 2 &\implies \int 2v\,dv = \int 3x ^ 2\,dx \\
            &\implies v ^ 2 = x ^ 3 + c &\text{($c = 0$ from initial conditions)} \\
            &\implies v ^ 2 = x ^ 3 \\
            &\implies v = x ^ {\frac{3}{2}} &\text{(positive since $x = v = 1$ at $t = 0$)}
        \end{align*}
        Now
        \begin{align*}
            \frac{dx}{dt} = x ^ {\frac{3}{2}} &\implies \int_{1}^{9} x ^ {-\frac{3}{2}}\,dx = \int_{0}^{T}\,dt \\
            &\implies T = \left[-2x ^ {-\frac{1}{2}}\right]_{2}^{9} = -2\left(\frac{1}{3} - 1\right) = \frac{4}{3}.
        \end{align*}
    \end{solution}
\end{example}

\begin{remark}
    For problems with $F = F(v)$ we can use $F(v) = m\frac{dv}{dt}$ as we saw before.
    Sometimes it may be more convenient to use $F(v) = mv\frac{dv}{dx}$,
    in situations where we want $v(x)$ rather than $v(t)$.
\end{remark}

\subsection{Two more general examples}

\begin{example}['Mixed' example]
    A unit mass moves along a line,
    acted on by a force $F(v, t) = -v + e ^ {-t}$ where $v$ is its velocity.
    Given that $v(0) = 0$,
    find $v(t)$.
    \begin{solution}
        The equation of motion is
        \[
        \dot{v} = -v + e ^ {-t},
        \]
        which is a linear ODE for $v(t)$.
        Solving it via an integrating factor $e ^ t$ gives
        \[
        v(t) = te ^ {-t}.
        \]
    \end{solution}
\end{example}

\begin{example}[Rocket: non-constant mass]
    A rocket of mass $m(t)$ burns fuel at a constant rate $\dot{m} = -c$,
    where $c > 0$.
    This produces a constant thrust $F = k|\dot{m}| = kc$.
    The rocket starts at rest at $x = 0$,
    with initial mass $m_0$.
    Find its subsequent position $x(t)$.
    \begin{solution}
        The constant force gives $mv = kct$.
        Now $m(t) = m_0 - ct$,
        so
        \[
        v(t) = \frac{kct}{m_0 - ct} = -k + \frac{km_0}{m_0 - ct}
        \]
        for $0 \leq t < \frac{m_0}{c}$.
        Integrating this gives
        \[
        x(t) = -kt + \frac{km_0}{c}\log\left(\frac{m_0}{m_0 - ct}\right)
        \]
        for $0 \leq t < \frac{m_0}{c}$.
        Note that for small $t$ we have
        \[
        v(t) \approx \frac{kct}{m_0};
        \]
        whereas when $t \rightarrow \frac{m_0}{c}$ we get $v \rightarrow \infty$ and $x \rightarrow \infty$.
    \end{solution}
\end{example}

\subsection{Energy}

\begin{definition}
    Consider the case $F = F(x)$,
    and write $V(x) = -\int F(x)\,dx$.
    Then
    \[
    \frac{d}{dx}\left(\frac{1}{2}mv ^ 2\right) = mv\frac{dv}{dx} = ma = F = -\frac{dV}{dx},
    \]
    and so
    \[
    E = \frac{1}{2}mv ^ 2 + V(x)
    \]
    is constant in time,
    since the chain rule gives
    \[
    \frac{dE}{dt} = \frac{dE}{dx}\dot{x} = 0.
    \]
    This quantity $E$ is the energy of the particle,
    we say that it is conserved.
    The term $K = \frac{1}{2}mv ^ 2$ is the kinetic energy,
    $V(x)$ is the potential energy;
    neither of these is conserved in general.
\end{definition}

\begin{remark}
    \begin{enumerate}[label = (\roman*)]
        \item A force of the form $F(x)$ is said to be conservative,
        since it conserves energy.

        \item The potential $V$,
        and hence the energy $E$,
        are only defined up to an arbitrary constant.

        \item The simplest example is where $F$ is constant,
        so $V$ is linear in $x$.

        \item Using conservation of energy can give us a good picture of the particle's motion,
        even when we cannot get $x(t)$ explicitly.

        \item A point $x$ where $F(x) = 0$,
        or equivalently $V'(x) = 0$,
        is called an equilibrium position.
    \end{enumerate}
\end{remark}

\begin{example}
    A unit-mass particle moves along the $x$-axis,
    subject to a force $F = -2x ^ 3$.
    Derive an expression for its energy $E$,
    such that $E = 0$ if the particle is stationary at $x = 0$.
    Then compute $E$ if its speed is $v = 1$ at $x = 1$.
    \begin{solution}
        Writing $V(x) = -\int F(x)\,dx$.
        \[
        V(x) = -\int -2x ^ 3\,dx = \frac{1}{2}x ^ 4 + C.
        \]
        Now we have
        \begin{align*}
            E &= \frac{1}{2}mv ^ 2 + V(x) \\
            &= \frac{1}{2}v ^ 2 + \frac{1}{2}x ^ 4 + C \\
            \intertext{But $E = 0, v = 0, x = 0$ so $0 = 0 + 0 + C$ hence $C = 0$} \\
            &= \frac{1}{2}\left(v ^ 2 + x ^ 4\right).
        \end{align*}
        Finally,
        we can see that $E$ with $v = 1$ and $x = 1$ is
        \[
        E = \frac{1}{2}\left(v ^ 2 + x ^ 4\right) = \frac{1}{2}(1 + 1) = 1.
        \]
    \end{solution}
\end{example}

\subsection{Motion in a potential}
Consider a particle moving along the $x$-axis,
in a potential $V(x)$.
Except in simple cases,
we won't know exactly what the trajectory $x(t)$ is.
But we can say something about what it looks like,

\begin{example}
    A double-well potential.
    A particle of mass $3$ moves in a potential
    \[
    V(x) = \frac{x ^ 4}{4} - \frac{x ^ 3}{3} - x ^ 2.
    \]
    At what points can the particle remain at rest
    (be in equilibrium)?
    If the particle starts at $x = 1$ with speed $u$,
    how large does $u$ have to be for it to reach $x = -1$?
    \begin{solution}
        We can see that the equilibrium points are at $V'(x) = 0$.
        Now
        \[
        0 = V'(x) = x(x - 2)(x + 1)
        \]
        so there are three equilibria,
        at $x = -1, 0, 2$.
        The potential $V(x)$ is a 'double well',
        with local maximum at $x = 0$,
        and local minima at $x = -1$ and $x = 2$.
        To get from $x = 1$ to $x = -1$,
        the particle needs enough initial kinetic energy to get over the 'hill' at $x = 0$.
        In other words,
        we need
        \[
        \frac{3}{2}u ^ 2 + V(1) > V(0),
        \]
        using $V(1) = -\frac{13}{12}$ and $V(0) = 0$ then gives $u > \sqrt{\frac{13}{18}}$.
    \end{solution}
\end{example}

\subsection{Forcing and resonance}
Say our car is travelling on a corrugated road surface.
This is modelled by adding a forcing term on the right-hand side,
such as $C\sin(pt)$.
Let us first look at a simple example with no damping,
namely
\[
m\ddot{x} + kx = m\sin(pt).
\]
The general solution is,
\[
x(t) = A\cos(\omega t) + B\sin(\omega t) + \frac{\sin(pt)}{\omega ^ 2 - p ^ 2},
\]
provided $|p| \neq \omega$.
We have a mixture of two oscillations with angular frequencies $\omega$ and $p$.
Note $x(t)$ is bounded.
If $p = \omega$,
the solution is
\[
x(t) = A\cos(\omega t) + B\sin(\omega t) - \frac{t\cos(\omega t)}{2\omega},
\]
This is resonance:
the applied force is synchronised with the natural oscillation of the mass-spring system.
The steady-state response is a periodic particular integral $x_{PI}$,
of the form $x(t) = A\cos(pt) + B\sin(pt)$.
Defining $\phi$ by $\tan(\phi) = \frac{2bp}{\omega ^ 2 - p ^ 2}$,
$x_{PI}$ can be written in the form $x_{PI}(t) = C\sin(pt - \phi)$.
The phase difference $\phi$ between forcing and response is due to a lag between when the car feels an upward force from the uneven road,
and when it actually moves up.
\begin{example}
    The behaviour $x(t)$ of a forced damped system is described by the equation $\ddot{x} + 4\dot{x} + x - \sin(t) = 0$.
    Find the steady-state response $x_S(t)$ of this system,
    and compute the phase difference between it and the external force.
    \begin{solution}
        The steady-state response is the periodic particular integral of the differential equation,
        and this is
        \begin{align*}
            x_S(t) = -\frac{1}{4}.
        \end{align*}
        Hence from
        \[
        \tan(\phi) = \frac{2bp}{\omega ^ 2 - p ^ 2}
        \]
        we get
        \[
        \tan(\phi) = \frac{2b}{1 ^ 2 - 1 ^ 2} \implies \phi = \frac{\pi}{2}.
        \]
        The phase difference between this and $\sin(t)$ is $\frac{\pi}{2}$
    \end{solution}
\end{example}

\subsection{Small Oscillations}
Motion near stable equilibria.
A conservative force $F(x) = -V'(x)$ has an equilibrium at $x = x_0$ if $F(x_0) = 0$.
A particle can sit at $x = x_0$,
at rest.
The equilibrium is stable if $x_0$ gives a local minimum of $V$,
so $V''(x_0) > 0$.
If,
on the other hand,
$V''(x_0) \leq 0$,
then the equilibrium is unstable.

\textbf{Simplest case}.
The simplest case is $V(x) = \frac{1}{2}kx ^ 2$,
with $k > 0$
(the simple harmonic oscillator).
There is a stable equilibrium at $x = 0$.
Then $m\ddot{x} = -kx$,
and the system oscillates with angular frequency $\omega = \sqrt{\frac{V''(0)}{m}} = \sqrt{\frac{k}{m}}$.
The period if $\frac{2\pi}{\omega} = 2\pi\sqrt{\frac{m}{k}}$.

\textbf{General potential}.
Suppose that $V(x)$ has a local minimum at $x_0$.
Write $x(t) = x_0 + \epsilon(t)$.
Thus we get approximate simple harmonic motion,
with angular frequency approximately $\omega = \sqrt{\frac{V''(x_0)}{m}}$ and period $\frac{2\pi}{\omega}$.
Note that $\epsilon$ remains small,
so this makes sense.
In effect,
we are approximating the graph of $V(x)$ by a parabola near $x_0$.

\subsection{Momentum and Collisions}
\textbf{Conservation of momentum}.
If there are no external forces,
and $N$ particles with momenta $p_1, p_2, \dotsc, p_N$,
then the total momentum
\[
p = p_1 + p_2 + \dotsi + p_{N}
\]
is conserved,
and does not change when the particles collide.

A collision which conserves energy is said to be elastic,
one which doesn't is inelastic.

\begin{example}
     A particle of mass $2m$ and speed $v$ collides head-on with a particle of mass m at rest.
     The two then continue to move in the same direction,
     with the heavier particle moving at speed $\frac{v}{2}$.
     What is the speed $u$ of the lighter particle,
     and what fraction of the initial energy is lost in the collision?

     \begin{solution}
         By conservation of momentum:
         \[
         2mv = 2m\frac{v}{2} + mu \iff u = v.
         \]
         Now,
         the initial energy was $E_0 = \frac{1}{2}(2m)v ^ 2 = mv ^ 2$,
         the final energy is $E_1 = \frac{1}{2}(2m)\left(\frac{v}{2}\right) ^ 2 + \frac{1}{2}(m)v ^ 2 = \frac{3}{4}mv ^ 2$.
         Hence the fraction of the initial energy lost in the collision is
         \[
         \frac{E_0 - E_1}{E_0} = \frac{mv ^ 2 - \frac{3}{4}mv ^ 2}{mv ^ 2} = 1 - \frac{3}{4} = \frac{1}{4}.
         \]
     \end{solution}
\end{example}

\section{Dynamics in Space}

\subsection{Velocity and relative motion}
A particle is an object with mass $m$,
located at a point in $3$D space.
Its position is described by a position vector $\mbf{r}$
(relative to some origin $\mbf{O}$).
We can use Cartesian coordinates $(x, y, z)$ by letting $\mbf{e}_1, \mbf{e}_2, \mbf{e}_3$ be the standard unit vectors,
and writing $\mbf{r} = x\mbf{e}_1 + y\mbf{e}_2 + z\mbf{e}_3$.
This system of axes is an example of a frame of reference.
For motion in a plane,
use $\mbf{r} = x\mbf{e}_1 + y\mbf{e}_2$.

The trajectory of a particle is given by $\mbf{r}(t) = x(t)\mbf{e}_1 + y(t)\mbf{e}_2 + z(t)\mbf{e}_3$.
Its velocity is $\mbf{v} = \dot{\mbf{r}}$,
where as usual the dot denotes the derivative with respect to time $t$.
If $\mbf{r} = x\mbf{e}_1 + y\mbf{e}_2 + z\mbf{e}_3$,
then $\mbf{v} = \dot{x}\mbf{e}_1 + \dot{y}\mbf{e}_2 + \dot{z}\mbf{e}_3$.
The speed of the particle $v = |\mbf{v}|$ is a real number with $v \geq 0$.
The acceleration is $\mbf{a} = \dot{\mbf{v}} = \ddot{\mbf{r}}$.

\begin{example}
    If $\mbf{r} = t\mbf{e}_1 + \mbf{e}_2 + t ^ 2\mbf{e}_3$ what is $\mbf{v}, \mbf{a}$ and $v$?

    \begin{solution}
        \begin{align*}
            \mbf{v} &= \dot{\mbf{r}} = \mbf{e}_1 + 2t\mbf{e}_3 \\
            \mbf{a} &= \dot{\mbf{v}} = 2\mbf{e}_3 \\
            v &= \sqrt{1 ^ 2 + (2t) ^ 2} = \sqrt{1 + 4t ^ 2}.
        \end{align*}
    \end{solution}
\end{example}

The momentum of a particle with mass $m$ and velocity $\mbf{v}$ is the vector $\mbf{p} = m\mbf{v}$.
If an external force $\mbf{F}$ acts on the particle,
then its equation of motion is $\dot{\mbf{p}} = \mbf{F}$.
If $m$ is constant,
this is equivalent to $m\ddot{\mbf{r}} = \mbf{F}$.

If particles $A$ and $B$ are located at $\mbf{r}_A$ and $\mbf{r}_B$ respectively,
then the position of $B$ relative to $A$ is $\mbf{r}_B - \mbf{r}_A$.
If they are moving with velocities $\mbf{v}_A$ and $\mbf{v}_B$ respectively,
then the velocity of $B$ relative to $A$ is $\mbf{v}_B - \mbf{v}_A$.

\begin{example}
    Two particles have positions $\mbf{r}_A = 2t\mbf{e}_1 + \mbf{e}_2 + t ^ 2\mbf{e}_3$ and $\mbf{r}_B = -t\mbf{e}_1$ respectively,
    what is their relative position and velocity?
    
    \begin{solution}
        The relative position is
        \[
        \mbf{r}_B - \mbf{r}_A = -t\mbf{e}_1 - 2t\mbf{e}_1 - \mbf{e}_2 - t ^ 2\mbf{e}_3 = -3t\mbf{e}_1 - \mbf{e}_2 - t ^ 2\mbf{e}_3.
        \]
        The relative velocity is
        \begin{align*}
            \mbf{v}_B - \mbf{v}_A &= \dot{\mbf{r}_B} - \dot{\mbf{r}_A} \\
            &= (-\mbf{e}_1) - (2\mbf{e}_1 + 2t\mbf{e}_3) \\
            &= -\mbf{e}_1 - 2\mbf{e}_1 - 2t\mbf{e}_3 \\
            &= -3\mbf{e}_1 - 2t\mbf{e}_3.
        \end{align*}
    \end{solution}
\end{example}

\begin{example}
    Bob is going north
    (direction $\mbf{e}_2$)
    with speed $10$,
    and feels a headwind of speed $25$.
    What is the wind velocity relative to the ground?
    If Bob now goes east
    (direction $\mbf{e}_1$)
    at speed $15$,
    what wind does he feel?

    \begin{solution}
        \[
        \mbf{v}_w - \mbf{v}_B = -25\mbf{e}_2 \iff \mbf{v}_w = -15\mbf{e}_2,
        \]
        so the wind velocity relative to the ground is $-15\mbf{e}_2$.

        Now we have that $\mbf{v}_B = 15\mbf{e}_1$,
        so
        \[
        \mbf{v}_w - \mbf{v}_B = -15\mbf{e}_2 - 15\mbf{e}_1
        \]
        hence he feels a speed of
        \[
        |\mbf{v}_w - \mbf{v}_B| = |-15\mbf{e}_2 - 15\mbf{e}_1| = \sqrt{15 ^ 2 + 15 ^ 2} = 15\sqrt{2}
        \]
        in the direction $-\frac{(\mbf{e}_1 + \mbf{e}_2)}{\sqrt{2}}$.
    \end{solution}
\end{example}

\subsection{Sliding down an inclined plane}
A mass $m$ is free to slide down a plane which is inclined at an angle $\alpha$ to the horizontal.
Gravity acts downwards,
there is a frictional force
(directed up the slope)
with magnitude $\mu \mbf{N}$,
where $\mu$ is a positive constant,
and $\mbf{N}$ is the normal force exerted by the inclined plane on the particle.
(The direction of $\mbf{N}$ is perpendicular to the plane).
Starting from rest,
how far does the particle travel in time $t$?
How large does $\alpha$ have to be for the mass to move at all?

\begin{solution}
    Take $\mbf{e}_1$ to point down the slope.
    Take $\mbf{e}_2$ to be perpendicular to $\mbf{e}_1$.

    Motion is in the $\mbf{e}_1$ direction.
    The weight of the particle is downwards,
    with magnitude $mg$.
    Hence its component in the $\mbf{e}_2$-direction is $-mg\cos{\alpha}$,
    and so $\mbf{N} = mg(\cos{\alpha})\mbf{e}_2$.
    Thus,
    the total force in the $\mbf{e}_1$ direction is
    \[
    F = -\mu mg\cos{\alpha} + mg\sin{\alpha}
    \]
    which is constant;
    so
    \[
    x(t) = \frac{1}{2}g(\sin{\alpha} - \mu\cos{\alpha})t ^ 2,
    \]
    provided $\tan{\alpha} > \mu$,
    else the particle does not move,
    and $x(t) = 0$ for all $t$.
\end{solution}

\subsection{Motion in an electromagnetic field}

A particle with mass $m$ and electric charge $q$,
placed in an electric field $\mbf{E}$ and a magnetic field $\mbf{B}$,
feels the Lorentz force
\[
\mbf{F} = q(\mbf{E} + \mbf{v} \times \mbf{B}).
\]

\begin{example}
    Take $\mbf{E} = \mbf{0}$ and $\mbf{B} = B\mbf{e}_3$ with $B$ constant,
    and with initial conditions $\mbf{r}(0) = \mbf{0}$ and $\dot{\mbf{r}}(0) = u\mbf{e}_1$.
    The equation of motion is
    \begin{align*}
        m\ddot{\mbf{r}} &= \mbf{F} \\
        &= q(\mbf{E} + \mbf{v} \times \mbf{B}) \\
        &= q(\mbf{v} \times \mbf{B}) \\
        &= Bq\left(\begin{pmatrix}
            \dot{x} \\ \dot{y} \\ \dot{z}
        \end{pmatrix} \times \begin{pmatrix}
            0 \\ 0 \\ 1
        \end{pmatrix}\right) \\
        &= Bq\begin{pmatrix}
            \dot{y} \\ \dot{x} \\ 0
        \end{pmatrix}
    \end{align*}
    so we have
    \[
    m\begin{pmatrix}
        \ddot{x} \\
        \ddot{y} \\
        \ddot{z}
    \end{pmatrix} = \begin{pmatrix}
        Bq\dot{y} \\
        Bq\dot{x} \\
        0
    \end{pmatrix}
    \]
    which is equivalent to
    \[
    m\ddot{x} = Bq\dot{y} \implies m\dot{x} = Bqy + C
    \]
    \[
    m\ddot{y} = Bq\dot{x} \implies m\dot{y} = -Bqx + D
    \]
    and
    $\ddot{z} = 0 \implies z = 0$.

    Applying the initial conditions $x(0) = 0, \dot{x}(0) = u, y(0) = 0, \dot{y}(0) = 0$
    \[
    m\dot{x}(0) = Bqy(0) + C \iff C = mu
    \]
    and
    \[
    m\dot{y}(0) = -Bq(0) + D \iff D = 0.
    \]
    Hence we get
    \[
    m\dot{x} = Bqy + mu
    \]
    \[
    m\dot{y} = -Bqx.
    \]
    So we have two simultaneous equations,
    solving for $x(t)$
    \begin{align*}
        m\dot{x} &= Bqy + mu \\
        &\implies \\
        m\ddot{x} &= Bq\dot{y} \\
        &= Bq\left(-\frac{Bq}{m}x\right) \\
        &= -\frac{(Bq) ^ 2}{m}x \\
        &\iff \\
        \ddot{x} + \left(\frac{Bq}{m}\right) ^ 2x &= 0 \\
        &\implies \\
        x(t) &= C\sin\left(\frac{Bq}{m}t\right) + D\cos\left(\frac{Bq}{m}t\right)
    \end{align*}
    applying the initial conditions we get
    \[
    x(t) = \frac{Bq}{m}u\sin\left(\frac{Bq}{m}t\right)
    \]
    then applying the equation for $x(t)$,
    integrating and applying initial conditions we get
    \[
    y(t) = \frac{Bq}{m}u\left(\cos\left(\frac{Bq}{m}t\right) - 1\right).
    \]
\end{example}

\subsection{Conservative forces and potential energy}
The kinetic energy of a particle of mass $m$ and speed $v$ is $\frac{1}{2}mv ^ 2$.
If the force $\mbf{F}$ depends only on position $\mbf{r}$,
and if there exists a real function $V(\mbf{r})$ such that
\[
\mbf{F} = \left(-\pd[V]{x}, -\pd[V]{y}, -\pd[V]{z}\right) = -\pd[V]{x}\mbf{e}_1 -\pd[V]{y}\mbf{e}_2 - \pd[V]{z}\mbf{e}_3,
\]
then the force is conservative,
and $V$ is called the potential energy.
The total energy is
\[
E(\mbf{r}, \mbf{v}) = \frac{1}{2}mv ^ 2 + V(\mbf{r}).
\]

\begin{proposition}
    In a conservative force field,
    the total energy is conserved.

    \begin{proof}
        Write $\mbf{v} = (v_x, v_y, v_z)$.
        Then
        \[
        \frac{dV}{dt} = \pd[V]{x}v_x + \pd[V]{y}v_y + \pd[V]{z}v_z = -\mbf{F} \cdot \mbf{v},
        \]
        by the chain rule.
        Secondly,
        \[
        \frac{d}{dt}\left(\frac{1}{2}mv ^ 2\right) = \frac{d}{dt}\left(\frac{1}{2}m(v_x ^ 2 + v_y ^ 2 + v_z ^ 2)\right) = m(v_x\dot{v}_x + v_y\dot{v}_y + v_z\dot{v}_z) = m\mbf{a} \cdot \mbf{v}.
        \]
        And so
        \[
        \frac{dE}{dt} = (m\mbf{a} - \mbf{F}) \cdot \mbf{v} = 0.
        \]
    \end{proof}
\end{proposition}

\begin{remark}
    The force $\mbf{F} = F_x\mbf{e}_1 + F_y\mbf{e}_2 + F_z\mbf{e}_3$ is conservative if and only if
    \[
    \pd[F_z]{y} = \pd[F_y]{z},\qquad\pd[F_x]{z} = \pd[F_z]{x},\qquad\pd[F_y]{x} = \pd[F_x]{y}.
    \]
\end{remark}

\begin{example}
    Take $\mbf{F} = (z - y)\mbf{e}_1 + (\alpha x - z)\mbf{e}_2 + (\beta x - y)\mbf{e}_3$,
    where $\alpha$ and $\beta$ are constants.
    For which value(s) of these constants,
    if any,
    is $\mbf{F}$ conservative?
    For these values,
    find $V$.

    \begin{solution}
        We can start by identifying that
        \[
        \pd[V]{x} = z - y \implies V(x, y, z) = (y - z)x + g(y, z)
        \]
        this is because $\mbf{F} = \left(-\pd[V]{x}, -\pd[V]{y}, -\pd[V]{z}\right)$ so $-\pd[V]{x} = (z - y)$.

        Now we can apply this $V$ since we know that
        \[
        \pd[V]{y} = z - \alpha x = (0 - 1)x + h(z) \iff h(z) - x
        \]
        so $h(z) = z$ and $\alpha = -1$.

        Similarly
        \[
        \pd[V]{z} = y - \beta x = -x + f(y) \iff f(y) - x
        \]
        so $f(y) = y$ and $\beta = 1$.

        Since $\pd{y}g(y, z) = h(z) = z$ and $\pd{z}g(y, z) = f(y) = y$ we have naturally that $g(y, z) = yz$.

        Using these we can see that
        \[
        V(x, y, z) = (y - z)x + yz = yx - xz + yz + C
        \]
        where $C$ is an arbitrary constant.
    \end{solution}
\end{example}

\subsection{The simple pendulum}













\end{document}