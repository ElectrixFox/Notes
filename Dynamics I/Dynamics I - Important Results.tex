\documentclass[10pt, a4paper]{article}
\usepackage{preamble}

\title{Dynamics I \\
    \large Important Results}
\author{Luke Phillips}
\date{April 2025}

\begin{document}

\maketitle

\newpage

\tableofcontents

\newpage

\section{Solving for force}
When we have $F = F(t)$ we get $m\ddot{x} = F(t)$,
that can be solved by direct integration,
i.e.
\[
m\ddot{x} = F(t).
\]

When we have $F = F(v)$ then we get
\[
m\dot{v} = F(v)
\]
we can then see
\[
m\dot{v} = F(v) \implies m\frac{dv}{dt} = F(v) \implies \frac{1}{F(v)}\,dv = \frac{1}{m}\,dt.
\]

To solve $F = F(x)$ then we get
\[
F(x) =m\ddot{x} = m\overbrace{\frac{d}{dt}v(x(t))}^{= \frac{d}{dx}(v(x))\cdot\frac{d}{dt}x(t)} = m\frac{dv}{dx}\frac{dx}{dt} = m\frac{dv}{dx}v
\]
which can be shown as
\[
F(x) = m\frac{dv}{dx}v \implies mv\frac{dv}{dx} = F(x) \implies mv\,dv = F(x)\,dx \implies \frac{1}{2}mv ^ 2 = \int F(x)\,dx.
\]
After finding $\frac{1}{2}mv ^ 2 = \int F(x)\,dx$ we can get
\[
\int\frac{1}{\sqrt{\frac{2}{m}\int F(x)\,dx}}\,dx = t + C.
\]

\newpage

\section{Energy}
We have $F = F(x)$ then $V(x) = -\int F(x)\,dx$.
\[
\frac{d}{dx}\left(\frac{1}{2}mv ^ 2\right) = mv\frac{dv}{dx} = ma = F = -\frac{dV}{dx},
\]
giving us
\[
E = \frac{1}{2}mv ^ 2 + V(x)
\]
then differentiating
\[
\frac{dE}{dt} = \frac{dE}{dx}\dot{x} = 0.
\]

\newpage

\section{Potential}

\textbf{Equilibrium}

the equilibrium are when the particle is at rest.
We can find the equilibria at $V'(x) = 0$.

To find if there is enough energy for a particle to reach a specific speed we get
\[
\frac{1}{2}mu ^ 2 + V(x_0) > V(x_1)
\]
with $x_0$ the initial point and $x_1$ the maxima.

An equilibrium is stable if $V''(x_0) > 0$.

\newpage

\section{Simple harmonic oscillator}
The potential is $V(x) = -\int F = \frac{1}{2}kx ^ 2$,
\[
x(t) = A\cos(\omega t) + B\sin(\omega t),
\]
with $\omega = \sqrt{k / m}$ the angular frequency,
\[
T = \frac{2\pi}{\omega} = 2\pi\frac{m}{k}.
\]

\subsection{Small oscillations}
A conservative force $F(x) = -V'(x)$.

To find the period of small oscillations $\omega = \sqrt{V''(x_0) / m}$ where we approximate things like $\sin{x}$ to $x$.

\end{document}